\documentclass{report}

% IMPORTS
\usepackage{ntheorem} % Theorem/Proof Umgebungen 
\usepackage[ngerman]{babel}
\usepackage{babelbib}
\usepackage[utf8x]{inputenc}
\usepackage{amsfonts}
\usepackage{amsmath}
\usepackage{amssymb}
\usepackage{amstext}
\usepackage{enumerate} % für kleine römische Nummerierung
\usepackage{url} % Für Hyperlinks
\usepackage[singlespacing]{setspace} % einfacher Zeilenabstand 
\usepackage{abstract}

% Für andere Kapitelüberschriften:
\usepackage{titlesec}
\titleformat{\chapter}
  {\normalfont\LARGE\bfseries}{\thechapter}{1em}{}
\titlespacing*{\chapter}{0pt}{3.5ex plus 1ex minus .2ex}{2.3ex plus .2ex}

\titleformat{\section} 
  {\normalfont\Large\bfseries}{\thesection}{0.8em}{}
\titlespacing*{\section}{0pt}{3.5ex plus 1ex minus .2ex}{2.3ex plus .2ex}

%
\selectbiblanguage{ngerman}
% Seitenränder 
\usepackage[a4paper]{geometry}
%zuvor [a4paper, , left=3cm, right=3cm, top=2cm]

% Für klickbares Inhaltsverzeichnis:
\usepackage{hyperref}
 
% NEWCOMMANDS
% Syntax: \newcommand{\shortcut}[AnzahlArgumente]{\wasrauskommensoll #fürArgumentNummer}

% Nützliches

\newcommand{\qed}{\hfill $\square$}
\newcommand{\qexampled}{\hfill $\lozenge$}
\newcommand{\norm}[1]{ \lvert\lvert #1 \rvert\rvert}
\newcommand{\abs}[1]{\left| #1 \right|}
\newcommand{\prob}[1]{P\left(\left\{ #1 \right\}\right)} 



% Shortcuts für Zahlenbereiche:
\newcommand{\N}{\mathbb{N}} % Natürliche Zahlen 
\newcommand{\R}{\mathbb{R}} % Reelle Zahlen
\newcommand{\Z}{\mathbb{Z}} % Ganze Zahlen
\newcommand{\C}{\mathbb{C}} % Komplexe Zahlen
\newcommand{\Q}{\mathbb{Q}} % Rationale Zahlen 

% Konvergenzarten
\newcommand{\schwach}{\overset{w}{\longrightarrow}} % Notation für schwache Konvergenz von Zufallsvariablen 
\newcommand{\stochastisch}{\overset{st}{\longrightarrow}} % Notation für stochastische Konvergenz von Zufallsvariablen
\newcommand{\fastsicher}{\overset{f.s.}{\longrightarrow}} % Notation für fast sichere Konvergenz von Zufallsvariablen 



% THEOREMSTYLES

\theorembodyfont{\upshape} 
\theoremstyle{changebreak} %Sorgt für Zeilenumbruch nach Titel und Nummer vor Name

\newtheorem{theorem}{Satz.} % Sätze
\numberwithin{theorem}{chapter}

\newtheorem{mydef}[theorem]{Definition.} % Definitionen
\numberwithin{theorem}{chapter} %sorgt für stringente Nummerierung

\newtheorem{example}[theorem]{Beispiel.} % Beispiele
\numberwithin{theorem}{chapter} %sorgt für stringente Nummerierung

\newtheorem{proposition}[theorem]{Proposition.} % Propositionen
\numberwithin{theorem}{chapter} %sorgt für stringente Nummerierung

\newtheorem{remark}[theorem]{Bemerkung.} %Bemerkungen
\numberwithin{theorem}{chapter} %sorgt für stringente Nummerierung

\newtheorem{lemma}[theorem]{Lemma.} %Lemmata
\numberwithin{theorem}{chapter} %sorgt für stringente Nummerierung

\newtheorem{corollary}[theorem]{Korollar.} %Korollare 
\numberwithin{theorem}{chapter} %sorgt für stringente Nummerierung

\newtheorem*{proof*}{Beweis.} %Beweise ohne Nummerierung! 

\setlength\parindent{0pt} %Kein Einzug bei neuem Paragraphen. 

\begin{document}
\begin{titlepage}
    \begin{center}
        \vspace*{2cm}
        \Huge
        \textbf{Der Satz von Itô-Nisio}
  
        \vfill
        \Large
        \textbf{Jonas Köppl}
  
        \vfill
  
        Bachelorarbeit im Studiengang Mathematik \\
        \ 
        \newline 
        Betreuer: Prof. Dr. Thomas Müller-Gronbach
        
        \vspace{0.8cm}
  
  
        Universität Passau\\
        Fakultät für Informatik und Mathematik\\
        Lehrstuhl für Mathematische Stochastik und ihre Anwendungen
        \ 
        \newline 
        
        Abgegeben am: \today
  
    \end{center}
 \end{titlepage}
\tableofcontents
\thispagestyle{empty}

In diesem ersten Kapitel befassen wir uns zunächst mit einigen grundlegenden Eigenschaften Borelscher $\sigma$-Algebren und darauf definierten Wahrscheinlichkeitsmaßen.
Da die algebraische Struktur eines Banachraums hierfür zunächst nicht von Bedeutung ist lassen sich viele Ergebnisse im allgemeineren Kontext von vollständigen metrischen Räumen zeigen. 
\textbf{TODO}
\setcounter{page}{2}
\chapter{Zufallsvariablen in Banachr"aumen}

\textbf{TODO: Einleitung}

\section{Borelmengen in metrischen Räumen}
Für einen metrischen Raum $(X,d)$ bezeichne im Folgenden $\mathcal{B}(X)$ die Borel-$\sigma$-algebra in $X$. 
Zudem werden für $x \in X$ und $r>0$ mit $B(x, r)$ bzw. $\overline{B}(x,r)$ die offene bzw. abgeschlossene Kugel um $x$ mit Radius $r$ bezeichnet.
\begin{proposition}
    Sei $(X,d)$ ein separabler metrischer Raum. Dann gilt
    \begin{align*}
        \mathcal{B}(X) = \sigma(\{B(x,r): x \in X, r > 0 \}) = \sigma(\{\overline{B}(x,r): x \in X, r > 0 \}). 
    \end{align*}
\end{proposition}
\begin{proof*}
    Setze 
    \begin{align*}
        \mathcal{A}_1 &:= \sigma(\{B(x,r): x \in X, r > 0 \}), \\\ 
        \mathcal{A}_2 &:= \sigma(\{\overline{B}(x,r): x \in X, r > 0 \}). 
    \end{align*}
    Man sieht leicht ein, dass $\mathcal{A}_2 = \mathcal{A}_1 \subseteq \mathcal{B}(X)$. Zu zeigen bleibt also nur die Inklusion $\mathcal{B}(X) \subseteq \mathcal{A}_1$.
    Sei dazu $U \subseteq X$ offen und $x \in U$. Nach Voraussetzung existiert eine abzählbare dichte Teilmenge $D \subseteq X$. Definiere 
    \begin{align*}
        R := \{(y,r) : y \in U \cap D, r > 0, r \in \Q, B(y,r) \subseteq U \}.
    \end{align*}
    Dann ist $R$ abzählbar und da $D$ dicht in $X$ liegt gilt $U = \bigcup_{(y,r) \in R}B(y,r)$. 
    Also gilt $U \in \mathcal{A}_1$ und da $\mathcal{B}(X)$ von den offenen Teilmengen von $X$ erzeugt wird folgt die Behauptung. \qed
\end{proof*}

\begin{proposition}
    Für $i \in \N$ sei $(X_i, d_i)$ ein separabler metrischer Raum. Dann gilt
    \begin{align*}
        \mathcal{B}(X_1 \times X_2 \times ...) = \otimes_{i=1}^{\infty}\mathcal{B}(X_i)
    \end{align*}
\end{proposition}

\begin{proof*}
    Setze $X:= \times_{k \in \N}X_k$ und bezeichne $p_k: X \to X_k$ die Projektion auf die k-te Komponente. Betrachte das Mengensystem
    \begin{align*}
        \mathcal{E} :&= \{ \bigcap_{k \in K}p_k^{-1}(O) | \forall k \in K: O_k \subseteq X_k \text{ offen}, K \subseteq \N \text{ endlich}\}. 
    \end{align*}
    Offensichtlich gilt $\otimes_{k \in \N}\mathcal{B}(X_k) = \sigma(\mathcal{E})$. 
    Ferner ist $X$ ein separabler metrischer Raum und $\mathcal{E}$ eine Basis der Produkttopologie auf $X$, vgl. \cite{querenburg}[3.7]. 
    Also lässt sich jede offene Menge $O \subset X$ als abzählbare Vereinigung von Elementen aus $\mathcal{E}$ darstellen.  
    Dies impliziert 
    $$
    \mathcal{B}(X) = \sigma(\mathcal{E}) = \otimes_{k \in N}\mathcal{B}(X_k).
    $$
    \qed
\end{proof*}

\section{Borelmaße auf metrischen Räumen}

Bis auf weiteres sei $(X,d)$ ein metrischer Raum mit Borel-$\sigma$-algebra $\mathcal{B}(X)$. 
Im Folgenden Abschnitt beschäftigen wir uns mit Maßen auf $\mathcal{B}(X)$, welche teilweise auch als \textit{Borel-Maße} bezeichnet werden. 
Die Bezeichnung wird in der Literatur allerdings nicht einheitlich verwendet. 

\begin{mydef}
    Ein Maß $\mu$ auf $\mathcal{B}(X)$ heißt \textit{regulär} , falls
    \begin{align*}
        \forall B \in \mathcal{B}(X): \quad \mu(B) &= \sup\{\mu(C): C \subseteq B, \ C \text{ abgeschlossen} \} \\\
                                                   &= \inf\{\mu(O): B \subseteq O, \ O \text{ offen} \}.  
    \end{align*}  
\end{mydef}

\begin{proposition}
    Sei $\mu$ ein Wahrscheinlichkeitsmaß auf $\mathcal{B}(X)$. Dann ist $\mu$ regulär. 
\end{proposition}

\begin{proof*}
    \textbf{TODO}
\end{proof*}

\begin{mydef}
    Ein Maß $\mu$ auf $\mathcal{B}(X)$ heißt \textit{straff}, falls es für alle $\varepsilon > 0$ eine kompakte Menge $K \subseteq X$ gibt mit 
    \begin{align*}
        \mu(K) \geq 1 - \varepsilon. 
    \end{align*}

\end{mydef}

\begin{corollary}
    Sei $\mu$ ein straffes Wahrscheinlichkeitsmaß auf $\mathcal{B}(X)$. Dann gilt
    \begin{align*}
        \forall A \in \mathcal{B}(X): \quad \mu(A) = \sup\{\mu(K): K \subseteq A,\ K \text{ kompakt}\}. 
    \end{align*}
\end{corollary}

\begin{proof*}
    Sei $A \in \mathcal{B}(X)$ und $\varepsilon > 0$. Wegen der Straffheit von $\mu$ existiert eine kompakte Menge $K_{\varepsilon} \subseteq X$ mit $\mu(K_{\varepsilon}) \geq 1 - \frac{\varepsilon}{2}$,
    und da $\mu$ nach Proposition 1.4 regulär ist gibt es eine abgeschlossene Menge $C \subseteq A$ mit $\mu(C) > \mu(A) - \frac{\varepsilon}{2}$. Dann ist die Menge $K_{\varepsilon} \cap C$ wiederum kompakt und es gilt
    \begin{align*}
        \mu(A) \geq \mu(K_{\varepsilon} \cap C) > \mu(C) - \frac{\varepsilon}{2} > \mu(A) - \varepsilon. 
    \end{align*} 
    \qed
\end{proof*}

\begin{remark}
    Ein Wahrscheinlichkeitsmaß $\mu$ auf $\mathcal{B}(X)$ mit der Eigenschaft
    \begin{align*}
        \forall A \in \mathcal{B}(X): \quad \mu(A) = \sup\{\mu(K): K \subseteq A, \ K \text{ kompakt}\}. 
    \end{align*}
    wird auch als \textit{Radon-Wahrscheinlichkeitsmaß} oder \textit{Radon-Maß} bezeichnet.
\end{remark}

\begin{proposition}
    Sei $(X,d)$ ein vollständiger separabler metrischer Raum. Dann ist jedes Wahrscheinlichkeitsmaß $\mu$ auf $\mathcal{B}(X)$ straff.
\end{proposition}

Wir verwenden zum Beweis der Proposition die folgende Charakterisierung kompakter Teilmengen metrischer Räume. Ein Beweis findet sich etwa in \cite{amann}. 

\begin{lemma}
    Sei $(X,d)$ ein metrischer Raum. Eine Menge $K \subseteq X$ ist genau dann kompakt, wenn sie die folgenden beiden Eigenschaften erfüllt:
    \begin{enumerate}[(i)]
        \item $K$ ist vollständig,
        \item $K$ ist total-beschränkt, d.h.
        \begin{align*}
            \forall  \varepsilon > 0 \ \exists x_1,...,x_n \in K: \  K \subseteq \cup_{i=1}^n B(x_i, \varepsilon). 
        \end{align*} 
\end{enumerate}
\end{lemma}

\begin{proof*}
    \textbf{TODO}
    Sei $\varepsilon > 0$. Nach Voraussetzung existiert eine abzählbare dichte Teilmenge $D = \{x_1, x_2,...\}$ von $X$. Also gilt insbesondere für $q \in \N $
    \begin{align*}
        \bigcup_{i\in \N}\overline{B}(x_i, 2^{-q}) = X.
    \end{align*}
    Wegen der $\sigma$-Stetigkeit von $\mu$ existiert also ein $N_q \in \N$ mit 
    \begin{align*}
        \mu(\cup_{i=1}^{N_q}\overline{B}(x_i, 2^{-q}) \geq 1 - \varepsilon 2^{-q}. 
    \end{align*}
    Setze nun 
    \begin{align*}
        K := \bigcap_{q \in \N}\bigcup_{i=1}^{N_q}\overline{B}(x_i, 2^{-q}). 
    \end{align*}
    Dann ist $K$ als Schnitt abgeschlossener Teilmengen abgeschlossen, und da $X$ vollständig ist, folgt daraus bereits die Vollständigkeit von $K$. 
    Ferner ist $K$ total-beschränkt, denn zu $\varepsilon > 0$ existiert ein $q \in \N$ mit $2^{-q} < \varepsilon$ und $K \subseteq \cup_{i=1}^{N_q}B(x_i, 2^{-q}) \subseteq \cup_{i=1}^{N_q}B(x_i, \varepsilon)$. 
    Zudem gilt
    \begin{align*}
        \mu(K)  = 1 - \mu(\cup_{q \in \N}\cap_{i=1}^{N_q}\overline{B}(x_i, 2^{-q})^c) 
                &\geq 1 - \sum_{q=1}^{\infty} \mu(\cap_{i=1}^{N_q}\overline{B}(x_i, 2^{-q})^c) \\\
                &\geq 1 - \sum_{q=1}^{\infty} \varepsilon 2^{-q} = 1 - \varepsilon.
    \end{align*}
    Also ist $\mu$ straff. \qed
\end{proof*}

\begin{proposition}
    Sei $(X,d)$ ein vollständiger metrischer Raum und $\mu$ ein Wahrscheinlichkeitsmaß auf $\mathcal{B}(X)$. Dann sind äquivalent
    \begin{enumerate}[(i)]
        \item $\mu$ ist straff.
        \item Es gibt eine separable Teilmenge $E \subseteq X$ mit $\mu(E) = 1$. 
    \end{enumerate}
\end{proposition}
\begin{proof*}
    zu (i) $\Rightarrow$ (ii): Für alle $n \in \N$ existiert $K_n \subseteq X$ kompakt mit $\mu(K_n) \geq 1 - \frac{1}{n}$, o.E. gelte $K_n \subseteq K_{n+1}$. Es folgt 
    \begin{align*}
        \mu\big(\cup_{n=1}^{\infty}K_n\big) = \lim_{n \to \infty}\mu\big(K_{n+1}\big) = 1. 
    \end{align*}
    Da kompakte Teilmengen metrischer Räume insbesondere separabel sind, ist $E := \cup_{n=1}^{\infty}K_n$ als abzählbare Vereinigung separabler Mengen ebenso separabel. 
    \newline 
    zu (ii) $\Rightarrow$ (i): 
    Analog zum Beweis von Proposition 1.8. \qed
\end{proof*}
\section{Meßbare Vektorräume}
\begin{mydef}
    Sei $X$ ein Vektorraum und $\mathcal{C}$ eine $\sigma$-Algebra auf $X$. Das Tupel $(X, \mathcal{C})$ heißt \textit{messbarer Vektorraum}, falls
    \begin{enumerate}[(a)]
        \item Die Abbildung 
        \begin{align*}
            + : X \times X \to X, \quad (x,y) \mapsto x + y
        \end{align*}
        ist $\mathcal{A}\otimes \mathcal{C}/\mathcal{C}$-messbar, und
        \item die Abbildung 
        \begin{align*}
            \cdot : \R \times X \to X, \quad  (\alpha, x) \mapsto \alpha x
        \end{align*}
        ist $\mathcal{B}(\R) \otimes \mathcal{C}/\mathcal{C}$-messbar. 
    \end{enumerate}
\end{mydef}

\begin{remark}
    Sei $(X, \mathcal{C})$ ein messbarer Vektorraum. Dann gilt
    \begin{enumerate}[(i)]
        \item Für alle $\alpha \in \R$ ist die Abbildung 
            $$f_{\alpha}: X \to X, x \mapsto \alpha x$$
        $\mathcal{C}/\mathcal{C}$-messbar. 
        \item Für alle $y \in X$ ist die Abbildung 
            $$g_y: X \to X, x \mapsto x + y$$
        $\mathcal{C}/\mathcal{C}$-messbar.
    \end{enumerate}
\end{remark}

Da die Komposition messbarer Abbildungen wiederum messbar ist erhält man unmittelbar
\begin{proposition}
    Sei $(X, \mathcal{C})$ ein messbarer Vektorraum und $(\Omega, \mathcal{A})$ ein messbarer Raum. 
    Sind $X,Y: \Omega \to X$ zwei $\mathcal{A}/\mathcal{C}$-messbare Abbildungen und $\alpha, \beta \in \R$, so ist auch $\alpha X + \beta Y$ $\mathcal{A}/\mathcal{C}$-messbar. 
\end{proposition}

\begin{proof*}
    Man beachte, dass für beliebige  messbare Räume $(\Omega_1, \mathcal{A}_1), (\Omega_2, \mathcal{A}_2)$, Mengen $A \in \mathcal{A}_1 \otimes \mathcal{A}_2$ und $\omega_1 \in \Omega_1$
    \begin{align*}
    A(\omega_1) = \{ \omega_2 : (\omega_1,\omega_2) \in A \} \in \mathcal{A}_2
    \end{align*}
    gilt. \qed
\end{proof*}
\begin{proposition}
    Sei $X$ ein separabler Banachraum. Dann ist $(X, \mathcal{B}(X))$ ein messbarer Vektorraum.
\end{proposition}
\begin{proof*}
    Nach Proposition 1.2 gilt $\mathcal{B}(X \times X) = \mathcal{B}(X) \otimes \mathcal{B}(X)$ und 
    $\mathcal{B}(\R \times X) = \mathcal{B}(\R) \otimes \mathcal{B}(X)$. Ferner sind die Abbildungen 
    \begin{align*}
        + &: X \times X \to X, \quad (x,y) \mapsto x + y, \\\
        \cdot &: \R \times X \to X, \quad (\alpha, x) \mapsto \alpha  x
    \end{align*}
    stetig bzgl. der jeweiligen Produkttopologien und somit insbesondere $\mathcal{B}(X \times X)/\mathcal{B}(X)$- bzw. 
    $\mathcal{B}(\R \times X)/\mathcal{B}(X)$-messbar. \qed
\end{proof*}

\begin{example}
    Für $d \in \N$ ist $(\R^d, \mathcal{B}(\R^d))$ ein messbarer Vektorraum. 
\end{example}
Im Folgenden sei $(X, \norm{\cdot})$ ein Banachraum und $(X', \norm{\cdot}_{op})$ der zugehörige Dualraum. 
\begin{proposition}
    Sei $\emptyset \neq \Gamma \subseteq X'$. Dann ist $(X, \sigma({\Gamma}))$ ein messbarer Vektorraum. 
\end{proposition}

\begin{proof*}
    \textbf{TODO}
\end{proof*}

\begin{proposition}
    Sei $X$ ein separabler Banachraum. Dann gilt $\sigma(X') = \mathcal{B}(X)$. 
\end{proposition}
\section{Zufallsvariablen mit Werten in Banachräumen}
Sei $(\Omega, \mathcal{A}, P)$ ein vollständiger Wahrscheinlichkeitsraum und $E$ ein Banachraum mit Norm $\norm{\cdot}$. 
\begin{mydef}
    Eine Abbildung $X: \Omega \to E$ heißt \textit{(Radon-)Zufallsvariable} falls 
    \begin{enumerate}[(a)]
        \item $X$ ist $\mathcal{A}/\mathcal{B}(E)$-messbar und
        \item es existiert ein separabler Untervektorraum $E_0 \subseteq E$ mit $P(\{X \in E_0\}) = 1$. 
    \end{enumerate}
\end{mydef}
\textbf{TODO: Erklärung warum Radon, Rückgriff auf Abschnitt 1.2}
\begin{proposition}
    \textbf{TODO: Charakterisierung von Radon-Zufallsvariablen}
\end{proposition}
\textbf{TODO: Einführung einfache Funktionen im Banach-Kontext}
\begin{proposition}
    \textbf{TODO: Summen von Radon-Variablen sind messbar, Approximation durch einfache Funktionen, Grenzwerte sind messbar}
\end{proposition}

Bezeichne $\mathcal{L}_0(\Omega, \mathcal{A}, P; E)$ den Raum der $E$-wertigen Radon-Zufallsvariablen auf $(\Omega,\mathcal{A},P)$ und sei $L_0(E)$ der Raum der Äquivalenzklassen bezüglich fast sicherer Gleichheit. 
Falls klar ist welcher Wahrscheinlichkeitsraum gemeint ist, so schreiben wir auch $\mathcal{L}_0(E)$ oder $L_0(E)$. 


\section{Charakteristische Funktionale}

\textbf{TODO}



\chapter{Konvergenzarten}
\textbf{Notation und Konventionen}\newline 
Sei $(E, \norm{\cdot})$ ein Banachraum und $(\Omega, \mathcal{A}, P)$ ein vollständiger Wahrscheinlichkeitsraum. 
\section{Fast sichere Konvergenz}
\begin{mydef}
    Seien $X, X_1, X_2,...$ $E$-wertige Zufallsvariablen. Die Folge $(X_n)_{n \in \N}$ \textit{konvergiert fast sicher} gegen $X$, falls
    $$
        P(\{ lim_{n \to \infty} \norm{X_n - X} = 0 \}).
    $$
Notation: $X_n \fastsicher X$. 
\end{mydef}
\section{Verteilungskonvergenz und charakteristische Funktionale}
\textbf{Konventionen und Notation} 
\newline 
Im Folgenden sei $(X,d)$ ein vollständiger und separabler metrischer Raum mit
$$d(x) \leq 1 \quad \text{für alle } x \in X.$$ 

\begin{mydef}
    Eine Folge $(\mu_n)_{n \in \N}$ von Wahrscheinlichkeitsmaßen auf $\mathcal{B}(X)$ heißt \textit{schwach konvergent} 
    gegen ein Wahrscheinlichkeitsmaß $\mu$ auf $\mathcal{B}(X)$, falls 
    $$
        \forall f \in C_b(X): \quad \lim_{n \to \infty} \int_Xfd\mu_n = \int_X fd\mu . 
    $$
    Bezeichnung: $\mu_n \rightharpoonup \mu$. 
    \newline
    Eine Folge $(X_n)_{n \in \N}$ von $E$-wertigen Zufallsvariablen \textit{konvergiert in Verteilung} gegen eine Zufallsvariable $X$,
    falls die Folge $(P^{X_n})_{n \in \N}$ von Verteilungen schwach gegen $P^X$ konvergiert. 
\end{mydef}





\chapter{Maximal-Ungleichungen und Konvergenz zufälliger Reihen}
Im Folgenden sei $E$ ein separabler Banachraum. 
\section{Maximalungleichungen}

\begin{mydef}
    Eine E-wertige Zufallsvariable $X$ heißt \textit{symmetrisch}, falls $-X$ die selbe Verteilung besitzt wie $X$, d.h.
    \begin{align*}
        \forall A \in \mathcal{B}(E): P(\{X \in A\}) = P(\{-X \in A\}). 
    \end{align*}
\end{mydef}

\begin{remark}
    Nach dem Eindeutigkeitssatz für charakteristische Funktionale ist eine Zufallsvariable $X \in \mathcal{L}_0(E)$ genau dann symmetrisch, wenn 
    $$
        \forall z \in E': \quad \widehat{\mu_X}(z) = \widehat{\mu_{-X}}(z). 
    $$
\end{remark}

\begin{mydef}%TODO: Wird das überhaupt gebraucht? Ggf schöner formulieren. 
    Eine Folge $(X_n)_{n \in \N}$ von E-wertigen Zufallsvariablen heißt \textit{symmetrisch}, 
    falls $(\varepsilon_1 X_1, \varepsilon_2 X_2,...)$ für jede Wahl von $\varepsilon_i = \pm 1$ 
    die gleiche Verteilung hat wie $(X_1,X_2,...)$. 
\end{mydef}

\begin{remark}
   Sind $X_1,X_2,...$ unabhängige E-wertige Zufallsvariablen, sodass $X_n$ für alle $n \in \N$ symmetrisch ist, dann ist $(X_1,X_2,...)$ symmetrisch. 
\end{remark}

\begin{theorem}[Lévys Maximal-Ungleichung]
    Seien $X_1,...,X_N \in L_0(E)$ unabhängige und symmetrische Zufallsvariablen und setze 
    \begin{align*}
        S_n := \sum_{i=1}^n X_i, \quad 1 \leq n \leq N. 
    \end{align*}
    Dann gilt für alle $t > 0$
    \begin{align}
        &P\big(\{ \max_{1 \leq n \leq N} \norm{S_n} > t \}\big) \leq 2 P\big(\{\norm{S_N} > t \}\big), \\\
        &P\big(\{ \max_{1 \leq n \leq N} \norm{X_n} > t \}\big) \leq 2 P\big(\{\norm{S_N} > t \}\big).
    \end{align}
\end{theorem}

\begin{proof*}
    zu $(3.1)$:
    Setze 
    $$
        T_1(\omega) := \inf\{ k \leq N: \norm{S_k} > t \} \in [0, \infty], \quad \omega \in \Omega. 
    $$
    Dann ist $T_1$ messbar, da für jedes $i =1,...,N$ die Menge $\{ \norm{S_i} \leq t\}$ messbar ist. 
    Wir zeigen zunächst
    \begin{align}
        \{T_1 = n\} \subseteq \{\norm{S_N} > t, T_1 = n\} \cup \{\norm{2S_n - S_N}>t, T_1=n\}, \quad n \in \{1,...,N\}, \\\
        \prob{\norm{S_N} > t, T_1=n} = \prob{\norm{2S_n - S_N} > t, T_1=n}, \quad n \in \{1,...,N\}. 
    \end{align}
    zu $(3.3)$:
    Für $\omega \in \Omega$ mit $T_1(\omega) = n$ und $\norm{S_N} \leq t$ liefert die umgekehrte Dreiecksungleichung
    $$
        \norm{2S_n - S_N} \geq 2\norm{S_n} - \norm{S_N} > 2t - t = t. 
    $$
    zu $(3.4)$: Setze $\varepsilon_1 = \varepsilon_2 = ... = \varepsilon_n = 1$ und $\varepsilon_{n+1} = ... = \varepsilon_N = -1$, sowie
    $$
        S'_j := \sum_{i=1}^N\varepsilon_i X_i. 
    $$
    Dann gilt $S_j = S'_j$ für alle $j \leq n$ und 
    $$
        2S_n - S_N = 2 \sum_{i=1}^n X_i - \sum_{i=1}^N X_I = \sum_{i=1}^nX_i - \sum_{i=n+1}^N X_i = S'_N. 
    $$
    Wegen der Symmetrie von $X_1,...,X_N$ sind $(S_1,...,S_N)$ und $(S'_1,...,S'_N)$ identisch verteilt. Also ergibt sich 
    \begin{align*}
        \prob{\norm{S_N} > t, T_1=n} &= \prob{\norm{S_1} \leq t,...,\norm{S_{n-1}}\leq t, \norm{S_n} > t, \norm{S_N} > t} \\\
                                   &= \prob{\norm{S'_1}\leq t,...,\norm{S'_{n-1}}\leq t, \norm{S'_n} > t, \norm{S'_N} > t} \\\
                                   &= \prob{\norm{2S_n - S_N} > t, T_1=n}. 
    \end{align*}
    Wir erhalten also mit $(3.3)$ und $(3.4)$ 
    $$
        \prob{T_1=n} \leq 2 \prob{\norm{S_N} > t, T_1=n}.
    $$
    Woraus wir schließlich $(3.1)$ folgern  
    \begin{align*}
        \prob{\max_{1 \leq n \leq N} \norm{S_n} > t} \leq \sum_{n=1}^{N} \prob{T_1=n} 
                                                     &\leq \sum_{n=1}^N2\prob{\norm{S_N} >t, T=n} \\\
                                                     &= 2 \prob{\norm{S_N}>t, T_1 \leq N} \leq 2 \prob{\norm{S_N} > t}. 
    \end{align*}
    zu $(3.2)$: 
    Setze 
    $$
        T_2(\omega) := \inf\{k \leq N: \norm{X_k(\omega)} > t\} \in [0, \infty], \quad \omega \in \Omega. 
    $$
    Analog zum Beweis von $(3.3)$ und $(3.4)$ zeigt man 
    \begin{align*}
        \{T_2 = n\} \subseteq \{\norm{S_N} > t, T_2 =n\} \cup \{\norm{2X_n - S_N} > t, T_2=n\}, \quad n \in \{1,...,N\}, \\\
        \prob{\norm{S_N} > t, T_2 = n} = \prob{\norm{2X_n - S_N}, T_2 = n}, \quad n \in \{1,...,N\}. 
    \end{align*}
    unter der Verwendung der Symmetrie von $X_1,...,X_N$ und folgert daraus schließlich 
    \begin{align*}
        \prob{\max_{1\leq n \leq N} \norm{X_n} > t} = \sum_{n=1}^N \prob{T_2 = n} 
                                                    &\leq \sum_{n=1}^N 2\prob{\norm{S_N} > t, T=n} \\\
                                                    &= 2 \prob{\norm{S_N}>t, T_2 \leq N} \leq 2 \prob{\norm{S_N} > t}. 
    \end{align*}
    \qed 
\end{proof*}

Für nicht-symmetrische Zufallsvariablen erhalten wir mit einer ähnlichen Beweismethode die folgende auf Giuseppe Ottaviani und Anatoli Skorohod zurückgehende Maximal-Ungleichung, vgl. \cite{ledoux-talagrand}[Lemma 6.2]. 
\begin{theorem}[Maximal-Ungleichung von Ottaviani-Skorohod]
    Seien $X_1,...,X_N$ unabhängige $E$-wertige Zufallsvariablen, $N \in \N$. Setze 
    $$
        S_k := \sum_{i=1}^kX_i, \quad k = 1,...,N. 
    $$
    Dann gilt für alle $s,t > 0$
    \begin{align}
        P(\{ \max_{1 \leq k \leq N} \norm{S_k} > s + t \}) \leq \frac{P(\{\norm{S_N} > t \})}{1 - \max_{1 \leq k \leq N}P(\{ \norm{S_N - S_k} > s \})} \ . 
    \end{align}
\end{theorem}

\begin{proof*}
    Setze 
    $$
        T(\omega) := \inf\{k \leq N: \norm{S_k(\omega)} > s + t\} \in [0, \infty], \quad \omega \in \Omega. 
    $$
    Dann ist $T$ messbar und $\{T = k\}$ hängt nur von $X_1,...,X_k$ ab. Weiter gilt 
    $$
        \sum_{k=1}^N \prob{T = k} = \prob{\max_{1\leq k \leq N}\norm{S_k} > s+t}.
    $$
    Für $\omega \in \Omega$ mit $T(\omega) = k$ und $\norm{S_N - S_k} \leq s$ gilt zudem $\norm{S_N} > t$, denn mit der umgekehrten Dreiecksungleichung erhält man
    $$
        s \geq \norm{S_N - S_k} \geq \abs{\norm{S_N}-\norm{S_k}} > (s+t) - \norm{S_N}.
    $$
    Die Unabhängigkeit von $X_1,...,X_N$ liefert schließlich
    \begin{align*}
        \prob{\norm{S_N} > t} &\geq \sum_{k=1}^N \prob{T=k, \norm{S_N} > t}  \\\
                              &\geq \sum_{k=1}^N \prob{T=k, \norm{S_N - S_k} \leq s} \\\
                              &\geq \min_{1\leq k \leq N}\prob{\norm{S_N-S_k} \leq s} \sum_{k=1}^N \prob{T=k}. 
    \end{align*}
    Umstellen und beachten von 
    $$
        \min_{1\leq k \leq N}\prob{\norm{S_N-S_k} \leq s} = 1 - \max_{1\leq k \leq N}\prob{\norm{S_N - s_k} > s}
    $$
    liefert nun die Behauptung. \qed
\end{proof*}
\section{Der Satz von Itô-Nisio}
    Wir wollen nun damit beginnen die bisher erarbeitete Technik zur Untersuchung der Konvergenz zufälliger Reihen in Banachräumen anzuwenden.
    Die vorliegenden Beweise orientieren sich an \cite{ito-nisio}, \cite{ledoux-talagrand}, \cite{li-queffelec} und \cite{van-neerven}.  
    \newline \ \newline 
    Sei $(X_n)_{n \in \N}$ eine Folge unabhängiger Zufallsvariablen in $\mathcal{L}_0(E)$. Für $n \in \N$ setze
    \begin{align*}
    S_n := \sum_{i=1}^n X_i, 
    \quad 
    \mu_n := P^{S_n} .
    \end{align*}
\begin{theorem}[Itô-Nisio]
    Es sind äquivalent
    \begin{enumerate}[(i)]
        \item $(S_n)_{n \in \N}$ konvergiert fast sicher, 
        \item $(S_n)_{n \in \N}$ konvergiert stochastisch, 
        \item $(S_n)_{n \in \N}$ konvergiert in Verteilung. 
    \end{enumerate}
\end{theorem}

\begin{proof*}%TODO: Formatierung% 
    Die Implikationen $(i) \Rightarrow (ii) \Rightarrow (iii)$ wurden bereits in Kapitel 2 gezeigt, es genügt also $(ii) \Rightarrow (i)$ und $(iii) \Rightarrow (ii)$ zu zeigen. 
    \newline 
    zu $(ii) \Rightarrow (i)$: \underline{Fall A}: Für alle $n \in \N$ ist $X_n$ symmetrisch verteilt. 
    \newline 
    Für $n, N \in \N$ mit $n < N$ setze 
    \begin{align*}
        Y_{n,N} &:= \max_{n < k \leq N}\norm{S_k - S_n}, \\\
        Y_n     &:= \lim_{N \to \infty} Y_{n,N} = \sup_{k > n} \norm{S_k - S_n}. 
    \end{align*}
    Seien $\varepsilon, t > 0$. Mit dem Cauchy-Kriterium für stochastische Konvergenz und Lévys Maximal-Ungleichung erhalten wir für $N > n \geq n_0 := n_0(\varepsilon,t) \in \N$
    $$
        P(\{ Y_{n, N} > t\}) \leq 2P(\{ \norm{S_N - S_n} > t\}) \leq \varepsilon . 
    $$
    Es folgt somit 
    $$
        P(\{Y_n > t \}) = \lim_{n \to \infty}P(\{Y_{n,N} > t \}) \leq \varepsilon
    $$
    für $n \geq n_0$. Also gilt $Y_n \stochastisch 0$, nach dem Cauchy-Kriterium für fast sichere Konvergenz folgt daraus die fast sichere Konvergenz von $(S_n)_{n \in \N}$. 
    \newline \underline{Fall B}: Allgemeiner Fall.
    \newline 
    Wir verwenden für diesen Fall die Beweistechnik der Symmetrisierung.  
    Betrachte hierzu den Produktraum $(\Omega \times \Omega, \mathcal{A}\otimes\mathcal{A}, P \times P)$. Für eine Zufallsvariable $X$ auf $(\Omega, \mathcal{A}, P)$ bezeichne
    $$
        \overline{X}(\omega_1, \omega_2) := X(\omega_1) - X(\omega_2), \quad (\omega_1, \omega_2) \in \Omega \times \Omega
    $$   
    die Symmetrisierung von $X$. Wie man mittels der charakteristischen Funktionale von $\overline{X}$ und $-\overline{X}$ leicht einsieht ist $\overline{X}$ tatsächlich symmetrisch. 
    Sei nun $S$ eine Zufallsvariable auf $(\Omega, \mathcal{A}, P)$ mit $S_n \stochastisch S$. 
    Dann folgt direkt $\overline{S_n} \stochastisch \overline{S}$, denn für $\varepsilon > 0$ gilt nach Konstruktion
    $$
        (P\times P)(\{ \norm{\overline{S_n} - \overline{S}} > \varepsilon \} ) \leq 2P(\{ \norm{S_n - S} > \frac{\varepsilon}{2} \}).
    $$
    Nach Fall A gilt also insbesondere $\overline{S_n} \fastsicher  \overline{S}$. Daher existiert eine Menge $\Omega^* \in \mathcal{A}\otimes\mathcal{A}$ mit \mbox{$(P\times P)(\Omega^*) = 1$} und
    $$  
        \forall (\omega_1, \omega_2) \in \Omega^*: \quad \overline{S_n}(\omega_1, \omega_2) \text{ konvergiert. }
    $$
    Mit dem Satz von Fubini erhalten wir
    $$
        0 = \int_{\Omega \times \Omega}1 - 1_{\Omega^*} d(P \times P) = \int_{\Omega}\int_{\Omega}1 - 1_{\Omega^*(\omega_2)}(\omega_1)dP(\omega_1)dP(\omega_2). 
    $$
    wobei $\Omega^*(\omega_2) =\{\omega_1\in\Omega: \ (\omega_1, \omega_2) \in \Omega^* \}$. Somit existiert ein $\omega_2 \in \Omega$ mit $P(\Omega^*(\omega_2)) = 1$ und es gilt
    $$
        \forall \omega_1 \in \Omega^*(\omega_2): \quad S_n(\omega_1) - S_n(\omega_2) \ \text{ konvergiert.}
    $$
    Setze nun $x_n := S_n(\omega_2)$, $n \in \N$. Dann existiert eine Zufallsvariable $L$ auf $(\Omega, \mathcal{A}, P)$ mit $S_n - x_n \fastsicher L$. Nach Voraussetzung erhalten wir also 
    $$
        x_n \stochastisch S - L
    $$
    wobei wir $x_n$ für $n \in \N$ als konstante Zufallsvariable auf $(\Omega, \mathcal{A}, P)$ auffassen. Folglich existiert ein $x \in E$ mit 
    $$
        \lim_{n \to \infty}x_n = x
    $$
    und insgesamt erhalten wir $S_n \fastsicher L + x$. 
    \newline 
    zu $(iii) \Rightarrow (ii)$: Für $1 \leq m < n$ bezeichne
    \begin{align*}
        \mu_{m,n} :&= P^{S_n - S_m}
    \end{align*}
    die Verteilung von $S_n - S_m$. Da $(\mu_n)_{n \in \N}$ nach Voraussetzung schwach gegen ein Wahrscheinlichkeitsmaß $\mu$ konvergiert
    ist die Menge $\{\mu_n: n \in \N \}$ relativ kompakt in $(\mathcal{M}(E), \rho)$ und somit nach dem Satz von Prokhorov gleichmäßig straff.
    Folglich existiert zu $\varepsilon > 0$ eine kompakte Teilmenge $K \subseteq E$ mit 
    $$
        \forall n \in \N: \quad \mu_n(K) \geq 1 - \varepsilon. 
    $$
    Wegen der Stetigkeit von 
    $$
        (x,y) \mapsto x - y, \quad (x,y) \in E \times E
    $$
    ist die Menge $\tilde{K} := \{x - y : x,y \in K \}$ wiederum kompakt, also insbesondere messbar, und es gilt
    \begin{align*}
        \mu_{m,n}(\tilde{K}) \geq P(\{S_n \in K, \ S_m \in K\}) \geq 1 - P(\{S_n \in K^c\}) - P(\{S_m \in K^c\}) \geq 1 - 2\varepsilon.
    \end{align*}
    Somit ist auch $\{\mu_{m,n}: m,n \in \N, m < n \}$ gleichmäßig straff und daher relativ kompakt in $(\mathcal{M}(E), \rho)$. 
    Wir zeigen nun
    \begin{align}
        \forall \varepsilon > 0 \ \exists N \in \N: \ \forall n > m \geq N: \quad \prob{\norm{S_n - S_m} < \varepsilon} = \mu_{m,n}(B(0,\varepsilon)) > 1 - \varepsilon.
    \end{align}
    Mit dem Cauchy-Kriterium folgt daraus die stochastische Konvergenz der Folge $(S_n)_{n \in \N}$. Angenommen $(3.6)$ ist nicht erfüllt, dann gilt
    \begin{align*}
        \exists \varepsilon > 0 \ \forall N \in \N: \ \exists n(N) > m(N) \geq N: \mu_{m(N), n(N)}(B(0, \varepsilon)) \leq 1 - \varepsilon.
    \end{align*}
    Da $\{\mu_{m,n}: m,n \in \N, m < n \}$ relativ kompakt ist existiert insbesondere eine Teilfolge von $(\mu_{m(N),n(N)})_{N \in \N}$ die schwach gegen ein Wahrscheinlichkeitsmaß $\nu$ auf $\mathcal{B}(E)$ konvergiert.
    Ohne Einschränkung können wir annehmen, dass bereits $\mu_{m(N),n(N)} \rightharpoonup \nu$ gilt. Da $B(0, \varepsilon)$ offen ist liefert das Portmanteau-Theorem
    \begin{align}
        \nu(B(0, \varepsilon)) \leq \liminf_{N \to \infty}\mu_{m(N),n(N)}(B(0,\varepsilon)) \leq 1 - \varepsilon. 
    \end{align}
    Andererseits gilt für $z \in E'$ wegen der Unabhängigkeit von $(X_n)_{n \in \N}$
    \begin{align*}
        \widehat{\mu_{n(N)}} = \mathbb{E}(e^{iz(S_{n(N)}})) &= \mathbb{E}(e^{iz(S_{m(N)})}e^{iz(S_{n(N)} - S_{m(N)})}) \\\
                                                   &= \mathbb{E}(e^{iz(S_{m(N)})})\mathbb{E}(e^{iz(S_{n(N)} - S_{m(N)})}) \\\
                                                   &= \widehat{\mu_{m(N}}(z)\widehat{\mu_{m(N),n(N)}}(z). 
    \end{align*}
    Mit Grenzübergang $N \to \infty$ folgt daraus wegen $\mu_N \rightharpoonup \mu$ und $\mu_{m(N),n(N)} \rightharpoonup \nu$
    \begin{align*}
        \forall z \in E': \quad \widehat{\mu}(z) = \widehat{\mu}(z) \widehat{\nu}(z)
    \end{align*}
    Wegen der Stetigkeit von $\widehat{\mu}$ und $\widehat{\mu}(0_{E'}) = 1$ existiert ein $r>0$ mit 
    \begin{align*}
        \widehat{\mu}(z) \neq 0, \quad \text{ für alle } z \in E' \text{ mit } \norm{z}_{op} \leq r. 
    \end{align*}
    Also muss 
    \begin{align*}
        \widehat{\nu}(z) = 1, \quad \text{ für alle } z \in E' \text{ mit } \norm{z}_{op} \leq r. 
    \end{align*}
    gelten. Mit Proposition $2.19$ folgt nun $\nu = \delta_0$. Im Widerspruch zu $(3.7)$. Es gilt folglich $(3.6)$ und $(S_n)_{n \in \N}$ konvergiert demnach stochastisch. 
    \qed
\end{proof*}

\begin{remark}
    Mittels der Maximal-Ungleichung von Ottaviani-Skorohod ist auch ein direkter Beweis der Implikation $(ii) \Rightarrow (i)$ möglich. Betrachte hierzu etwa die Ereignisse 
    $$
        A_N := \bigcap_{m \in \N}\{\sup_{n > m} \norm{S_n - S_m} > \frac{1}{N}\}, \quad N \in \N.                                                                                                                        
    $$
    Dann ist $A := (\cup_{N=1}^{\infty} A_N)^c$ das Ereignis, dass $(S_n)_{n \in \N}$ eine Cauchy-Folge ist und man zeigt 
    $$
        \forall N \in \N: \quad P(A_N) = 0. 
    $$
    Ein Beweis mit dieser Vorgehensweise für den skalaren Fall findet sich etwa \cite{bauer}[Theorem 14.2, S.109]. Der allgemeine Fall funktioniert vollkommen analog. \qexampled
\end{remark}




Unter der zusätzlichen Voraussetzung, dass die unabhängigen Zufallsvariablen $(X_n)_{n \in \N}$ symmetrisch verteilt sind, lässt sich der Satz von Itô-Nisio auf drei noch schwächere Annahmen als die Verteilungskonvergenz von $(S_n)_{n \in \N}$ erweitern. 
Für den Beweis dient uns unter anderem der \mbox{folgende Satz.} 
\begin{theorem}
    Sei $(X_n)_{n \in \N}$ eine Folge unabhängiger Radon-Zufallsvariablen und sei $(\mu_n)_{n \in \N}$ gleichmäßig straff. Dann existiert eine Folge $(c_n)_{n \in \N}$ in $E$, sodass $(S_n - c_n)_{n \in \N}$ fast sicher konvergiert.
\end{theorem}

\begin{proof*}
    Betrachte den Produktraum $(\Omega \times \Omega, \mathcal{A} \otimes \mathcal{A}, P \times P)$ und definiere für $n \in \N$
    \begin{align*}
        \widetilde{X}_n&: \Omega \times \Omega \to E, \ (\omega_1, \omega_2) \mapsto X_n(\omega_1), \\\
        \widetilde{Y}_n&: \Omega \times \Omega \to E, \ (\omega_1, \omega_2) \mapsto X_n(\omega_2),
    \end{align*}
    sowie 
    \begin{align*}
        \widetilde{S}_n := \sum_{i = 1}^n \widetilde{X}_i, \quad \widetilde{T}_n := \sum_{i = 1}^n \widetilde{Y}_n, \quad U_n := \widetilde{S}_n - \widetilde{T}_n, \quad \mu_{U_n} := (P\times P)^{U_n}. 
    \end{align*}
    Nach Konstruktion sind die Zufallsvariablen $\widetilde{S}_n$, $\widetilde{T}_n$ und $S_n$ für festes $n \in \N$ identisch verteilt. Wir zeigen zunächst, dass $\{\mu_{U_n}: \ n \in \N\}$ gleichmäßig straff ist. 
    Sei dazu $\varepsilon > 0$. Dann existiert nach Voraussetzung eine kompakte Menge $\tilde{K} \subseteq E$ mit 
    $$
        \forall n \in \N: \mu_n(\tilde{K}) \geq 1 - \varepsilon. 
    $$
    Wegen der Stetigkeit der Abbildung 
    $$
        E \times E \to E, \ (x,y) \mapsto x - y
    $$
    ist auch die Menge $K := \{ x - y : x,y \in \tilde{K}\}$ kompakt und somit insbesondere messbar. Ferner gilt
    \begin{align*}
        \mu_{U_n}(K) = (P \times P)\{\widetilde{S}_n - \widetilde{T}_n \in K\} &\geq (P \times P)\{\widetilde{S}_n \in \tilde{K}, \widetilde{T}_n \in \tilde{K}\} \\\
                                                              &\geq 1 - (P \times P)\{\widetilde{S}_n \notin \tilde{K}\} - (P \times P)\{\widetilde{T}_n \notin \tilde{K}\} \\\
                                                              &= 1 - 2\mu_n(\tilde{K}^c) \\\
                                                              &\geq 1 - 2 \varepsilon.                                              
    \end{align*}
    Demnach ist $\{\mu_{U_n}: \ n \in \N\}$ straff. 
    Als nächstes zeigen wir, dass $(\widehat{\mu_{U_n}}(f))_{n \in \N}$ für alle $f \in E'$ konvergiert. Sei dazu $f \in E'$ beliebig aber fest. 
    Die Unabhängigkeit von $(\widetilde{Y}_n)_{n \in \N}$ und $(\widetilde{X}_n)_{n \in \N}$ liefert direkt die Unabhängigkeit von $(\widetilde{X}_n - \widetilde{Y}_n)_{n \in \N}$
    und nach Konstruktion sind $\tilde{Y}_n$ und $\tilde{X}_n$ für $n \in \N$ identisch verteilt. Es gilt demzufolge  
    \begin{align*}
        \widehat{\mu_{U_n}}(f) = E(e^{if(U_n)}) &= \mathbb{E}\big(\prod_{j=1}^n(e^{if(\widetilde{X}_j-\widetilde{Y}_j)})\big) \\\
                                                &= \prod_{j=1}^n \mathbb{E}(e^{if(\widetilde{X}_j-\widetilde{Y}_j)})
                                                 = \prod_{j=1}^n \mathbb{E}(e^{if(\widetilde{X}_j)})\mathbb{E}(e^{-if(\widetilde{Y}_j)})
                                                 = \prod_{j=1}^n \abs{\mathbb{E}(e^{if(\widetilde{X}_j)})}^2.
    \end{align*}
    Wegen $0 \leq \abs{\mathbb{E}(e^{if(\widetilde{X}_j)})} \leq 1$ für alle $j \in \N$ folgt daraus die Konvergenz von $(\widehat{\mu_{U_n}}(f))_{n \in \N}$. 
    Wir zeigen nun, dass die Folge $(\mu_{U_n})_{n \in \N}$ schwach konvergiert. 
    Wie zuvor gezeigt wurde, ist $\{\mu_{U_n}: n \in \N\}$ gleichmäßig straff, also nach Satz $1.27$ relativ kompakt in $(\mathcal{M}(E), \rho)$. 
    Es genügt also zu zeigen, dass alle konvergenten Teilfolgen von $(\mu_{U_n})_{n \in \N}$ gegen den gleichen Grenzwert konvergieren. 
    Seien dazu $(\nu_n)_{n \in \N}$ und $(\eta_n)_{n \in \N}$ zwei konvergente Teilfolgen von $(\mu_{U_n})_{n \in \N}$ mit schwachem Grenzwert $\nu$ bzw. $\eta$. 
    Insbesondere gilt also nach Satz $2.17$ für alle $f \in E'$
    $$
        \lim_{n \to \infty} \widehat{\nu_n}(f) = \widehat{\nu}(f) \text{ und } \lim_{n \to \infty} \widehat{\eta_n}(f) = \widehat{\eta}(f). 
    $$
    Da für alle $f \in E'$ die Folge $(\widehat{\mu_{U_n}}(f))_{n \in \N}$ konvergiert, gilt 
    $$
        \forall f \in E': \ \nu(f) = \lim_{n \to \infty} \widehat{\nu_n}(f) = \lim_{n \to \infty}\widehat{\mu_{U_n}}(f) = \lim_{n \to \infty}\widehat{\eta_n}(f) = \widehat{\eta}(f). 
    $$
    Nach Satz $2.16$ folgt daraus $\eta = \nu$. Also ist die Folge $(\mu_{U_n})_{n \in \N}$ schwach konvergent, d.h. $(U_n)_{n \in \N}$ konvergiert in Verteilung. 
    Aus Satz $3.6$ folgt nun, dass $(U_n)_{n \in \N}$ fast sicher konvergiert. 
    Daher existiert eine Menge $\Omega^* \in \mathcal{A} \otimes \mathcal{A}$ mit $(P\times P)(\Omega^*) = 1$ und 
    $$
        \forall (\omega_1, \omega_2) \in \Omega^*: \quad (U_n(\omega_1, \omega_2))_{n \in \N} = (S_n(\omega_1) - S_n(\omega_2))_{n \in \N} \text{ konvergiert.}
    $$
    Mit dem Satz von Fubini erhalten wir wie im Beweis von Satz $3.6 \ (ii) \Rightarrow (i)$ ein $\omega' \in \Omega$, sodass \mbox{$(S_n - S_n(\omega'))_{n \in \N}$} fast sicher konvergiert. 
    Die Folge $(c_n)_{n \in \N}$, definiert durch $c_n := S_n(\omega')$, $n \in \N$, erfüllt nun die gewünschte Eigenschaft. \qed

\end{proof*}

\begin{theorem}[Satz von Itô-Nisio für Folgen symmetrischer Zufallsvariablen]
    Sei $(X_n)_{n \in \N}$ eine Folge unabhängiger und symmetrischer Zufallsvariablen in $\mathcal{L}_0(E)$. Dann sind äquivalent:
    \begin{enumerate}[(i)]
        \item $(S_n)_{n \in \N}$ konvergiert fast sicher.
        \item $(S_n)_{n \in \N}$ konvergiert stochastisch. 
        \item $(S_n)_{n \in \N}$ konvergiert in Verteilung. 
        \item $(\mu_n)_{n \in \N}$ ist gleichmäßig straff.
        \item Es gibt eine Zufallsvariable $S \in \mathcal{L}_0(E)$, sodass 
        $$
            \forall f \in E': \quad f(S_n) \stochastisch f(S).
        $$
        \item Es gibt ein Wahrscheinlichkeitsmaß $\mu$ auf $\mathcal{B}(E)$, sodass 
        $$
            \forall f \in E': \quad \lim_{n \to \infty}\widehat{\mu_n}(f) = \widehat{\mu}(f). 
        $$
    \end{enumerate}
\end{theorem}

\begin{proof*}
    Die Äquivalenz $(i) \iff (ii) \iff (iii)$ folgt aus Satz $3.6$. Die Implikation $(iii) \Rightarrow (iv)$ gilt nach Satz $1.27$. Ferner erhalten wir $(i) \Rightarrow (v)$ aus Proposition $2.12$.
    Wir zeigen noch $(v)\Rightarrow (vi)$, $(vi) \Rightarrow (iv)$ und $(v) \Rightarrow (iv) \Rightarrow (i)$. 
    \newline
    Zu $(v)\Rightarrow (vi)$: Setze $\mu := P^S$ und sei $f \in E'$. Wegen Satz $2.6$ können wir ohne Einschränkung annehmen, dass $(f(S_n))_{n \in \N}$ fast sicher gegen $f(S)$ konvergiert. 
    Der Satz von der dominierten Konvergenz liefert dann 
    $$
        \lim_{n \to \infty}\widehat{\mu_n}(f) = \lim_{n \to \infty}\mathbb{E}(e^{if(S_n)}) = \mathbb{E}(e^{f(S)}) = \widehat{\mu}(f).
    $$
    \newline
    Zu $(iv) \Rightarrow (i)$:
    Nach Satz $3.8$ existiert eine Folge $(c_n)_{n \in \N}$ in $E$, sodass $(S_n - c_n)_{n \in \N}$ fast sicher konvergiert. Setze nun $P^X := P^{(X_1,X_2,...)}$ und $P^{-X} :=P^{(-X_1,-X_2,...)}$. 
    Dann sind $P^X$ und $P^{-X}$ jeweils Wahrscheinlichkeitsmaße auf dem Produktraum $(E^{\N}, \otimes_{n \in \N}B(E))$ und wegen der Unabhängigkeit und Symmetrie von $(X_n)_{n \in \N}$ erhalten wir direkt $P^X = P^{-X}$. Daher gilt für $N \in \N$ und $\varepsilon >0$
    \begin{align*}
        &\quad \ \prob{\sup_{n \geq N}\norm{(S_n - c_n) - (S_N- c_N)} > \varepsilon} \\\
                &= P^X\big(\{(y_n)_{n \in \N}\in E^{\N}: \  \sup_{n \geq N}\norm{y_n + y_{n-1} + ... + y_{N+1} + (c_n - c_N)} > \varepsilon \}\big) \\\
                &= P^{-X}\big(\{(y_n)_{n \in \N}\in E^{\N}: \ \sup_{n \geq N}\norm{y_n + y_{n-1} + ... + y_{N+1} + (c_n - c_N)} > \varepsilon \}\big) \\\
                &= \prob{\sup_{n \geq N}\norm{(-S_n - c_n) - (-S_N - c_N)} > \varepsilon}.
    \end{align*}
    Nach Satz $2.4$ konvergiert also auch $(-S_n - c_n)_{n \in \N}$ fast sicher. Daraus folgt die fast sichere Konvergenz von $(S_n)_{n \in \N}$, denn für $n \in \N$ gilt
    $$
       S_n =  \frac{1}{2}((S_n - c_n) - (-S_n -c_n)). 
    $$
    Zu $(v) \Rightarrow (iv)$: Wegen der Unabhängigkeit von $(X_n)_{n \in \N}$ sind für alle $f \in E'$ und $m \geq n$ die Zufallsvariablen $f(S_m - S_n)$ und $f(S_n)$ unabhängig. 
    Nach Proposition $2.19$ sind somit $f(S-S_n)$ und $f(S_n)$ für alle $n \in \N$ unabhängig. Zusammen mit der Symmetrie von $S_n$ ergibt sich deshalb für $f \in E'$
    \begin{align*}
        \widehat{P^S}(f) = \mathbb{E}(e^{if(S)}) &= \mathbb{E}(e^{if(S-S_n)})\mathbb{E}(e^{if(S_n)})  \\\
                                        &= \mathbb{E}(e^{if(S-S_n)})\mathbb{E}(e^{if(-S_n)}) = \mathbb{E}(e^{if(S-2S_n)}) = \widehat{P^{S-2S_n}}(f).
    \end{align*}
    Folglich sind $S$ und $S - 2S_n$ nach Satz $2.16$ für alle $n \in \N$  identisch verteilt. Da $P^S$ straff ist, existiert zu $\varepsilon >0$ eine kompakte Menge $K \subseteq E'$ mit $\prob{S \in K} \geq 1 - \varepsilon$. 
    Aus Stetigkeitsgründen ist auch die Menge $L := \{\frac{1}{2}(x-y): x,y \in K\}$ kompakt und es gilt für alle $n \in \N$
    $$
        \prob{S_n \in L} \geq \prob{S \in K, \ S-2S_n \in K} \geq 1 - \prob{S \notin K} - \prob{S-2S_n \notin K} \geq 1 - 2\varepsilon. 
    $$ 
    Somit ist $(\mu_n)_{n \in \N}$ gleichmäßig straff. 
    \newline
    Zu $(vi) \Rightarrow (iv)$: 
    Sei $f \in E'$ beliebig aber fest. Betrachte die Abbildungen 
    \begin{align*}
        \varphi_n &: \R \to \C, \quad t \mapsto \int_E e^{itf(x)}\mu_n(dx) = \widehat{\mu_n}(tf), \quad n \in \N, \\\
        \varphi   &: \R \to \C, \quad t \mapsto \int_E e^{itf(x)}\mu(dx) = \widehat{\mu}(tf). 
    \end{align*}
    Dann ist $\varphi_n$ für alle $n \in \N$ die charakteristische Funktion von $\mu_n^{f}$ und $\varphi$ die charakteristische Funktion von $\mu^f$. Nach Voraussetzung konvergiert $(\varphi_n)_{n \in \N}$ punktweise gegen $\varphi$ und nach 
    dem Stetigkeitssatz von Lévy, vgl. \cite[Satz 8.7.5]{gs}, gilt daher $\mu_n^f \rightharpoonup \mu^f$. Für festes $f \in E'$ ist $\{\mu_n^f : n \in \N\}$ somit insbesondere relativ kompakt.
     Nach Satz $1.33$ genügt es folglich zu zeigen, dass $(\mu_n)_{n \in \N}$ flach konzentriert ist. 
    Da $\{\mu\}$ flach konzentriert ist, genügt es dafür zu zeigen, dass für jeden endlichdimensionalen Untervektorraum $F \subseteq E$ und alle $\varepsilon > 0$ gilt 
    $$
        \mu_n(\{x \in E: \ \inf_{y \in F}\norm{x-y} > \varepsilon \}) \leq 2 \mu(\{x \in E: \ \inf_{y \in F}\norm{x-y} > \varepsilon \}).
    $$
    Man beachte hierbei, dass die Menge $\{x \in E: \ \inf_{y \in F}\norm{x-y} > \varepsilon \}$ als offene Menge insbesondere messbar ist. 
    Sei also $F \subseteq E$ ein endlichdimensionaler Untervektorraum und $\varepsilon >0$. Nach Korollar $A.13$ existiert eine Folge $(f_n)_{n \in \N}$ in $E'$ mit 
    $$
        \forall x \in E: \quad d(x,F) := \inf_{y \in F}\norm{x-y} = \sup_{n \in \N}\abs{f_n(x)}. 
    $$
    Sei zunächst $m \in \N$ festgewählt. Mittels charakteristischer Funktionen prüft man leicht, dass $\mu_n^{(f_1,...,f_m)} \rightharpoonup \mu^{(f_1,...,f_m)}$. 
    Wegen der Linearität von $f_1,...,f_m$ können wir Satz $3.6$ auf die Folge $(T_n^{(m)})_{n \in \N} := ((f_1,...,f_m)\circ S_n)_{n \in \N}$ anwenden und erhalten eine $\R^m$-wertige Zufallsvariable $T^{(m)}$ auf $(\Omega, \mathcal{A}, P)$ mit 
    $T_n^{(m)} \stochastisch T^{(m)}$. Insbesondere gilt daher $T_n^{(m)} \schwach T^{(m)}$ und somit \mbox{$P^{T^{(m)}} = \mu^{(f_1,...,f_m)}$}. 
    Ferner ist $T_n$ wegen der Linearität von $f_1,...,f_m$ und der Symmetrie von $S_n$ für alle $n \in \N$ symmetrisch . Mit Korollar $3.4$, angewendet auf $(T_n^{(m)})_{n \in \N}$, erhalten wir im Banachraum $(\R^m, \norm{\cdot}_{\infty})$ für fast alle $\varepsilon >0$
    \begin{align*}
        \prob{\max_{1 \leq i \leq m}\abs{f_i(S_n)} > \varepsilon} &= \prob{\norm{T_n^{(m)}}_{\infty} > \varepsilon} \\\
                                                                  &\leq 2 \prob{\norm{T^{(m)}}_{\infty} > \varepsilon} = 2\mu\big(\{x \in E: \max_{1\leq i \leq m}\abs{f_i(x)} > \varepsilon\}\big).
    \end{align*}
    Mit der $\sigma$-Stetigkeit von Wahrscheinlichkeitsmaßen folgt schließlich
    \begin{align*}
        \mu_n(\{x \in E: \ \inf_{y \in F}\norm{x-y} > \varepsilon \}) = \prob{d(S_n, F) > \varepsilon}
                                                                     &= \lim_{m \to \infty}\prob{\max_{1 \leq i \leq m}\abs{f_i(S_n)} > \varepsilon} \\\
                                                                    &\leq \lim_{m \to \infty} 2 \mu(\{x \in E: \max_{1\leq i \leq m}\abs{f_i(x)} > \varepsilon\}) \\\
                                                                    &= 2 \mu(\{x \in E: \ \inf_{y \in F}\norm{x-y} > \varepsilon \}).
    \end{align*}
    \qed
\end{proof*}

\begin{remark}
    Auf die Annahme der Symmetrie in Satz $3.9$ kann im Allgemeinen nicht verzichtet werden. Sei etwa $(E, \langle\cdot,\cdot\rangle)$ ein Hilbertraum mit Orthonormalbasis $(e_n)_{n \in \N}$.
    Man bemerke zunächst, dass nach dem Darstellungsatz von Fréchet-Riesz, vgl. \cite[Theorem V.3.6]{werner}, für jedes $f \in E'$ ein $z \in E$ mit $f = \langle \cdot, z \rangle$ existiert. 
    Setze nun 
    $$
        X_1(\omega) = e_1, \quad X_n(\omega) = e_n - e_{n-1}, \ n \geq 2, \quad \omega \in \Omega. 
    $$
    Dann gilt offensichtlich $S_n(\omega) = e_n$ für alle $\omega \in \Omega$ und da $(e_n)_{n \in \N}$ eine Orthonormalbasis von $E$ ist, gilt für alle $z \in E$
    $$
        \lim_{n \to \infty}\langle z,S_n \rangle = 0 = \langle z,S \rangle,
    $$
    wobei $S(\omega) = 0$ für alle $\omega \in \Omega$. Nach dem Satz von Fréchet-Riesz gilt also 
    $$
        \forall f \in E': \quad f(S_n) \stochastisch f(S). 
    $$
    Wegen $\norm{S_n} = \norm{e_n} =1$ für alle $n \in \N$ konvergiert $(S_n)_{n \in \N}$ aber weder fast-sicher noch stochastisch gegen $S$. 
    \qexampled 
\end{remark}

Für Folgen $(a_n)_{n \in \N}$ in $[0, \infty)$ kennt man aus der reellen Analysis die Äquivalenz
$$
    \bigg(\sum_{i=1}^n a_i\bigg)_{n \in \N} \text{ konvergiert. } \iff \bigg(\sum_{i=1}^n a_i\bigg)_{n \in \N} \text{ ist beschränkt.}
$$
Mit Hilfe des Satzes von Itô-Nisio können wir nun ein ähnliches Resultat für symmetrische Zufallsvariablen formulieren. 

\begin{mydef}
    Eine Folge $(X_n)_{n \in \N}$ in $\mathcal{L}_0(E)$ heißt \textit{stochastisch beschränkt}, falls
    $$
        \forall \varepsilon >0 \ \exists R > 0: \quad \sup_{n \in \N}\prob{\norm{X_n} > R} < \varepsilon. 
    $$
\end{mydef}

\begin{lemma}
    Sei $(X_n)_{n \in \N}$ eine Folge von Radon-Zufallsvariablen und $X \in \mathcal{L}_0(E)$ mit $X_n \stochastisch X$. Dann ist $(X_n)_{n \in \N}$ stochastisch beschränkt. 
\end{lemma}
\begin{proof*}
    Sei $\varepsilon > 0$. Dann existiert nach Satz $2.8$ ein $N \in \N$ mit 
    $$
        \sup_{n \geq N}\prob{\norm{X_n - X} > \varepsilon} < \frac{\varepsilon}{2}.
    $$
    Wie man leicht einsieht, existiert ferner ein $R_0 > 0$ mit 
    $$
        \prob{ \norm{X} > R_0} < \frac{\varepsilon}{2} \ \text{ und } \ \max_{1 \leq n \leq N-1}\prob{\norm{X_n} > R_0} < \varepsilon.
    $$
    Für $R := \max\{R_0, \varepsilon\}$ gilt also 
    \begin{align*}
        \sup_{n \geq N}\prob{\norm{X_n} > R} &\leq \sup_{n \geq N}\prob{\norm{X_n - X} + \norm{X} > R} \\\
                                            &\leq \sup_{n \geq N}\prob{\norm{X_n - X} > \frac{R}{2}} + \prob{\norm{X} > \frac{R}{2}} \\\
                                            &< \sup_{n \geq N}\prob{\norm{X_n - X} > \frac{R}{2}} + \frac{\varepsilon}{2} < \varepsilon. 
    \end{align*}
    Zudem gilt nach Konstruktion
    $$
        \max_{1\leq n \leq N-1}\prob{\norm{X_n} > R} < \varepsilon. 
    $$
    Insgesamt erhalten wir somit 
    $$
        \sup_{n \in \N} \prob{\norm{X_n}>R}= \max\bigg(\max_{1\leq n \leq N-1}\prob{\norm{X_n} > R},\sup_{n \geq N}\prob{\norm{X_n}> R}\bigg) < \varepsilon. 
    $$
    \qed 
\end{proof*}
\begin{corollary}
    Sei $d \in \N$ und $(X_n)_{n \in \N}$ eine unabhängige und symmetrische Folge $\R^d$-wertiger Zufallsvariablen. Dann sind äquivalent: 
    \begin{enumerate}[(i)]
        \item $(S_n)_{n \in \N}$ konvergiert fast sicher. 
        \item $(S_n)_{n \in \N}$ ist stochastisch beschränkt. 
    \end{enumerate} 
\end{corollary}
\begin{proof*}
    Zu $(i) \Rightarrow (ii)$ Nach Korollar $2.7$ konvergiert $(S_n)_{n \in \N}$ insbesondere stochastisch und nach Lemma $3.12$ ist $(S_n)_{n \in \N}$ folglich stochastisch beschränkt.   
    \newline Zu $(ii) \Rightarrow (i)$: Aus $(ii)$ erhalten wir unmittelbar
    $$
        \forall \varepsilon > 0 \ \exists R > 0: \quad \inf_{n \in \N}\prob{S_n \in \overline{B}(0,R)} \geq 1 - \varepsilon.
    $$
    Da abgeschlossene und beschränkte Teilmengen des $\R^d$ nach dem Satz von Heine-Borel kompakt sind, folgt daraus die gleichmäßige Straffheit von $(\mu_n)_{n \in \N}$. 
    Nach Satz $3.9$ konvergiert $(S_n)_{n \in \N}$ also fast sicher. \qed 
\end{proof*}




%\section{Das Kontraktions Prinzip}
Im Folgenden sei $(X_n)_{n \in \N}$ eine unabhängige Folge symmetrischer Zufallsvariablens aus $\mathcal{L}_0(E)$ und $(\lambda_n)_{n \in \N}$ eine beschränkte Folge in $\R$. 
Für $n \in \N$ setze
$$
    S_n := \sum_{i=1}^nX_i, \quad T_n = \sum_{i=1}^n\lambda_iX_i. 
$$
Seien ferner $S_0 = T_0 = 0$. 
\begin{lemma}
    Für alle $t > 0$ und $N \in \N$ gilt
    \begin{align}
        \prob{\norm{T_N} > t} \leq 2 \prob{\norm{S_N} > t}. 
    \end{align}
\end{lemma}

\begin{proof*}
    \textbf{TODO}
\end{proof*}

\begin{theorem}[Kontraktions-Prinzip, Qualitative Version]
    Falls $(S_n)_{n \in \N}$ fast sicher konvergiert, dann konvergiert auch $(T_n)_{n \in \N}$ fast sicher. 
\end{theorem}
\begin{proof*}
    Sei $\varepsilon > 0$. Nach Lemma $3.11$ gilt für $m < n$
    $$
        \prob{\norm{T_n - T_m} > \varepsilon} \leq 2\prob{\norm{S_n - S_m} > \varepsilon}. 
    $$
    Nach dem Cauchy-Kriterium für stochastische Konvergenz konvergiert $(T_n)_{n \in \N}$ also stochastisch und nach dem Satz von Itô-Nisio insbesondere fast sicher. \qed
\end{proof*}

\begin{remark}
    Auf die Bedingung der Symmetrie kann nicht verzichtet werden, wie uns die deterministische Folge $X_n = (-1)^n \frac{1}{n}$ und $(\lambda_n)_{n \in \N} = ((-1)^{n})_{n \in \N}$ zeigen. 
    \newline 
    \textbf{TODO: Weitere Gegenbeispiele siehe Li,Queffelec}
\end{remark}
 
\appendix
\chapter{Funktionalanalytische und topologische Hilfsmittel}
\section{Separabilit"at}

\textbf{TODO}
\section{Der Satz von Hahn-Banach}
\textbf{Notation und Konventionen}\newline
Für einen normierten Vektorraum $(X, \norm{\cdot})$ sei $(X', \norm{\cdot}_op)$ der zugehörige Dualraum. Ferner bezeichne
$$
    B_{X'} := \{f \in X' \ : \ \norm{f}_op \leq 1 \}
$$
die abgeschlossene Einheitskugel in $X'$. 

\begin{mydef}
    Sei $X$ ein Vektorraum. Eine Abbildung $p:X \to \R$ heißt \textit{sublinear}, falls
    \begin{enumerate}[(a)]
        \item $p(\lambda x) = \lambda p(x)$ für alle $\lambda \geq 0, x \in X$,
        \item $p(x+y) \leq p(x) + p(y)$ für alle $x,y \in X$. 
    \end{enumerate}
\end{mydef}

\begin{proposition}
    Sei $X$ ein normierter Vektorraum und $F \subseteq X$ ein abgeschlossener Untervektorraum. Dann ist die Abbildung
    $$
        d(\cdot, F):X \to \R, y \mapsto d(y,F) := \inf_{x \in F}\norm{x-y}
    $$
    sublinear. 
\end{proposition}

\begin{proof*}
    Da $F$ ein Untervektorraum von $X$ ist, gilt für $\lambda \geq 0$ und $y \in X$ 
    $$
        d(\lambda y, F) = \inf_{x \in F}\norm{\lambda y - x} = \inf_{x \in F}\norm{\lambda y - \lambda x} = \lambda \inf_{x \in F}\norm{y-x} = \lambda d(y,F). 
    $$
    Ferner liefert die Dreiecksungleichung für $y,z \in X$
    $$
        d(y+z,F) = \inf_{x \in F}\norm{y+z - x} \leq \inf_{x \in F}(\norm{y - \frac{x}{2}} + \norm{z - \frac{x}{2}}) \leq \inf_{x \in F}\norm{y-x} + \inf_{x \in F}\norm{z-x} = d(y,F) + d(z,F). 
    $$
    Also ist $d(\cdot, F)$ sublinear. \qed 
\end{proof*}

\begin{theorem}[Hahn-Banach, Version der Linearen Algebra]
    Sei $X$ ein Vektorraum und $U$ ein Untervektorraum von $X$. Ferner seien $p: X \to \R$ sublinear und $f: U \to \R$ linear mit 
    $$
       \forall x \in U: \quad f(x) \leq p(x).
    $$
    Dann existiert eine lineare Fortsetzung $F:X \to \R$, mit $F|_U = f$ und
    $$
    \forall x \in X: \quad F(x) \leq p(x). 
    $$ 
\end{theorem}

\begin{theorem}[Hahn-Banach, Fortsetzungsversion]
    Sei $X$ ein normierter Raum und $U$ ein Untervektorraum. 
    Zu jedem stetigen linearen Funktional $f:U \to \R$ existiert ein stetiges lineares Funktional $F:X \to \R$ mit 
    $$
        F|_U = f \text{ und } \norm{F}_{op} = \norm{f}_{op}.
    $$
    Jedes stetige Funktional kann also normgleich fortgesetzt werden.
\end{theorem}

\begin{corollary}
    In jedem normierten Raum $X$ existiert zu jedem $x \in X$, $x \neq 0$, ein Funktional $f_x \in X'$ mit $\norm{f_x}_{op} = 1$ und $f_x(x) = \norm{x}$. 
    Speziell trennt $X'$  die Punkte von $X$, d.h.
    $$
        \forall x_1, x_2 \in X \text{ mit } x_1 \neq x_2 \ \exists f \in X': \quad f(x_1) \neq f(x_2). 
    $$
\end{corollary}

\begin{corollary}
    In jedem normierten Raum $X$ gilt 
    $$
        \forall x \in X : \quad \norm{x} = \sup_{f \in B_{X'}}\abs{f(x)}.
    $$
\end{corollary}

\begin{corollary}
    Falls $X$ separabel ist, so existiert eine abzählbare Menge $D \subseteq B_{X'}$ mit 
    $$
        \forall x \in X : \quad \norm{x} = \sup_{f \in D}\abs{f(x)}. 
    $$
\end{corollary}

\begin{proof*}
    Sei $E \subseteq X$ eine abzählbare dichte Teilmenge von $X$. Nach Korollar $A.10$ existiert für jedes $x \in E$ ein $f_x \in X'$ mit 
    $\norm{f_x}_{op} = 1$ und $f_x(x) = \norm{x}$. Setze also $D:=\{f_x : x \in E\}$. Sei nun $x \in X$ beliebig und $\varepsilon > 0$. Dann gilt zunächst 
    $$
        \forall n \in \N: \quad \abs{f_n(x)} \leq \norm{f_n}_{op} \norm{x} = \norm{x}. 
    $$
    Also
    $$
        \sup_{f \in D}\abs{f(x)} \leq \norm{x}. 
    $$
    Ferner existiert eine Folge $(y_n)_{n \in \N}$ in $E$ mit $\lim_{n \to \infty}\norm{x-y_n} = 0$. Insbesondere existiert $N \in \N$, sodass für alle $n \geq N$ 
    \begin{align}
        \norm{y_n - x} < \frac{\varepsilon}{2}.
    \end{align}
    Zudem gilt für alle $n \geq N$
    \begin{align}
        \abs{f_{y_n}(y_n)} \leq \abs{f_{y_n}(y_n)-f_{y_n}(x)} + \abs{f_{y_n}(x)} \leq \frac{\varepsilon}{2} + \abs{f_{y_n}(x)}.
    \end{align}
    Insgesamt ergibt sich also wegen $(A.1)$ und $(A.2)$ für $n \geq N$
    $$
        \norm{x} \leq \norm{x - y_n} + \norm{y_n} \leq \frac{\varepsilon}{2} + \abs{f_{y_n}(y_n)} \leq \abs{f_{y_n}(x)} + \varepsilon. 
    $$
    Da $\varepsilon > 0$ und $x \in X$ beliebig gewählt waren, gilt also 
    $$
        \forall x \in X: \ \norm{x} = \sup_{f \in D}\abs{f(x)}.  
    $$
\end{proof*}

\begin{corollary}
    Sei $X$ separabel und $F \subseteq X$ ein abgeschlossener Untervektorraum.
    Dann existiert eine Folge $(f_n)_{n \in \N}$ in $X'$ mit $\norm{f_n}_{op} \leq 1$ für alle $n \in \N$ und 
    $$
        \forall x \in X: \quad d(x,F) = \inf_{y \in F}\norm{x-y} = \sup_{n \in \N}\abs{f_n(x)}. 
    $$
\end{corollary}
\begin{proof*}
    Da $X$ separabel ist, existiert eine abzählbare dichte Teilmenge $D = \{x_1,x_2,...\} \subseteq X$. Sei nun $n \in \N$. Dann gilt $lin(\{x_n\}) = \{\lambda x_n: \lambda \in \R\}$ und die Abbildung
    $$
        g_n: lin(\{x_n\}) \to \R, \quad \lambda x_n \mapsto \lambda d(x_n,F)
    $$
    ist, wie man leicht nachrechnet, linear. Nach Proposition $A.7$ ist ferner die Abbildung 
    $$
        d(\cdot, F):X \to \R, y \mapsto d(y,F) := \inf_{x \in F}\norm{x-y}
    $$
    sublinear. Für $z = \lambda x_n \in lin(\{x_n\})$ gilt zudem 
    $$
        g_n(z) = \lambda d(x_n,F) \leq \abs{\lambda} \inf_{y \in F}\norm{x-y} = \inf_{y \in F}\norm{\lambda x - \lambda y} = d(z,F). 
    $$
    Also existiert nach Satz $A.8$ für alle $n \in \N$ ein Funktional $f_n \in X'$ mit $f_n|_{lin(\{x_n\})} = g_n$ und 
    $$
        \forall y \in X: \quad f_n(y) \leq d(y,F) = \inf_{x \in F}\norm{x-y} \leq \norm{y}. 
    $$
    Somit gilt insbesondere $\norm{f_n}_{op} \leq 1$. Betrachte nun die so gewonnene Folge $(f_n)_{n \in \N}$ in $X'$. Für $x \in X$ gilt dann nach Konstruktion
    $$
        \forall n \in \N: \quad f_n(x) \leq d(x,F). 
    $$
    Daraus folgt direkt
    $$
        \sup_{n \in \N}\abs{f_n(x)} \leq d(x,F).
    $$
    Sei nun $\varepsilon > 0$. Da $D$ dicht in $X$ liegt, gibt es ein $n \in \N$, sodass $\norm{x_n - x} < \frac{\varepsilon}{2}$. Daher gilt wegen $\norm{f_n}_{op} \leq 1$
    $$
        \abs{f_n(x_n)} \leq \abs{f_n(x_n) - f_n(y)} + \abs{f_n(y)} \leq \abs{f_n(y)} + \frac{\varepsilon}{2}. 
    $$
    Daraus folgt schließlich wegen der sublinearität von $d(\cdot,F)$
    $$
        d(x,F) \leq d(x-x_n,F) + d(x_n,F) \leq \norm{x-x_n} + \abs{f_n(x_n)} \leq \abs{f_n(y)} + \varepsilon. 
    $$
    Da $\varepsilon > 0$ beliebig gewählt war, folgt insgesamt
    $$
        \sup_{n \in \N}\abs{f_n(x)} = d(x,F).
    $$
    \qed
\end{proof*}


\addcontentsline{toc}{chapter}{Literaturverzeichnis}
\bibliography{literatur}
\bibliographystyle{babplain-lf}


\chapter*{Eigenständigkeitserklärung}

Hiermit bestätige ich, Jonas Köppl, dass ich die
vorliegende Arbeit selbstständig und ohne unzulässige Hilfe verfasst und keine
anderen als die angegebenen Quellen und Hilfsmittel benutzt sowie die wörtlich und
sinngemäß übernommenen Passagen aus anderen Werken kenntlich gemacht habe.
Die Arbeit ist weder von mir noch von einer anderen Person an der Universität
Passau oder an einer anderen Hochschule zur Erlangung eines akademischen
Grades bereits eingereicht worden.\\

\begin{flushleft}
Ort, Datum:\\\
\newline 
Unterschrift:
\end{flushleft}
\addcontentsline{toc}{chapter}{Eigenständigkeitserklärung}

\end{document}