\documentclass{report}

% IMPORTS
\usepackage{ntheorem} % Theorem/Proof Umgebungen 
\usepackage[ngerman]{babel}
\usepackage[utf8x]{inputenc}
\usepackage{amsfonts}
\usepackage{amsmath}
\usepackage{amssymb}
\usepackage{amstext}
\usepackage{enumerate} % für kleine römische Nummerierung
\usepackage{url} % Für Hyperlinks
\usepackage[singlespacing]{setspace} %1,5 Zeilenabstand 

% Für andere Kapitelüberschriften:
\usepackage{titlesec}
\titleformat{\chapter}
  {\normalfont\LARGE\bfseries}{\thechapter}{1em}{}
\titlespacing*{\chapter}{0pt}{3.5ex plus 1ex minus .2ex}{2.3ex plus .2ex}

% Seitenränder 
\usepackage[a4paper, left=4cm, right=4cm, top=2cm]{geometry}

% NEWCOMMANDS
% Syntax: \newcommand{\shortcut}[AnzahlArgumente]{\wasrauskommensoll #fürArgumentNummer}

% Nützliches

\newcommand{\qed}{\hfill $\square$}
\newcommand{\norm}[1]{\lvert \lvert #1 \rvert \rvert}
% Shortcuts für Zahlenbereiche:
\newcommand{\N}{\mathbb{N}} % Natürliche Zahlen 
\newcommand{\R}{\mathbb{R}} % Reelle Zahlen
\newcommand{\Z}{\mathbb{Z}} % Ganze Zahlen
\newcommand{\C}{\mathbb{C}} % Komplexe Zahlen
\newcommand{\Q}{\mathbb{Q}} % Rationale Zahlen 

% Oft verwen



% THEOREMSTYLES

\theorembodyfont{\upshape}
\theoremstyle{changebreak} %Sorgt für Zeilenumbruch nach Titel und Nummer vor Name

\newtheorem{theorem}{Satz} % Sätze
\numberwithin{theorem}{chapter}

\newtheorem{mydef}[theorem]{Definition} % Definitionen
\numberwithin{theorem}{chapter} %sorgt für stringente Nummerierung

\newtheorem{example}[theorem]{Beispiel} % Beispiele
\numberwithin{theorem}{chapter} %sorgt für stringente Nummerierung

\newtheorem{proposition}[theorem]{Proposition} % Propositionen
\numberwithin{theorem}{chapter} %sorgt für stringente Nummerierung

\newtheorem{remark}[theorem]{Bemerkung} %Bemerkungen
\numberwithin{theorem}{chapter} %sorgt für stringente Nummerierung

\newtheorem{lemma}[theorem]{Lemma} %Lemmata
\numberwithin{theorem}{chapter} %sorgt für stringente Nummerierung

\newtheorem{corollary}[theorem]{Korollar} %Korollare
\numberwithin{theorem}{chapter} %sorgt für stringente Nummerierung

\newtheorem*{proof*}{Beweis.} %Beweise ohne Nummerierung! 

\setlength\parindent{0pt} %Kein Einzug bei neuem Paragraphen. 

\begin{document}

\tableofcontents

\chapter{Ma\ss theoretische Vorbereitungen}
Bevor wir uns im späteren Verlauf der Arbeit mit zufälligen Reihen in Banachräumen beschäftigen können, benötigen wir ein paar maßtheoretische Vorbereitungen. 
Wir beginnen hierbei mit einigen grundlegenden Eigenschaften Borelscher $\sigma$-Algebren und darauf definierten Wahrscheinlichkeitsmaßen. 
Danach gehen wir kurz auf messbare Vektorräume ein und führen dann den Begriff der Radon-Zufallsvariable mit Werten in einem Banachraum ein. 
Da die zusätzliche algebraische Struktur eines Banachraums für unsere Betrachtung zunächst nicht von Bedeutung ist, 
werden wir uns in den ersten Abschnitten dieses Kapitels mit dem allgemeineren Fall eines (vollständigen) metrischen Raumes beschäftigen. 
Die Darstellung der ersten drei Abschnitte orientiert sich an den beiden Standardwerken \cite{parthasarathy} und \cite{billingsley}. Der kleine Exkurs zur Topologie stammt aus \cite{preuss}. 
Die Abschnitte zu messbaren Vektorräumen und Zufallsvariablen mit Werten in Banachräumen fußen auf \cite{vakhania} und \cite{ledoux-talagrand}. 
\section{Borelmengen in metrischen Räumen}
Für einen metrischen Raum $(X,d)$ bezeichne im Folgenden $\mathcal{B}(X)$ die Borel-$\sigma$-algebra in $X$. 
Zudem werden für $x \in X$ und $r>0$ mit $B(x, r)$ bzw. $\overline{B}(x,r)$ die offene bzw. abgeschlossene Kugel um $x$ mit Radius $r$ bezeichnet.
\begin{proposition}
    Sei $(X,d)$ ein separabler metrischer Raum. Dann gilt
    \begin{align*}
        \mathcal{B}(X) = \sigma(\{B(x,r): x \in X, r > 0 \}) = \sigma(\{\overline{B}(x,r): x \in X, r > 0 \}). 
    \end{align*}
\end{proposition}
\begin{proof*}
    Setze 
    \begin{align*}
        \mathcal{A}_1 &:= \sigma(\{B(x,r): x \in X, r > 0 \}), \\\ 
        \mathcal{A}_2 &:= \sigma(\{\overline{B}(x,r): x \in X, r > 0 \}). 
    \end{align*}
    Man sieht leicht ein, dass $\mathcal{A}_2 = \mathcal{A}_1 \subseteq \mathcal{B}(X)$. Zu zeigen bleibt also nur die Inklusion $\mathcal{B}(X) \subseteq \mathcal{A}_1$.
    Sei dazu $U \subseteq X$ offen und $x \in U$. Nach Voraussetzung existiert eine abzählbare dichte Teilmenge $D \subseteq X$. Definiere 
    \begin{align*}
        R := \{(y,r) : y \in U \cap D, r > 0, r \in \Q, B(y,r) \subseteq U \}.
    \end{align*}
    Dann ist $R$ abzählbar und da $D$ dicht in $X$ liegt gilt $U = \bigcup_{(y,r) \in R}B(y,r)$. 
    Also gilt $U \in \mathcal{A}_1$ und da $\mathcal{B}(X)$ von den offenen Teilmengen von $X$ erzeugt wird folgt die Behauptung. \qed
\end{proof*}

\begin{proposition}
    Für $i \in \N$ sei $(X_i, d_i)$ ein separabler metrischer Raum. Dann gilt
    \begin{align*}
        \mathcal{B}(X_1 \times X_2 \times ...) = \otimes_{i=1}^{\infty}\mathcal{B}(X_i)
    \end{align*}
\end{proposition}

\begin{proof*}
    Setze $X:= \times_{k \in \N}X_k$ und bezeichne $p_k: X \to X_k$ die Projektion auf die k-te Komponente. Betrachte das Mengensystem
    \begin{align*}
        \mathcal{E} :&= \{ \bigcap_{k \in K}p_k^{-1}(O) | \forall k \in K: O_k \subseteq X_k \text{ offen}, K \subseteq \N \text{ endlich}\}. 
    \end{align*}
    Offensichtlich gilt $\otimes_{k \in \N}\mathcal{B}(X_k) = \sigma(\mathcal{E})$. 
    Ferner ist $X$ ein separabler metrischer Raum und $\mathcal{E}$ eine Basis der Produkttopologie auf $X$, vgl. \cite{querenburg}[3.7]. 
    Also lässt sich jede offene Menge $O \subset X$ als abzählbare Vereinigung von Elementen aus $\mathcal{E}$ darstellen.  
    Dies impliziert 
    $$
    \mathcal{B}(X) = \sigma(\mathcal{E}) = \otimes_{k \in N}\mathcal{B}(X_k).
    $$
    \qed
\end{proof*}

\section{Borelmaße auf metrischen Räumen}
Im Folgenden Abschnitt beschäftigen wir uns mit Wahrscheinlichkeitsmaßen auf der Borelschen $\sigma$-Algebra $\mathcal{B}(X)$ eines metrischen Raums $(X,d)$, 
welche teilweise auch als \textit{Borelsche Wahrscheinlichkeitsmaße} bezeichnet werden. 
Zunächst interessieren wir uns hierbei für Regularitätseigenschaften solcher Maße, welche uns die spätere Arbeit erleichtern werden. 
Im zweiten Teil dieses Abschnitts untersuchen wir Folgen von Wahrscheinlichkeitsmaßen und den Begriff der schwachen Konvergenz. 
\newline \ \newline 
\textbf{Notation und Konventionen} 
\newline
Bis auf weiteres sei $(X,d)$ ein metrischer Raum mit Borel-$\sigma$-algebra $\mathcal{B}(X)$. 
Für eine Teilmenge $A \subseteq X$ sei $\mathring{A}$ das \textit{Innere}, $\overline{A}$ der \textit{Abschluss} und 
$$
    \partial A := \overline{A} \setminus \mathring{A}
$$
der \textit{Rand der Menge}. 
Setze ferner 
$$
    C_b(X;\R) := \{f:X \to \R \ | \ f \text{ ist stetig und beschränkt} \}.
$$ 
Falls klar ist welcher metrische Raum gemeint ist, so schreiben wir auch $C_b(\R)$ statt $C_b(X;\R)$. 
\newline 



\begin{mydef}
    Ein Maß $\mu$ auf $\mathcal{B}(X)$ heißt \textit{regulär} , falls
    \begin{align*}
        \forall B \in \mathcal{B}(X): \quad \mu(B) &= \sup\{\mu(C): C \subseteq B, \ C \text{ abgeschlossen} \} \\\
                                                   &= \inf\{\mu(O): B \subseteq O, \ O \text{ offen} \}.  
    \end{align*}  
\end{mydef}

\begin{proposition}
    Sei $\mu$ ein Wahrscheinlichkeitsmaß auf $\mathcal{B}(X)$. Dann ist $\mu$ regulär. 
\end{proposition}

\begin{proof*}
    Wir verwenden zum Beweis das Good-Set-Principle. Setze dazu
    $$
        \mathcal{R}:=\big\{A \in \mathcal{B}(X):  \mu(A) = \sup\{\mu(C): C \subseteq A, \ C \text{ abgeschlossen}\} =\inf\{\mu(O): A \subseteq O, \ O \text{ offen}\} \big\}.
    $$
    Wir zeigen zunächst, dass $\mathcal{R}$ eine $\sigma$-Algebra ist. Offensichtlich gilt $\emptyset \in \mathcal{R}$. Sei nun $A \in \mathcal{R}$ und $\varepsilon > 0$. 
    Dann existieren eine offene Menge $O$ und eine abgeschlossene Menge $C$ mit $C \subseteq A \subseteq O$ und 
    $$
        \mu(0) - \varepsilon < \mu(A) < \mu(C) + \varepsilon.
    $$
    Es gilt also $O^c \subseteq A^c \subseteq C^c$ und 
    $$
        \mu(C^c) - \varepsilon = 1 - \mu(C) - \varepsilon < 1 - \mu(A) = \mu(A^c) = 1 - \mu(A) < 1 - \mu(O) + \varepsilon = \mu(O^c) + \varepsilon. 
    $$
    Da $O^c$ abgeschlossen ist und $C^c$ offen ist, folgt $A^c \in \mathcal{R}$. 
    \newline 
    Seien nun $A_1, A_2,... \in \mathcal{R}$ und $\varepsilon > 0$ Für jedes $n \in \N$ existieren dann eine offene Menge $O_n$ und eine abgeschlossene Menge $C_n$ mit $C_n \subseteq A_n \subseteq O_n$ und 
    $$
        \mu(O_n) - 2^{-n}\varepsilon < \mu(A_n) < \mu(C_n) + 2^{n+1}\varepsilon.
    $$
    Es gilt also
    $$
        \bigcup_{n = 1}^{\infty} C_n \subseteq \bigcup_{n=1}^{\infty} A_n \subseteq \bigcup_{n=1}^{\infty}C_n
    $$
    und 
    \begin{align}
        \mu\left(\bigcup_{n =1}^{\infty}U_n\right) - \mu\left(\bigcup_{n=1}^{\infty}A_n\right) &\leq \mu\left(\bigcup_{n =1}^{\infty}U_n\setminus \bigcup_{n =1}^{\infty} A_n\right) \nonumber \\\
                                                                                &\leq \mu\left(\bigcup_{n =1}^{\infty}(U_n\setminus A_n)\right) 
                                                                                \leq \sum_{i=1}^{\infty}\mu(U_n\setminus A_n)
                                                                                < \sum_{i=1}^{\infty}2^{-n}\varepsilon = \varepsilon. 
    \end{align}
    Wegen 
    $$
    \mu(\cup_{n = 1}^{\infty}C_n) = \lim_{k \to \infty}\mu(\cup_{n=1}^kC_n),
    $$
    existiert zudem ein $k \in \N$ mit 
    $$
    \mu(\cup_{n=1}^{\infty}C_n) - \mu(\cup_{n=1}^kC_n) < \frac{\varepsilon}{2}.
    $$ 
    Die Menge $C:= \cup_{n=1}^kC_n$ ist als endliche Vereinigung abgeschlossener Mengen abgeschlossen und nach Konstruktion in $\cup_{n=1}^{\infty}A_n$ enthalten. Ferner gilt 
    \begin{align}
        \mu(\bigcup_{n=1}^{\infty}A_n) - \mu(C) &< \mu(\bigcup_{n=1}^{\infty}A_n) - \mu(\bigcup_{n=1}^{\infty}C_n) + \frac{\varepsilon}{2} \nonumber \\\
                                                &\leq \mu(\bigcup_{n=1}^{\infty}A_n \setminus \bigcup_{n=1}^{\infty}C_n)  + \frac{\varepsilon}{2} \nonumber \\\
                                                &\leq \mu(\bigcup_{n=1}^{\infty}(A_n \setminus C_n)) + \frac{\varepsilon}{2} \nonumber \\\
                                                &\leq \sum_{n=1}^{\infty}\mu(A_n \setminus C_n) + \frac{\varepsilon}{2} = \varepsilon. 
    \end{align}
    Da $\varepsilon$ beliebig gewählt war folgt aus $(1.1)$ und $(1.2)$, dass $\cup_{n=1}^{\infty}A_n \in \mathcal{R}$. Folglich ist $\mathcal{R}$ eine $\sigma$-Algebra. 
    Es bleibt noch zu zeigen, dass $\mathcal{R}$ alle abgeschlossenen Mengen enthält. Sei also $\emptyset \neq A \subseteq X$ abgeschlossen. Die Bedingung 
    $$
        \mu(A) = \sup\{\mu(C): C \subseteq A, C\text{ abgeschlossen }\}
    $$
    folgt direkt aus der Monotonie von $\mu$. Um die zweite Bedingung zu zeigen setze für $n \in \N$
    $$
        O_n := \big\{x \in X: \inf_{y \in  A}d(x,y) < \frac{1}{n}\big\}.
    $$
    Dann ist $O_n$ für jedes $n \in \N$ offen und es gilt $O_1 \supseteq O_2 \supseteq ...$, sowie $\cap_{n=1}^{\infty}O_n = A$, da $A$ abgeschlossen ist. 
    Mit der $\sigma$-Stetigkeit von $\mu$ folgt letztendlich 
    $$
        \mu(A) \leq \inf\{\mu(O): A \subseteq O, O \text{ offen}\} \leq \inf_{n \in \N}\mu(O_n) = \lim_{n \to \infty}\mu(O_n) = \mu(A). 
    $$
    Da die abgeschlossenen Mengen $\mathcal{B}(X)$ erzeugen gilt $\mathcal{R} = \mathcal{B}(X)$ und folglich ist $\mu$ regulär. \qed
\end{proof*}
\begin{mydef}
    Ein Maß $\mu$ auf $\mathcal{B}(X)$ heißt \textit{straff}, falls es für alle $\varepsilon > 0$ eine kompakte Menge $K \subseteq X$ gibt mit 
    \begin{align*}
        \mu(K) \geq 1 - \varepsilon. 
    \end{align*}

\end{mydef}

\begin{corollary}
    Sei $\mu$ ein straffes Wahrscheinlichkeitsmaß auf $\mathcal{B}(X)$. Dann gilt
    \begin{align*}
        \forall A \in \mathcal{B}(X): \quad \mu(A) = \sup\{\mu(K): K \subseteq A,\ K \text{ kompakt}\}. 
    \end{align*}
\end{corollary}

\begin{proof*}
    Sei $A \in \mathcal{B}(X)$ und $\varepsilon > 0$. Wegen der Straffheit von $\mu$ existiert eine kompakte Menge $K_{\varepsilon} \subseteq X$ mit $\mu(K_{\varepsilon}) \geq 1 - \frac{\varepsilon}{2}$,
    und da $\mu$ nach Proposition $1.4$ regulär ist gibt es eine abgeschlossene Menge $C \subseteq A$ mit $\mu(C) > \mu(A) - \frac{\varepsilon}{2}$. Die Menge $K_{\varepsilon} \cap C$ ist wiederum kompakt und es gilt
    \begin{align*}
        \mu(A) \geq \mu(K_{\varepsilon} \cap C) > \mu(C) - \frac{\varepsilon}{2} > \mu(A) - \varepsilon. 
    \end{align*} 
    \qed
\end{proof*}

\begin{remark}
    Ein Wahrscheinlichkeitsmaß $\mu$ auf $\mathcal{B}(X)$ mit der Eigenschaft
    \begin{align*}
        \forall A \in \mathcal{B}(X): \quad \mu(A) = \sup\{\mu(K): K \subseteq A, \ K \text{ kompakt}\}. 
    \end{align*}
    wird auch als \textit{Radon-Wahrscheinlichkeitsmaß} oder \textit{Radon-Maß} bezeichnet.
\end{remark}

\begin{proposition}
    Sei $(X,d)$ ein vollständiger separabler metrischer Raum. Dann ist jedes Wahrscheinlichkeitsmaß $\mu$ auf $\mathcal{B}(X)$ straff.
\end{proposition}

Wir verwenden zum Beweis der Proposition die folgende Charakterisierung kompakter Teilmengen metrischer Räume. 
Der Beweis wird mittels der in metrischen Räumen geltenden Äquivalenz von Überdeckungskompaktheit und Folgenkompaktheit  geführt und findet sich etwa in \cite{amann}[Theorem 3.10]. 

\begin{lemma}
    Sei $(X,d)$ ein metrischer Raum. Eine Menge $K \subseteq X$ ist genau dann kompakt, wenn sie die folgenden beiden Eigenschaften erfüllt
    \begin{enumerate}[(a)]
        \item $K$ ist vollständig,
        \item $K$ ist total-beschränkt, d.h.
        \begin{align*}
            \forall  \varepsilon > 0 \ \exists x_1,...,x_n \in K: \  K \subseteq \cup_{i=1}^n B(x_i, \varepsilon). 
        \end{align*} 
\end{enumerate}
\end{lemma}

\begin{proof*}
    Sei $\varepsilon > 0$. Nach Voraussetzung existiert eine abzählbare dichte Teilmenge $D = \{x_1, x_2,...\}$ von $X$. Also gilt insbesondere für $q \in \N $
    \begin{align*}
        \bigcup_{i = 1}^{\infty}\overline{B}(x_i, 2^{-q}) = X.
    \end{align*}
    Wegen der $\sigma$-Stetigkeit von $\mu$ existiert also ein $N_q \in \N$ mit 
    \begin{align*}
        \mu\big(\bigcup_{i=1}^{N_q}\overline{B}(x_i, 2^{-q})\big) \geq 1 - \varepsilon 2^{-q}. 
    \end{align*}
    Setze nun 
    \begin{align*}
        K := \bigcap_{q = 1}^{\infty}\bigcup_{i=1}^{N_q}\overline{B}(x_i, 2^{-q}). 
    \end{align*}
    Dann ist $K$ als Schnitt abgeschlossener Teilmengen abgeschlossen, und da $X$ vollständig ist, folgt daraus bereits die Vollständigkeit von $K$. 
    Ferner ist $K$ total-beschränkt, denn zu $\varepsilon > 0$ existiert ein $q \in \N$ mit $2^{-q} < \varepsilon$ und $K \subseteq \cup_{i=1}^{N_q}B(x_i, 2^{-q}) \subseteq \cup_{i=1}^{N_q}B(x_i, \varepsilon)$. 
    Zudem gilt
    \begin{align*}
        \mu(K)  = 1 - \mu\big(\bigcup_{q = 1}^{\infty}\bigcap_{i=1}^{N_q}\overline{B}(x_i, 2^{-q})^c\big) 
                &\geq 1 - \sum_{q=1}^{\infty} \mu\big(\bigcap_{i=1}^{N_q}\overline{B}(x_i, 2^{-q})^c\big) \\\
                &\geq 1 - \sum_{q=1}^{\infty} \varepsilon 2^{-q} = 1 - \varepsilon.
    \end{align*}
    Also ist $\mu$ straff. \qed
\end{proof*}

\begin{proposition}
    Sei $(X,d)$ ein vollständiger metrischer Raum und $\mu$ ein Wahrscheinlichkeitsmaß auf $\mathcal{B}(X)$. Dann sind äquivalent
    \begin{enumerate}[(i)]
        \item $\mu$ ist straff.
        \item Es gibt eine separable Teilmenge $E \subseteq X$ mit $\mu(E) = 1$. 
    \end{enumerate}
\end{proposition}
\begin{proof*}
    zu (i) $\Rightarrow$ (ii): Für alle $n \in \N$ existiert $K_n \subseteq X$ kompakt mit $\mu(K_n) \geq 1 - \frac{1}{n}$, o.E. gelte $K_n \subseteq K_{n+1}$. Es folgt 
    \begin{align*}
        \mu\big(\cup_{n=1}^{\infty}K_n\big) = \lim_{n \to \infty}\mu\big(K_ n\big) = 1. 
    \end{align*}
    Da kompakte Teilmengen metrischer Räume insbesondere separabel sind, ist $E := \cup_{n=1}^{\infty}K_n$ als abzählbare Vereinigung separabler Mengen ebenso separabel. 
    \newline 
    zu (ii) $\Rightarrow$ (i): 
    Analog zum Beweis von Proposition 1.8. \qed
\end{proof*}

Insgesamt haben wir also gezeigt

\begin{theorem}
    Für ein Wahrscheinlichkeitsmaß $\mu$ auf der Borelschen $\sigma$-Algebra eines vollständigen metrischen Raumes $(X,d)$ sind äquivalent
    \begin{enumerate}[(i)]
        \item $\mu$ ist straff.
        \item Es gibt eine separable Menge $E \subseteq X$ mit $\mu(E) = 1$.
        \item $\mu$ ist ein Radon-Maß, d.h.
        $$
        \forall A \in \mathcal{B}(X): \quad \mu(A) = \sup\{\mu(K): K \subseteq A, \ K \text{ kompakt}\}.
        $$   
    \end{enumerate}
\end{theorem}

Nachdem wir uns nun ausgiebig mit den Eigenschaften einzelner Maße beschäftigt haben, möchten wir uns jetzt mit Folgen von Wahrscheinlichkeitsmaßen und deren Konvergenz beschäftigen. 

\begin{mydef}
    Eine Folge $(\mu_n)_{n \in \N}$ von Wahrscheinlichkeitsmaßen auf $\mathcal{B}(X)$ heißt \textit{schwach konvergent} 
    gegen ein Wahrscheinlichkeitsmaß $\mu$ auf $\mathcal{B}(X)$, falls 
    $$
        \forall f \in C_b(X): \quad \lim_{n \to \infty} \int_Xfd\mu_n = \int_X fd\mu . 
    $$
    Bezeichnung: $\mu_n \rightharpoonup \mu$. 
\end{mydef}

Als nützliches Hilfsmittel für viele Beweise dient der folgende Satz, der meist als \textit{Portmanteau-Theorem} bezeichnet wird. 

\begin{theorem}[Portmanteau-Theorem]
    Sei $(X,d)$ ein metrischer Raum und seien $\mu, \mu_1, \mu_2,...$ Wahrscheinlichkeitsmaße auf $\mathcal{B}(X)$. Dann sind äquivalent
    \begin{enumerate}[(i)]
        \item $(\mu_n)_{n \in \N}$ konvergiert schwach gegen $\mu$.
        \item Für alle abgeschlossenen Teilmengen $A \subseteq X$ gilt 
        $$
            \limsup_{n \to \infty} \mu_n(A) \leq \mu(A).
        $$
        \item Für alle offenen Teilmengen $B \subseteq X$ gilt 
        $$
            \liminf_{n \to \infty} \mu_n(B) \geq \mu(B).
        $$
        \item Für alle Borelmengen $C \in \mathcal{B}(X)$ mit $\mu(\partial C) = 0$ gilt 
        $$
            \lim_{n \to \infty}\mu_n(C) = \mu(C).
        $$
    \end{enumerate}
\end{theorem}

\begin{proof*}
    Zu $(i) \Rightarrow (ii)$: Sei $A \subseteq X$ abgeschlossen und $\varepsilon > 0$. Da die Aussage für $A = \emptyset$ trivialerweise erfüllt ist
    können wir ohne Beschränkung der Allgemeinheit annehmen, dass $A \neq \emptyset$. Für $m \in \N$ setze
    \begin{align*}
        U_m := \{x \in X: d(x,A) < \frac{1}{m}\}.
    \end{align*}
    Dann sind die Mengen $U_m$ offen und es für alle $m \in \N$ gilt $A \subseteq U_m$. Da $A$ abgeschlossen ist erhalten wir zudem $ A = \cap_{m \in \N} U_m$. 
    Aufgrund der $\sigma$-Stetigkeit von $\mu$ existiert ein $k \in \N$ mit 
    $$
        \mu(U_k) < \mu(A) + \varepsilon . 
    $$
    Betrachte nun die Abbildung 
    $$
        f:X \to \R, \quad x \mapsto \max\{1 - kd(x,A), 0\}.
    $$
    Offensichtlich ist $f$ beschränkt und nach der umgekehrten Dreiecksungleichung auch stetig. Wegen $1_A \leq f \leq 1_{U_k}$ erhalten wir zusammen mit der Voraussetzung 
    $$
    \limsup_{n \to \infty} \mu_n(A) \leq \lim_{n \to \infty} \int_X fd\mu_n = \int_X fd\mu \leq \mu(U_k) \leq \mu(A) + \varepsilon.
    $$
    Da diese Ungleichung für alle $\varepsilon > 0$ erfüllt ist folgt die Behauptung. 
    \newline 
    Zu $(ii) \iff (iii)$: Folgt unmittelbar durch Komplementbildung. 
    \newline
    Zu $(iii) \Rightarrow (iv)$: 
    Sei $C \in \mathcal{B}(X)$ mit $\mu(\partial C) = 0$. Dann gilt insbesondere $\mu(\overline{C}) = \mu(\mathring{C}))$. Da $(iii)$ auch $(ii)$ impliziert erhalten wir somit
    $$
        \mu(C) = \mu(\mathring{C})) \leq \liminf_{n \to \infty} \mu_n(C) \leq \limsup_{n \to \infty} \mu_n(C) \leq \mu(\overline{C}) = \mu(C).
    $$
    \newline 
    Zu $(iv) \Rightarrow (i)$: 
    Sei $f \in C_b(X)$ beschränkt durch $M > 0$. Wegen der Linearität des Integrals können wir ohne Einschränkung annehmen, dass $f \geq 0$. 
    Wegen der Stetigkeit von $f$ erhalten wir zunächst für alle $t > 0$
    $$
        \partial\{ f > t \} \subseteq \{f = t \}. 
    $$
    Da $\mu$ ein Wahrscheinlichkeitsmaß ist, gibt es eine abzählbare Menge $C \subseteq \R$ mit 
    $$
        \forall t \in \R \setminus C: \quad \mu(\{f = t \}) = 0. 
    $$
    Also gilt für alle $t \in \R \setminus C$ nach Voraussetzung 
    $$
        \lim_{n \to \infty} \mu_n(\{f > t \}) = \mu(\{f > t \})
    $$
    Mit dem Satz von Cavalieri, vgl. \cite{gs}[1.8.20], erhalten wir schließlich per dominierter Konvergenz
    $$
        \lim_{n \to \infty} \int_X fd\mu_n = \lim_{n \to \infty} \int_0^M \mu_n(\{f > t \})d\lambda(t) = \int_0^M \mu(\{f > t \}) d\lambda(t) = \int_Xfd\mu. 
    $$
    \qed 
\end{proof*}
\section{Die Prokhorov Metrik}
Nachdem wir im letzten Abschnitt damit begonnen haben uns mit der schwachen Konvergenz von Wahrscheinlichkeitsmaßen zu beschäftigen, 
wollen wir nun ein weiteres Hilfsmittel zur Untersuchung von schwacher Konvergenz einführen. 
\newline 
Für einen metrischen Raum $(X,d)$ bezeichne $\mathcal{M}(X)$ die Menge aller Wahrscheinlichkeitsmaße auf der Borelschen $\sigma$-Algebra $\mathcal{B}(X)$. 
\begin{mydef}
    Eine Familie $M \subseteq \mathcal{M}(X)$ von Wahrscheinlichkeitsmaßen heißt \textit{gleichmäßig straff}, 
    falls es für alle $\varepsilon > 0$ eine kompakte Menge $K \subseteq X$ gibt mit 
    $$
        \forall \mu \in M: \mu(K) \geq 1-\varepsilon. 
    $$
\end{mydef}
Ziel dieses Abschnitts ist der Satz von Prokhorov, der uns eine für spätere Beweise wichtige Charakterisierung der gleichmäßigen Straffheit liefert. 

Ein wichtiges Resultat ist die Folgende auf Prokhorov zurückgehende Charakterisierung. Ein Beweis findet sich etwa in \cite{parthasarathy}[Theorem 6.7]. 

\begin{theorem}[Satz von Prokhorov]
    Sei $(X,d)$ ein vollständiger separabler metrischer Raum und $M \subseteq \mathcal{M}(X)$. Dann sind äquivalent
    \begin{enumerate}[(i)]
        \item $M$ ist relativ kompakt,
        \item $M$ ist gleichmäßig straff. 
    \end{enumerate}
\end{theorem}
\newpage
\section{Flache Konzentrierung von Wahrscheinlichkeitsmaßen}
Quelle: \cite{vakhania}, Originalpaper: de Acosta \textbf{(TODO: zitieren)}
Für eine nichtleere Teilmenge $A \subseteq E$ setze 
$$
    A^{\varepsilon} := \{x \in E: \inf_{y\in A}\norm{x-y} < \varepsilon \}.
$$
Ferner sei daran erinnert, dass jeder endlich dimensionale Untervektorraum  $S \subseteq E$ abgeschlossen ist und eine Menge $A \subseteq S$ genau dann kompakt ist, wenn sie abgeschlossen und beschränkt ist. 
Insbesondere besitzt jede beschränkte Folge in $S$ eine konvergente Teilfolge. 
\begin{mydef}
    Eine Familie $M \subseteq \mathcal{M}(E)$ von Wahrscheinlichkeitsmaßen auf $\mathcal{B}(E)$ heißt \textit{flach konzentriert}, falls es für alle $\varepsilon > 0$ einen endlichdimensionalen 
    Untervektorraum $S \subseteq E$ gibt mit 
    $$
        \forall \mu \in M: \quad \mu(S^{\varepsilon}) \geq 1 - \varepsilon.
    $$ 
\end{mydef}

\begin{lemma}
    Eine Teilmenge $A$ von $E$ ist genau dann relativ kompakt, wenn $A$ beschränkt ist und es für alle $\varepsilon > 0$ einen endlichdimensionalen Untervektorraum $S \subseteq E$ gibt mit 
    \begin{align*}
        A \subseteq S^{\varepsilon}
    \end{align*}
\end{lemma}

\begin{proof*}
    zu $\Rightarrow$: Sei $A \subseteq E$ relativ kompakt. Dann ist $\overline{A}$ kompakt und folglich beschränkt, woraus wir direkt die Beschränktheit von $A$ erhalten. 
    Ferner ist $\overline{A}$ separabel, also existiert eine abzählbare dichte Teilmenge $\{x_1, x_2,...\} \subseteq A$. 
    Für $\varepsilon > 0$ ist daher $(B(x_n, \varepsilon))_{n \in \N}$ eine offene Überdeckung von $\overline{A}$. Wegen der Kompaktheit existiert $I \subseteq \N$ endlich mit 
    $$
        \overline{A} \subseteq \bigcup_{i \in I}B(x_i, \varepsilon).
    $$
    Sei also $S$ der von $\{x_i : i \in I\}$ erzeugte endlich dimensionale Untervektorraum von $E$. Dann gilt 
    $$
        A \subseteq \overline{A} \subseteq \bigcup_{i \in I}B(x_i, \varepsilon) \subseteq S^{\varepsilon}.
    $$
    zu $\Leftarrow$: Wir zeigen, dass jede Folge in $A$ eine konvergente Teilfolge besitzt, der Grenzwert der Teilfolge muss hierbei nicht in $A$ liegen. 
    Sei dazu $(x^{(0)}_n)_{n \in \N}$ eine Folge in $A$, $\varepsilon > 0$ und $S \subseteq E$ ein endlichdimensionaler Untervektorraum mit $A \subseteq S^{\varepsilon}$. 
    Dann existieren insbesondere eine Folge $(y_n)_{n \in \N}$ in $S$ mit 
    $$
        \forall n \in \N: \quad d(x^{(0)}_n, y_n) \leq 2\varepsilon.
    $$
    Aus der Beschränktheit von $(x^{(0)}_n)_{n \in \N}$ erhalten wir direkt die Beschränktheit von $(y_n)_{n \in \N}$ und da $S$ endlichdimensional ist existiert eine konvergente Teilfolge $(y_{n_k})_{k \in \N}$.
    Es gilt also für $k,m \geq N(\varepsilon) \in \N$
    \begin{align*}
        \norm{x^{(0)}_{n_k} - x^{(0)}_{n_m}} \leq \norm{x^{(0)}_{n_k} - y_{n_k}} + \norm{y_{n_k} - y_{n_m}} + \norm{x^{(0)}_{n_m} - y_{n_m}} \leq 5\varepsilon. 
    \end{align*}
    Ohne Einschränkung können wir durch entfernen endlich vieler Folgenglieder annehmen, dass 
    $$
        \forall k,m \in \N: \quad \norm{x^{(0)}_{n_k} - x^{(0)}_{n_m}} \leq 5\varepsilon. 
    $$
    Durch obiges Verfahren können wir für $N \in \N$ und $\varepsilon_N = \frac{1}{N}$ induktiv eine Teilfolge $(x^{(N)}_n)_{n \in \N}$ von $(x^{(N-1)}_n)_{n \in \N}$ gewinnen mit
    $$
        \forall m,n \in \N: \quad \norm{x^{(N)}_n - x^{(N)}_m} \leq \frac{5}{N}.
    $$
    Durch bilden der Diagonalfolge $(x^{(N)}_N)_{N \in \N}$ erhalten wir somit eine Teilfolge der Ausgangsfolge $(x^{(0)}_n)_{n \in \N}$ die eine Cauchy-Folge ist und daher in $E$ konvergiert. 
    \qed 
\end{proof*}

\begin{lemma}
    Sei $S \subseteq E$ ein endlichdimensionaler Untervektorraum und $l_1,...,l_n \in E'$ Funktionale mit 
    \begin{align}
        \forall x,y \in S \ \exists k \in \{1,...,n\}: \quad l_k(x) \neq l_k(y).
    \end{align}
    Dann ist die Menge 
    $$
        B := S^{\varepsilon} \cap \{x \in E: \abs{l_1(x)} \leq r_1,...,\abs{l_n(x)}\leq r_n\}
    $$
    für alle $\varepsilon, r_1,...,r_n \in (0, \infty)$ beschränkt. 
\end{lemma}

\begin{proof*}
    Wegen $(1.1)$ definiert 
    $$
        p(x) := \max_{1\leq k \leq n} \abs{l_k(x)}, \quad x \in S,
    $$
    eine Norm auf $S$. Da $S$ endlichdimensional ist, ist diese insbesondere äquivalent zur Einschränkung von $\norm{\cdot}$ auf $S$. 
    Angenommen die Menge $B$ ist nicht beschränkt. Dann existiert eine Folge $(x_n)_{n \in \N}$ in $B$ mit 
    $$
        \lim_{n \to \infty} \norm{x_n} = \infty. 
    $$
    Wegen der Normäquivalenz existiert dann ein $k \in \{1,...,n\}$ mit 
    $$
        \lim_{n \to \infty}\abs{l_k(x_n)} = \infty. 
    $$
    Im Widerspruch zur Definition von $B$. \qed
\end{proof*}

\begin{theorem}
    Sei $\Gamma \subseteq E'$, sodass 
    \begin{align}
        \forall x,y \in E \ \exists l \in \Gamma: \quad l(x) \neq l(y).
    \end{align}
    Eine Menge $M \subseteq \mathcal{M}(E)$ von Wahrscheinlichkeitsmaßen ist genau dann relativ kompakt in $(\mathcal{M}, \rho)$ wenn die folgenden beiden Bedingungen erfüllt sind
    \begin{enumerate}[(a)]
        \item Für alle $l \in \Gamma$ ist $\{\mu^l : \mu \in \Gamma\} \subseteq \mathcal{M}(\R)$ relativ kompakt,
        \item $M$ ist flach konzentriert. 
    \end{enumerate}
\end{theorem}

\begin{proof*}
    zu $\Rightarrow$: 
    Sei $M \subseteq \mathcal{M}(E)$ relativ kompakt. Nach dem Satz von Prokhorov ist $M$ dann insbesondere gleichmäßig straff. 
    Also existiert zu $\varepsilon > 0$ eine kompakte Menge $K \subseteq E$ mit 
    $$
        \forall \mu \in M: \quad \mu(K) \geq 1 - \varepsilon.   
    $$ 
    Aus der Stetigkeit von $l \in \Gamma$ erhalten wir somit direkt die gleichmäßige Straffheit von $\{\mu^l : \mu \in \Gamma\}$. Erneutes anwenden des Satzes von Prokhorov liefert $(a)$. 
    Da $K$ insbesondere relativ kompakt ist liefert \textbf{Lemma zwei davor, TODO} einen endlichdimensionalen Untervektorraum $S \subseteq E$ mit $K \subseteq S^{\varepsilon}$. Somit gilt
    $$
        \forall \mu \in M: \quad \mu(S^{\varepsilon}) \geq \mu(K) \geq 1 - \varepsilon.
    $$
    Folglich ist $M$ flach konzentriert. 
    \newline 
    zu $\Leftarrow$: 
    \textbf{TODO}
\end{proof*}

\begin{remark}%TODO: evtl. weiter nach oben verschieben und noch erklären wieso es für S endlich viele derartige Funktionale gibt. 
    Man sagt eine Familie von Abbildungen mit der Eigenschaft $(1.2)$ (bzw. $(1.1)$ ) \textit{trenne die Punkte von E} (bzw $S$). Die Existenz einer solchen Menge $\Gamma \subseteq E'$ ist durch den Satz von Hahn-Banach sichergestellt. 
    Insbesondere erfüllt $E'$ selbst$(1.1)$. 
\end{remark}
\section{Messbare Vektorräume}
Bislang haben wir uns fast ausschließlich mit dem Zusammenspiel von Maßen und den topologischen Eigenschaften der zugrunde liegenden Räume beschäftigt.
In Banachräumen steht uns aber auch die algebraische Struktur eines Vektorraums zur Verfügung, allerdings ist per se nicht klar, ob die algebraischen Operationen mit der messbaren Struktur kompatibel, also messbar, sind. 
Diese Überlegung führt direkt zur Definition eines \textit{messbaren Vektorraums}. 

\begin{mydef}
    Sei $X$ ein Vektorraum und $\mathcal{C}$ eine $\sigma$-Algebra auf $X$. Das Tupel $(X, \mathcal{C})$ heißt \textit{messbarer Vektorraum}, falls die folgenden beiden Bedingungen erfüllt sind: 
    \begin{enumerate}[(a)]
        \item Die Abbildung 
        \begin{align*}
            + : X \times X \to X, \quad (x,y) \mapsto x + y
        \end{align*}
        ist $\mathcal{C}\otimes \mathcal{C}/\mathcal{C}$-messbar und
        \item die Abbildung 
        \begin{align*}
            \cdot : \R \times X \to X, \quad  (\alpha, x) \mapsto \alpha x
        \end{align*}
        ist $\mathcal{B}(\R) \otimes \mathcal{C}/\mathcal{C}$-messbar. 
    \end{enumerate}
\end{mydef}

\begin{remark}
    Sei $(X, \mathcal{C})$ ein messbarer Vektorraum. Dann gilt:
    \begin{enumerate}[(i)]
        \item Für alle $\alpha \in \R$ ist die Abbildung 
            $$f_{\alpha}: X \to X, \quad x \mapsto \alpha x$$
        $\mathcal{C}/\mathcal{C}$-messbar. 
        \item Für alle $y \in X$ ist die Abbildung 
            $$g_y: X \to X, \quad x \mapsto x + y$$
        $\mathcal{C}/\mathcal{C}$-messbar.
    \end{enumerate}
\end{remark}
\begin{proof*}
    Aus der Messbarkeit der Skalarmultiplikation und Vektoraddition  folgt zunächst
    \begin{align*}
        \forall A \in \mathcal{C}: \quad &A^{(1)} := \{(\alpha, x) \in \R \times X: \ \alpha x \in A\} \in \mathcal{B}(\R) \otimes \mathcal{C}, \\\
                                         &A^{(2)} := \{(x,y) \in X \times X: \ x + y \in A\} \in \mathcal{C} \otimes \mathcal{C}. 
    \end{align*}
    Man beachte nun, dass für beliebige  messbare Räume $(\Omega_1, \mathcal{A}_1), (\Omega_2, \mathcal{A}_2)$, Mengen $A \in \mathcal{A}_1 \otimes \mathcal{A}_2$ und $\omega_1 \in \Omega_1$ 
    \begin{align*}
    A(\omega_1) = \{ \omega_2 \in \Omega_2 : (\omega_1,\omega_2) \in A \} \in \mathcal{A}_2
    \end{align*}
    gilt. In unserem Fall erhalten wir für festes $\alpha \in R$ und $y \in X$
    \begin{align*}
        &f_{\alpha}^{-1}(A) = \{x \in X: \ \alpha x \in A\} = A^{(1)}(\alpha) \in \mathcal{C},  \\\
        &g_y^{-1}(A) = \{ x \in X: \ x + y \in A\} = A^{(2)}(y)  \in \mathcal{C}. 
    \end{align*}
    \qed
\end{proof*}
Da die Komposition messbarer Abbildungen wiederum messbar ist, erhält man unmittelbar
\begin{proposition}
    Sei $(X, \mathcal{C})$ ein messbarer Vektorraum und $(\Omega, \mathcal{A})$ ein messbarer Raum. 
    Sind $X,Y: \Omega \to X$ zwei $\mathcal{A}/\mathcal{C}$-messbare Abbildungen und $\alpha, \beta \in \R$, so ist auch $\alpha X + \beta Y$ $\mathcal{A}/\mathcal{C}$-messbar. 
\end{proposition}

Bislang wissen wir noch nicht einmal, ob es überhaupt nicht-triviale Beispiele messbarer Vektorräume gibt. Das wollen wir nun ändern.  
\begin{proposition}
    Sei $X$ ein separabler Banachraum. Dann ist $(X, \mathcal{B}(X))$ ein messbarer Vektorraum.
\end{proposition}
\begin{proof*}
    Nach Proposition $1.7$ gilt $\mathcal{B}(X \times X) = \mathcal{B}(X) \otimes \mathcal{B}(X)$ und 
    $\mathcal{B}(\R \times X) = \mathcal{B}(\R) \otimes \mathcal{B}(X)$. Ferner sind die Abbildungen 
    \begin{align*}
        + &: X \times X \to X, \quad (x,y) \mapsto x + y, \\\
        \cdot &: \R \times X \to X, \quad (\alpha, x) \mapsto \alpha  x
    \end{align*}
    stetig bzgl. der jeweiligen Produkttopologien und somit insbesondere $\mathcal{B}(X \times X)/\mathcal{B}(X)$- bzw. 
    $\mathcal{B}(\R \times X)/\mathcal{B}(X)$-messbar. \qed
\end{proof*}

\begin{example}
    Für $d \in \N$ ist $(\R^d, \mathcal{B}(\R^d))$ ein messbarer Vektorraum. 
\end{example}

Im Folgenden bezeichne $(X', \norm{\cdot}_{op})$ den Dualraum eines normierten Vektorraums $(X, \norm{\cdot})$.

\begin{proposition}
    Sei $\emptyset \neq \Gamma \subseteq X'$. Dann ist $(X, \sigma({\Gamma}))$ ein messbarer Vektorraum. 
\end{proposition}

\begin{proof*}%TODO: überarbeiten und schöneren Beweis finden. 
    Zur Messbarkeit der Addition: Es genügt zu zeigen, dass 
    $$
        \forall f \in \Gamma : \big(g: X \times X \to \R, \ (x,y) \mapsto f(x+y)\big) \text{ ist messbar.}
    $$
    Sei dazu $f \in \Gamma$. Wegen der Linearität von $f$ gilt für $(x,y) \in X \times X$
    $$
        f(x + y) = (f(x) + f(y)).
    $$
    Weiter ist die Abbildung 
    $$
        h: X \times X \to \R, (x,y) \mapsto f(x) + f(y)
    $$
    als Komposition der messbaren Funktionen 
    \begin{align*}
        &h_1: X \times X \to \R \times \R, (x,y) \mapsto (f(x),f(y)), \\\
        &h_2: \R \times \R \to \R, (x,y) \mapsto x+y
    \end{align*}
    messbar. Die Abbildung $h_2$ ist messbar, weil $(\R, \mathcal{B}(\R))$ nach Beispiel $1.39$ ein messbarer Vektorraum ist. 
    Die Messbarkeit der Skalarmultiplikation zeigt man ähnlich. \qed 
\end{proof*}

\begin{proposition}
    Sei $E$ ein separabler Banachraum. Dann gilt $\sigma(E') = \mathcal{B}(E)$. 
\end{proposition}

\begin{proof*}%TODO: Überprüfen. Reicht das wirklich schon so?
    Da alle $f \in E'$ stetig sind, gilt offensichtlich $\sigma(E') \subseteq \mathcal{B}(E)$. 
    Wegen der Separabilität von $E$ wird $\mathcal{B}(E)$ nach Proposition $1.1$ von den abgeschlossenen Kugeln erzeugt und weil $(E, \sigma(E'))$ nach Proposition $1.40$ ein messbarer Vektorraum ist, genügt es nach Bemerkung $1.36$ zu zeigen, 
    dass $\overline{B}(0,1)$ in $\sigma(E')$ enthalten ist. Da $E$ separabel ist, existiert nach Korollar $A.12$ eine Folge $(f_n)_{n \in \N}$ in $E'$ mit
    $$
        \forall x \in E: \quad \norm{x} = \sup_{n \in \N}\abs{f_n(x)}.
    $$
    Es gilt also
    $$
        \overline{B}(0,1) = \{ x \in E: \norm{x} \leq 1 \} = \{x \in E: \ \sup_{n \in \N}\abs{f_n(x)} \leq 1 \} = \bigcap_{n=1}^{\infty}\{x \in E: \abs{f_n(x)} \leq 1\} \in \sigma(E').
    $$
    \qed 
\end{proof*}

\section{Zufallsvariablen mit Werten in Banachräumen}
Sei $(\Omega, \mathcal{A}, P)$ ein vollständiger Wahrscheinlichkeitsraum und $E$ ein Banachraum mit Norm $\norm{\cdot}$. 
\begin{mydef}
    Eine Abbildung $X: \Omega \to E$ heißt \textit{(Radon-)Zufallsvariable} falls 
    \begin{enumerate}[(a)]
        \item $X$ ist $\mathcal{A}/\mathcal{B}(E)$-messbar und
        \item es existiert ein separabler Untervektorraum $E_0 \subseteq E$ mit $P(\{X \in E_0\}) = 1$. 
    \end{enumerate}
\end{mydef}
\textbf{TODO: Erklärung warum Radon, Rückgriff auf Abschnitt 1.2}
\begin{proposition}
    \textbf{TODO: Charakterisierung von Radon-Zufallsvariablen}
\end{proposition}
\textbf{TODO: Einführung einfache Funktionen im Banach-Kontext}
\begin{proposition}
    \textbf{TODO: Summen von Radon-Variablen sind messbar, Approximation durch einfache Funktionen, Grenzwerte sind messbar}
\end{proposition}

Bezeichne $\mathcal{L}_0(\Omega, \mathcal{A}, P; E)$ den Raum der $E$-wertigen Radon-Zufallsvariablen auf $(\Omega,\mathcal{A},P)$ und sei $L_0(E)$ der Raum der Äquivalenzklassen bezüglich fast sicherer Gleichheit. 
Falls klar ist welcher Wahrscheinlichkeitsraum gemeint ist, so schreiben wir auch $\mathcal{L}_0(E)$ oder $L_0(E)$. 







\chapter{Konvergenzarten}

\section{Fast-sichere Konvergenz}
\section{Stochastische Konvergenz}
\section{Verteilungskonvergenz}
\chapter{Maximal-Ungleichungen und Konvergenz zufälliger Reihen}
\textbf{Notation und Konventionen}\newline 
Im Folgenden sei $E$ ein separabler Banachraum und $(\Omega, \mathcal{A}, P)$ ein vollständiger Wahrscheinlichkeitsraum. Wegen der Separabilität von $E$ ist jede messbare Abbildung $X: (\Omega, \mathcal{A}) \to (E, \mathcal{B}(E))$ eine Radon-Zufallsvariable. 

\section{Maximalungleichungen}

Bezeichne $L_0(E)$ den Vektorraum der E-wertigen-Zufallsvariablen. 

\begin{mydef}
    Eine E-wertige Zufallsvariable $X$ heißt \textit{symmetrisch}, falls $-X$ die selbe Verteilung besitzt wie $X$, d.h.
    \begin{align*}
        \forall A \in \mathcal{B}(E): P(\{X \in A\}) = P(\{-X \in A\}). 
    \end{align*}
\end{mydef}

\begin{remark}
    Nach dem Eindeutigkeitssatz für charakteristische Funktionale ist eine Zufallsvariable $X \in L_0(E)$ genau dann symmetrisch, wenn 
    $$
        \forall z \in E': \quad \widehat{\mu_X}(z) = \widehat{\mu_{-X}}(z). 
    $$
\end{remark}

\begin{mydef}%TODO: Wird das überhaupt gebraucht? Ggf schöner formulieren. 
    Eine Folge $(X_n)_{n \in \N}$ von E-wertigen Zufallsvariablen heißt \textit{symmetrisch}, 
    falls $(\varepsilon_1 X_1, \varepsilon_2 X_2,...)$ für jede Wahl von $\varepsilon_i = \pm 1$ 
    die gleiche Verteilung hat wie $(X_1,X_2,...)$. 
\end{mydef}

\begin{remark}
   Sind $X_1,X_2,...$ unabhängige E-wertige Zufallsvariablen, sodass $X_n$ für alle $n \in \N$ symmetrisch ist, dann ist $(X_1,X_2,...)$ symmetrisch. 
\end{remark}

\begin{theorem}[Lévys Maximal-Ungleichung]
    Seien $X_1,...,X_N \in L_0(E)$ unabhängige und symmetrische Zufallsvariablen und setze 
    \begin{align*}
        S_n := \sum_{i=1}^n X_i, \quad 1 \leq n \leq N. 
    \end{align*}
    Dann gilt für alle $t > 0$
    \begin{align}
        &P\big(\{ \max_{1 \leq n \leq N} \norm{S_n} > t \}\big) \leq 2 P\big(\{\norm{S_N} > t \}\big), \\\
        &P\big(\{ \max_{1 \leq n \leq N} \norm{X_n} > t \}\big) \leq 2 P\big(\{\norm{S_N} > t \}\big).
    \end{align}
\end{theorem}

\begin{proof*}
    \textbf{TODO}
\end{proof*}

Für nicht-symmetrische Zufallsvariablen erhalten wir mit einer ähnlichen Beweismethode die folgende auf Giuseppe Ottaviani und Anatoli Skorohod zurückgehende Maximal-Ungleichung, vgl. \cite{ledoux-talagrand}[Lemma 6.2]. 
\begin{theorem}[Maximal-Ungleichung von Ottaviani-Skorohod]
    Seien $X_1,...,X_N$ unabhängige $E$-wertige Zufallsvariablen, $N \in \N$. Setze 
    $$
        S_k := \sum_{i=1}^kX_i, \quad k = 1,...,N. 
    $$
    Dann gilt für alle $s,t > 0$
    \begin{align}
        P(\{ \max_{1 \leq k \leq N} \norm{S_k} > s + t \}) \leq \frac{P(\{\norm{S_N} > t \})}{1 - \max_{1 \leq k \leq N}P(\{ \norm{S_N - S_k} > s \})} \ . 
    \end{align}
\end{theorem}

\begin{proof*}
    \textbf{TODO}
\end{proof*}
\newpage
\section{Der Satz von Itô-Nisio}
    Wir wollen nun damit beginnen die bisher erarbeitete Technik zur Untersuchung der Konvergenz zufälliger Reihen in Banachräumen anzuwenden.
    Die vorliegenden Beweise orientieren sich an \cite{ito-nisio}, \cite{ledoux-talagrand}, \cite{li-queffelec} und \cite{van-neerven}.  
    \newline \ \newline 
    Sei $(X_n)_{n \in \N}$ eine Folge unabhängiger Zufallsvariablen in $\mathcal{L}_0(E)$. Für $n \in \N$ setze
    \begin{align*}
    S_n := \sum_{i=1}^n X_i, 
    \quad 
    \mu_n := P^{S_n} .
    \end{align*}
\begin{theorem}[Itô-Nisio]
    Es sind äquivalent
    \begin{enumerate}[(i)]
        \item $(S_n)_{n \in \N}$ konvergiert fast sicher, 
        \item $(S_n)_{n \in \N}$ konvergiert stochastisch, 
        \item $(S_n)_{n \in \N}$ konvergiert in Verteilung. 
    \end{enumerate}
\end{theorem}

\begin{proof*}%TODO: Formatierung% 
    Die Implikationen $(i) \Rightarrow (ii) \Rightarrow (iii)$ wurden bereits in Kapitel 2 gezeigt, es genügt also $(ii) \Rightarrow (i)$ und $(iii) \Rightarrow (ii)$ zu zeigen. 
    \newline 
    Zu $(ii) \Rightarrow (i)$: \underline{Fall A}: Für alle $n \in \N$ ist $X_n$ symmetrisch verteilt. 
    \newline 
    Für $n, N \in \N$ mit $n < N$ setze 
    \begin{align*}
        Y_{n,N} &:= \max_{n < k \leq N}\norm{S_k - S_n}, \\\
        Y_n     &:= \lim_{N \to \infty} Y_{n,N} = \sup_{k > n} \norm{S_k - S_n}. 
    \end{align*}
    Seien $\varepsilon, t > 0$. Mit dem Cauchy-Kriterium für stochastische Konvergenz und Lévys Maximal-Ungleichung erhalten wir für $N > n \geq n_0 := n_0(\varepsilon,t) \in \N$
    $$
        P(\{ Y_{n, N} > t\}) \leq 2P(\{ \norm{S_N - S_n} > t\}) \leq \varepsilon . 
    $$
    Es folgt somit 
    $$
        P(\{Y_n > t \}) = \lim_{n \to \infty}P(\{Y_{n,N} > t \}) \leq \varepsilon
    $$
    für $n \geq n_0$. Also gilt $Y_n \stochastisch 0$, nach dem Cauchy-Kriterium für fast sichere Konvergenz folgt daraus die fast sichere Konvergenz von $(S_n)_{n \in \N}$. 
    \newline \underline{Fall B}: Allgemeiner Fall.
    \newline 
    Wir verwenden für diesen Fall die Beweistechnik der Symmetrisierung.  
    Betrachte hierzu den Produktraum $(\Omega \times \Omega, \mathcal{A}\otimes\mathcal{A}, P \times P)$. Für eine Zufallsvariable $X$ auf $(\Omega, \mathcal{A}, P)$ bezeichne
    $$
        \overline{X}(\omega_1, \omega_2) := X(\omega_1) - X(\omega_2), \quad (\omega_1, \omega_2) \in \Omega \times \Omega
    $$   
    die Symmetrisierung von $X$. Wie man mittels der charakteristischen Funktionale von $\overline{X}$ und $-\overline{X}$ leicht einsieht ist $\overline{X}$ tatsächlich symmetrisch. 
    Sei nun $S$ eine Zufallsvariable auf $(\Omega, \mathcal{A}, P)$ mit $S_n \stochastisch S$. 
    Dann folgt direkt $\overline{S_n} \stochastisch \overline{S}$, denn für $\varepsilon > 0$ gilt nach Konstruktion
    $$
        (P\times P)(\{ \norm{\overline{S_n} - \overline{S}} > \varepsilon \} ) \leq 2P(\{ \norm{S_n - S} > \frac{\varepsilon}{2} \}).
    $$
    Nach Fall A gilt also insbesondere $\overline{S_n} \fastsicher  \overline{S}$. Daher existiert eine Menge $\Omega^* \in \mathcal{A}\otimes\mathcal{A}$ mit \mbox{$(P\times P)(\Omega^*) = 1$} und
    $$  
        \forall (\omega_1, \omega_2) \in \Omega^*: \quad \overline{S_n}(\omega_1, \omega_2) \text{ konvergiert. }
    $$
    Mit dem Satz von Fubini erhalten wir
    $$
        0 = \int_{\Omega \times \Omega}1 - 1_{\Omega^*} d(P \times P) = \int_{\Omega}\int_{\Omega}1 - 1_{\Omega^*(\omega_2)}(\omega_1)dP(\omega_1)dP(\omega_2). 
    $$
    wobei $\Omega^*(\omega_2) =\{\omega_1\in\Omega: \ (\omega_1, \omega_2) \in \Omega^* \}$. Somit existiert ein $\omega_2 \in \Omega$ mit $P(\Omega^*(\omega_2)) = 1$ und es gilt
    $$
        \forall \omega_1 \in \Omega^*(\omega_2): \quad S_n(\omega_1) - S_n(\omega_2) \ \text{ konvergiert.}
    $$
    Setze nun $x_n := S_n(\omega_2)$, $n \in \N$. Dann existiert eine Zufallsvariable $L$ auf $(\Omega, \mathcal{A}, P)$ mit $S_n - x_n \fastsicher L$. Nach Voraussetzung erhalten wir also 
    $$
        x_n \stochastisch S - L,
    $$
    wobei wir $x_n$ für $n \in \N$ als konstante Zufallsvariable auf $(\Omega, \mathcal{A}, P)$ auffassen. Folglich existiert ein $x \in E$ mit 
    $$
        \lim_{n \to \infty}x_n = x
    $$
    und insgesamt erhalten wir $S_n \fastsicher L + x$. 
    \newline 
    Zu $(iii) \Rightarrow (ii)$: Für $1 \leq m < n$ setze
    \begin{align*}
        \mu_{m,n} :&= P^{S_n - S_m}
    \end{align*}
    Da $(\mu_n)_{n \in \N}$ nach Voraussetzung schwach gegen ein Wahrscheinlichkeitsmaß $\mu$ konvergiert
    ist die Menge $\{\mu_n: n \in \N \}$ relativ kompakt in $(\mathcal{M}(E), \rho)$ und somit nach dem Satz von Prokhorov gleichmäßig straff.
    Folglich existiert zu $\varepsilon > 0$ eine kompakte Teilmenge $K \subseteq E$ mit 
    $$
        \forall n \in \N: \quad \mu_n(K) \geq 1 - \varepsilon. 
    $$
    Wegen der Stetigkeit von 
    $$
        (x,y) \mapsto x - y, \quad (x,y) \in E \times E
    $$
    ist die Menge $\tilde{K} := \{x - y : x,y \in K \}$ wiederum kompakt, also insbesondere messbar, und es gilt
    \begin{align*}
        \mu_{m,n}(\tilde{K}) \geq P(\{S_n \in K, \ S_m \in K\}) \geq 1 - P(\{S_n \in K^c\}) - P(\{S_m \in K^c\}) \geq 1 - 2\varepsilon.
    \end{align*}
    Somit ist auch $\{\mu_{m,n}: m,n \in \N, m < n \}$ gleichmäßig straff und daher relativ kompakt in $(\mathcal{M}(E), \rho)$. 
    Wir zeigen nun
    \begin{align}
        \forall \varepsilon > 0 \ \exists N \in \N: \ \forall n > m \geq N: \quad \prob{\norm{S_n - S_m} < \varepsilon} = \mu_{m,n}(B(0,\varepsilon)) > 1 - \varepsilon.
    \end{align}
    Mit dem Satz $2.9$ folgt daraus die stochastische Konvergenz der Folge $(S_n)_{n \in \N}$. Angenommen $(3.6)$ ist nicht erfüllt, dann gilt
    \begin{align*}
        \exists \varepsilon > 0 \ \forall N \in \N: \ \exists n(N) > m(N) \geq N: \mu_{m(N), n(N)}(B(0, \varepsilon)) \leq 1 - \varepsilon.
    \end{align*}
    Da $\{\mu_{m,n}: m,n \in \N, m < n \}$ relativ kompakt ist existiert insbesondere eine Teilfolge von $(\mu_{m(N),n(N)})_{N \in \N}$ die schwach gegen ein Wahrscheinlichkeitsmaß $\nu$ auf $\mathcal{B}(E)$ konvergiert.
    Ohne Einschränkung können wir annehmen, dass bereits $\mu_{m(N),n(N)} \rightharpoonup \nu$ gilt. Da $B(0, \varepsilon)$ offen ist liefert das Portmanteau-Theorem
    \begin{align}
        \nu(B(0, \varepsilon)) \leq \liminf_{N \to \infty}\mu_{m(N),n(N)}(B(0,\varepsilon)) \leq 1 - \varepsilon. 
    \end{align}
    Andererseits gilt für $f \in E'$ wegen der Unabhängigkeit von $(X_n)_{n \in \N}$
    \begin{align*}
        \widehat{\mu_{n(N)}}(f) = \mathbb{E}(e^{if(S_{n(N)})})) &= \mathbb{E}(e^{if(S_{m(N)})}e^{if(S_{n(N)} - S_{m(N)})}) \\\
                                                   &= \mathbb{E}(e^{if(S_{m(N)})})\mathbb{E}(e^{if(S_{n(N)} - S_{m(N)})}) \\\
                                                   &= \widehat{\mu_{m(N}}(f)\widehat{\mu_{m(N),n(N)}}(f). 
    \end{align*}
    Mit Grenzübergang $N \to \infty$ folgt daraus wegen $\mu_N \rightharpoonup \mu$ und $\mu_{m(N),n(N)} \rightharpoonup \nu$
    \begin{align*}
        \forall z \in E': \quad \widehat{\mu}(z) = \widehat{\mu}(z) \widehat{\nu}(z).
    \end{align*}
    Wegen der Stetigkeit von $\widehat{\mu}$ und $\widehat{\mu}(0_{E'}) = 1$ existiert ein $r>0$ mit 
    \begin{align*}
        \widehat{\mu}(z) \neq 0, \quad \text{ für alle } z \in E' \text{ mit } \norm{z}_{op} \leq r. 
    \end{align*}
    Also gilt 
    \begin{align*}
        \widehat{\nu}(z) = 1, \quad \text{ für alle } z \in E' \text{ mit } \norm{z}_{op} \leq r. 
    \end{align*}
    Aus Proposition $2.19$ folgt nun $\nu = \delta_0$. Im Widerspruch zu $(3.7)$. Es gilt folglich $(3.6)$ und $(S_n)_{n \in \N}$ konvergiert demnach stochastisch. 
    \qed
\end{proof*}

\begin{remark}
    Mittels der Maximal-Ungleichung von Ottaviani-Skorohod ist auch ein direkter Beweis der Implikation $(ii) \Rightarrow (i)$ möglich. Betrachte hierzu etwa die Ereignisse 
    $$
        A_N := \bigcap_{m \in \N}\{\sup_{n > m} \norm{S_n - S_m} > \frac{1}{N}\}, \quad N \in \N.                                                                                                                        
    $$
    Dann ist $A := (\cup_{N=1}^{\infty} A_N)^c$ das Ereignis, dass $(S_n)_{n \in \N}$ eine Cauchy-Folge ist und man zeigt 
    $$
        \forall N \in \N: \quad P(A_N) = 0. 
    $$
    Ein Beweis mit dieser Vorgehensweise für den skalaren Fall findet sich etwa \cite{bauer}[Theorem 14.2, S.109]. Der allgemeine Fall funktioniert vollkommen analog. \qexampled
\end{remark}




\begin{theorem}
    Sei $(\mu_n)_{n \in \N}$ gleichmäßig straff. Dann existiert eine Folge $(c_n)_{n \in \N}$ in $E$ sodass $(S_n - c_n)_{n \in \N}$ fast sicher konvergiert.
\end{theorem}

\begin{proof*}
    Betrachte den Produktraum $(\Omega \times \Omega, \mathcal{A} \otimes \mathcal{A}, P \times P)$ und definiere 
    \begin{align*}
        \widetilde{X}_n&: \Omega \times \Omega \to E, \ (\omega_1, \omega_2) \mapsto X_n(\omega_1), \\\
        \widetilde{Y}_n&: \Omega \times \Omega \to E, \ (\omega_1, \omega_2) \mapsto X_n(\omega_2),
    \end{align*}
    sowie 
    \begin{align*}
        \widetilde{S}_n := \sum_{i = 1}^n \widetilde{X}_i, \quad \widetilde{T}_n := \sum_{i = 1}^n \widetilde{Y}_n, \quad U_n := \widetilde{S}_n - \widetilde{T}_n, \quad \mu_{U_n} := (P\times P)^{U_n}. 
    \end{align*}
    Nach Konstruktion sind $\widetilde{S}_n$,$\widetilde{T}_n$ und $S_n$ identisch verteilt. Wir zeigen zunächst, dass $(\mu_{U_n})_{n \in \N}$ gleichmäßig straff ist. 
    Sei dazu $\varepsilon > 0$ Dann existiert nach Voraussetzung eine kompakte Menge $K_{\varepsilon} \subseteq E$ mit 
    $$
        \forall n \in \N: \mu_n(K_{\varepsilon}) \geq 1 - \varepsilon. 
    $$
    Wegen der Stetigkeit der Abbildung 
    $$
        E \times E \to E, \ (x,y) \mapsto x - y
    $$
    ist auch die Menge $K := \{ x - y : x,y \in K_{\varepsilon} \}$ kompakt und somit insbesondere messbar. Ferner gilt
    \begin{align*}
        \mu_{U_n}(K) = \prob{\widetilde{S}_n - \widetilde{T}_n \in K} &\geq \prob{\widetilde{S}_n \in K_{\varepsilon}, \widetilde{T}_n \in \varepsilon} \\\
                                                              &\geq 1 - \prob{\widetilde{S}_n \notin K_{\varepsilon}} - \prob{\widetilde{T}_n \notin K_{\varepsilon}} \\\
                                                              &= 1 - 2\mu_n(K_{\varepsilon}^c) \\\
                                                              &\geq 1 - 2 \varepsilon.                                              
    \end{align*}
    Also ist $(\mu_{U_n})_{n \in \N}$ straff. 
    Als nächstes zeigen wir, dass $(\widehat{\mu_{U_n}}(f))_{n \in \N}$ für alle $f \in E'$ konvergiert. Sei dazu $f \in E'$ beliebig aber fest. 
    Die Unabhängigkeit von $(\widetilde{Y}_n)_{n \in \N}$ und $(\widetilde{X}_n)_{n \in \N}$ liefert direkt die Unabhängigkeit von $(\widetilde{X}_n - \widetilde{Y}_n)_{n \in \N}$
    und nach Konstruktion sind $Y_n$ und $X_n$ identisch verteilt. Also folgt 
    \begin{align*}
        \widehat{\mu_{U_n}}(f) = E(e^{if(U_n)}) = \mathbb{E}\big(\prod_{j=1}^n(e^{if(\widetilde{X}_j-\widetilde{Y}_j)})\big) &= \prod_{j=1}^n \mathbb{E}(e^{if(\widetilde{X}_j-\widetilde{Y}_j)}) \\\ 
                                                                                                            &= \prod_{j=1}^n \mathbb{E}(e^{if(\widetilde{X}_j)})\mathbb{E}(e^{-if(\widetilde{Y}_j)}) \\\
                                                                                                            &= \prod_{j=1}^n \abs{\mathbb{E}(e^{if(\widetilde{X}_j)})}^2
    \end{align*}
    Wegen $0 \leq \abs{E(e^{if(\widetilde{X}_j)})} \leq 1$ für alle $j \in \N$ folgt somit die Konvergenz von $(\widehat{\mu_{U_n}}(f))_{n \in \N}$. 
    Nach Kapitel 2 konvergiert $(U_n)_{n \in \N}$ also in Verteilung und somit nach dem Satz von Itô-Nisio insbesondere fast sicher. 
    Daher existiert eine Menge $\Omega^* \in \mathcal{A} \otimes \mathcal{A}$ mit $(P\times P)(\Omega^*) = 1$ und 
    $$
        \forall (\omega_1, \omega_2) \in \Omega^*: \quad U_n(\omega_1, \omega_2) = S_n(\omega_1) - S_n(\omega_2) \text{ konvergiert.}
    $$
    Mit dem Satz von Fubini erhalten wir wie im Beweis des Satzes von Itô-Nisio ein $\omega' \in \Omega$, sodass $S_n - S_n(\omega')$ fast sicher konvergiert. 
    Also erfüllt die Folge $(c_n)_{n \in \N}$ definiert durch $c_n := S_n(\omega')$, $n \in \N$, die gewünschte Eigenschaft. \qed

\end{proof*}

\begin{theorem}[Satz von Itô-Nisio für symmetrische Folgen]
    Sei $(X_n)_{n \in \N}$ eine Folge unabhängiger und symmetrischer Zufallsvariablen in $\mathcal{L}_0(E)$. Dann sind äquivalent
    \begin{enumerate}[(i)]
        \item $(S_n)_{n \in \N}$ konvergiert fast sicher, 
        \item $(S_n)_{n \in \N}$ konvergiert stochastisch, 
        \item $(S_n)_{n \in \N}$ konvergiert in Verteilung, 
        \item $(\mu_n)_{n \in \N}$ ist gleichmäßig straff, 
        \item Es gibt eine Zufallsvariable $S \in \mathcal{L}_0(E)$, sodass 
        $$
            \forall f \in E': \quad f(S_n) \stochastisch f(S),
        $$
        \item Es gibt ein Wahrscheinlichkeitsmaß $\mu$ auf $\mathcal{B}(E)$, sodass 
        $$
            \forall f \in E': \quad \lim_{n \to \infty}\widehat{\mu_n}(f) = \widehat{\mu}(f). 
        $$
    \end{enumerate}
\end{theorem}

\begin{proof*}
    Die Äquivalenz $(i) \iff (ii) \iff (iii)$ wurde bereits im allgemeinen Fall nicht-symmetrischer Zufallsvariablen gezeigt und die Implikationen $(iii) \Rightarrow (iv)$, $(i) \Rightarrow (v) \Rightarrow (vi)$ sind klar. 
    Wir zeigen noch $(vi) \Rightarrow (iv)$ und $(v) \Rightarrow (iv) \Rightarrow (i)$. 
    \newline
    zu $(iv) \Rightarrow (i)$:
    Nach Satz $3.8$ existiert eine Folge $(c_n)_{n \in \N}$ in $E$, sodass $(S_n - c_n)_{n \in \N}$ fast sicher konvergiert. Setze nun $P^X := P^{(X_1,X_2,...)}$ und $P^{-X} :=P^{(-X_1,-X_2,...)}$. 
    Wegen der Unabhängigkeit und Symmetrie von $(X_n)_{n \in \N}$ erhalten wir direkt $P^X = P^{-X}$. Weiter gilt für $N \in \N$ und $\varepsilon >0$
    \begin{align*}
        &\quad \ \prob{\sup_{n \geq N}\norm{(S_n - c_n) - (S_N- c_N)} > \varepsilon} \\\
                &= P^X(\{(y_n)_{n \in \N}\in E^{\N}: \  \sup_{n \geq N}\norm{y_n + y_{n-1} + ... + y_{N+1} + (c_n - c_N)} > \varepsilon \}) \\\
                &= P^{-X}(\{(y_n)_{n \in \N}\in E^{\N}: \ \sup_{n \geq N}\norm{y_n + y_{n-1} + ... + y_{N+1} + (c_n - c_N)} > \varepsilon \}) \\\
                &= \prob{\sup_{n \geq N}\norm{(-S_n - c_n) - (-S_N - c_N)} > \varepsilon}.
    \end{align*}
    Also konvergiert wegen dem Cauchy-Kriterium für fast sichere Konvergenz auch $(-S_n - c_n)_{n \in \N}$ fast sicher. Daraus folgt die fast sichere Konvergenz von $(S_n)_{n \in \N}$, denn für $n \in \N$ gilt
    $$
       S_n =  \frac{1}{2}((S_n - c_n) - (-S_n -c_n)). 
    $$
    zu $(v) \Rightarrow (iv)$: Wegen der Unabhängigkeit von $(X_n)_{n \in \N}$ sind für alle $f \in E'$ und $m \geq n$ die Zufallsvariablen $f(S_m - S_n)$ und $f(S_n)$ unabhängig. 
    Nach Proposition $2.18$ sind also $f(S-S_n)$ und $f(S_n)$ für alle $n \in \N$ unabhängig. Zusammen mit der Symmetrie von $S_n$ ergibt sich also für $f \in E'$
    \begin{align*}
        \widehat{P^S}(f) = \mathbb{E}(e^{if(S)}) &= \mathbb{E}(e^{if(S-S_n)})\mathbb{E}(e^{if(S_n)})  \\\
                                        &= \mathbb{E}(e^{if(S-S_n)})\mathbb{E}(e^{if(-S_n)}) = \mathbb{E}(e^{if(S-2S_n)}) = \widehat{P^{S-2S_n}}(f).
    \end{align*}
    Also sind $S$ und $S - 2S_n$ nach dem Eindeutigkeitssatz für alle $n \in \N$  identisch verteilt. Da $P^S$ straff ist existiert zu $\varepsilon >0$ eine kompakte Menge $K \subseteq E'$ mit $\prob{S \in K} \geq 1 - \varepsilon$. 
    Aus Stetigkeitsgründen ist auch die Menge $L := \{\frac{1}{2}(x-y): x,y \in K\}$ kompakt und es gilt für alle $n \in \N$
    $$
        \prob{S_n \in L} \geq \prob{S \in K, \ S-2S_n \in K} \geq 1 - \prob{S \notin K} - \prob{S-2S_n \notin K} \geq 1 - 2\varepsilon. 
    $$ 
    Also ist $(\mu_n)_{n \in \N}$ gleichmäßig straff. 
    \newline
    zu $(vi) \Rightarrow (iv)$: 
    Sei $f \in E'$ beliebig aber fest. Betrachte die Abbildungen 
    \begin{align*}
        \varphi_n &: \R \to \C, \quad t \mapsto \int_E e^{itf(x)}\mu_n(dx) = \widehat{\mu_n}(tf), \quad n \in \N, \\\
        \varphi   &: \R \to \C, \quad t \mapsto \int_E e^{itf(x)}\mu(dx) = \widehat{\mu}(tf). 
    \end{align*}
    Dann ist $\varphi_n$ für alle $n \in \N$ die charakteristische Funktion von $\mu_n^{f}$ und $\varphi$ die charakteristische Funktion von $\mu^f$. Nach Voraussetzung konvergiert $(\varphi_n)_{n \in \N}$ punktweise gegen $\varphi$ und nach 
    dem Stetigkeitssatz von Lévy, vgl. \cite{gs}[Satz 8.7.5, S.357], gilt also $\mu_n^f \rightharpoonup \mu^f$. Für festes $f \in E'$ ist $\{\mu_n^f : n \in \N\}$ also insbesondere relativ kompakt. Nach dem Satz von de Acosta genügt es folglich zu zeigen, dass $(\mu_n)_{n \in \N}$ flach konzentriert ist. 
    Da $\{\mu\}$ flach konzentriert ist genügt es dafür zu zeigen, dass für jeden endlich dimensionalen Untervektorraum $F \subseteq E$ und alle $\varepsilon > 0$ gilt 
    $$
        \mu_n((F^{\varepsilon})^c) \leq 2 \mu((F^{\varepsilon})^c).
    $$
    Sei also $F \subseteq E$ ein endlich dimensionaler Untervektorraum und $\varepsilon >0$. Nach dem Satz von Hahn-Banach existiert eine Folge $(f_n)_{n \in \N}$ in $E'$ mit 
    $$
        \forall x \in E: \quad d(x,F) := \inf_{y \in F}\norm{x-y} = \sup_{n \in \N}\abs{f_n(x)}. 
    $$
    Sei zunächst $m \in \N$ festgewählt. Mittels charakteristischer Funktionen prüft man leicht, dass $\mu_n^{(f_1,...,f_m)} \rightharpoonup \mu^{(f_1,...,f_m)}$. 
    Wegen der Linearität von $f_1,...,f_m$ können wir den Satz von Itô-Nisio auf die Folge $(T_n)_{n \in \N} := ((f_1,...,f_m)\circ S_n)_{n \in \N}$ anwenden und erhalten eine $\R^m$-wertige Zufallsvariable $T$ auf $(\Omega, \mathcal{A}, P)$ mit 
    $T_n \stochastisch T$. Insbesondere gilt also $T_n \schwach T$ und somit $P^T = \mu^{(f_1,...,f_,)}$. 
    Ferner erhält die Linearität von $f_1,...,f_m$ die Symmetrie und mit der Lévy Ungleichung für Grenzwerte in Verteilung angewendet auf $(T_n)_{n \in \N}$ erhalten wir im Banachraum $(\R^m, \norm{\cdot}_{\infty})$
    \begin{align*}
        \prob{\max_{1 \leq i \leq m}\abs{f_i(S_n)} > \varepsilon} &= \prob{\norm{T_n}_{\infty} > \varepsilon} \\\
                                                                  &\leq 2 \prob{\norm{T}_{\infty} > \varepsilon} = 2\mu\big(\{x \in E: \max_{1\leq i \leq m}\abs{f_i(x)} > \varepsilon\}\big)
    \end{align*}
    Es gilt folglich wegen der $\sigma$-Stetigkeit von Wahrscheinlichkeitsmaßen 
    \begin{align*}
        \mu_n((F^{\varepsilon})^c) = \prob{d(S_n, F) > \varepsilon} &= \lim_{m \to \infty}\prob{\max_{1 \leq i \leq m}\abs{f_i(S_n)} > \varepsilon} \\\
                                                                    &\leq \lim_{m \to \infty} 2 \mu(\{x \in E: \max_{1\leq i \leq m}\abs{f_i(x)} > \varepsilon\})
                                                                    = 2 \mu((F^{\varepsilon})^c).
    \end{align*}
    \qed
\end{proof*}

\begin{remark}
    Auf die Annahme der Symmetrie in Satz $3.10$ kann im Allgemeinen nicht verzichtet werden. Sei $E$ ein Hilbertraum mit Orthonormalbasis $(e_n)_{n \in \N}$.
    Nach dem Darstellungsatz von Riesz können wir den Dualraum $E'$ mit $E$ identifizieren. Setze nun 
    $$
        X_1(\omega) = e_1, \quad X_n(\omega) = e_n - e_{n-1}, \ n \in \N, \quad \omega \in \Omega. 
    $$
    Dann gilt offensichtlich $S_n = e_n$ und da $(e_n)_{n \in \N}$ eine Orthonormalbasis von $E$ ist gilt für alle $z \in E$
    $$
        \lim_{n \to \infty}\langle z,S_n \rangle = 0 = \langle z,S \rangle,
    $$
    wobei $S(\omega) = 0$ für alle $\omega \in \Omega$. Wegen $\norm{S_n} = \norm{e_n} =1$ für alle $n \in \N$ konvergiert $(S_n)_{n \in \N}$ aber weder fast-sicher noch stochastisch gegen $S$.
    \qexampled 
\end{remark}

Für Folgen $(a_n)_{n \in \N}$ in $[0, \infty)$ kennt man aus der reellen Analysis die Äquivalenz
$$
    \bigg(\sum_{i=1}^n a_i\bigg)_{n \in \N} \text{ konvergiert. } \iff \bigg(\sum_{i=1}^n a_i\bigg)_{n \in \N} \text{ ist beschränkt.}
$$
Mit Hilfe des Satzes von Itô-Nisio können wir nun ein ähnliches Resultat für symmetrische Zufallsvariablen formulieren. 

\begin{mydef}
    Eine Folge $(X_n)_{n \in \N}$ in $\mathcal{L}_0(E)$ heißt \textit{stochastisch beschränkt}, falls
    $$
        \forall \varepsilon >0 \ \exists R > 0: \quad \sup_{n \in \N}\prob{\norm{X_n} > R} < \varepsilon. 
    $$
\end{mydef}

\begin{corollary}
    Sei $d \in \N$ und $(X_n)_{n \in \N}$ eine unabhängige Folge $\R^d$-wertiger symmetrischer Zufallsvariablen. Dann sind äquivalent 
    \begin{enumerate}[(i)]
        \item $(S_n)_{n \in \N}$ konvergiert fast sicher.
        \item $(S_n)_{n \in \N}$ ist stochastisch beschränkt. 
    \end{enumerate} 
\end{corollary}
\begin{proof*}
    $(i) \Rightarrow (ii)$ ist klar. Aus $(ii)$ erhalten wir unmittelbar
    $$
        \forall \varepsilon > 0 \ \exists R > 0: \quad \inf_{n \in \N}\prob{S_n \in \overline{B}(0,R)} \geq 1 - \varepsilon.
    $$
    Da abgeschlossene und beschränkte Teilmengen des $\R^d$ nach dem Satz von Heine-Borel kompakt sind folgt daraus die gleichmäßige Straffheit von $(\mu_n)_{n \in \N}$. 
    Nach dem Satz von Itô-Nisio konvergiert $(S_n)_{n \in \N}$ also fast sicher. \qed 
\end{proof*}
Wir wollen nun noch eine zweite Anwendung von Satz $3.9$ geben, die Darstellung orientiert sich hierbei an \cite{li-queffelec}. 
Im Folgenden sei $(X_n)_{n \in \N}$ eine unabhängige Folge symmetrischer Zufallsvariablens aus $\mathcal{L}_0(E)$ und $(\lambda_n)_{n \in \N}$ eine beschränkte Folge in $\R$. 
Für $n \in \N$ setze
$$
    S_n := \sum_{i=1}^nX_i, \quad T_n = \sum_{i=1}^n\lambda_iX_i. 
$$
Aus technischen Gründen seien ferner $S_0 = T_0 = 0$. 
Unser Ziel ist es, zu zeigen, dass die fast sichere Konvergenz von $(S_n)_{n \in \N}$ auch die fast sichere Konvergenz von $(T_n)_{n \in \N}$ impliziert. 
Der Beweis beruht hauptsächlich auf der folgenden Abschätzung. 
\begin{lemma}
    Für alle $t > 0$ und $N \in \N$ gilt
    \begin{align}
        \prob{\norm{T_N} > t} \leq 2 \prob{\norm{S_N} > t}. 
    \end{align}
\end{lemma}

\begin{proof*}
    Wegen der Symmetrie von $X_n$ sind $\lambda_n X_n$ und $\abs{\lambda_n}X_n)$ für $n \in \N$ identisch verteilt. Ferner ist $(\lambda_n)_{n \in \N}$ nach Voraussetzung beschränkt. 
    Wir können also ohne Einschränkung annehmen, dass $0 \leq \lambda_n \leq 1$ für alle $n \in \N$. 
    \newline 
    \underline{Fall A}: $\lambda_1 \geq \lambda_2 \geq... \geq \lambda_N$. 
    \newline 
    Wie man leicht nachrechnet, gilt  
    $$
        T_N = \sum_{n=1}^N \lambda_n(S_n - S_{n-1}) = \sum_{n=1}^{N-1}(\lambda_n - \lambda_{n+1})S_n + \lambda_N S_N. 
    $$
    Mit der Dreiecksungleichung erhalten wir somit
    \begin{align*}
        \norm{T_N} &\leq \sum_{n=1}^{N-1}(\lambda_n - \lambda_{n+1})\norm{S_n} + \lambda_N \norm{S_N} \\\
                   &\leq \max_{1 \leq n \leq N}\norm{S_n} \big(\sum_{n=1}^{N-1}(\lambda_n - \lambda_{n+1}) + \lambda_N\big) \\\
                   &= \max_{1\leq n \leq N}\norm{S_n} \lambda_1 \leq \max_{1\leq n \leq N}\norm{S_n}.
    \end{align*}
    Mit Lévys Maximalungleichung $(3.1)$ folgt daraus
    $$
        \prob{\norm{T_N} > t} \leq \prob{\max_{1 \leq n \leq N}\norm{S_n} > t} \leq 2 \prob{\norm{S_N} > t}.
    $$
    \underline{Fall B}: Allgemeiner Fall. 
    \newline 
    Mittels einer Permutation $\sigma$ erhält man $\lambda_{\sigma(1)} \geq ... \geq \lambda_{\sigma(N)}$. Man beachte schließlich, dass 
    $$
        \sum_{n=1}^N\lambda_{\sigma(n)}X_{\sigma(n)} = T_N \quad \text{ und } \quad \sum_{n=1}^N X_{\sigma(n)} = S_N.
    $$ 
    Aus Fall A folgt nun die Behauptung. \qed
\end{proof*}

\begin{theorem}[Kontraktions-Prinzip, Qualitative Version]
    Falls $(S_n)_{n \in \N}$ fast sicher konvergiert, dann konvergiert auch $(T_n)_{n \in \N}$ fast sicher. 
\end{theorem}
\begin{proof*}
    Sei $\varepsilon > 0$. Nach Lemma $3.13$ gilt für $m < n$
    $$
        \prob{\norm{T_n - T_m} > \varepsilon} \leq 2\prob{\norm{S_n - S_m} > \varepsilon}. 
    $$
    Nach dem Cauchy-Kriterium für stochastische Konvergenz konvergiert $(T_n)_{n \in \N}$ somit stochastisch und nach Satz $3.8$ insbesondere fast sicher. \qed
\end{proof*}

\begin{remark}
    In Satz $3.15$ kann nicht auf die Symmetrie der Zufallsvariablen $X_1,X_2,...$ verzichtet werden. Betrachte dazu etwa die Folge $(X_n)_{n \in \N}$ definiert durch
    $$
        X_n(\omega) := (-1)^n \frac{1}{n},  \quad \omega \in \Omega, n \in \N, 
    $$ 
    und die beschränkte Folge $((-1)^n)_{n \in \N}$. Dann ist die Folge $(\sum_{k=1}^n X_n)_{n \in \N}$ fast sicher konvergent, aber die Folge $(\sum_{k=1}^n\lambda_k X_k)_{n \in \N}$ divergiert fast sicher. 
    \qexampled
\end{remark}
 

\bibliography{literatur}
\bibliographystyle{plaindin}

\end{document}