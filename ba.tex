\documentclass{report}

% IMPORTS
\usepackage{ntheorem} % Theorem/Proof Umgebungen 
\usepackage[ngerman]{babel}
\usepackage[utf8x]{inputenc}
\usepackage{amsfonts}
\usepackage{amsmath}
\usepackage{amssymb}
\usepackage{amstext}
\usepackage{enumerate} % für kleine römische Nummerierung
\usepackage{url} % Für Hyperlinks
\usepackage[singlespacing]{setspace} %1,5 Zeilenabstand 

% Für andere Kapitelüberschriften:
\usepackage{titlesec}
\titleformat{\chapter}
  {\normalfont\LARGE\bfseries}{\thechapter}{1em}{}
\titlespacing*{\chapter}{0pt}{3.5ex plus 1ex minus .2ex}{2.3ex plus .2ex}

% Seitenränder 
\usepackage[a4paper, left=4cm, right=4cm, top=2cm]{geometry}

% NEWCOMMANDS
% Syntax: \newcommand{\shortcut}[AnzahlArgumente]{\wasrauskommensoll #fürArgumentNummer}

% Nützliches

\newcommand{\qed}{\hfill $\square$}
\newcommand{\norm}[1]{\lvert \lvert #1 \rvert \rvert}
% Shortcuts für Zahlenbereiche:
\newcommand{\N}{\mathbb{N}} % Natürliche Zahlen 
\newcommand{\R}{\mathbb{R}} % Reelle Zahlen
\newcommand{\Z}{\mathbb{Z}} % Ganze Zahlen
\newcommand{\C}{\mathbb{C}} % Komplexe Zahlen
\newcommand{\Q}{\mathbb{Q}} % Rationale Zahlen 

% Oft verwen



% THEOREMSTYLES

\theorembodyfont{\upshape}
\theoremstyle{changebreak} %Sorgt für Zeilenumbruch nach Titel und Nummer vor Name

\newtheorem{theorem}{Satz} % Sätze
\numberwithin{theorem}{chapter}

\newtheorem{mydef}[theorem]{Definition} % Definitionen
\numberwithin{theorem}{chapter} %sorgt für stringente Nummerierung

\newtheorem{example}[theorem]{Beispiel} % Beispiele
\numberwithin{theorem}{chapter} %sorgt für stringente Nummerierung

\newtheorem{proposition}[theorem]{Proposition} % Propositionen
\numberwithin{theorem}{chapter} %sorgt für stringente Nummerierung

\newtheorem{remark}[theorem]{Bemerkung} %Bemerkungen
\numberwithin{theorem}{chapter} %sorgt für stringente Nummerierung

\newtheorem{lemma}[theorem]{Lemma} %Lemmata
\numberwithin{theorem}{chapter} %sorgt für stringente Nummerierung

\newtheorem{corollary}[theorem]{Korollar} %Korollare
\numberwithin{theorem}{chapter} %sorgt für stringente Nummerierung

\newtheorem*{proof*}{Beweis.} %Beweise ohne Nummerierung! 

\setlength\parindent{0pt} %Kein Einzug bei neuem Paragraphen. 

\begin{document}

\tableofcontents

\chapter{Ma\ss theoretische Vorbereitungen}
Bevor wir uns im späteren Verlauf der Arbeit mit zufälligen Reihen in Banachräumen beschäftigen können, benötigen wir ein paar maßtheoretische Vorbereitungen. 
Wir beginnen hierbei mit einigen grundlegenden Eigenschaften Borelscher $\sigma$-Algebren und darauf definierten Wahrscheinlichkeitsmaßen. 
Danach gehen wir kurz auf messbare Vektorräume ein und führen dann den Begriff der Radon-Zufallsvariable mit Werten in einem Banachraum ein. 
Da die zusätzliche algebraische Struktur eines Banachraums für unsere Betrachtung zunächst nicht von Bedeutung ist, 
werden wir uns in den ersten Abschnitten dieses Kapitels mit dem allgemeineren Fall eines (vollständigen) metrischen Raumes beschäftigen. 
Die Darstellung der ersten drei Abschnitte orientiert sich an den beiden Standardwerken \cite{parthasarathy} und \cite{billingsley}. Der kleine Exkurs zur Topologie stammt aus \cite{preuss}. 
Die Abschnitte zu messbaren Vektorräumen und Zufallsvariablen mit Werten in Banachräumen fußen auf \cite{vakhania} und \cite{ledoux-talagrand}. 
\section{Borelmengen in metrischen Räumen}
Für einen metrischen Raum $(X,d)$ bezeichne im Folgenden $\mathcal{B}(X)$ die Borel-$\sigma$-algebra in $X$. 
Zudem werden für $x \in X$ und $r>0$ mit $B(x, r)$ bzw. $\overline{B}(x,r)$ die offene bzw. abgeschlossene Kugel um $x$ mit Radius $r$ bezeichnet.
\begin{proposition}
    Sei $(X,d)$ ein separabler metrischer Raum. Dann gilt
    \begin{align*}
        \mathcal{B}(X) = \sigma(\{B(x,r): x \in X, r > 0 \}) = \sigma(\{\overline{B}(x,r): x \in X, r > 0 \}). 
    \end{align*}
\end{proposition}
\begin{proof*}
    Setze 
    \begin{align*}
        \mathcal{A}_1 &:= \sigma(\{B(x,r): x \in X, r > 0 \}), \\\ 
        \mathcal{A}_2 &:= \sigma(\{\overline{B}(x,r): x \in X, r > 0 \}). 
    \end{align*}
    Man sieht leicht ein, dass $\mathcal{A}_2 = \mathcal{A}_1 \subseteq \mathcal{B}(X)$. Zu zeigen bleibt also nur die Inklusion $\mathcal{B}(X) \subseteq \mathcal{A}_1$.
    Sei dazu $U \subseteq X$ offen und $x \in U$. Nach Voraussetzung existiert eine abzählbare dichte Teilmenge $D \subseteq X$. Definiere 
    \begin{align*}
        R := \{(y,r) : y \in U \cap D, r > 0, r \in \Q, B(y,r) \subseteq U \}.
    \end{align*}
    Dann ist $R$ abzählbar und da $D$ dicht in $X$ liegt gilt $U = \bigcup_{(y,r) \in R}B(y,r)$. 
    Also gilt $U \in \mathcal{A}_1$ und da $\mathcal{B}(X)$ von den offenen Teilmengen von $X$ erzeugt wird folgt die Behauptung. \qed
\end{proof*}

\begin{proposition}
    Für $i \in \N$ sei $(X_i, d_i)$ ein separabler metrischer Raum. Dann gilt
    \begin{align*}
        \mathcal{B}(X_1 \times X_2 \times ...) = \otimes_{i=1}^{\infty}\mathcal{B}(X_i)
    \end{align*}
\end{proposition}

\begin{proof*}
    Setze $X:= \times_{k \in \N}X_k$ und bezeichne $p_k: X \to X_k$ die Projektion auf die k-te Komponente. Betrachte das Mengensystem
    \begin{align*}
        \mathcal{E} :&= \{ \bigcap_{k \in K}p_k^{-1}(O) | \forall k \in K: O_k \subseteq X_k \text{ offen}, K \subseteq \N \text{ endlich}\}. 
    \end{align*}
    Offensichtlich gilt $\otimes_{k \in \N}\mathcal{B}(X_k) = \sigma(\mathcal{E})$. 
    Ferner ist $X$ ein separabler metrischer Raum und $\mathcal{E}$ eine Basis der Produkttopologie auf $X$, vgl. \cite{querenburg}[3.7]. 
    Also lässt sich jede offene Menge $O \subset X$ als abzählbare Vereinigung von Elementen aus $\mathcal{E}$ darstellen.  
    Dies impliziert 
    $$
    \mathcal{B}(X) = \sigma(\mathcal{E}) = \otimes_{k \in N}\mathcal{B}(X_k).
    $$
    \qed
\end{proof*}

\section{Borelmaße auf metrischen Räumen}
Im Folgenden Abschnitt beschäftigen wir uns mit Wahrscheinlichkeitsmaßen auf der Borelschen $\sigma$-Algebra $\mathcal{B}(X)$ eines metrischen Raums $(X,d)$, 
welche teilweise auch als \textit{Borelsche Wahrscheinlichkeitsmaße} bezeichnet werden. 
Zunächst interessieren wir uns hierbei für Regularitätseigenschaften solcher Maße, welche uns die spätere Arbeit erleichtern werden. 
Im zweiten Teil dieses Abschnitts untersuchen wir Folgen von Wahrscheinlichkeitsmaßen und den Begriff der schwachen Konvergenz. 
\newline \ \newline 
\textbf{Notation und Konventionen} 
\newline
Bis auf weiteres sei $(X,d)$ ein metrischer Raum mit Borel-$\sigma$-algebra $\mathcal{B}(X)$. 
Für eine Teilmenge $A \subseteq X$ sei $\mathring{A}$ das \textit{Innere}, $\overline{A}$ der \textit{Abschluss} und 
$$
    \partial A := \overline{A} \setminus \mathring{A}
$$
der \textit{Rand der Menge}. 
Setze ferner 
$$
    C_b(X;\R) := \{f:X \to \R \ | \ f \text{ ist stetig und beschränkt} \}.
$$ 
Falls klar ist welcher metrische Raum gemeint ist, so schreiben wir auch $C_b(\R)$ statt $C_b(X;\R)$. 
\newline 



\begin{mydef}
    Ein Maß $\mu$ auf $\mathcal{B}(X)$ heißt \textit{regulär} , falls
    \begin{align*}
        \forall B \in \mathcal{B}(X): \quad \mu(B) &= \sup\{\mu(C): C \subseteq B, \ C \text{ abgeschlossen} \} \\\
                                                   &= \inf\{\mu(O): B \subseteq O, \ O \text{ offen} \}.  
    \end{align*}  
\end{mydef}

\begin{proposition}
    Sei $\mu$ ein Wahrscheinlichkeitsmaß auf $\mathcal{B}(X)$. Dann ist $\mu$ regulär. 
\end{proposition}

\begin{proof*}
    Wir verwenden zum Beweis das Good-Set-Principle. Setze dazu
    $$
        \mathcal{R}:=\big\{A \in \mathcal{B}(X):  \mu(A) = \sup\{\mu(C): C \subseteq A, \ C \text{ abgeschlossen}\} =\inf\{\mu(O): A \subseteq O, \ O \text{ offen}\} \big\}.
    $$
    Wir zeigen zunächst, dass $\mathcal{R}$ eine $\sigma$-Algebra ist. Offensichtlich gilt $\emptyset \in \mathcal{R}$. Sei nun $A \in \mathcal{R}$ und $\varepsilon > 0$. 
    Dann existieren eine offene Menge $O$ und eine abgeschlossene Menge $C$ mit $C \subseteq A \subseteq O$ und 
    $$
        \mu(0) - \varepsilon < \mu(A) < \mu(C) + \varepsilon.
    $$
    Es gilt also $O^c \subseteq A^c \subseteq C^c$ und 
    $$
        \mu(C^c) - \varepsilon = 1 - \mu(C) - \varepsilon < 1 - \mu(A) = \mu(A^c) = 1 - \mu(A) < 1 - \mu(O) + \varepsilon = \mu(O^c) + \varepsilon. 
    $$
    Da $O^c$ abgeschlossen ist und $C^c$ offen ist, folgt $A^c \in \mathcal{R}$. 
    \newline 
    Seien nun $A_1, A_2,... \in \mathcal{R}$ und $\varepsilon > 0$ Für jedes $n \in \N$ existieren dann eine offene Menge $O_n$ und eine abgeschlossene Menge $C_n$ mit $C_n \subseteq A_n \subseteq O_n$ und 
    $$
        \mu(O_n) - 2^{-n}\varepsilon < \mu(A_n) < \mu(C_n) + 2^{n+1}\varepsilon.
    $$
    Es gilt also
    $$
        \bigcup_{n = 1}^{\infty} C_n \subseteq \bigcup_{n=1}^{\infty} A_n \subseteq \bigcup_{n=1}^{\infty}C_n
    $$
    und 
    \begin{align}
        \mu\left(\bigcup_{n =1}^{\infty}U_n\right) - \mu\left(\bigcup_{n=1}^{\infty}A_n\right) &\leq \mu\left(\bigcup_{n =1}^{\infty}U_n\setminus \bigcup_{n =1}^{\infty} A_n\right) \nonumber \\\
                                                                                &\leq \mu\left(\bigcup_{n =1}^{\infty}(U_n\setminus A_n)\right) 
                                                                                \leq \sum_{i=1}^{\infty}\mu(U_n\setminus A_n)
                                                                                < \sum_{i=1}^{\infty}2^{-n}\varepsilon = \varepsilon. 
    \end{align}
    Wegen 
    $$
    \mu(\cup_{n = 1}^{\infty}C_n) = \lim_{k \to \infty}\mu(\cup_{n=1}^kC_n),
    $$
    existiert zudem ein $k \in \N$ mit 
    $$
    \mu(\cup_{n=1}^{\infty}C_n) - \mu(\cup_{n=1}^kC_n) < \frac{\varepsilon}{2}.
    $$ 
    Die Menge $C:= \cup_{n=1}^kC_n$ ist als endliche Vereinigung abgeschlossener Mengen abgeschlossen und nach Konstruktion in $\cup_{n=1}^{\infty}A_n$ enthalten. Ferner gilt 
    \begin{align}
        \mu(\bigcup_{n=1}^{\infty}A_n) - \mu(C) &< \mu(\bigcup_{n=1}^{\infty}A_n) - \mu(\bigcup_{n=1}^{\infty}C_n) + \frac{\varepsilon}{2} \nonumber \\\
                                                &\leq \mu(\bigcup_{n=1}^{\infty}A_n \setminus \bigcup_{n=1}^{\infty}C_n)  + \frac{\varepsilon}{2} \nonumber \\\
                                                &\leq \mu(\bigcup_{n=1}^{\infty}(A_n \setminus C_n)) + \frac{\varepsilon}{2} \nonumber \\\
                                                &\leq \sum_{n=1}^{\infty}\mu(A_n \setminus C_n) + \frac{\varepsilon}{2} = \varepsilon. 
    \end{align}
    Da $\varepsilon$ beliebig gewählt war folgt aus $(1.1)$ und $(1.2)$, dass $\cup_{n=1}^{\infty}A_n \in \mathcal{R}$. Folglich ist $\mathcal{R}$ eine $\sigma$-Algebra. 
    Es bleibt noch zu zeigen, dass $\mathcal{R}$ alle abgeschlossenen Mengen enthält. Sei also $\emptyset \neq A \subseteq X$ abgeschlossen. Die Bedingung 
    $$
        \mu(A) = \sup\{\mu(C): C \subseteq A, C\text{ abgeschlossen }\}
    $$
    folgt direkt aus der Monotonie von $\mu$. Um die zweite Bedingung zu zeigen setze für $n \in \N$
    $$
        O_n := \big\{x \in X: \inf_{y \in  A}d(x,y) < \frac{1}{n}\big\}.
    $$
    Dann ist $O_n$ für jedes $n \in \N$ offen und es gilt $O_1 \supseteq O_2 \supseteq ...$, sowie $\cap_{n=1}^{\infty}O_n = A$, da $A$ abgeschlossen ist. 
    Mit der $\sigma$-Stetigkeit von $\mu$ folgt letztendlich 
    $$
        \mu(A) \leq \inf\{\mu(O): A \subseteq O, O \text{ offen}\} \leq \inf_{n \in \N}\mu(O_n) = \lim_{n \to \infty}\mu(O_n) = \mu(A). 
    $$
    Da die abgeschlossenen Mengen $\mathcal{B}(X)$ erzeugen gilt $\mathcal{R} = \mathcal{B}(X)$ und folglich ist $\mu$ regulär. \qed
\end{proof*}
\begin{mydef}
    Ein Maß $\mu$ auf $\mathcal{B}(X)$ heißt \textit{straff}, falls es für alle $\varepsilon > 0$ eine kompakte Menge $K \subseteq X$ gibt mit 
    \begin{align*}
        \mu(K) \geq 1 - \varepsilon. 
    \end{align*}

\end{mydef}

\begin{corollary}
    Sei $\mu$ ein straffes Wahrscheinlichkeitsmaß auf $\mathcal{B}(X)$. Dann gilt
    \begin{align*}
        \forall A \in \mathcal{B}(X): \quad \mu(A) = \sup\{\mu(K): K \subseteq A,\ K \text{ kompakt}\}. 
    \end{align*}
\end{corollary}

\begin{proof*}
    Sei $A \in \mathcal{B}(X)$ und $\varepsilon > 0$. Wegen der Straffheit von $\mu$ existiert eine kompakte Menge $K_{\varepsilon} \subseteq X$ mit $\mu(K_{\varepsilon}) \geq 1 - \frac{\varepsilon}{2}$,
    und da $\mu$ nach Proposition $1.4$ regulär ist gibt es eine abgeschlossene Menge $C \subseteq A$ mit $\mu(C) > \mu(A) - \frac{\varepsilon}{2}$. Die Menge $K_{\varepsilon} \cap C$ ist wiederum kompakt und es gilt
    \begin{align*}
        \mu(A) \geq \mu(K_{\varepsilon} \cap C) > \mu(C) - \frac{\varepsilon}{2} > \mu(A) - \varepsilon. 
    \end{align*} 
    \qed
\end{proof*}

\begin{remark}
    Ein Wahrscheinlichkeitsmaß $\mu$ auf $\mathcal{B}(X)$ mit der Eigenschaft
    \begin{align*}
        \forall A \in \mathcal{B}(X): \quad \mu(A) = \sup\{\mu(K): K \subseteq A, \ K \text{ kompakt}\}. 
    \end{align*}
    wird auch als \textit{Radon-Wahrscheinlichkeitsmaß} oder \textit{Radon-Maß} bezeichnet.
\end{remark}

\begin{proposition}
    Sei $(X,d)$ ein vollständiger separabler metrischer Raum. Dann ist jedes Wahrscheinlichkeitsmaß $\mu$ auf $\mathcal{B}(X)$ straff.
\end{proposition}

Wir verwenden zum Beweis der Proposition die folgende Charakterisierung kompakter Teilmengen metrischer Räume. 
Der Beweis wird mittels der in metrischen Räumen geltenden Äquivalenz von Überdeckungskompaktheit und Folgenkompaktheit  geführt und findet sich etwa in \cite{amann}[Theorem 3.10]. 

\begin{lemma}
    Sei $(X,d)$ ein metrischer Raum. Eine Menge $K \subseteq X$ ist genau dann kompakt, wenn sie die folgenden beiden Eigenschaften erfüllt
    \begin{enumerate}[(a)]
        \item $K$ ist vollständig,
        \item $K$ ist total-beschränkt, d.h.
        \begin{align*}
            \forall  \varepsilon > 0 \ \exists x_1,...,x_n \in K: \  K \subseteq \cup_{i=1}^n B(x_i, \varepsilon). 
        \end{align*} 
\end{enumerate}
\end{lemma}

\begin{proof*}
    Sei $\varepsilon > 0$. Nach Voraussetzung existiert eine abzählbare dichte Teilmenge $D = \{x_1, x_2,...\}$ von $X$. Also gilt insbesondere für $q \in \N $
    \begin{align*}
        \bigcup_{i = 1}^{\infty}\overline{B}(x_i, 2^{-q}) = X.
    \end{align*}
    Wegen der $\sigma$-Stetigkeit von $\mu$ existiert also ein $N_q \in \N$ mit 
    \begin{align*}
        \mu\big(\bigcup_{i=1}^{N_q}\overline{B}(x_i, 2^{-q})\big) \geq 1 - \varepsilon 2^{-q}. 
    \end{align*}
    Setze nun 
    \begin{align*}
        K := \bigcap_{q = 1}^{\infty}\bigcup_{i=1}^{N_q}\overline{B}(x_i, 2^{-q}). 
    \end{align*}
    Dann ist $K$ als Schnitt abgeschlossener Teilmengen abgeschlossen, und da $X$ vollständig ist, folgt daraus bereits die Vollständigkeit von $K$. 
    Ferner ist $K$ total-beschränkt, denn zu $\varepsilon > 0$ existiert ein $q \in \N$ mit $2^{-q} < \varepsilon$ und $K \subseteq \cup_{i=1}^{N_q}B(x_i, 2^{-q}) \subseteq \cup_{i=1}^{N_q}B(x_i, \varepsilon)$. 
    Zudem gilt
    \begin{align*}
        \mu(K)  = 1 - \mu\big(\bigcup_{q = 1}^{\infty}\bigcap_{i=1}^{N_q}\overline{B}(x_i, 2^{-q})^c\big) 
                &\geq 1 - \sum_{q=1}^{\infty} \mu\big(\bigcap_{i=1}^{N_q}\overline{B}(x_i, 2^{-q})^c\big) \\\
                &\geq 1 - \sum_{q=1}^{\infty} \varepsilon 2^{-q} = 1 - \varepsilon.
    \end{align*}
    Also ist $\mu$ straff. \qed
\end{proof*}

\begin{proposition}
    Sei $(X,d)$ ein vollständiger metrischer Raum und $\mu$ ein Wahrscheinlichkeitsmaß auf $\mathcal{B}(X)$. Dann sind äquivalent
    \begin{enumerate}[(i)]
        \item $\mu$ ist straff.
        \item Es gibt eine separable Teilmenge $E \subseteq X$ mit $\mu(E) = 1$. 
    \end{enumerate}
\end{proposition}
\begin{proof*}
    zu (i) $\Rightarrow$ (ii): Für alle $n \in \N$ existiert $K_n \subseteq X$ kompakt mit $\mu(K_n) \geq 1 - \frac{1}{n}$, o.E. gelte $K_n \subseteq K_{n+1}$. Es folgt 
    \begin{align*}
        \mu\big(\cup_{n=1}^{\infty}K_n\big) = \lim_{n \to \infty}\mu\big(K_ n\big) = 1. 
    \end{align*}
    Da kompakte Teilmengen metrischer Räume insbesondere separabel sind, ist $E := \cup_{n=1}^{\infty}K_n$ als abzählbare Vereinigung separabler Mengen ebenso separabel. 
    \newline 
    zu (ii) $\Rightarrow$ (i): 
    Analog zum Beweis von Proposition 1.8. \qed
\end{proof*}

Insgesamt haben wir also gezeigt

\begin{theorem}
    Für ein Wahrscheinlichkeitsmaß $\mu$ auf der Borelschen $\sigma$-Algebra eines vollständigen metrischen Raumes $(X,d)$ sind äquivalent
    \begin{enumerate}[(i)]
        \item $\mu$ ist straff.
        \item Es gibt eine separable Menge $E \subseteq X$ mit $\mu(E) = 1$.
        \item $\mu$ ist ein Radon-Maß, d.h.
        $$
        \forall A \in \mathcal{B}(X): \quad \mu(A) = \sup\{\mu(K): K \subseteq A, \ K \text{ kompakt}\}.
        $$   
    \end{enumerate}
\end{theorem}

Nachdem wir uns nun ausgiebig mit den Eigenschaften einzelner Maße beschäftigt haben, möchten wir uns jetzt mit Folgen von Wahrscheinlichkeitsmaßen und deren Konvergenz beschäftigen. 

\begin{mydef}
    Eine Folge $(\mu_n)_{n \in \N}$ von Wahrscheinlichkeitsmaßen auf $\mathcal{B}(X)$ heißt \textit{schwach konvergent} 
    gegen ein Wahrscheinlichkeitsmaß $\mu$ auf $\mathcal{B}(X)$, falls 
    $$
        \forall f \in C_b(X): \quad \lim_{n \to \infty} \int_Xfd\mu_n = \int_X fd\mu . 
    $$
    Bezeichnung: $\mu_n \rightharpoonup \mu$. 
\end{mydef}

Als nützliches Hilfsmittel für viele Beweise dient der folgende Satz, der meist als \textit{Portmanteau-Theorem} bezeichnet wird. 

\begin{theorem}[Portmanteau-Theorem]
    Sei $(X,d)$ ein metrischer Raum und seien $\mu, \mu_1, \mu_2,...$ Wahrscheinlichkeitsmaße auf $\mathcal{B}(X)$. Dann sind äquivalent
    \begin{enumerate}[(i)]
        \item $(\mu_n)_{n \in \N}$ konvergiert schwach gegen $\mu$.
        \item Für alle abgeschlossenen Teilmengen $A \subseteq X$ gilt 
        $$
            \limsup_{n \to \infty} \mu_n(A) \leq \mu(A).
        $$
        \item Für alle offenen Teilmengen $B \subseteq X$ gilt 
        $$
            \liminf_{n \to \infty} \mu_n(B) \geq \mu(B).
        $$
        \item Für alle Borelmengen $C \in \mathcal{B}(X)$ mit $\mu(\partial C) = 0$ gilt 
        $$
            \lim_{n \to \infty}\mu_n(C) = \mu(C).
        $$
    \end{enumerate}
\end{theorem}

\begin{proof*}
    Zu $(i) \Rightarrow (ii)$: Sei $A \subseteq X$ abgeschlossen und $\varepsilon > 0$. Da die Aussage für $A = \emptyset$ trivialerweise erfüllt ist
    können wir ohne Beschränkung der Allgemeinheit annehmen, dass $A \neq \emptyset$. Für $m \in \N$ setze
    \begin{align*}
        U_m := \{x \in X: d(x,A) < \frac{1}{m}\}.
    \end{align*}
    Dann sind die Mengen $U_m$ offen und es für alle $m \in \N$ gilt $A \subseteq U_m$. Da $A$ abgeschlossen ist erhalten wir zudem $ A = \cap_{m \in \N} U_m$. 
    Aufgrund der $\sigma$-Stetigkeit von $\mu$ existiert ein $k \in \N$ mit 
    $$
        \mu(U_k) < \mu(A) + \varepsilon . 
    $$
    Betrachte nun die Abbildung 
    $$
        f:X \to \R, \quad x \mapsto \max\{1 - kd(x,A), 0\}.
    $$
    Offensichtlich ist $f$ beschränkt und nach der umgekehrten Dreiecksungleichung auch stetig. Wegen $1_A \leq f \leq 1_{U_k}$ erhalten wir zusammen mit der Voraussetzung 
    $$
    \limsup_{n \to \infty} \mu_n(A) \leq \lim_{n \to \infty} \int_X fd\mu_n = \int_X fd\mu \leq \mu(U_k) \leq \mu(A) + \varepsilon.
    $$
    Da diese Ungleichung für alle $\varepsilon > 0$ erfüllt ist folgt die Behauptung. 
    \newline 
    Zu $(ii) \iff (iii)$: Folgt unmittelbar durch Komplementbildung. 
    \newline
    Zu $(iii) \Rightarrow (iv)$: 
    Sei $C \in \mathcal{B}(X)$ mit $\mu(\partial C) = 0$. Dann gilt insbesondere $\mu(\overline{C}) = \mu(\mathring{C}))$. Da $(iii)$ auch $(ii)$ impliziert erhalten wir somit
    $$
        \mu(C) = \mu(\mathring{C})) \leq \liminf_{n \to \infty} \mu_n(C) \leq \limsup_{n \to \infty} \mu_n(C) \leq \mu(\overline{C}) = \mu(C).
    $$
    \newline 
    Zu $(iv) \Rightarrow (i)$: 
    Sei $f \in C_b(X)$ beschränkt durch $M > 0$. Wegen der Linearität des Integrals können wir ohne Einschränkung annehmen, dass $f \geq 0$. 
    Wegen der Stetigkeit von $f$ erhalten wir zunächst für alle $t > 0$
    $$
        \partial\{ f > t \} \subseteq \{f = t \}. 
    $$
    Da $\mu$ ein Wahrscheinlichkeitsmaß ist, gibt es eine abzählbare Menge $C \subseteq \R$ mit 
    $$
        \forall t \in \R \setminus C: \quad \mu(\{f = t \}) = 0. 
    $$
    Also gilt für alle $t \in \R \setminus C$ nach Voraussetzung 
    $$
        \lim_{n \to \infty} \mu_n(\{f > t \}) = \mu(\{f > t \})
    $$
    Mit dem Satz von Cavalieri, vgl. \cite{gs}[1.8.20], erhalten wir schließlich per dominierter Konvergenz
    $$
        \lim_{n \to \infty} \int_X fd\mu_n = \lim_{n \to \infty} \int_0^M \mu_n(\{f > t \})d\lambda(t) = \int_0^M \mu(\{f > t \}) d\lambda(t) = \int_Xfd\mu. 
    $$
    \qed 
\end{proof*}
\section{Die Prokhorov Metrik}
Nachdem wir im letzten Abschnitt damit begonnen haben uns mit der schwachen Konvergenz von Wahrscheinlichkeitsmaßen zu beschäftigen, 
wollen wir nun ein weiteres Hilfsmittel zur Untersuchung von schwacher Konvergenz einführen. 
\newline 
Für einen metrischen Raum $(X,d)$ bezeichne $\mathcal{M}(X)$ die Menge aller Wahrscheinlichkeitsmaße auf der Borelschen $\sigma$-Algebra $\mathcal{B}(X)$. 
\begin{mydef}
    Eine Familie $M \subseteq \mathcal{M}(X)$ von Wahrscheinlichkeitsmaßen heißt \textit{gleichmäßig straff}, 
    falls es für alle $\varepsilon > 0$ eine kompakte Menge $K \subseteq X$ gibt mit 
    $$
        \forall \mu \in M: \mu(K) \geq 1-\varepsilon. 
    $$
\end{mydef}
Ziel dieses Abschnitts ist der Satz von Prokhorov, der uns eine für spätere Beweise wichtige Charakterisierung der gleichmäßigen Straffheit liefert. 

Ein wichtiges Resultat ist die Folgende auf Prokhorov zurückgehende Charakterisierung. Ein Beweis findet sich etwa in \cite{parthasarathy}[Theorem 6.7]. 

\begin{theorem}[Satz von Prokhorov]
    Sei $(X,d)$ ein vollständiger separabler metrischer Raum und $M \subseteq \mathcal{M}(X)$. Dann sind äquivalent
    \begin{enumerate}[(i)]
        \item $M$ ist relativ kompakt,
        \item $M$ ist gleichmäßig straff. 
    \end{enumerate}
\end{theorem}
\newpage
\section{Flache Konzentrierung von Wahrscheinlichkeitsmaßen}
Quelle: \cite{vakhania}, Originalpaper: de Acosta \textbf{(TODO: zitieren)}
Für eine nichtleere Teilmenge $A \subseteq E$ setze 
$$
    A^{\varepsilon} := \{x \in E: \inf_{y\in A}\norm{x-y} < \varepsilon \}.
$$
Ferner sei daran erinnert, dass jeder endlich dimensionale Untervektorraum  $S \subseteq E$ abgeschlossen ist und eine Menge $A \subseteq S$ genau dann kompakt ist, wenn sie abgeschlossen und beschränkt ist. 
Insbesondere besitzt jede beschränkte Folge in $S$ eine konvergente Teilfolge. 
\begin{mydef}
    Eine Familie $M \subseteq \mathcal{M}(E)$ von Wahrscheinlichkeitsmaßen auf $\mathcal{B}(E)$ heißt \textit{flach konzentriert}, falls es für alle $\varepsilon > 0$ einen endlichdimensionalen 
    Untervektorraum $S \subseteq E$ gibt mit 
    $$
        \forall \mu \in M: \quad \mu(S^{\varepsilon}) \geq 1 - \varepsilon.
    $$ 
\end{mydef}

\begin{lemma}
    Eine Teilmenge $A$ von $E$ ist genau dann relativ kompakt, wenn $A$ beschränkt ist und es für alle $\varepsilon > 0$ einen endlichdimensionalen Untervektorraum $S \subseteq E$ gibt mit 
    \begin{align*}
        A \subseteq S^{\varepsilon}
    \end{align*}
\end{lemma}

\begin{proof*}
    zu $\Rightarrow$: Sei $A \subseteq E$ relativ kompakt. Dann ist $\overline{A}$ kompakt und folglich beschränkt, woraus wir direkt die Beschränktheit von $A$ erhalten. 
    Ferner ist $\overline{A}$ separabel, also existiert eine abzählbare dichte Teilmenge $\{x_1, x_2,...\} \subseteq A$. 
    Für $\varepsilon > 0$ ist daher $(B(x_n, \varepsilon))_{n \in \N}$ eine offene Überdeckung von $\overline{A}$. Wegen der Kompaktheit existiert $I \subseteq \N$ endlich mit 
    $$
        \overline{A} \subseteq \bigcup_{i \in I}B(x_i, \varepsilon).
    $$
    Sei also $S$ der von $\{x_i : i \in I\}$ erzeugte endlich dimensionale Untervektorraum von $E$. Dann gilt 
    $$
        A \subseteq \overline{A} \subseteq \bigcup_{i \in I}B(x_i, \varepsilon) \subseteq S^{\varepsilon}.
    $$
    zu $\Leftarrow$: Wir zeigen, dass jede Folge in $A$ eine konvergente Teilfolge besitzt, der Grenzwert der Teilfolge muss hierbei nicht in $A$ liegen. 
    Sei dazu $(x^{(0)}_n)_{n \in \N}$ eine Folge in $A$, $\varepsilon > 0$ und $S \subseteq E$ ein endlichdimensionaler Untervektorraum mit $A \subseteq S^{\varepsilon}$. 
    Dann existieren insbesondere eine Folge $(y_n)_{n \in \N}$ in $S$ mit 
    $$
        \forall n \in \N: \quad d(x^{(0)}_n, y_n) \leq 2\varepsilon.
    $$
    Aus der Beschränktheit von $(x^{(0)}_n)_{n \in \N}$ erhalten wir direkt die Beschränktheit von $(y_n)_{n \in \N}$ und da $S$ endlichdimensional ist existiert eine konvergente Teilfolge $(y_{n_k})_{k \in \N}$.
    Es gilt also für $k,m \geq N(\varepsilon) \in \N$
    \begin{align*}
        \norm{x^{(0)}_{n_k} - x^{(0)}_{n_m}} \leq \norm{x^{(0)}_{n_k} - y_{n_k}} + \norm{y_{n_k} - y_{n_m}} + \norm{x^{(0)}_{n_m} - y_{n_m}} \leq 5\varepsilon. 
    \end{align*}
    Ohne Einschränkung können wir durch entfernen endlich vieler Folgenglieder annehmen, dass 
    $$
        \forall k,m \in \N: \quad \norm{x^{(0)}_{n_k} - x^{(0)}_{n_m}} \leq 5\varepsilon. 
    $$
    Durch obiges Verfahren können wir für $N \in \N$ und $\varepsilon_N = \frac{1}{N}$ induktiv eine Teilfolge $(x^{(N)}_n)_{n \in \N}$ von $(x^{(N-1)}_n)_{n \in \N}$ gewinnen mit
    $$
        \forall m,n \in \N: \quad \norm{x^{(N)}_n - x^{(N)}_m} \leq \frac{5}{N}.
    $$
    Durch bilden der Diagonalfolge $(x^{(N)}_N)_{N \in \N}$ erhalten wir somit eine Teilfolge der Ausgangsfolge $(x^{(0)}_n)_{n \in \N}$ die eine Cauchy-Folge ist und daher in $E$ konvergiert. 
    \qed 
\end{proof*}

\begin{lemma}
    Sei $S \subseteq E$ ein endlichdimensionaler Untervektorraum und $l_1,...,l_n \in E'$ Funktionale mit 
    \begin{align}
        \forall x,y \in S \ \exists k \in \{1,...,n\}: \quad l_k(x) \neq l_k(y).
    \end{align}
    Dann ist die Menge 
    $$
        B := S^{\varepsilon} \cap \{x \in E: \abs{l_1(x)} \leq r_1,...,\abs{l_n(x)}\leq r_n\}
    $$
    für alle $\varepsilon, r_1,...,r_n \in (0, \infty)$ beschränkt. 
\end{lemma}

\begin{proof*}
    Wegen $(1.1)$ definiert 
    $$
        p(x) := \max_{1\leq k \leq n} \abs{l_k(x)}, \quad x \in S,
    $$
    eine Norm auf $S$. Da $S$ endlichdimensional ist, ist diese insbesondere äquivalent zur Einschränkung von $\norm{\cdot}$ auf $S$. 
    Angenommen die Menge $B$ ist nicht beschränkt. Dann existiert eine Folge $(x_n)_{n \in \N}$ in $B$ mit 
    $$
        \lim_{n \to \infty} \norm{x_n} = \infty. 
    $$
    Wegen der Normäquivalenz existiert dann ein $k \in \{1,...,n\}$ mit 
    $$
        \lim_{n \to \infty}\abs{l_k(x_n)} = \infty. 
    $$
    Im Widerspruch zur Definition von $B$. \qed
\end{proof*}

\begin{theorem}
    Sei $\Gamma \subseteq E'$, sodass 
    \begin{align}
        \forall x,y \in E \ \exists l \in \Gamma: \quad l(x) \neq l(y).
    \end{align}
    Eine Menge $M \subseteq \mathcal{M}(E)$ von Wahrscheinlichkeitsmaßen ist genau dann relativ kompakt in $(\mathcal{M}, \rho)$ wenn die folgenden beiden Bedingungen erfüllt sind
    \begin{enumerate}[(a)]
        \item Für alle $l \in \Gamma$ ist $\{\mu^l : \mu \in \Gamma\} \subseteq \mathcal{M}(\R)$ relativ kompakt,
        \item $M$ ist flach konzentriert. 
    \end{enumerate}
\end{theorem}

\begin{proof*}
    zu $\Rightarrow$: 
    Sei $M \subseteq \mathcal{M}(E)$ relativ kompakt. Nach dem Satz von Prokhorov ist $M$ dann insbesondere gleichmäßig straff. 
    Also existiert zu $\varepsilon > 0$ eine kompakte Menge $K \subseteq E$ mit 
    $$
        \forall \mu \in M: \quad \mu(K) \geq 1 - \varepsilon.   
    $$ 
    Aus der Stetigkeit von $l \in \Gamma$ erhalten wir somit direkt die gleichmäßige Straffheit von $\{\mu^l : \mu \in \Gamma\}$. Erneutes anwenden des Satzes von Prokhorov liefert $(a)$. 
    Da $K$ insbesondere relativ kompakt ist liefert \textbf{Lemma zwei davor, TODO} einen endlichdimensionalen Untervektorraum $S \subseteq E$ mit $K \subseteq S^{\varepsilon}$. Somit gilt
    $$
        \forall \mu \in M: \quad \mu(S^{\varepsilon}) \geq \mu(K) \geq 1 - \varepsilon.
    $$
    Folglich ist $M$ flach konzentriert. 
    \newline 
    zu $\Leftarrow$: 
    \textbf{TODO}
\end{proof*}

\begin{remark}%TODO: evtl. weiter nach oben verschieben und noch erklären wieso es für S endlich viele derartige Funktionale gibt. 
    Man sagt eine Familie von Abbildungen mit der Eigenschaft $(1.2)$ (bzw. $(1.1)$ ) \textit{trenne die Punkte von E} (bzw $S$). Die Existenz einer solchen Menge $\Gamma \subseteq E'$ ist durch den Satz von Hahn-Banach sichergestellt. 
    Insbesondere erfüllt $E'$ selbst$(1.1)$. 
\end{remark}
\section{Messbare Vektorräume}
Bislang haben wir uns fast ausschließlich mit dem Zusammenspiel von Maßen und den topologischen Eigenschaften der zugrunde liegenden Räume beschäftigt.
In Banachräumen steht uns aber auch die algebraische Struktur eines Vektorraums zur Verfügung, allerdings ist per se nicht klar, ob die algebraischen Operationen mit der messbaren Struktur kompatibel, also messbar, sind. 
Diese Überlegung führt direkt zur Definition eines \textit{messbaren Vektorraums}. 

\begin{mydef}
    Sei $X$ ein Vektorraum und $\mathcal{C}$ eine $\sigma$-Algebra auf $X$. Das Tupel $(X, \mathcal{C})$ heißt \textit{messbarer Vektorraum}, falls die folgenden beiden Bedingungen erfüllt sind: 
    \begin{enumerate}[(a)]
        \item Die Abbildung 
        \begin{align*}
            + : X \times X \to X, \quad (x,y) \mapsto x + y
        \end{align*}
        ist $\mathcal{C}\otimes \mathcal{C}/\mathcal{C}$-messbar und
        \item die Abbildung 
        \begin{align*}
            \cdot : \R \times X \to X, \quad  (\alpha, x) \mapsto \alpha x
        \end{align*}
        ist $\mathcal{B}(\R) \otimes \mathcal{C}/\mathcal{C}$-messbar. 
    \end{enumerate}
\end{mydef}

\begin{remark}
    Sei $(X, \mathcal{C})$ ein messbarer Vektorraum. Dann gilt:
    \begin{enumerate}[(i)]
        \item Für alle $\alpha \in \R$ ist die Abbildung 
            $$f_{\alpha}: X \to X, \quad x \mapsto \alpha x$$
        $\mathcal{C}/\mathcal{C}$-messbar. 
        \item Für alle $y \in X$ ist die Abbildung 
            $$g_y: X \to X, \quad x \mapsto x + y$$
        $\mathcal{C}/\mathcal{C}$-messbar.
    \end{enumerate}
\end{remark}
\begin{proof*}
    Aus der Messbarkeit der Skalarmultiplikation und Vektoraddition  folgt zunächst
    \begin{align*}
        \forall A \in \mathcal{C}: \quad &A^{(1)} := \{(\alpha, x) \in \R \times X: \ \alpha x \in A\} \in \mathcal{B}(\R) \otimes \mathcal{C}, \\\
                                         &A^{(2)} := \{(x,y) \in X \times X: \ x + y \in A\} \in \mathcal{C} \otimes \mathcal{C}. 
    \end{align*}
    Man beachte nun, dass für beliebige  messbare Räume $(\Omega_1, \mathcal{A}_1), (\Omega_2, \mathcal{A}_2)$, Mengen $A \in \mathcal{A}_1 \otimes \mathcal{A}_2$ und $\omega_1 \in \Omega_1$ 
    \begin{align*}
    A(\omega_1) = \{ \omega_2 \in \Omega_2 : (\omega_1,\omega_2) \in A \} \in \mathcal{A}_2
    \end{align*}
    gilt. In unserem Fall erhalten wir für festes $\alpha \in R$ und $y \in X$
    \begin{align*}
        &f_{\alpha}^{-1}(A) = \{x \in X: \ \alpha x \in A\} = A^{(1)}(\alpha) \in \mathcal{C},  \\\
        &g_y^{-1}(A) = \{ x \in X: \ x + y \in A\} = A^{(2)}(y)  \in \mathcal{C}. 
    \end{align*}
    \qed
\end{proof*}
Da die Komposition messbarer Abbildungen wiederum messbar ist, erhält man unmittelbar
\begin{proposition}
    Sei $(X, \mathcal{C})$ ein messbarer Vektorraum und $(\Omega, \mathcal{A})$ ein messbarer Raum. 
    Sind $X,Y: \Omega \to X$ zwei $\mathcal{A}/\mathcal{C}$-messbare Abbildungen und $\alpha, \beta \in \R$, so ist auch $\alpha X + \beta Y$ $\mathcal{A}/\mathcal{C}$-messbar. 
\end{proposition}

Bislang wissen wir noch nicht einmal, ob es überhaupt nicht-triviale Beispiele messbarer Vektorräume gibt. Das wollen wir nun ändern.  
\begin{proposition}
    Sei $X$ ein separabler Banachraum. Dann ist $(X, \mathcal{B}(X))$ ein messbarer Vektorraum.
\end{proposition}
\begin{proof*}
    Nach Proposition $1.7$ gilt $\mathcal{B}(X \times X) = \mathcal{B}(X) \otimes \mathcal{B}(X)$ und 
    $\mathcal{B}(\R \times X) = \mathcal{B}(\R) \otimes \mathcal{B}(X)$. Ferner sind die Abbildungen 
    \begin{align*}
        + &: X \times X \to X, \quad (x,y) \mapsto x + y, \\\
        \cdot &: \R \times X \to X, \quad (\alpha, x) \mapsto \alpha  x
    \end{align*}
    stetig bzgl. der jeweiligen Produkttopologien und somit insbesondere $\mathcal{B}(X \times X)/\mathcal{B}(X)$- bzw. 
    $\mathcal{B}(\R \times X)/\mathcal{B}(X)$-messbar. \qed
\end{proof*}

\begin{example}
    Für $d \in \N$ ist $(\R^d, \mathcal{B}(\R^d))$ ein messbarer Vektorraum. 
\end{example}

Im Folgenden bezeichne $(X', \norm{\cdot}_{op})$ den Dualraum eines normierten Vektorraums $(X, \norm{\cdot})$.

\begin{proposition}
    Sei $\emptyset \neq \Gamma \subseteq X'$. Dann ist $(X, \sigma({\Gamma}))$ ein messbarer Vektorraum. 
\end{proposition}

\begin{proof*}%TODO: überarbeiten und schöneren Beweis finden. 
    Zur Messbarkeit der Addition: Es genügt zu zeigen, dass 
    $$
        \forall f \in \Gamma : \big(g: X \times X \to \R, \ (x,y) \mapsto f(x+y)\big) \text{ ist messbar.}
    $$
    Sei dazu $f \in \Gamma$. Wegen der Linearität von $f$ gilt für $(x,y) \in X \times X$
    $$
        f(x + y) = (f(x) + f(y)).
    $$
    Weiter ist die Abbildung 
    $$
        h: X \times X \to \R, (x,y) \mapsto f(x) + f(y)
    $$
    als Komposition der messbaren Funktionen 
    \begin{align*}
        &h_1: X \times X \to \R \times \R, (x,y) \mapsto (f(x),f(y)), \\\
        &h_2: \R \times \R \to \R, (x,y) \mapsto x+y
    \end{align*}
    messbar. Die Abbildung $h_2$ ist messbar, weil $(\R, \mathcal{B}(\R))$ nach Beispiel $1.39$ ein messbarer Vektorraum ist. 
    Die Messbarkeit der Skalarmultiplikation zeigt man ähnlich. \qed 
\end{proof*}

\begin{proposition}
    Sei $E$ ein separabler Banachraum. Dann gilt $\sigma(E') = \mathcal{B}(E)$. 
\end{proposition}

\begin{proof*}%TODO: Überprüfen. Reicht das wirklich schon so?
    Da alle $f \in E'$ stetig sind, gilt offensichtlich $\sigma(E') \subseteq \mathcal{B}(E)$. 
    Wegen der Separabilität von $E$ wird $\mathcal{B}(E)$ nach Proposition $1.1$ von den abgeschlossenen Kugeln erzeugt und weil $(E, \sigma(E'))$ nach Proposition $1.40$ ein messbarer Vektorraum ist, genügt es nach Bemerkung $1.36$ zu zeigen, 
    dass $\overline{B}(0,1)$ in $\sigma(E')$ enthalten ist. Da $E$ separabel ist, existiert nach Korollar $A.12$ eine Folge $(f_n)_{n \in \N}$ in $E'$ mit
    $$
        \forall x \in E: \quad \norm{x} = \sup_{n \in \N}\abs{f_n(x)}.
    $$
    Es gilt also
    $$
        \overline{B}(0,1) = \{ x \in E: \norm{x} \leq 1 \} = \{x \in E: \ \sup_{n \in \N}\abs{f_n(x)} \leq 1 \} = \bigcap_{n=1}^{\infty}\{x \in E: \abs{f_n(x)} \leq 1\} \in \sigma(E').
    $$
    \qed 
\end{proof*}

\section{Zufallsvariablen mit Werten in Banachräumen}
Sei $(\Omega, \mathcal{A}, P)$ ein vollständiger Wahrscheinlichkeitsraum und $E$ ein Banachraum mit Norm $\norm{\cdot}$. 
\begin{mydef}
    Eine Abbildung $X: \Omega \to E$ heißt \textit{(Radon-)Zufallsvariable} falls 
    \begin{enumerate}[(a)]
        \item $X$ ist $\mathcal{A}/\mathcal{B}(E)$-messbar und
        \item es existiert ein separabler Untervektorraum $E_0 \subseteq E$ mit $P(\{X \in E_0\}) = 1$. 
    \end{enumerate}
\end{mydef}
\textbf{TODO: Erklärung warum Radon, Rückgriff auf Abschnitt 1.2}
\begin{proposition}
    \textbf{TODO: Charakterisierung von Radon-Zufallsvariablen}
\end{proposition}
\textbf{TODO: Einführung einfache Funktionen im Banach-Kontext}
\begin{proposition}
    \textbf{TODO: Summen von Radon-Variablen sind messbar, Approximation durch einfache Funktionen, Grenzwerte sind messbar}
\end{proposition}

Bezeichne $\mathcal{L}_0(\Omega, \mathcal{A}, P; E)$ den Raum der $E$-wertigen Radon-Zufallsvariablen auf $(\Omega,\mathcal{A},P)$ und sei $L_0(E)$ der Raum der Äquivalenzklassen bezüglich fast sicherer Gleichheit. 
Falls klar ist welcher Wahrscheinlichkeitsraum gemeint ist, so schreiben wir auch $\mathcal{L}_0(E)$ oder $L_0(E)$. 







\chapter{Symmetrische Zufallsvariablen und Lévys Ungleichung}
Bezeichne $L_0(E)$ den Vektorraum der E-wertigen-Zufallsvariablen. 
\begin{mydef}
    Eine E-wertige Zufallsvariable $X$ heißt \textit{symmetrisch}, falls $-X$ die selbe Verteilung hat wie $X$, d.h.
    \begin{align*}
        \forall A \in \mathcal{B}(E): P(\{X \in A\}) = P(\{-X \in A\}). 
    \end{align*}
\end{mydef}

\begin{mydef}
    Eine Folge $X_1,X_2,...$ von E-wertigen Zufallsvariablen heißt \textit{symmetrisch}, 
    falls $(\varepsilon_1 X_1, \varepsilon_2 X_2,...)$ für jede Wahl von $\varepsilon_i = \pm 1$ 
    die gleiche Verteilung hat wie $(X_1,X_2,...)$. 
\end{mydef}

\begin{remark}
   Sind $X_1,X_2,...$ unabhängige E-wertige Zufallsvariablen, sodass $X_n$ für alle $n \in \N$ symmetrisch ist, dann ist $(X_1,X_2,...)$ symmetrisch. 
\end{remark}

\begin{theorem}[Lévy's maximal inequality]
    Seien $X_1,...,X_N \in L_0(E)$ unabhängige und symmetrische Zufallsvariablen und setze 
    \begin{align*}
        S_n := \sum_{i=1}^n X_i, \quad 1 \leq n \leq N. 
    \end{align*}
    Dann gilt für alle $t > 0$
    \begin{align}
        &P\big(\{ \max_{1 \leq n \leq N} \norm{S_n} > t \}\big) \leq 2 P\big(\{\norm{S_N} > t \}\big), \\\
        &P\big(\{ \max_{1 \leq n \leq N} \norm{X_n} > t \}\big) \leq 2 P\big(\{\norm{S_N} > t \}\big).
    \end{align}
\end{theorem}


\bibliography{literatur}
\bibliographystyle{plaindin}

\end{document}