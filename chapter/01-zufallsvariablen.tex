\chapter{Ma\ss theoretische Vorbereitungen}
Bevor wir uns im späteren Verlauf der Arbeit mit zufälligen Reihen in Banachräumen beschäftigen können, benötigen wir ein paar maßtheoretische Vorbereitungen. 
Wir beginnen hierbei mit einigen grundlegenden Eigenschaften Borelscher $\sigma$-Algebren und darauf definierten Wahrscheinlichkeitsmaßen. 
Danach gehen wir kurz auf messbare Vektorräume ein und führen dann den Begriff der Radon-Zufallsvariable mit Werten in einem Banachraum ein. 
Da die zusätzliche algebraische Struktur eines Banachraums für unsere Betrachtung zunächst nicht von Bedeutung ist, 
werden wir uns in den ersten Abschnitten dieses Kapitels mit dem allgemeineren Fall eines (vollständigen) metrischen Raumes beschäftigen. 
Die Darstellung der ersten drei Abschnitte orientiert sich an den beiden Standardwerken \cite{parthasarathy} und \cite{billingsley}. Der kleine Exkurs zur Topologie stammt aus \cite{preuss}. 
Die Abschnitte zu messbaren Vektorräumen und Zufallsvariablen mit Werten in Banachräumen fußen auf \cite{vakhania} und \cite{ledoux-talagrand}. 
\section{Borelmengen in metrischen Räumen}
Für einen metrischen Raum $(X,d)$ bezeichne im Folgenden $\mathcal{B}(X)$ die Borel-$\sigma$-algebra in $X$. 
Zudem werden für $x \in X$ und $r>0$ mit $B(x, r)$ bzw. $\overline{B}(x,r)$ die offene bzw. abgeschlossene Kugel um $x$ mit Radius $r$ bezeichnet.
\begin{proposition}
    Sei $(X,d)$ ein separabler metrischer Raum. Dann gilt
    \begin{align*}
        \mathcal{B}(X) = \sigma(\{B(x,r): x \in X, r > 0 \}) = \sigma(\{\overline{B}(x,r): x \in X, r > 0 \}). 
    \end{align*}
\end{proposition}
\begin{proof*}
    Setze 
    \begin{align*}
        \mathcal{A}_1 &:= \sigma(\{B(x,r): x \in X, r > 0 \}), \\\ 
        \mathcal{A}_2 &:= \sigma(\{\overline{B}(x,r): x \in X, r > 0 \}). 
    \end{align*}
    Man sieht leicht ein, dass $\mathcal{A}_2 = \mathcal{A}_1 \subseteq \mathcal{B}(X)$. Zu zeigen bleibt also nur die Inklusion $\mathcal{B}(X) \subseteq \mathcal{A}_1$.
    Sei dazu $U \subseteq X$ offen und $x \in U$. Nach Voraussetzung existiert eine abzählbare dichte Teilmenge $D \subseteq X$. Definiere 
    \begin{align*}
        R := \{(y,r) : y \in U \cap D, r > 0, r \in \Q, B(y,r) \subseteq U \}.
    \end{align*}
    Dann ist $R$ abzählbar und da $D$ dicht in $X$ liegt gilt $U = \bigcup_{(y,r) \in R}B(y,r)$. 
    Also gilt $U \in \mathcal{A}_1$ und da $\mathcal{B}(X)$ von den offenen Teilmengen von $X$ erzeugt wird folgt die Behauptung. \qed
\end{proof*}

\begin{proposition}
    Für $i \in \N$ sei $(X_i, d_i)$ ein separabler metrischer Raum. Dann gilt
    \begin{align*}
        \mathcal{B}(X_1 \times X_2 \times ...) = \otimes_{i=1}^{\infty}\mathcal{B}(X_i)
    \end{align*}
\end{proposition}

\begin{proof*}
    Setze $X:= \times_{k \in \N}X_k$ und bezeichne $p_k: X \to X_k$ die Projektion auf die k-te Komponente. Betrachte das Mengensystem
    \begin{align*}
        \mathcal{E} :&= \{ \bigcap_{k \in K}p_k^{-1}(O) | \forall k \in K: O_k \subseteq X_k \text{ offen}, K \subseteq \N \text{ endlich}\}. 
    \end{align*}
    Offensichtlich gilt $\otimes_{k \in \N}\mathcal{B}(X_k) = \sigma(\mathcal{E})$. 
    Ferner ist $X$ ein separabler metrischer Raum und $\mathcal{E}$ eine Basis der Produkttopologie auf $X$, vgl. \cite{querenburg}[3.7]. 
    Also lässt sich jede offene Menge $O \subset X$ als abzählbare Vereinigung von Elementen aus $\mathcal{E}$ darstellen.  
    Dies impliziert 
    $$
    \mathcal{B}(X) = \sigma(\mathcal{E}) = \otimes_{k \in N}\mathcal{B}(X_k).
    $$
    \qed
\end{proof*}

\section{Borelmaße auf metrischen Räumen}

Bis auf weiteres sei $(X,d)$ ein metrischer Raum mit Borel-$\sigma$-algebra $\mathcal{B}(X)$. 
Im Folgenden Abschnitt beschäftigen wir uns mit Maßen auf $\mathcal{B}(X)$, welche teilweise auch als \textit{Borel-Maße} bezeichnet werden. 
Die Bezeichnung wird in der Literatur allerdings nicht einheitlich verwendet. 

\begin{mydef}
    Ein Maß $\mu$ auf $\mathcal{B}(X)$ heißt \textit{regulär} , falls
    \begin{align*}
        \forall B \in \mathcal{B}(X): \quad \mu(B) &= \sup\{\mu(C): C \subseteq B, \ C \text{ abgeschlossen} \} \\\
                                                   &= \inf\{\mu(O): B \subseteq O, \ O \text{ offen} \}.  
    \end{align*}  
\end{mydef}

\begin{proposition}
    Sei $\mu$ ein Wahrscheinlichkeitsmaß auf $\mathcal{B}(X)$. Dann ist $\mu$ regulär. 
\end{proposition}

\begin{proof*}
    \textbf{TODO}
\end{proof*}

\begin{mydef}
    Ein Maß $\mu$ auf $\mathcal{B}(X)$ heißt \textit{straff}, falls es für alle $\varepsilon > 0$ eine kompakte Menge $K \subseteq X$ gibt mit 
    \begin{align*}
        \mu(K) \geq 1 - \varepsilon. 
    \end{align*}

\end{mydef}

\begin{corollary}
    Sei $\mu$ ein straffes Wahrscheinlichkeitsmaß auf $\mathcal{B}(X)$. Dann gilt
    \begin{align*}
        \forall A \in \mathcal{B}(X): \quad \mu(A) = \sup\{\mu(K): K \subseteq A,\ K \text{ kompakt}\}. 
    \end{align*}
\end{corollary}

\begin{proof*}
    Sei $A \in \mathcal{B}(X)$ und $\varepsilon > 0$. Wegen der Straffheit von $\mu$ existiert eine kompakte Menge $K_{\varepsilon} \subseteq X$ mit $\mu(K_{\varepsilon}) \geq 1 - \frac{\varepsilon}{2}$,
    und da $\mu$ nach Proposition 1.4 regulär ist gibt es eine abgeschlossene Menge $C \subseteq A$ mit $\mu(C) > \mu(A) - \frac{\varepsilon}{2}$. Dann ist die Menge $K_{\varepsilon} \cap C$ wiederum kompakt und es gilt
    \begin{align*}
        \mu(A) \geq \mu(K_{\varepsilon} \cap C) > \mu(C) - \frac{\varepsilon}{2} > \mu(A) - \varepsilon. 
    \end{align*} 
    \qed
\end{proof*}

\begin{remark}
    Ein Wahrscheinlichkeitsmaß $\mu$ auf $\mathcal{B}(X)$ mit der Eigenschaft
    \begin{align*}
        \forall A \in \mathcal{B}(X): \quad \mu(A) = \sup\{\mu(K): K \subseteq A, \ K \text{ kompakt}\}. 
    \end{align*}
    wird auch als \textit{Radon-Wahrscheinlichkeitsmaß} oder \textit{Radon-Maß} bezeichnet.
\end{remark}

\begin{proposition}
    Sei $(X,d)$ ein vollständiger separabler metrischer Raum. Dann ist jedes Wahrscheinlichkeitsmaß $\mu$ auf $\mathcal{B}(X)$ straff.
\end{proposition}

Wir verwenden zum Beweis der Proposition die folgende Charakterisierung kompakter Teilmengen metrischer Räume. Ein Beweis findet sich etwa in \cite{amann}. 

\begin{lemma}
    Sei $(X,d)$ ein metrischer Raum. Eine Menge $K \subseteq X$ ist genau dann kompakt, wenn sie die folgenden beiden Eigenschaften erfüllt:
    \begin{enumerate}[(i)]
        \item $K$ ist vollständig,
        \item $K$ ist total-beschränkt, d.h.
        \begin{align*}
            \forall  \varepsilon > 0 \ \exists x_1,...,x_n \in K: \  K \subseteq \cup_{i=1}^n B(x_i, \varepsilon). 
        \end{align*} 
\end{enumerate}
\end{lemma}

\begin{proof*}
    \textbf{TODO}
    Sei $\varepsilon > 0$. Nach Voraussetzung existiert eine abzählbare dichte Teilmenge $D = \{x_1, x_2,...\}$ von $X$. Also gilt insbesondere für $q \in \N $
    \begin{align*}
        \bigcup_{i\in \N}\overline{B}(x_i, 2^{-q}) = X.
    \end{align*}
    Wegen der $\sigma$-Stetigkeit von $\mu$ existiert also ein $N_q \in \N$ mit 
    \begin{align*}
        \mu(\cup_{i=1}^{N_q}\overline{B}(x_i, 2^{-q}) \geq 1 - \varepsilon 2^{-q}. 
    \end{align*}
    Setze nun 
    \begin{align*}
        K := \bigcap_{q \in \N}\bigcup_{i=1}^{N_q}\overline{B}(x_i, 2^{-q}). 
    \end{align*}
    Dann ist $K$ als Schnitt abgeschlossener Teilmengen abgeschlossen, und da $X$ vollständig ist, folgt daraus bereits die Vollständigkeit von $K$. 
    Ferner ist $K$ total-beschränkt, denn zu $\varepsilon > 0$ existiert ein $q \in \N$ mit $2^{-q} < \varepsilon$ und $K \subseteq \cup_{i=1}^{N_q}B(x_i, 2^{-q}) \subseteq \cup_{i=1}^{N_q}B(x_i, \varepsilon)$. 
    Zudem gilt
    \begin{align*}
        \mu(K)  = 1 - \mu(\cup_{q \in \N}\cap_{i=1}^{N_q}\overline{B}(x_i, 2^{-q})^c) 
                &\geq 1 - \sum_{q=1}^{\infty} \mu(\cap_{i=1}^{N_q}\overline{B}(x_i, 2^{-q})^c) \\\
                &\geq 1 - \sum_{q=1}^{\infty} \varepsilon 2^{-q} = 1 - \varepsilon.
    \end{align*}
    Also ist $\mu$ straff. \qed
\end{proof*}

\begin{proposition}
    Sei $(X,d)$ ein vollständiger metrischer Raum und $\mu$ ein Wahrscheinlichkeitsmaß auf $\mathcal{B}(X)$. Dann sind äquivalent
    \begin{enumerate}[(i)]
        \item $\mu$ ist straff.
        \item Es gibt eine separable Teilmenge $E \subseteq X$ mit $\mu(E) = 1$. 
    \end{enumerate}
\end{proposition}
\begin{proof*}
    zu (i) $\Rightarrow$ (ii): Für alle $n \in \N$ existiert $K_n \subseteq X$ kompakt mit $\mu(K_n) \geq 1 - \frac{1}{n}$, o.E. gelte $K_n \subseteq K_{n+1}$. Es folgt 
    \begin{align*}
        \mu\big(\cup_{n=1}^{\infty}K_n\big) = \lim_{n \to \infty}\mu\big(K_{n+1}\big) = 1. 
    \end{align*}
    Da kompakte Teilmengen metrischer Räume insbesondere separabel sind, ist $E := \cup_{n=1}^{\infty}K_n$ als abzählbare Vereinigung separabler Mengen ebenso separabel. 
    \newline 
    zu (ii) $\Rightarrow$ (i): 
    Analog zum Beweis von Proposition 1.8. \qed
\end{proof*}
\section{Die Prokhorov Metrik}
Nachdem wir im letzten Abschnitt damit begonnen haben uns mit der schwachen Konvergenz von Wahrscheinlichkeitsmaßen zu beschäftigen, 
wollen wir nun ein weiteres Hilfsmittel zur Untersuchung von schwacher Konvergenz einführen. 
\newline 
Für einen metrischen Raum $(X,d)$ bezeichne $\mathcal{M}(X)$ die Menge aller Wahrscheinlichkeitsmaße auf der Borelschen $\sigma$-Algebra $\mathcal{B}(X)$. 
\begin{mydef}
    Eine Familie $M \subseteq \mathcal{M}(X)$ von Wahrscheinlichkeitsmaßen heißt \textit{gleichmäßig straff}, 
    falls es für alle $\varepsilon > 0$ eine kompakte Menge $K \subseteq X$ gibt mit 
    $$
        \forall \mu \in M: \mu(K) \geq 1-\varepsilon. 
    $$
\end{mydef}
Ziel dieses Abschnitts ist der Satz von Prokhorov, der uns eine für spätere Beweise wichtige Charakterisierung der gleichmäßigen Straffheit liefert. 

Ein wichtiges Resultat ist die Folgende auf Prokhorov zurückgehende Charakterisierung. Ein Beweis findet sich etwa in \cite{parthasarathy}[Theorem 6.7]. 

\begin{theorem}[Satz von Prokhorov]
    Sei $(X,d)$ ein vollständiger separabler metrischer Raum und $M \subseteq \mathcal{M}(X)$. Dann sind äquivalent
    \begin{enumerate}[(i)]
        \item $M$ ist relativ kompakt,
        \item $M$ ist gleichmäßig straff. 
    \end{enumerate}
\end{theorem}
\newpage
\section{Flache Konzentrierung von Wahrscheinlichkeitsmaßen}
Für unsere spätere Betrachtung von Zufallsvariablen mit Werten in einem Banachraum lohnt es sich hier, noch eine weitere Charakterisierung der relativen Kompaktheit von Familien von Wahrscheinlichkeitsmaßen zu diskutieren. 
Der Begriff der \textit{flachen Konzentrierung} von Wahrscheinlichkeitsmaßen wurde zuerst von A.D. de Acosta betrachtet, vgl. \cite{acosta}. 
Die Darstellung hier orientiert sich an \cite{vakhania}. 
\newline \ \newline 
\textbf{Notation und Konventionen} \newline 
Im Folgenden sei $(E, \norm{\cdot})$ ein separabler reeller Banachraum und $(E', \norm{\cdot}_{op})$ der zugehörige Dualraum. Der metrische Raum der Wahrscheinlichkeitsmaße auf $\mathcal{B}(E)$ werde mit $(\mathcal{M}(E), \rho)$ bezeichnet, wobei 
$\rho$ die durch $(1.6)$ definierte Prokhorov-Metrik ist. Für ein Maß $\mu$ auf $\mathcal{B}(E)$, einen messbaren Raum $(X, \mathcal{X})$ und eine messbare Abbildung $f:E \to X$ bezeichne $\mu^f$ das \textit{Bildmaß} von $\mu$ unter $f$. Dieses ist definiert als 
$$
    \mu^f(A) := \mu(f^{-1}(A)), \quad A \in \mathcal{X}. 
$$
Es sei ferner daran erinnert, dass jeder endlichdimensionale Untervektorraum  $S \subseteq E$ abgeschlossen ist und eine Menge $A \subseteq S$ nach dem Satz von Heine-Borel genau dann kompakt ist, wenn sie abgeschlossen und beschränkt ist. 
Insbesondere besitzt jede beschränkte Folge in $S$ eine konvergente Teilfolge. 
\begin{mydef}
    Eine Menge $M \subseteq \mathcal{M}(E)$ von Wahrscheinlichkeitsmaßen auf $\mathcal{B}(E)$ heißt \textit{flach konzentriert}, falls es für alle $\varepsilon > 0$ einen endlichdimensionalen 
    Untervektorraum $S \subseteq E$ gibt mit 
    $$
        \forall \mu \in M: \quad \mu(S^{\varepsilon}) \geq 1 - \varepsilon.
    $$ 
\end{mydef}

\begin{lemma}
    Eine Teilmenge $A$ von $E$ ist genau dann relativ kompakt, wenn $A$ beschränkt ist und es für alle $\varepsilon > 0$ einen endlichdimensionalen Untervektorraum $S \subseteq E$ gibt mit 
    \begin{align*}
        A \subseteq S^{\varepsilon}. 
    \end{align*}
\end{lemma}

\begin{proof*}
    Zu $\Rightarrow$: Sei $A \subseteq E$ relativ kompakt. Dann ist $\overline{A}$ kompakt und folglich beschränkt, woraus wir direkt die Beschränktheit von $A$ erhalten. 
    Ferner ist $\overline{A}$ separabel, also existiert eine abzählbare dichte Teilmenge $\{x_1, x_2,...\} \subseteq \overline{A}$. 
    Für $\varepsilon > 0$ ist $(B(x_n, \varepsilon))_{n \in \N}$ somit eine offene Überdeckung von $\overline{A}$. Wegen der Kompaktheit von $\overline{A}$ existiert $I \subseteq \N$ endlich mit 
    $$
        \overline{A} \subseteq \bigcup_{i \in I}B(x_i, \varepsilon).
    $$
    Sei daher $S$ der von $\{x_i : i \in I\}$ erzeugte endlichdimensionale Untervektorraum von $E$. Dann gilt 
    $$
        A \subseteq \overline{A} \subseteq \bigcup_{i \in I}B(x_i, \varepsilon) \subseteq S^{\varepsilon}.
    $$
    Zu $\Leftarrow$: Wir zeigen, dass jede Folge in $A$ eine konvergente Teilfolge besitzt, der Grenzwert der Teilfolge muss hierbei nicht in $A$ liegen. 
    Sei dazu $(x^{(0)}_n)_{n \in \N}$ eine Folge in $A$, $\varepsilon > 0$ und $S \subseteq E$ ein endlichdimensionaler Untervektorraum mit $A \subseteq S^{\varepsilon}$. 
    Dann existiert insbesondere eine Folge $(y_n)_{n \in \N}$ in $S$ mit 
    $$
        \forall n \in \N: \quad d(x^{(0)}_n, y_n) \leq \varepsilon.
    $$
    Aus der Beschränktheit von $(x^{(0)}_n)_{n \in \N}$ erhalten wir direkt die Beschränktheit von $(y_n)_{n \in \N}$ und da $S$ endlichdimensional ist, existiert eine konvergente Teilfolge $(y_{n_k})_{k \in \N}$.
    Es existiert demnach $N(\varepsilon) \in \N$, sodass für $k,m \geq N(\varepsilon) \in \N$
    \begin{align*}
        \norm{x^{(0)}_{n_k} - x^{(0)}_{n_m}} \leq \norm{x^{(0)}_{n_k} - y_{n_k}} + \norm{y_{n_k} - y_{n_m}} + \norm{x^{(0)}_{n_m} - y_{n_m}} \leq 3\varepsilon
    \end{align*}
    gilt. 
    Durch Entfernen endlich vieler Folgenglieder können wir ohne Einschränkung annehmen, dass 
    $$
        \forall k,m \in \N: \quad \norm{x^{(0)}_{n_k} - x^{(0)}_{n_m}} \leq 3\varepsilon. 
    $$
    Durch obiges Verfahren können wir für $N \in \N$ und $\varepsilon_N = \frac{1}{N}$ induktiv eine Teilfolge $(x^{(N)}_n)_{n \in \N}$ von $(x^{(N-1)}_n)_{n \in \N}$ gewinnen mit
    $$
        \forall m,n \in \N: \quad \norm{x^{(N)}_n - x^{(N)}_m} \leq \frac{3}{N}.
    $$
    Durch Bilden der Diagonalfolge $(x^{(N)}_N)_{N \in \N}$ erhalten wir somit eine Teilfolge der Ausgangsfolge $(x^{(0)}_n)_{n \in \N}$, die eine Cauchy-Folge ist und daher in $E$ konvergiert. 
    \qed 
\end{proof*}

\begin{lemma}
    Sei $S \subseteq E$ ein endlichdimensionaler Untervektorraum und $f_1,...,f_n \in E'$ Funktionale mit 
    \begin{align}
        \forall x,y \in S: \ x \neq y \ \Rightarrow \ \exists k \in \{1,...,n\}: \quad f_k(x) \neq f_k(y).
    \end{align}
    Dann ist die Menge 
    $$
        B := \overline{S^{\varepsilon}} \cap \{x \in E: \abs{f_1(x)} \leq r_1,...,\abs{f_n(x)}\leq r_n\}
    $$
    für alle $\varepsilon, r_1,...,r_n \in (0, \infty)$ beschränkt. 
\end{lemma}

\begin{proof*}
    Wegen $(1.12)$ definiert 
    $$
        p(x) := \max_{1\leq k \leq n} \abs{f_k(x)}, \quad x \in S,
    $$
    eine Norm auf $S$. Da $S$ endlichdimensional ist, ist diese insbesondere äquivalent zur Einschränkung von $\norm{\cdot}$ auf $S$. 
    Angenommen die Menge $B$ ist nicht beschränkt. Dann existiert eine Folge $(x_m)_{m \in \N}$ in $B$ mit 
    $$
        \lim_{m \to \infty} \norm{x_m} = \infty. 
    $$
    Wegen $B \subseteq \overline{S^{\varepsilon}}$ gibt es somit eine Folge $(y_m)_{m \in \N}$ in $S$ mit 
    $$
        \forall m \in \N: \quad \norm{x_m - y_m} \leq \varepsilon. 
    $$
    Mittels der Dreiecksungleichung erhält man daraus
    $$
        \lim_{m \to \infty}\norm{y_m} = \infty. 
    $$
    Wegen der Normäquivalenz existiert folglich ein $k \in \{1,...,n\}$ mit 
    \begin{align}
        \lim_{m \to \infty}\abs{f_k(y_m)} = \infty. 
    \end{align}
    Da $f_1,...,f_n$ stetig und linear sind, existiert ferner ein $K > 0$ mit 
    $$
        \forall j \in \{1,...,n\} \ \forall x,y \in E: \quad \abs{f_j(x)-f_j(y)} \leq K \norm{x-y}. 
    $$
    Nach Definition von $B$ gilt 
    $$
        \forall m \in \N: \quad \abs{f_k(x_m)} \leq r_k.
    $$
    Also folgt für alle $m \in \N$ aus der Dreiecksungleichung
    $$
        \abs{f_k(y_m)} \leq \abs{f_k(y_m) - f_k(x_m)} + \abs{f_k(x_m)} \leq K \varepsilon + r_k. 
    $$
    Im Widerspruch zu $(1.13)$. \qed
\end{proof*}

\begin{remark}%TODO: evtl. weiter nach oben verschieben und noch erklären wieso es für S endlich viele derartige Funktionale gibt. 
    Sei $S \subseteq E$ ein endlichdimensionaler Untervektorraum von $E$ mit $dim(S) = r$. 
    Man sagt eine Familie von Abbildungen mit der Eigenschaft $(1.12)$ \textit{trenne die Punkte von S}. 
    Ist $\{x_1,...,x_r\}$ eine Basis von $S$, so lässt sich jedes Element $x \in S$ eindeutig darstellen als $x = \sum_{i=1}^r \lambda_i(x) x_i$, wobei $\lambda_1(x),...,\lambda_r(x) \in \R$.
    Definiere nun für $k \in \{1,...,r\}$
    $$
        \tilde{f}_k: S \to \R, \quad x \mapsto \lambda_k(x).
    $$
    Dann sind $\tilde{f}_1,...,\tilde{f}_r$ offensichtlich linear und da $S$ endlichdimensional ist, folgt daraus bereits die Stetigkeit. 
    Ferner trennen $\tilde{f}_1,...,\tilde{f}_r$ die Punkte von $S$, da $\{x_1,...,x_r\}$ eine Basis von $S$ ist. Mit Satz $A.9$ erhalten wir nun Funktionale $f_1,...,f_r \in E'$ mit 
    $f_j|S = \tilde{f}_j$ für alle $j =1,...,r$. Da $\tilde{f}_1,...,\tilde{f}_r$ die Punkte von S trennen, gilt dies insbesondere auch für $f_1,...,f_r \in E'$. 
    Wir haben also gezeigt, dass es zu jedem endlichdimensionalen Untervektorraum $S$ von $E$ tatsächlich endlich viele stetige lineare Funktionale gibt, die die Punkte von $S$ trennen. 
\end{remark}

\begin{theorem}[Satz von de Acosta]
    Eine Menge $M \subseteq \mathcal{M}(E)$ von Wahrscheinlichkeitsmaßen ist genau dann relativ kompakt in $(\mathcal{M}(E), \rho)$, wenn die folgenden beiden Bedingungen erfüllt sind:
    \begin{enumerate}[(a)]
        \item Für alle $f \in E'$ ist $\{\mu^f : \mu \in M\} \subseteq \mathcal{M}(\R)$ relativ kompakt und
        \item $M$ ist flach konzentriert. 
    \end{enumerate}
\end{theorem}

\begin{proof*}
    Zu $\Rightarrow$: 
    Sei $M \subseteq \mathcal{M}(E)$ relativ kompakt. Nach Satz $1.27$ ist $M$ dann insbesondere gleichmäßig straff. 
    Daher existiert zu $\varepsilon > 0$ eine kompakte Menge $K \subseteq E$ mit 
    $$
        \forall \mu \in M: \quad \mu(K) \geq 1 - \varepsilon.   
    $$ 
    Da alle $f \in E'$ stetig sind, ist jeweils auch $f(K)$ kompakt und es gilt
    $$
        \forall \mu \in M: \quad \mu^f(f(K)) = \mu(f^{-1}(f(K))) \geq \mu(K) \geq 1 - \varepsilon. 
    $$
    Also ist $\{\mu^f : \mu \in M\}$ für alle $f \in E'$ gleichmäßig straff. Erneutes Anwenden von Satz $1.27$  liefert $(a)$. 
    Da $K$ insbesondere relativ kompakt ist, liefert Lemma $1.30$ einen endlichdimensionalen Untervektorraum $S \subseteq E$ mit $K \subseteq S^{\varepsilon}$. Somit gilt
    $$
        \forall \mu \in M: \quad \mu(S^{\varepsilon}) \geq \mu(K) \geq 1 - \varepsilon.
    $$
    Folglich ist $M$ flach konzentriert. 
    \newline 
    Zu $\Leftarrow$: 
    Sei $\varepsilon > 0$. Dann existiert für alle $n \in \N$ ein endlichdimensionaler Untervektorraum $S_{n, \varepsilon} \subseteq E$ mit 
    \begin{align}
        \forall \mu \in M: \quad \mu(S_{n, \varepsilon}^{\frac{\varepsilon}{2^{n+1}}}) \geq 1 - \frac{\varepsilon}{2^{n+1}}.
    \end{align}
    Nach Bemerkung $1.32$ existieren $f_1^{(n)},...,f_{k_n}^{(n)} \in E'$ mit 
    $$
        \forall x,y \in S_{n, \varepsilon}: \quad x \neq y \Rightarrow \ \exists j \in \{1,...,k_n\}: \quad f_j(x) \neq f_j(y). 
    $$
    Nach Voraussetzung $(a)$ können wir zudem $r_1^{(n)},...,r_{k_n}^{(n)} \in (0, \infty)$ so wählen, dass
    \begin{align}
        \inf_{\mu \in M} \mu^{f_i}\big([-r_i^{(n)}, r_i^{(n)}]\big) \geq 1 - \frac{\varepsilon}{k_n 2^{n+1}}, \quad i=1,...,k_n.
    \end{align}
    Setze ferner
    $$
        F_{n,\varepsilon} := \overline{\big(S_{n,\varepsilon}^{\frac{\varepsilon}{2^{n+1}}}\big)}.
    $$
    Dann ist die Menge 
    $$
        K := \bigcap_{n =1}^{\infty}\big(F_{n, \varepsilon} \cap\{x \in E: \ \abs{f_1^{(n)}}\leq r_1^{(n)},...,\abs{f_{k_n}^{(n)}}\leq r_{k_n}^{(n)}\}\big)
    $$
    nach Lemma $1.31$ beschränkt und als Schnitt abgeschlossener Mengen abgeschlossen. Ferner gilt für alle $n \in \N$
    $$
        K \subseteq  F_{n, \varepsilon} \subseteq S_{n,\varepsilon}^{\frac{\varepsilon}{2^n}}.
    $$
    Also ist $K$ nach Lemma $1.30$ kompakt. Zudem gilt für $\mu \in M$ nach $(1.14)$ und $(1.15)$
    \begin{align*}
        \mu(K^c) &\leq \sum_{n=1}^{\infty}\mu(F_{n, \varepsilon}^c) + \sum_{i=1}^{k_n}\mu^{f_i^{(n)}}\big([-r_i^{(n)}, r_i^{(n)}]^c\big) \\\
                 &\leq \sum_{n=1}^{\infty}\mu(F_{n, \varepsilon}^c) + \frac{\varepsilon}{2^{n+1}} \\\
                 &\leq \sum_{n=1}^{\infty}(\frac{\varepsilon}{2^{n+1}} + \frac{\varepsilon}{2^{n+1}}) = \varepsilon. 
    \end{align*}
    Folglich ist $M$ gleichmäßig straff und somit nach Satz $1.27$ relativ kompakt. \qed
\end{proof*}

\begin{corollary}
    Eine Menge $M \subseteq \mathcal{M}(E)$ ist genau dann relativ kompakt, wenn $M$ flach konzentriert ist und für alle $\varepsilon > 0$ eine beschränkte Menge $L \subseteq E$ existiert mit
    $$
        \forall \mu \in M: \quad \mu(L) \geq 1 - \varepsilon. 
    $$
\end{corollary}

\begin{proof*}
    Zu $\Rightarrow$: Nach Satz $1.27$ und Satz $1.33$ ist $M$ flach konzentriert und gleichmäßig straff. Somit existiert für jedes $\varepsilon > 0$ eine kompakte Menge $K \subseteq E$ mit 
    $$
        \forall \mu \in M: \quad \mu(K) \geq 1 - \varepsilon. 
    $$
    Als kompakte Menge ist $K$ insbesondere beschränkt und die Behauptung ist gezeigt.
    \newline  
    Zu $\Leftarrow$: Nach Voraussetzung ist $M$ flach konzentriert, es genügt daher nach Satz $1.33$ zu zeigen, dass $\{\mu^f : \ \mu \in M\}$ für alle $f \in E'$ relativ kompakt ist. Sei dazu $\varepsilon > 0$ und $f \in E'$. 
    Dann existiert eine beschränkte Teilmenge $L \subseteq E$ mit
    $$
        \forall \mu \in M: \quad \mu(L) \geq 1 - \varepsilon.
    $$
    Da $f$ ein stetiger linearer Operator ist, existiert ein $R > 0$ mit 
    $$
        \forall x \in E: \quad \abs{f(x)} \leq R \norm{x}. 
    $$
    Somit folgt aus der Beschränktheit von $L$, dass $f(L)$ eine beschränkte Teilmenge von $\R$ ist und nach dem Satz von Heine-Borel ist $\overline{f(L)}$ kompakt. Es gilt ferner
    $$
        \forall \mu \in M: \quad \mu^f(\overline{f(L)}) = \mu(\{x \in E: \ f(x) \in \overline{f(L)}\}) \geq \mu(L) \geq 1 - \varepsilon. 
    $$
    Also ist $\{\mu^f : \ \mu \in M\}$ gleichmäßig straff und nach Satz $1.27$ relativ kompakt. \qed
\end{proof*}


\section{Meßbare Vektorräume}
\begin{mydef}
    Sei $X$ ein Vektorraum und $\mathcal{C}$ eine $\sigma$-Algebra auf $X$. Das Tupel $(X, \mathcal{C})$ heißt \textit{messbarer Vektorraum}, falls
    \begin{enumerate}[(a)]
        \item Die Abbildung 
        \begin{align*}
            + : X \times X \to X, \quad (x,y) \mapsto x + y
        \end{align*}
        ist $\mathcal{A}\otimes \mathcal{C}/\mathcal{C}$-messbar, und
        \item die Abbildung 
        \begin{align*}
            \cdot : \R \times X \to X, \quad  (\alpha, x) \mapsto \alpha x
        \end{align*}
        ist $\mathcal{B}(\R) \otimes \mathcal{C}/\mathcal{C}$-messbar. 
    \end{enumerate}
\end{mydef}

\begin{remark}
    Sei $(X, \mathcal{C})$ ein messbarer Vektorraum. Dann gilt
    \begin{enumerate}[(i)]
        \item Für alle $\alpha \in \R$ ist die Abbildung 
            $$f_{\alpha}: X \to X, x \mapsto \alpha x$$
        $\mathcal{C}/\mathcal{C}$-messbar. 
        \item Für alle $y \in X$ ist die Abbildung 
            $$g_y: X \to X, x \mapsto x + y$$
        $\mathcal{C}/\mathcal{C}$-messbar.
    \end{enumerate}
\end{remark}

Da die Komposition messbarer Abbildungen wiederum messbar ist erhält man unmittelbar
\begin{proposition}
    Sei $(X, \mathcal{C})$ ein messbarer Vektorraum und $(\Omega, \mathcal{A})$ ein messbarer Raum. 
    Sind $X,Y: \Omega \to X$ zwei $\mathcal{A}/\mathcal{C}$-messbare Abbildungen und $\alpha, \beta \in \R$, so ist auch $\alpha X + \beta Y$ $\mathcal{A}/\mathcal{C}$-messbar. 
\end{proposition}

\begin{proof*}
    Man beachte, dass für beliebige  messbare Räume $(\Omega_1, \mathcal{A}_1), (\Omega_2, \mathcal{A}_2)$, Mengen $A \in \mathcal{A}_1 \otimes \mathcal{A}_2$ und $\omega_1 \in \Omega_1$
    \begin{align*}
    A(\omega_1) = \{ \omega_2 : (\omega_1,\omega_2) \in A \} \in \mathcal{A}_2
    \end{align*}
    gilt. \qed
\end{proof*}
\begin{proposition}
    Sei $X$ ein separabler Banachraum. Dann ist $(X, \mathcal{B}(X))$ ein messbarer Vektorraum.
\end{proposition}
\begin{proof*}
    Nach Proposition 1.2 gilt $\mathcal{B}(X \times X) = \mathcal{B}(X) \otimes \mathcal{B}(X)$ und 
    $\mathcal{B}(\R \times X) = \mathcal{B}(\R) \otimes \mathcal{B}(X)$. Ferner sind die Abbildungen 
    \begin{align*}
        + &: X \times X \to X, \quad (x,y) \mapsto x + y, \\\
        \cdot &: \R \times X \to X, \quad (\alpha, x) \mapsto \alpha  x
    \end{align*}
    stetig bzgl. der jeweiligen Produkttopologien und somit insbesondere $\mathcal{B}(X \times X)/\mathcal{B}(X)$- bzw. 
    $\mathcal{B}(\R \times X)/\mathcal{B}(X)$-messbar. \qed
\end{proof*}

\begin{example}
    Für $d \in \N$ ist $(\R^d, \mathcal{B}(\R^d))$ ein messbarer Vektorraum. 
\end{example}
Im Folgenden sei $(X, \norm{\cdot})$ ein Banachraum und $(X', \norm{\cdot}_{op})$ der zugehörige Dualraum. 
\begin{proposition}
    Sei $\emptyset \neq \Gamma \subseteq X'$. Dann ist $(X, \sigma({\Gamma}))$ ein messbarer Vektorraum. 
\end{proposition}

\begin{proof*}
    \textbf{TODO}
\end{proof*}

\begin{proposition}
    Sei $X$ ein separabler Banachraum. Dann gilt $\sigma(X') = \mathcal{B}(X)$. 
\end{proposition}
\section{Zufallsvariablen mit Werten in Banachräumen}
Sei $(\Omega, \mathcal{A}, P)$ ein vollständiger Wahrscheinlichkeitsraum und $E$ ein Banachraum mit Norm $\norm{\cdot}$. 
\begin{mydef}
    Eine Abbildung $X: \Omega \to E$ heißt \textit{(Radon-)Zufallsvariable} falls 
    \begin{enumerate}[(a)]
        \item $X$ ist $\mathcal{A}/\mathcal{B}(E)$-messbar und
        \item es existiert ein separabler Untervektorraum $E_0 \subseteq E$ mit $P(\{X \in E_0\}) = 1$. 
    \end{enumerate}
\end{mydef}
\textbf{TODO: Erklärung warum Radon, Rückgriff auf Abschnitt 1.2}
\begin{proposition}
    \textbf{TODO: Charakterisierung von Radon-Zufallsvariablen}
\end{proposition}
\textbf{TODO: Einführung einfache Funktionen im Banach-Kontext}
\begin{proposition}
    \textbf{TODO: Summen von Radon-Variablen sind messbar, Approximation durch einfache Funktionen, Grenzwerte sind messbar}
\end{proposition}

Bezeichne $\mathcal{L}_0(\Omega, \mathcal{A}, P; E)$ den Raum der $E$-wertigen Radon-Zufallsvariablen auf $(\Omega,\mathcal{A},P)$ und sei $L_0(E)$ der Raum der Äquivalenzklassen bezüglich fast sicherer Gleichheit. 
Falls klar ist welcher Wahrscheinlichkeitsraum gemeint ist, so schreiben wir auch $\mathcal{L}_0(E)$ oder $L_0(E)$. 






