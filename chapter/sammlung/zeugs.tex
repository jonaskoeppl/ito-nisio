Wie bereits aus dem skalaren Fall bekannt, lässt sich die fast sichere Konvergenz nicht durch eine Metrik beschreiben. Die stochastische Konvergenz allerdings schon. 
Definiere dazu eine Abbildung 
    $$d_P : L_0(E) \times L_0(E) \to [0,\infty)$$
 durch
\begin{align}
    d_P([X],[Y]) := \mathbb{E}\bigg(\frac{\norm{X-Y}}{1+\norm{X+Y}}\bigg), \quad [X], [Y] \in L_0(E).  
\end{align}
Dann ist $d_P$ wegen der Stetigkeit von $\norm{\cdot}$ und $\big\lvert\frac{x}{1+x}\big\rvert \leq 1$ für alle $x \in [0, \infty)$ wohldefiniert
und liefert eine weitere Charakterisierung der stochastischen Konvergenz. 
\begin{proposition}
    \begin{enumerate}[(i)]
        \item $d_P$ ist eine Metrik auf $L_0(E)$,
        \item $X_n \stochastisch X$ $\iff$ $\lim_{n \to \infty}d_P([X_n], [X]) = 0$,
        \item $(L_0(E), d_P)$ ist vollständig. 
    \end{enumerate}
\end{proposition}


\begin{mydef}%TODO: Wird das überhaupt gebraucht? Ggf schöner formulieren. 
    Eine Folge $(X_n)_{n \in \N}$ von E-wertigen Zufallsvariablen heißt \textit{symmetrisch}, 
    falls $(\varepsilon_1 X_1, \varepsilon_2 X_2,...)$ für jede Folge $(\varepsilon_i)_{i \in \N} \in \{-1,1\}^{\N}$ die gleiche Verteilung wie $(X_1,X_2,...)$ auf dem Produktraum $(E^{\N}, \otimes_{i \in \N}\mathcal{B}(E))$ hat. 
\end{mydef}

\begin{remark}
   Sind $X_1,X_2,...$ unabhängige E-wertige Zufallsvariablen, sodass $X_n$ für alle $n \in \N$ symmetrisch ist, dann ist $(X_n)_{n \in \N}$ symmetrisch. 
\end{remark}