\section{Zufallsvariablen mit Werten in Banachräumen}
Sei $(\Omega, \mathcal{A}, P)$ ein vollständiger Wahrscheinlichkeitsraum und $E$ ein Banachraum mit Norm $\norm{\cdot}$. 
\begin{mydef}
    Eine Abbildung $X: \Omega \to E$ heißt \textit{(Radon-)Zufallsvariable} falls 
    \begin{enumerate}[(a)]
        \item $X$ ist $\mathcal{A}/\mathcal{B}(E)$-messbar und
        \item es existiert ein separabler Untervektorraum $E_0 \subseteq E$ mit $P(\{X \in E_0\}) = 1$. 
    \end{enumerate}
\end{mydef}

\begin{remark}
    Nach Abschnitt $1.2$ sind für eine $\mathcal{A}/\mathcal{B}(X)$-messbare Abbildung $X: \Omega \to E$ äquivalent
    \begin{enumerate}[(i)]
        \item es existiert ein separabler Untervektorraum $E_0 \subseteq E$ mit $P(\{X \in E_0\}) = 1$,
        \item $P^X$ ist ein Radon-Maß auf $\mathcal{B}(X)$, 
        \item $P^X$ ist straff. 
    \end{enumerate}
    Hiermit erklärt sich auch die Bezeichnung Radon-Zufallsvariable. 
\end{remark}

\textbf{TODO: Einführung einfache Funktionen im Banach-Kontext}
\begin{proposition}
    \textbf{TODO: Summen von Radon-Variablen sind messbar, Approximation durch einfache Funktionen, Grenzwerte sind messbar}
\end{proposition}

Bezeichne $\mathcal{L}_0(\Omega, \mathcal{A}, P; E)$ den Raum der $E$-wertigen Radon-Zufallsvariablen auf $(\Omega,\mathcal{A},P)$ und sei $L_0(\Omega, \mathcal{A}, P;E)$ der Raum der Äquivalenzklassen bezüglich fast sicherer Gleichheit. 
Falls klar ist welcher Wahrscheinlichkeitsraum gemeint ist, so schreiben wir auch $\mathcal{L}_0(E)$ oder $L_0(E)$. 

