\section{Zufallsvariablen mit Werten in Banachräumen}
\textbf{Notation und Konventionen} \newline
Sei $(\Omega, \mathcal{A}, P)$ ein vollständiger Wahrscheinlichkeitsraum und $E$ ein Banachraum mit Norm $\norm{\cdot}$. 
\begin{mydef}
    Eine Abbildung $X: \Omega \to E$ heißt \textit{(E-wertige) Radon-Zufallsvariable}, falls sie die folgenden beiden Bedingungen erfüllt: 
    \begin{enumerate}[(a)]
        \item $X$ ist $\mathcal{A}/\mathcal{B}(E)$-messbar und
        \item es existiert ein separabler Untervektorraum $E_0 \subseteq E$ mit $P(\{X \in E_0\}) = 1$. 
    \end{enumerate}
\end{mydef}

\begin{remark}
    Nach Bemerkung $1.18$ sind für eine $\mathcal{A}/\mathcal{B}(X)$-messbare Abbildung $X: \Omega \to E$ äquivalent:
    \begin{enumerate}[(i)]
        \item Es existiert ein abgeschlossener separabler Untervektorraum $E_0 \subseteq E$ mit $P(\{X \in E_0\}) = 1$.
        \item $P^X$ ist ein Radon-Maß auf $\mathcal{B}(X)$. 
        \item $P^X$ ist straff. 
    \end{enumerate}
    Hiermit erklärt sich auch die aus \cite{ledoux-talagrand} stammende Bezeichnung Radon-Zufallsvariable. 
    Ist ferner $(X_n)_{n \N}$ eine Folge von Radon-Zufallsvariablen mit Werten in $E$, so dass die zugehörige Folge der Verteilungen $(P^{X_n})_{n \in \N}$ gleichmäßig straff ist, dann nennen wir $(X_n)_{n \in \N}$ gleichmäßig straff. 
\end{remark}

Bezeichne $\mathcal{L}_0(\Omega, \mathcal{A}, P; E)$ den Raum der $E$-wertigen Radon-Zufallsvariablen auf $(\Omega,\mathcal{A},P)$ und sei $L_0(\Omega, \mathcal{A}, P;E)$ der Raum der Äquivalenzklassen bezüglich fast sicherer Gleichheit. 
Falls klar ist welcher Wahrscheinlichkeitsraum gemeint ist, so schreiben wir auch $\mathcal{L}_0(E)$ bzw. $L_0(E)$. 

\begin{proposition}
   Für alle $X,Y \in \mathcal{L}_0(E)$ und $\alpha, \beta \in \R$ gilt $\alpha X + \beta Y \in \mathcal{L}_0(E)$.
\end{proposition}

\begin{proof*}
    Da $X,Y$ Radon-Zufallsvariablen sind, existieren zwei abgeschlossene separable Untervektorräume $E_X$ und $E_Y$ mit 
    $$
        \prob{X \in E_X} = \prob{Y \in E_Y} = 1. 
    $$
    Nach Korollar $A.4$ ist dann auch $E_0 := \overline{lin(E_X \cup E_Y)}$ separabel und es gilt 
    $$
        \prob{X \in E_0} = \prob{Y \in E_0} = 1. 
    $$
    Da $\mathcal{A}$ vollständig ist, können wir also ohne Beschränkung der Allgemeinheit annehmen, dass $E$ selbst separabel ist. 
    Nach Proposition $1.38$ ist dann $(E, \mathcal{B}(E))$ ein messbarer Vektorraum und die Behauptung folgt nun aus Proposition $1.37$. \qed 
\end{proof*}

Wie im skalaren Fall lassen sich Radon-Zufallsvariablen durch sogenannte \textit{einfache Zufallsvariablen} approximieren. 

\begin{mydef}
    Eine Abbildung $X:\Omega \to E$ von der Form 
    $$
        X = \sum_{i=1}^n 1_{A_i}x_i
    $$
    mit $x_1,...,x_n \in E$ und $A_1,..., A_n \in \mathcal{A}$ heißt \textit{einfache Zufallsvariable}. Offensichtlich gilt $X \in \mathcal{L}_0(E)$. 
\end{mydef}


\begin{proposition}
    Sei $X \in \mathcal{L}_0(E)$ und $Y: \Omega \to E$ eine Abbildung. 
    \begin{enumerate}[(i)]
        \item Dann existiert eine Folge einfacher Zufallsvariablen $(X_n)_{n \in \N}$ mit $X_n \fastsicher X$ und 
        $$
            \norm{X_n(\omega)} \leq 2 \norm{X(\omega)}
        $$
        für alle $n \in \N$ und fast alle $\omega \in \Omega$. 
        \item Sei $(Y_n)_{n \in \N}$ eine Folge in $\mathcal{L}_0(E)$, sodass eine Menge $\Omega^*$ existiert mit 
        $$
            \forall \omega \in \Omega^*: \quad \lim_{n \to \infty}Y_n(\omega) = Y(\omega). 
        $$
        Dann gilt $Y \in \mathcal{L}_0(E)$. 
    \end{enumerate}
\end{proposition}

\begin{proof*}
    Zu $(i)$:
    Da $X$ eine Radon-Zufallsvariable ist, existiert ein abgeschlossener separabler Untervektorraum $E_0 \subseteq E$ mit $\prob{X \in E_0} = 1$. 
    Da $\mathcal{A}$ vollständig ist können wir daher ohne Einschränkung annehmen, dass $E$ selbst bereits separabel ist. 
    Sei $\{x_1,x_2,... \} \subseteq E$ eine dichte Teilmenge. Für $n \in \N$ betrachte die Abbildung
    $$
        T_n: \Omega \to \N, \quad \omega \mapsto \inf\{k \leq n: \norm{X(\omega)-x_k} = \min_{1\leq l \leq n}\norm{X(\omega)-x_l}\}.
    $$ 
    Aus der Messbarkeit von $X$ und der Stetigkeit von $\norm{\cdot - x}$ für $x \in E$ folgt direkt die Messbarkeit von $T_n$,
    da für alle $i,j \in \{1,...,n\}$ die Menge $\{\norm{X - x_i} < \norm{X - x_j}\}$ messbar ist. Setze nun
    $$
        X_n := \sum_{k=1}^n 1_{\{T_n = k\}}x_k, \quad n \in \N.
    $$
    Dann ist $(X_n)_{n \in \N}$ eine Folge einfacher Zufallsvariablen und es gilt für $n \in \N$ und $\omega \in \Omega$
    $$
        \norm{X(\omega)-X_n(\omega)} = \min_{1\leq k \leq n}\norm{X(\omega)-x_k}.
    $$
    Da $\{x_1,x_2,...\}$ dicht in $E$ liegt, folgt für alle $\omega \in \Omega$
    $$
        \lim_{n \to \infty}\norm{X(\omega) - X_n(\omega)} = \lim_{n \to \infty}\min_{1\leq k \leq n}\norm{X(\omega)- x_k} = \inf_{n \in \N} \norm{X(\omega)- x_n} = 0. 
    $$
    Zu $(ii)$: Da $Y_n$ für jedes $n \in \N$ eine Radon-Zufallsvariable ist, existiert eine Folge $(E_n)_{n \in \N}$ von abgeschlossenen separablen Untervektorräumen von $E$ mit 
    $$
        \forall n \in \N: \quad \prob{Y_n \in E_n} = 1. 
    $$
    Nach Korollar $A.4$ ist $E_0 := \overline{lin(\cup_{n \in \N}E_n)}$ ein abgeschlossener separabler Untervektorraum von $E$ und es gilt  
    $$
        \forall n \in \N: \quad \prob{Y_n \in E_0} = 1. 
    $$
    Da $\mathcal{A}$ vollständig ist, können wir daher ohne Einschränkung annehmen, dass $E$ separabel ist. 
    Ferner können wir wegen der Vollständigkeit von $\mathcal{A}$ annehmen, dass $\lim_{n \to \infty} Y_n(\omega) = Y(\omega)$ für alle $\omega \in \Omega$. 
    Wir zeigen, dass für jede abgeschlossene Menge $A \subseteq E$ das Urbild $Y^{-1}(A)$ in $\mathcal{A}$ liegt. 
    Da $\mathcal{B}(E)$ von den abgeschlossenen Mengen erzeugt wird folgt daraus die Behauptung. 
    Für eine abgeschlossene Menge $\emptyset \neq A \subseteq E$ betrachte für $k \in \N$ die offene, also auch messbare, Menge 
    $$
        A_k := \{x \in E: \inf_{y\in A}\norm{x-y} < \frac{1}{k}\}.
    $$
    Da $A$ abgeschlossen ist gilt 
    \begin{align*}
        \{Y \in A\} = \bigcap_{k=1}^{\infty}\liminf_{n \to \infty}\{Y_n \in A_k\} = \bigcap_{k=1}^{\infty}\bigcup_{m=1}^{\infty}\bigcap_{n=m}^{\infty}\{Y_n \in A_k\}\in \mathcal{A}.
    \end{align*}     
    \qed 
\end{proof*}

\begin{remark}
    Neben dem Begriff der Radon-Zufallsvariable findet man in der Literatur auch den der \textit{starken Messbarkeit}.
    Eine Abbildung $f:\Omega \to E$ heißt \textit{stark messbar}, falls es eine Folge $(f_n)_{n \in \N}$ einfacher Zufallsvariablen gibt, die punktweise gegen $f$ konvergiert. 
    Proposition $1.46$ zeigt, dass für eine Abbildung $X: \Omega \to E$ die folgenden beiden Bedingungen äquivalent sind:
    \begin{enumerate}[(i)]
        \item $X$ ist eine Radon-Zufallsvariable. 
        \item Es existiert eine stark messbare Abbildung $f:\Omega \to E$ mit $\prob{X=f} = 1$. 
    \end{enumerate}
    Das Konzept der starken Messbarkeit und die Approximation durch einfache Zufallsvariablen sind insbesondere für die Definition des Bochner-Integrals nützlich. 
    Eine ausführliche Konstruktion des Bochner-Integrals findet sich etwa in \cite{van-neerven1}. 
\end{remark}


