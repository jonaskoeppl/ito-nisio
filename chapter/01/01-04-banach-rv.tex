\section{Zufallsvariablen mit Werten in Banachräumen}
Sei $(\Omega, \mathcal{A}, P)$ ein vollständiger Wahrscheinlichkeitsraum und $E$ ein Banachraum mit Norm $\norm{\cdot}$. 
\begin{mydef}
    Eine Abbildung $X: \Omega \to E$ heißt \textit{(Radon-)Zufallsvariable} oder auch \textit{E-wertige Zufallsvariable}, falls 
    \begin{enumerate}[(a)]
        \item $X$ ist $\mathcal{A}/\mathcal{B}(E)$-messbar und
        \item es existiert ein separabler Untervektorraum $E_0 \subseteq E$ mit $P(\{X \in E_0\}) = 1$. 
    \end{enumerate}
\end{mydef}

\begin{remark}
    Nach Abschnitt $1.2$ sind für eine $\mathcal{A}/\mathcal{B}(X)$-messbare Abbildung $X: \Omega \to E$ äquivalent
    \begin{enumerate}[(i)]
        \item Es existiert ein separabler Untervektorraum $E_0 \subseteq E$ mit $P(\{X \in E_0\}) = 1$,
        \item $P^X$ ist ein Radon-Maß auf $\mathcal{B}(X)$, 
        \item $P^X$ ist straff. 
    \end{enumerate}
    Hiermit erklärt sich auch die in \cite{ledoux-talagrand} verwendete Bezeichnung Radon-Zufallsvariable. 
\end{remark}

Bezeichne $\mathcal{L}_0(\Omega, \mathcal{A}, P; E)$ den Raum der $E$-wertigen Radon-Zufallsvariablen auf $(\Omega,\mathcal{A},P)$ und sei $L_0(\Omega, \mathcal{A}, P;E)$ der Raum der Äquivalenzklassen bezüglich fast sicherer Gleichheit. 
Falls klar ist welcher Wahrscheinlichkeitsraum gemeint ist, so schreiben wir auch $\mathcal{L}_0(E)$ bzw. $L_0(E)$. 

\begin{proposition}
   Für alle $X,Y \in \mathcal{L}_0(E)$ und $\alpha, \beta \in \R$ gilt $\alpha X + \beta Y \in \mathcal{L}_0(E)$.
\end{proposition}

\begin{proof*}
    Da $\mathcal{A}$ vollständig ist und $X,Y$ fast sicher Werte in einem separablen Unterraum von $E$ annehmen, sei $E$ ohne Beschränkung der Allgemeinheit separabel. 
    Nach Proposition $1.30$ ist dann $(E, \mathcal{B}(E))$ ein messbarer Vektorraum und die Behauptung folgt nun aus Proposition $1.29$. \qed 
\end{proof*}

Wie auch im skalaren Fall lassen sich Radon-Zufallsvariablen durch sogenannte \textit{einfache Zufallsvariablen} approximieren. 

\begin{mydef}
    Eine Abbildung $X:\Omega \to E$ von der Form 
    $$
        X = \sum_{i=1}^n 1_{A_i}x_i
    $$
    mit $x_1,...,x_n \in E$ und $A_1,..., A_n \in \mathcal{A}$ heißt \textit{einfache Zufallsvariable}. Offensichtlich gilt $X \in \mathcal{L}_0(E)$. 
\end{mydef}


\begin{proposition}
    Sei $X \in \mathcal{L}_0(E)$ und $Y: \Omega \to E$ eine Abbildung. 
    \begin{enumerate}[(i)]
        \item Es existiert eine Folge $(X_n)_{n \in \N}$ mit $X_n \fastsicher X$ und 
        $$
            \norm{X_n(\omega)} \leq 2 \norm{X(\omega)}
        $$
        für alle $n \in \N$ und fast alle $\omega \in \Omega$. 
        \item Sei $(Y_n)_{n \in \N}$ eine Folge in $\mathcal{L}_0(E)$ sodass eine Menge $\Omega^*$ existiert mit 
        $$
            \forall \omega \in \Omega^*: \quad \lim_{n \to \infty}Y_n(\omega) = Y(\omega). 
        $$
        Dann gilt $Y \in \mathcal{L}_0(E)$. 
    \end{enumerate}
\end{proposition}

\begin{proof*}
    Wegen der Vollständigkeit von $\mathcal{A}$ und da das lineare Erzeugnis abzählbar vieler separabler Unterräume wieder ein separabler Unterraum ist können wir erneut ohne Einschränkung annehmen, dass $E$ selbst bereits separabel ist.
    \newline 
    zu $(i)$:
    Sei $\{x_1,x_2,... \} \subseteq E$ eine dichte Teilmenge. Für $n \in \N$ betrachte die Abbildung
    $$
        T_n: \Omega \to \N, \quad \omega \mapsto \inf\{k \leq n: \norm{X(\omega)-x_k} = \min_{1\leq l \leq n}\norm{X(\omega)-x_l}\}.
    $$ 
    Aus der Messbarkeit von $X$ und der Stetigkeit von $\norm{\cdot - x}$ für $x \in E$ folgt direkt die Messbarkeit von $T_n$,
    da für alle $i,j \in \{1,..,n\}$ die Menge $\{\norm{X - x_i} < \norm{X - x_j}\}$ messbar ist. Setze nun
    $$
        X_n := \sum_{k=1}^n 1_{\{T_n = k\}}x_k, \quad n \in \N.
    $$
    Dann ist $(X_n)_{n \in \N}$ eine Folge einfacher Zufallsvariablen und es gilt für $n \in \N$ und $\omega \in \Omega$
    $$
        \norm{X(\omega)-X_n(\omega)} = \min_{1\leq k \leq n}\norm{X(\omega)-x_k}.
    $$
    Da $\{x_1,x_2,...\}$ dicht in $E$ liegt folgt 
    $$
        \lim_{n \to \infty}{X(\omega) - X_n(\omega)} = \lim_{n \to \infty}\min_{1\leq k \leq n}\norm{X(\omega)- x_k} = \inf_{n \in \N} \norm{X(\omega)- x_n} = 0. 
    $$
    zu $(ii)$: Wegen der Vollständigkeit von $\mathcal{A}$ können wir annehmen, dass $X_n(\omega) \to Y(\omega)$ für alle $\omega \in \Omega$. 
    Wir zeigen, dass für jede abgeschlossene Menge $A \subseteq E$ das Urbild $Y^{-1}(A)$ in $\mathcal{A}$ liegt. 
    Da $\mathcal{B}(E)$ von den abgeschlossenen Mengen erzeugt wird folgt daraus die Behauptung. 
    Für $\emptyset \neq A \subseteq E$ abgeschlossen betrachte für $k \in \N$ die offene, also auch messbare, Menge 
    $$
        A_k := \{x \in E: \inf_{y\in A}\norm{x-y} < \frac{1}{k}\}.
    $$
    Dann gilt wie man leicht prüft
    \begin{align*}
        \{Y \in A\} = \bigcap_{k=1}^{\infty}\liminf_{n \to \infty}\{X_n \in A_k\} \in \mathcal{A}.
    \end{align*}     
    \qed 
\end{proof*}


