Bevor wir uns im späteren Verlauf der Arbeit mit zufälligen Reihen in Banachräumen beschäftigen können benötigen wir ein paar maßtheoretische Vorbereitungen. 
Wir beginnen mit einigen grundlegenden Eigenschaften Borelscher $\sigma$-Algebren und darauf definierten Wahrscheinlichkeitsmaßen. 
Später gehen wir kurz auf messbare Vektorräume ein und führen dann den Begriff der Zufallsvariable mit Werten in einem Banachraum ein. 
Da die zusätzliche algebraische Struktur eines Banachraums für unsere Betrachtung zunächst nicht von Bedeutung ist, 
werden wir uns in den ersten Abschnitten mit dem allgemeineren Fall eines (vollständigen) metrischen Raumes beschäftigen.
\textbf{TODO}