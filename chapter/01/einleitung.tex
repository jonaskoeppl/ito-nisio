\begin{abstract}
    \thispagestyle{plain}
    \addcontentsline{toc}{chapter}{Zusammenfassung}
    Sei $(X_n)_{n \in \N}$ eine Folge unabhängiger reellwertiger Zufallsvariablen und bezeichne
    $$
        S_n := \sum_{i=1}^nX_i, \quad n \in \N,
    $$
    die Partialsummen. 
    Nach einem auf P. Lévy \cite{levy} zurückgehenden Satz sind für die Folge $(S_n)_{n \in \N}$ die folgenden Aussagen äquivalent:
    \begin{enumerate}[(i)]
        \item $(S_n)_{n \in \N}$ konvergiert fast sicher.
        \item $(S_n)_{n \in \N}$ konvergiert stochastisch.
        \item $(S_n)_{n \in \N}$ konvergiert in Verteilung.
    \end{enumerate}
    Im Jahr $1968$ gelang es K.Itô und M. Nisio in \cite{ito-nisio}, dieses Resultat auf Folgen von Radon-Zufallsvariablen mit Werten in einem beliebigen separablen Banachraum zu verallgemeinern und im Fall unabhängiger symmetrischer Zufallsvariablen um weitere äquivalente Eigenschaften zu erweitern.
    Ziel dieser Arbeit ist es, zunächst eine (größtenteils) in sich geschlossene Einführung in die für den Beweis benötigten Hilfsmittel zu geben, und anschließend den Satz von Itô-Nisio zu beweisen. 
    Im Anschluss an den zentralen Beweis werden zudem zwei leichte, aber interessante Anwendungen diskutiert. 
    Neben den Inhalten der Grundvorlesungen zur Linearen Algebra und Analysis werden Grundlagen der maßtheoretischen Wahrscheinlichkeitstheorie im Umfang von \cite[Kapitel I]{gs} als Vorkenntnisse angenommen.
\end{abstract}
