\section{Messbare Vektorräume}
Bislang haben wir uns fast ausschließlich mit dem Zusammenspiel von Maßen und den topologischen Eigenschaften der zugrunde liegenden Räume beschäftigt.
In Banachräumen steht uns aber auch die algebraische Struktur eines Vektorraums zur Verfügung, allerdings ist per se nicht klar, ob die algebraischen Operationen mit der messbaren Struktur kompatibel, also messbar, sind. 
Diese Überlegung führt direkt zur Definition eines \textit{messbaren Vektorraums}. 

\begin{mydef}
    Sei $X$ ein Vektorraum und $\mathcal{C}$ eine $\sigma$-Algebra auf $X$. Das Tupel $(X, \mathcal{C})$ heißt \textit{messbarer Vektorraum}, falls die folgenden beiden Bedingungen erfüllt sind: 
    \begin{enumerate}[(a)]
        \item Die Abbildung 
        \begin{align*}
            + : X \times X \to X, \quad (x,y) \mapsto x + y
        \end{align*}
        ist $\mathcal{C}\otimes \mathcal{C}/\mathcal{C}$-messbar und
        \item die Abbildung 
        \begin{align*}
            \cdot : \R \times X \to X, \quad  (\alpha, x) \mapsto \alpha x
        \end{align*}
        ist $\mathcal{B}(\R) \otimes \mathcal{C}/\mathcal{C}$-messbar. 
    \end{enumerate}
\end{mydef}

\begin{remark}
    Sei $(X, \mathcal{C})$ ein messbarer Vektorraum. Dann gilt:
    \begin{enumerate}[(i)]
        \item Für alle $\alpha \in \R$ ist die Abbildung 
            $$f_{\alpha}: X \to X, \quad x \mapsto \alpha x$$
        $\mathcal{C}/\mathcal{C}$-messbar. 
        \item Für alle $y \in X$ ist die Abbildung 
            $$g_y: X \to X, \quad x \mapsto x + y$$
        $\mathcal{C}/\mathcal{C}$-messbar.
    \end{enumerate}
\end{remark}
\begin{proof*}
    Aus der Messbarkeit der Skalarmultiplikation und Vektoraddition  folgt zunächst
    \begin{align*}
        \forall A \in \mathcal{C}: \quad &A^{(1)} := \{(\alpha, x) \in \R \times X: \ \alpha x \in A\} \in \mathcal{B}(\R) \otimes \mathcal{C}, \\\
                                         &A^{(2)} := \{(x,y) \in X \times X: \ x + y \in A\} \in \mathcal{C} \otimes \mathcal{C}. 
    \end{align*}
    Man beachte nun, dass für beliebige  messbare Räume $(\Omega_1, \mathcal{A}_1), (\Omega_2, \mathcal{A}_2)$, Mengen $A \in \mathcal{A}_1 \otimes \mathcal{A}_2$ und $\omega_1 \in \Omega_1$ 
    \begin{align*}
    A(\omega_1) = \{ \omega_2 \in \Omega_2 : (\omega_1,\omega_2) \in A \} \in \mathcal{A}_2
    \end{align*}
    gilt. In unserem Fall erhalten wir für festes $\alpha \in R$ und $y \in X$
    \begin{align*}
        &f_{\alpha}^{-1}(A) = \{x \in X: \ \alpha x \in A\} = A^{(1)}(\alpha) \in \mathcal{C},  \\\
        &g_y^{-1}(A) = \{ x \in X: \ x + y \in A\} = A^{(2)}(y)  \in \mathcal{C}. 
    \end{align*}
    \qed
\end{proof*}
Da die Komposition messbarer Abbildungen wiederum messbar ist, erhält man unmittelbar
\begin{proposition}
    Sei $(X, \mathcal{C})$ ein messbarer Vektorraum und $(\Omega, \mathcal{A})$ ein messbarer Raum. 
    Sind $X,Y: \Omega \to X$ zwei $\mathcal{A}/\mathcal{C}$-messbare Abbildungen und $\alpha, \beta \in \R$, so ist auch $\alpha X + \beta Y$ $\mathcal{A}/\mathcal{C}$-messbar. 
\end{proposition}

Bislang wissen wir noch nicht einmal, ob es überhaupt nicht-triviale Beispiele messbarer Vektorräume gibt. Das wollen wir nun ändern.  
\begin{proposition}
    Sei $X$ ein separabler Banachraum. Dann ist $(X, \mathcal{B}(X))$ ein messbarer Vektorraum.
\end{proposition}
\begin{proof*}
    Nach Proposition $1.7$ gilt $\mathcal{B}(X \times X) = \mathcal{B}(X) \otimes \mathcal{B}(X)$ und 
    $\mathcal{B}(\R \times X) = \mathcal{B}(\R) \otimes \mathcal{B}(X)$. Ferner sind die Abbildungen 
    \begin{align*}
        + &: X \times X \to X, \quad (x,y) \mapsto x + y, \\\
        \cdot &: \R \times X \to X, \quad (\alpha, x) \mapsto \alpha  x
    \end{align*}
    stetig bzgl. der jeweiligen Produkttopologien und somit insbesondere $\mathcal{B}(X \times X)/\mathcal{B}(X)$- bzw. 
    $\mathcal{B}(\R \times X)/\mathcal{B}(X)$-messbar. \qed
\end{proof*}

\begin{example}
    Für $d \in \N$ ist $(\R^d, \mathcal{B}(\R^d))$ ein messbarer Vektorraum. 
\end{example}

Im Folgenden bezeichne $(X', \norm{\cdot}_{op})$ den Dualraum eines normierten Vektorraums $(X, \norm{\cdot})$.

\begin{proposition}
    Sei $\emptyset \neq \Gamma \subseteq X'$. Dann ist $(X, \sigma({\Gamma}))$ ein messbarer Vektorraum. 
\end{proposition}

\begin{proof*}%TODO: überarbeiten und schöneren Beweis finden. 
    Zur Messbarkeit der Addition: Es genügt zu zeigen, dass 
    $$
        \forall f \in \Gamma : \big(g: X \times X \to \R, \ (x,y) \mapsto f(x+y)\big) \text{ ist messbar.}
    $$
    Sei dazu $f \in \Gamma$. Wegen der Linearität von $f$ gilt für $(x,y) \in X \times X$
    $$
        f(x + y) = (f(x) + f(y)).
    $$
    Weiter ist die Abbildung 
    $$
        h: X \times X \to \R, (x,y) \mapsto f(x) + f(y)
    $$
    als Komposition der messbaren Funktionen 
    \begin{align*}
        &h_1: X \times X \to \R \times \R, (x,y) \mapsto (f(x),f(y)), \\\
        &h_2: \R \times \R \to \R, (x,y) \mapsto x+y
    \end{align*}
    messbar. Die Abbildung $h_2$ ist messbar, weil $(\R, \mathcal{B}(\R))$ nach Beispiel $1.39$ ein messbarer Vektorraum ist. 
    Die Messbarkeit der Skalarmultiplikation zeigt man ähnlich. \qed 
\end{proof*}

\begin{proposition}
    Sei $E$ ein separabler Banachraum. Dann gilt $\sigma(E') = \mathcal{B}(E)$. 
\end{proposition}

\begin{proof*}%TODO: Überprüfen. Reicht das wirklich schon so?
    Da alle $f \in E'$ stetig sind, gilt offensichtlich $\sigma(E') \subseteq \mathcal{B}(E)$. 
    Wegen der Separabilität von $E$ wird $\mathcal{B}(E)$ nach Proposition $1.1$ von den abgeschlossenen Kugeln erzeugt und weil $(E, \sigma(E'))$ nach Proposition $1.40$ ein messbarer Vektorraum ist, genügt es nach Bemerkung $1.36$ zu zeigen, 
    dass $\overline{B}(0,1)$ in $\sigma(E')$ enthalten ist. Da $E$ separabel ist, existiert nach Korollar $A.12$ eine Folge $(f_n)_{n \in \N}$ in $E'$ mit
    $$
        \forall x \in E: \quad \norm{x} = \sup_{n \in \N}\abs{f_n(x)}.
    $$
    Es gilt also
    $$
        \overline{B}(0,1) = \{ x \in E: \norm{x} \leq 1 \} = \{x \in E: \ \sup_{n \in \N}\abs{f_n(x)} \leq 1 \} = \bigcap_{n=1}^{\infty}\{x \in E: \abs{f_n(x)} \leq 1\} \in \sigma(E').
    $$
    \qed 
\end{proof*}
