\section{Meßbare Vektorräume}
\begin{mydef}
    Sei $X$ ein Vektorraum und $\mathcal{C}$ eine $\sigma$-Algebra auf $X$. Das Tupel $(X, \mathcal{C})$ heißt \textit{messbarer Vektorraum}, falls
    \begin{enumerate}[(a)]
        \item Die Abbildung 
        \begin{align*}
            + : X \times X \to X, \quad (x,y) \mapsto x + y
        \end{align*}
        ist $\mathcal{A}\otimes \mathcal{C}/\mathcal{C}$-messbar, und
        \item die Abbildung 
        \begin{align*}
            \cdot : \R \times X \to X, \quad  (\alpha, x) \mapsto \alpha x
        \end{align*}
        ist $\mathcal{B}(\R) \otimes \mathcal{C}/\mathcal{C}$-messbar. 
    \end{enumerate}
\end{mydef}

\begin{remark}
    Sei $(X, \mathcal{C})$ ein messbarer Vektorraum. Dann gilt
    \begin{enumerate}[(i)]
        \item Für alle $\alpha \in \R$ ist die Abbildung 
            $$f_{\alpha}: X \to X, x \mapsto \alpha x$$
        $\mathcal{C}/\mathcal{C}$-messbar. 
        \item Für alle $y \in X$ ist die Abbildung 
            $$g_y: X \to X, x \mapsto x + y$$
        $\mathcal{C}/\mathcal{C}$-messbar.
    \end{enumerate}
\end{remark}

Da die Komposition messbarer Abbildungen wiederum messbar ist erhält man unmittelbar
\begin{proposition}
    Sei $(X, \mathcal{C})$ ein messbarer Vektorraum und $(\Omega, \mathcal{A})$ ein messbarer Raum. 
    Sind $X,Y: \Omega \to X$ zwei $\mathcal{A}/\mathcal{C}$-messbare Abbildungen und $\alpha, \beta \in \R$, so ist auch $\alpha X + \beta Y$ $\mathcal{A}/\mathcal{C}$-messbar. 
\end{proposition}

\begin{proof*}
    Man beachte, dass für beliebige  messbare Räume $(\Omega_1, \mathcal{A}_1), (\Omega_2, \mathcal{A}_2)$, Mengen $A \in \mathcal{A}_1 \otimes \mathcal{A}_2$ und $\omega_1 \in \Omega_1$
    \begin{align*}
    A(\omega_1) = \{ \omega_2 : (\omega_1,\omega_2) \in A \} \in \mathcal{A}_2
    \end{align*}
    gilt. \qed
\end{proof*}
\begin{proposition}
    Sei $X$ ein separabler Banachraum. Dann ist $(X, \mathcal{B}(X))$ ein messbarer Vektorraum.
\end{proposition}
\begin{proof*}
    Nach Proposition 1.2 gilt $\mathcal{B}(X \times X) = \mathcal{B}(X) \otimes \mathcal{B}(X)$ und 
    $\mathcal{B}(\R \times X) = \mathcal{B}(\R) \otimes \mathcal{B}(X)$. Ferner sind die Abbildungen 
    \begin{align*}
        + &: X \times X \to X, \quad (x,y) \mapsto x + y, \\\
        \cdot &: \R \times X \to X, \quad (\alpha, x) \mapsto \alpha  x
    \end{align*}
    stetig bzgl. der jeweiligen Produkttopologien und somit insbesondere $\mathcal{B}(X \times X)/\mathcal{B}(X)$- bzw. 
    $\mathcal{B}(\R \times X)/\mathcal{B}(X)$-messbar. \qed
\end{proof*}

\begin{example}
    Für $d \in \N$ ist $(\R^d, \mathcal{B}(\R^d))$ ein messbarer Vektorraum. 
\end{example}
Im Folgenden sei $(X, \norm{\cdot})$ ein Banachraum und $(X', \norm{\cdot}_{op})$ der zugehörige Dualraum. 
\begin{proposition}
    Sei $\emptyset \neq \Gamma \subseteq X'$. Dann ist $(X, \sigma({\Gamma}))$ ein messbarer Vektorraum. 
\end{proposition}

\begin{proof*}%TODO: überarbeiten und schöneren Beweis finden. 
    Es genügt zu zeigen, dass 
    $$
        \forall f \in \Gamma : g: X \times X \to \R, \ (x,y) \mapsto f(x+y) \text{ ist messbar.}
    $$
    Sei dazu $f \in \Gamma$. Es gilt wegen der Linearität von $f$ für $(x,y) \in X \times X$
    $$
        f(x + y) = (f(x) + f(y)).
    $$
    Weiter ist die Abbildung 
    $$
        h: X \times X \to \R, (x,y) \mapsto f(x) + f(y)
    $$
    als Komposition messbarer Funktionen messbar, da $(\R, \mathcal{B}(\R))$ ein messbarer Vektorraum ist. 
    Die Messbarkeit der Skalarmultiplikation zeigt man analog. \qed 
\end{proof*}

\begin{proposition}
    Sei $E$ ein separabler Banachraum. Dann gilt $\sigma(E') = \mathcal{B}(E)$. 
\end{proposition}

\begin{proof*}%TODO: Überprüfen. Reicht das wirklich schon so?
    Da alle $f \in E'$ stetig sind gilt offensichtlich $\sigma(E') \subseteq \mathcal{B}(E)$. 
    Wegen der Separabilität von $E$ wird $\mathcal{B}(E)$ von den abgeschlossenen Kugeln erzeugt und da $(E, \sigma(E'))$ nach vorheriger Proposition ein messbarer Vektorraum ist genügt es zu zeigen, 
    dass $\overline{B}(0,1)$ in $\sigma(E')$ enthalten ist. Nach dem Satz von Hahn-Banach existiert eine Folge $(f_n)_{n \in \N}$ in $E'$ mit
    $$
        \forall x \in E: \quad \norm{x} = \sup_{n \in \N}\abs{f_n(x)}.
    $$
    Schließlich gilt 
    $$
        \overline{B}(0,1) = \{ x \in E: \norm{x} \leq 1 \} = \{x \in E: \ \sup_{n \in \N}\abs{f_n(x)} \leq 1 \} = \bigcap_{n=1}^{\infty}\{x \in E: \abs{f_n(x)} \leq 1\} \in \sigma(E').
    $$
    \qed 
\end{proof*}