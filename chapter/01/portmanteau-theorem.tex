
\begin{theorem}[Portmanteau-Theorem]
    Sei $(X,d)$ ein metrischer Raum und seien $\mu, \mu_1, \mu_2,...$ Wahrscheinlichkeitsmaße auf $\mathcal{B}(X)$. Dann sind äquivalent:
    \begin{enumerate}[(i)]
        \item $(\mu_n)_{n \in \N}$ konvergiert schwach gegen $\mu$.
        \item Für alle abgeschlossenen Teilmengen $A \subseteq X$ gilt 
        $$
            \limsup_{n \to \infty} \mu_n(A) \leq \mu(A).
        $$
        \item Für alle offenen Teilmengen $B \subseteq X$ gilt 
        $$
            \liminf_{n \to \infty} \mu_n(B) \geq \mu(B).
        $$
        \item Für alle Borelmengen $C \in \mathcal{B}(X)$ mit $\mu(\partial C) = 0$ gilt 
        $$
            \lim_{n \to \infty}\mu_n(C) = \mu(C).
        $$
    \end{enumerate}
\end{theorem}

\begin{proof*}
    Zu $(i) \Rightarrow (ii)$: Sei $A \subseteq X$ abgeschlossen und $\varepsilon > 0$. Da die Aussage für $A = \emptyset$ trivialerweise erfüllt ist,
    können wir ohne Beschränkung der Allgemeinheit annehmen, dass $A \neq \emptyset$. Für $m \in \N$ setze
    \begin{align*}
        U_m := \{x \in X: \inf_{y \in A}d(x,y) < \frac{1}{m}\}.
    \end{align*}
    Dann ist die Menge $U_m$ für alle $m \in \N$ offen und es gilt $U_m \downarrow A$, weil $A$ abgeschlossen ist.  
    Aufgrund der $\sigma$-Stetigkeit von $\mu$ existiert also ein $k \in \N$ mit 
    $$
        \mu(U_k) < \mu(A) + \varepsilon . 
    $$
    Betrachte nun die Abbildung 
    $$
        f:X \to \R, \quad x \mapsto \max\{1 - k \inf_{y \in A}d(x,y), 0\}.
    $$
    Offensichtlich ist $f$ beschränkt. Ferner ist die Abbildung 
    $$
        d(\cdot, A): X \to \R, \quad x \mapsto d(x,A) := \inf_{y \in A}d(x,y)
    $$
    nach der umgekehrten Dreiecksungleichung stetig. Da die Abbildung
    $$
        x \mapsto \max\{x,0\}, \quad x \in \R,
    $$
    stetig ist, ist somit $f$ als Komposition stetiger Funktionen stetig. 
    Wegen $1_A \leq f \leq 1_{U_k}$ erhalten wir zusammen mit der Voraussetzung 
    $$
    \limsup_{n \to \infty} \mu_n(A) \leq \lim_{n \to \infty} \int_X fd\mu_n = \int_X fd\mu \leq \mu(U_k) \leq \mu(A) + \varepsilon.
    $$
    Da diese Ungleichung für alle $\varepsilon > 0$ erfüllt ist, folgt die Behauptung. 
    \newline 
    Zu $(ii) \iff (iii)$: Folgt unmittelbar durch Komplementbildung. 
    \newline
    Zu $(iii) \Rightarrow (iv)$: 
    Sei $C \in \mathcal{B}(X)$ mit $\mu(\partial C) = 0$. Dann gilt insbesondere $\mu(\overline{C}) = \mu(C) = \mu(\mathring{C})$. Da $(iii)$ auch $(ii)$ impliziert, erhalten wir somit
    $$
        \mu(C) = \mu(\mathring{C}) \leq \liminf_{n \to \infty} \mu_n(C) \leq \limsup_{n \to \infty} \mu_n(C) \leq \mu(\overline{C}) = \mu(C).
    $$
    \newline 
    Zu $(iv) \Rightarrow (i)$: 
    Sei $f \in C_b(X)$ beschränkt durch $M > 0$. Wegen der Linearität des Integrals können wir ohne Einschränkung annehmen, dass $f \geq 0$. 
    Wegen der Stetigkeit von $f$ erhalten wir für alle $t > 0$
    $$
        \partial\{ f > t \} \subseteq \{f = t \}. 
    $$
    Da $\mu$ ein Wahrscheinlichkeitsmaß ist, gibt es eine abzählbare Menge $C \subseteq \R$ mit 
    $$
        \forall t \in \R \setminus C: \quad \mu(\{f = t \}) = 0. 
    $$
    Also gilt für alle $t \in \R \setminus C$ nach Voraussetzung 
    $$
        \lim_{n \to \infty} \mu_n(\{f > t \}) = \mu(\{f > t \}).
    $$
    Mit dem Prinzip von Cavalieri, vgl. \cite[Satz 1.8.20]{gs}, erhalten wir schließlich per dominierter Konvergenz
    $$
        \lim_{n \to \infty} \int_X fd\mu_n = \lim_{n \to \infty} \int_0^M \mu_n(\{f > t \})d\lambda(t) = \int_0^M \mu(\{f > t \}) d\lambda(t) = \int_Xfd\mu. 
    $$
    \qed 
\end{proof*}