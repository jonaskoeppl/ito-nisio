\section{Borelmaße auf metrischen Räumen}
Im Folgenden Abschnitt beschäftigen wir uns mit Wahrscheinlichkeitsmaßen auf der Borelschen $\sigma$-Algebra $\mathcal{B}(X)$ eines metrischen Raums $(X,d)$, 
welche teilweise auch als \textit{Borelsche Wahrscheinlichkeitsmaße} bezeichnet werden. 
Zunächst interessieren wir uns hierbei für Regularitätseigenschaften solcher Maße, welche uns die spätere Arbeit erleichtern werden. 
Im zweiten Teil dieses Abschnitts untersuchen wir Folgen von Wahrscheinlichkeitsmaßen und den Begriff der schwachen Konvergenz. 
\newline \ \newline 
\textbf{Notation und Konventionen} 
\newline
Bis auf weiteres sei $(X,d)$ ein metrischer Raum mit Borel-$\sigma$-algebra $\mathcal{B}(X)$. 
Für eine Teilmenge $A \subseteq X$ sei $\mathring{A}$ das \textit{Innere}, $\overline{A}$ der \textit{Abschluss} und 
$$
    \partial A := \overline{A} \setminus \mathring{A}
$$
der \textit{Rand der Menge}. 
Setze ferner 
$$
    C_b(X;\R) := \{f:X \to \R \ | \ f \text{ ist stetig und beschränkt} \}.
$$ 
Falls klar ist welcher metrische Raum gemeint ist, so schreiben wir auch $C_b(\R)$ statt $C_b(X;\R)$. 
\newline 



\begin{mydef}
    Ein Maß $\mu$ auf $\mathcal{B}(X)$ heißt \textit{regulär} , falls
    \begin{align*}
        \forall B \in \mathcal{B}(X): \quad \mu(B) &= \sup\{\mu(C): C \subseteq B, \ C \text{ abgeschlossen} \} \\\
                                                   &= \inf\{\mu(O): B \subseteq O, \ O \text{ offen} \}.  
    \end{align*}  
\end{mydef}

\begin{proposition}
    Sei $\mu$ ein Wahrscheinlichkeitsmaß auf $\mathcal{B}(X)$. Dann ist $\mu$ regulär. 
\end{proposition}

\begin{proof*}
    Wir verwenden zum Beweis das Good-Set-Principle. Setze dazu
    $$
        \mathcal{R}:=\big\{A \in \mathcal{B}(X):  \mu(A) = \sup\{\mu(C): C \subseteq A, \ C \text{ abgeschlossen}\} =\inf\{\mu(O): A \subseteq O, \ O \text{ offen}\} \big\}.
    $$
    Wir zeigen zunächst, dass $\mathcal{R}$ eine $\sigma$-Algebra ist. Offensichtlich gilt $\emptyset \in \mathcal{R}$. Sei nun $A \in \mathcal{R}$ und $\varepsilon > 0$. 
    Dann existieren eine offene Menge $O$ und eine abgeschlossene Menge $C$ mit $C \subseteq A \subseteq O$ und 
    $$
        \mu(0) - \varepsilon < \mu(A) < \mu(C) + \varepsilon.
    $$
    Es gilt also $O^c \subseteq A^c \subseteq C^c$ und 
    $$
        \mu(C^c) - \varepsilon = 1 - \mu(C) - \varepsilon < 1 - \mu(A) = \mu(A^c) = 1 - \mu(A) < 1 - \mu(O) + \varepsilon = \mu(O^c) + \varepsilon. 
    $$
    Da $O^c$ abgeschlossen ist und $C^c$ offen ist, folgt $A^c \in \mathcal{R}$. 
    \newline 
    Seien nun $A_1, A_2,... \in \mathcal{R}$ und $\varepsilon > 0$ Für jedes $n \in \N$ existieren dann eine offene Menge $O_n$ und eine abgeschlossene Menge $C_n$ mit $C_n \subseteq A_n \subseteq O_n$ und 
    $$
        \mu(O_n) - 2^{-n}\varepsilon < \mu(A_n) < \mu(C_n) + 2^{n+1}\varepsilon.
    $$
    Es gilt also
    $$
        \bigcup_{n = 1}^{\infty} C_n \subseteq \bigcup_{n=1}^{\infty} A_n \subseteq \bigcup_{n=1}^{\infty}C_n
    $$
    und 
    \begin{align}
        \mu\left(\bigcup_{n =1}^{\infty}U_n\right) - \mu\left(\bigcup_{n=1}^{\infty}A_n\right) &\leq \mu\left(\bigcup_{n =1}^{\infty}U_n\setminus \bigcup_{n =1}^{\infty} A_n\right) \nonumber \\\
                                                                                &\leq \mu\left(\bigcup_{n =1}^{\infty}(U_n\setminus A_n)\right) 
                                                                                \leq \sum_{i=1}^{\infty}\mu(U_n\setminus A_n)
                                                                                < \sum_{i=1}^{\infty}2^{-n}\varepsilon = \varepsilon. 
    \end{align}
    Wegen 
    $$
    \mu(\cup_{n = 1}^{\infty}C_n) = \lim_{k \to \infty}\mu(\cup_{n=1}^kC_n),
    $$
    existiert zudem ein $k \in \N$ mit 
    $$
    \mu(\cup_{n=1}^{\infty}C_n) - \mu(\cup_{n=1}^kC_n) < \frac{\varepsilon}{2}.
    $$ 
    Die Menge $C:= \cup_{n=1}^kC_n$ ist als endliche Vereinigung abgeschlossener Mengen abgeschlossen und nach Konstruktion in $\cup_{n=1}^{\infty}A_n$ enthalten. Ferner gilt 
    \begin{align}
        \mu(\bigcup_{n=1}^{\infty}A_n) - \mu(C) &< \mu(\bigcup_{n=1}^{\infty}A_n) - \mu(\bigcup_{n=1}^{\infty}C_n) + \frac{\varepsilon}{2} \nonumber \\\
                                                &\leq \mu(\bigcup_{n=1}^{\infty}A_n \setminus \bigcup_{n=1}^{\infty}C_n)  + \frac{\varepsilon}{2} \nonumber \\\
                                                &\leq \mu(\bigcup_{n=1}^{\infty}(A_n \setminus C_n)) + \frac{\varepsilon}{2} \nonumber \\\
                                                &\leq \sum_{n=1}^{\infty}\mu(A_n \setminus C_n) + \frac{\varepsilon}{2} = \varepsilon. 
    \end{align}
    Da $\varepsilon$ beliebig gewählt war folgt aus $(1.1)$ und $(1.2)$, dass $\cup_{n=1}^{\infty}A_n \in \mathcal{R}$. Folglich ist $\mathcal{R}$ eine $\sigma$-Algebra. 
    Es bleibt noch zu zeigen, dass $\mathcal{R}$ alle abgeschlossenen Mengen enthält. Sei also $\emptyset \neq A \subseteq X$ abgeschlossen. Die Bedingung 
    $$
        \mu(A) = \sup\{\mu(C): C \subseteq A, C\text{ abgeschlossen }\}
    $$
    folgt direkt aus der Monotonie von $\mu$. Um die zweite Bedingung zu zeigen setze für $n \in \N$
    $$
        O_n := \big\{x \in X: \inf_{y \in  A}d(x,y) < \frac{1}{n}\big\}.
    $$
    Dann ist $O_n$ für jedes $n \in \N$ offen und es gilt $O_1 \supseteq O_2 \supseteq ...$, sowie $\cap_{n=1}^{\infty}O_n = A$, da $A$ abgeschlossen ist. 
    Mit der $\sigma$-Stetigkeit von $\mu$ folgt letztendlich 
    $$
        \mu(A) \leq \inf\{\mu(O): A \subseteq O, O \text{ offen}\} \leq \inf_{n \in \N}\mu(O_n) = \lim_{n \to \infty}\mu(O_n) = \mu(A). 
    $$
    Da die abgeschlossenen Mengen $\mathcal{B}(X)$ erzeugen gilt $\mathcal{R} = \mathcal{B}(X)$ und folglich ist $\mu$ regulär. \qed
\end{proof*}
\begin{mydef}
    Ein Maß $\mu$ auf $\mathcal{B}(X)$ heißt \textit{straff}, falls es für alle $\varepsilon > 0$ eine kompakte Menge $K \subseteq X$ gibt mit 
    \begin{align*}
        \mu(K) \geq 1 - \varepsilon. 
    \end{align*}

\end{mydef}

\begin{corollary}
    Sei $\mu$ ein straffes Wahrscheinlichkeitsmaß auf $\mathcal{B}(X)$. Dann gilt
    \begin{align*}
        \forall A \in \mathcal{B}(X): \quad \mu(A) = \sup\{\mu(K): K \subseteq A,\ K \text{ kompakt}\}. 
    \end{align*}
\end{corollary}

\begin{proof*}
    Sei $A \in \mathcal{B}(X)$ und $\varepsilon > 0$. Wegen der Straffheit von $\mu$ existiert eine kompakte Menge $K_{\varepsilon} \subseteq X$ mit $\mu(K_{\varepsilon}) \geq 1 - \frac{\varepsilon}{2}$,
    und da $\mu$ nach Proposition $1.4$ regulär ist gibt es eine abgeschlossene Menge $C \subseteq A$ mit $\mu(C) > \mu(A) - \frac{\varepsilon}{2}$. Die Menge $K_{\varepsilon} \cap C$ ist wiederum kompakt und es gilt
    \begin{align*}
        \mu(A) \geq \mu(K_{\varepsilon} \cap C) > \mu(C) - \frac{\varepsilon}{2} > \mu(A) - \varepsilon. 
    \end{align*} 
    \qed
\end{proof*}

\begin{remark}
    Ein Wahrscheinlichkeitsmaß $\mu$ auf $\mathcal{B}(X)$ mit der Eigenschaft
    \begin{align*}
        \forall A \in \mathcal{B}(X): \quad \mu(A) = \sup\{\mu(K): K \subseteq A, \ K \text{ kompakt}\}. 
    \end{align*}
    wird auch als \textit{Radon-Wahrscheinlichkeitsmaß} oder \textit{Radon-Maß} bezeichnet.
\end{remark}

\begin{proposition}
    Sei $(X,d)$ ein vollständiger separabler metrischer Raum. Dann ist jedes Wahrscheinlichkeitsmaß $\mu$ auf $\mathcal{B}(X)$ straff.
\end{proposition}

Wir verwenden zum Beweis der Proposition die folgende Charakterisierung kompakter Teilmengen metrischer Räume. 
Der Beweis wird mittels der in metrischen Räumen geltenden Äquivalenz von Überdeckungskompaktheit und Folgenkompaktheit  geführt und findet sich etwa in \cite{amann}[Theorem 3.10]. 

\begin{lemma}
    Sei $(X,d)$ ein metrischer Raum. Eine Menge $K \subseteq X$ ist genau dann kompakt, wenn sie die folgenden beiden Eigenschaften erfüllt
    \begin{enumerate}[(a)]
        \item $K$ ist vollständig,
        \item $K$ ist total-beschränkt, d.h.
        \begin{align*}
            \forall  \varepsilon > 0 \ \exists x_1,...,x_n \in K: \  K \subseteq \cup_{i=1}^n B(x_i, \varepsilon). 
        \end{align*} 
\end{enumerate}
\end{lemma}

\begin{proof*}
    Sei $\varepsilon > 0$. Nach Voraussetzung existiert eine abzählbare dichte Teilmenge $D = \{x_1, x_2,...\}$ von $X$. Also gilt insbesondere für $q \in \N $
    \begin{align*}
        \bigcup_{i = 1}^{\infty}\overline{B}(x_i, 2^{-q}) = X.
    \end{align*}
    Wegen der $\sigma$-Stetigkeit von $\mu$ existiert also ein $N_q \in \N$ mit 
    \begin{align*}
        \mu\big(\bigcup_{i=1}^{N_q}\overline{B}(x_i, 2^{-q})\big) \geq 1 - \varepsilon 2^{-q}. 
    \end{align*}
    Setze nun 
    \begin{align*}
        K := \bigcap_{q = 1}^{\infty}\bigcup_{i=1}^{N_q}\overline{B}(x_i, 2^{-q}). 
    \end{align*}
    Dann ist $K$ als Schnitt abgeschlossener Teilmengen abgeschlossen, und da $X$ vollständig ist, folgt daraus bereits die Vollständigkeit von $K$. 
    Ferner ist $K$ total-beschränkt, denn zu $\varepsilon > 0$ existiert ein $q \in \N$ mit $2^{-q} < \varepsilon$ und $K \subseteq \cup_{i=1}^{N_q}B(x_i, 2^{-q}) \subseteq \cup_{i=1}^{N_q}B(x_i, \varepsilon)$. 
    Zudem gilt
    \begin{align*}
        \mu(K)  = 1 - \mu\big(\bigcup_{q = 1}^{\infty}\bigcap_{i=1}^{N_q}\overline{B}(x_i, 2^{-q})^c\big) 
                &\geq 1 - \sum_{q=1}^{\infty} \mu\big(\bigcap_{i=1}^{N_q}\overline{B}(x_i, 2^{-q})^c\big) \\\
                &\geq 1 - \sum_{q=1}^{\infty} \varepsilon 2^{-q} = 1 - \varepsilon.
    \end{align*}
    Also ist $\mu$ straff. \qed
\end{proof*}

\begin{proposition}
    Sei $(X,d)$ ein vollständiger metrischer Raum und $\mu$ ein Wahrscheinlichkeitsmaß auf $\mathcal{B}(X)$. Dann sind äquivalent
    \begin{enumerate}[(i)]
        \item $\mu$ ist straff.
        \item Es gibt eine separable Teilmenge $E \subseteq X$ mit $\mu(E) = 1$. 
    \end{enumerate}
\end{proposition}
\begin{proof*}
    zu (i) $\Rightarrow$ (ii): Für alle $n \in \N$ existiert $K_n \subseteq X$ kompakt mit $\mu(K_n) \geq 1 - \frac{1}{n}$, o.E. gelte $K_n \subseteq K_{n+1}$. Es folgt 
    \begin{align*}
        \mu\big(\cup_{n=1}^{\infty}K_n\big) = \lim_{n \to \infty}\mu\big(K_ n\big) = 1. 
    \end{align*}
    Da kompakte Teilmengen metrischer Räume insbesondere separabel sind, ist $E := \cup_{n=1}^{\infty}K_n$ als abzählbare Vereinigung separabler Mengen ebenso separabel. 
    \newline 
    zu (ii) $\Rightarrow$ (i): 
    Analog zum Beweis von Proposition 1.8. \qed
\end{proof*}

Insgesamt haben wir also gezeigt

\begin{theorem}
    Für ein Wahrscheinlichkeitsmaß $\mu$ auf der Borelschen $\sigma$-Algebra eines vollständigen metrischen Raumes $(X,d)$ sind äquivalent
    \begin{enumerate}[(i)]
        \item $\mu$ ist straff.
        \item Es gibt eine separable Menge $E \subseteq X$ mit $\mu(E) = 1$.
        \item $\mu$ ist ein Radon-Maß, d.h.
        $$
        \forall A \in \mathcal{B}(X): \quad \mu(A) = \sup\{\mu(K): K \subseteq A, \ K \text{ kompakt}\}.
        $$   
    \end{enumerate}
\end{theorem}

Nachdem wir uns nun ausgiebig mit den Eigenschaften einzelner Maße beschäftigt haben, möchten wir uns jetzt mit Folgen von Wahrscheinlichkeitsmaßen und deren Konvergenz beschäftigen. 

\begin{mydef}
    Eine Folge $(\mu_n)_{n \in \N}$ von Wahrscheinlichkeitsmaßen auf $\mathcal{B}(X)$ heißt \textit{schwach konvergent} 
    gegen ein Wahrscheinlichkeitsmaß $\mu$ auf $\mathcal{B}(X)$, falls 
    $$
        \forall f \in C_b(X): \quad \lim_{n \to \infty} \int_Xfd\mu_n = \int_X fd\mu . 
    $$
    Bezeichnung: $\mu_n \rightharpoonup \mu$. 
\end{mydef}

Als nützliches Hilfsmittel für viele Beweise dient der folgende Satz, der meist als \textit{Portmanteau-Theorem} bezeichnet wird. 

\begin{theorem}[Portmanteau-Theorem]
    Sei $(X,d)$ ein metrischer Raum und seien $\mu, \mu_1, \mu_2,...$ Wahrscheinlichkeitsmaße auf $\mathcal{B}(X)$. Dann sind äquivalent:
    \begin{enumerate}[(i)]
        \item $(\mu_n)_{n \in \N}$ konvergiert schwach gegen $\mu$.
        \item Für alle abgeschlossenen Teilmengen $A \subseteq X$ gilt 
        $$
            \limsup_{n \to \infty} \mu_n(A) \leq \mu(A).
        $$
        \item Für alle offenen Teilmengen $B \subseteq X$ gilt 
        $$
            \liminf_{n \to \infty} \mu_n(B) \geq \mu(B).
        $$
        \item Für alle Borelmengen $C \in \mathcal{B}(X)$ mit $\mu(\partial C) = 0$ gilt 
        $$
            \lim_{n \to \infty}\mu_n(C) = \mu(C).
        $$
    \end{enumerate}
\end{theorem}

\begin{proof*}
    Zu $(i) \Rightarrow (ii)$: Sei $A \subseteq X$ abgeschlossen und $\varepsilon > 0$. Da die Aussage für $A = \emptyset$ trivialerweise erfüllt ist,
    können wir ohne Beschränkung der Allgemeinheit annehmen, dass $A \neq \emptyset$. Für $m \in \N$ setze
    \begin{align*}
        U_m := \{x \in X: \inf_{y \in A}d(x,y) < \frac{1}{m}\}.
    \end{align*}
    Dann ist die Menge $U_m$ für alle $m \in \N$ offen und es gilt $U_m \downarrow A$, weil $A$ abgeschlossen ist.  
    Aufgrund der $\sigma$-Stetigkeit von $\mu$ existiert also ein $k \in \N$ mit 
    $$
        \mu(U_k) < \mu(A) + \varepsilon . 
    $$
    Betrachte nun die Abbildung 
    $$
        f:X \to \R, \quad x \mapsto \max\{1 - k \inf_{y \in A}d(x,y), 0\}.
    $$
    Offensichtlich ist $f$ beschränkt. Ferner ist die Abbildung 
    $$
        d(\cdot, A): X \to \R, \quad x \mapsto d(x,A) := \inf_{y \in A}d(x,y)
    $$
    nach der umgekehrten Dreiecksungleichung stetig. Da die Abbildung
    $$
        (x,y) \mapsto \max\{x,y\}, \quad (x,y) \in \R^2,
    $$
    stetig ist, ist somit $f$ als Komposition stetiger Funktionen stetig. 
    Wegen $1_A \leq f \leq 1_{U_k}$ erhalten wir zusammen mit der Voraussetzung 
    $$
    \limsup_{n \to \infty} \mu_n(A) \leq \lim_{n \to \infty} \int_X fd\mu_n = \int_X fd\mu \leq \mu(U_k) \leq \mu(A) + \varepsilon.
    $$
    Da diese Ungleichung für alle $\varepsilon > 0$ erfüllt ist, folgt die Behauptung. 
    \newline 
    Zu $(ii) \iff (iii)$: Folgt unmittelbar durch Komplementbildung. 
    \newline
    Zu $(iii) \Rightarrow (iv)$: 
    Sei $C \in \mathcal{B}(X)$ mit $\mu(\partial C) = 0$. Dann gilt insbesondere $\mu(\overline{C}) = \mu(C) = \mu(\mathring{C})$. Da $(iii)$ auch $(ii)$ impliziert, erhalten wir somit
    $$
        \mu(C) = \mu(\mathring{C}) \leq \liminf_{n \to \infty} \mu_n(C) \leq \limsup_{n \to \infty} \mu_n(C) \leq \mu(\overline{C}) = \mu(C).
    $$
    \newline 
    Zu $(iv) \Rightarrow (i)$: 
    Sei $f \in C_b(X)$ beschränkt durch $M > 0$. Wegen der Linearität des Integrals können wir ohne Einschränkung annehmen, dass $f \geq 0$. 
    Wegen der Stetigkeit von $f$ erhalten wir für alle $t > 0$
    $$
        \partial\{ f > t \} \subseteq \{f = t \}. 
    $$
    Da $\mu$ ein Wahrscheinlichkeitsmaß ist, gibt es eine abzählbare Menge $C \subseteq \R$ mit 
    $$
        \forall t \in \R \setminus C: \quad \mu(\{f = t \}) = 0. 
    $$
    Also gilt für alle $t \in \R \setminus C$ nach Voraussetzung 
    $$
        \lim_{n \to \infty} \mu_n(\{f > t \}) = \mu(\{f > t \}).
    $$
    Mit dem Prinzip von Cavalieri, vgl. \cite[Satz 1.8.20]{gs}, erhalten wir schließlich per dominierter Konvergenz
    $$
        \lim_{n \to \infty} \int_X fd\mu_n = \lim_{n \to \infty} \int_0^M \mu_n(\{f > t \})d\lambda(t) = \int_0^M \mu(\{f > t \}) d\lambda(t) = \int_Xfd\mu. 
    $$
    \qed 
\end{proof*}