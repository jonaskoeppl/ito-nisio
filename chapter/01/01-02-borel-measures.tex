\section{Borelmaße auf metrischen Räumen}

Bis auf weiteres sei $(X,d)$ ein metrischer Raum mit Borel-$\sigma$-algebra $\mathcal{B}(X)$. 
Im Folgenden Abschnitt beschäftigen wir uns mit Maßen auf $\mathcal{B}(X)$, welche teilweise auch als \textit{Borel-Maße} bezeichnet werden. 
Die Bezeichnung wird in der Literatur allerdings nicht einheitlich verwendet. 

\begin{mydef}
    Ein Maß $\mu$ auf $\mathcal{B}(X)$ heißt \textit{regulär} , falls
    \begin{align*}
        \forall B \in \mathcal{B}(X): \quad \mu(B) &= \sup\{\mu(C): C \subseteq B, \ C \text{ abgeschlossen} \} \\\
                                                   &= \inf\{\mu(O): B \subseteq O, \ O \text{ offen} \}.  
    \end{align*}  
\end{mydef}

\begin{proposition}
    Sei $\mu$ ein Wahrscheinlichkeitsmaß auf $\mathcal{B}(X)$. Dann ist $\mu$ regulär. 
\end{proposition}

\begin{proof*}
    \textbf{TODO}
\end{proof*}

\begin{mydef}
    Ein Maß $\mu$ auf $\mathcal{B}(X)$ heißt \textit{straff}, falls es für alle $\varepsilon > 0$ eine kompakte Menge $K \subseteq X$ gibt mit 
    \begin{align*}
        \mu(K) \geq 1 - \varepsilon. 
    \end{align*}

\end{mydef}

\begin{corollary}
    Sei $\mu$ ein straffes Wahrscheinlichkeitsmaß auf $\mathcal{B}(X)$. Dann gilt
    \begin{align*}
        \forall A \in \mathcal{B}(X): \quad \mu(A) = \sup\{\mu(K): K \subseteq A,\ K \text{ kompakt}\}. 
    \end{align*}
\end{corollary}

\begin{proof*}
    Sei $A \in \mathcal{B}(X)$ und $\varepsilon > 0$. Wegen der Straffheit von $\mu$ existiert eine kompakte Menge $K_{\varepsilon} \subseteq X$ mit $\mu(K_{\varepsilon}) \geq 1 - \frac{\varepsilon}{2}$,
    und da $\mu$ nach Proposition 1.4 regulär ist gibt es eine abgeschlossene Menge $C \subseteq A$ mit $\mu(C) > \mu(A) - \frac{\varepsilon}{2}$. Dann ist die Menge $K_{\varepsilon} \cap C$ wiederum kompakt und es gilt
    \begin{align*}
        \mu(A) \geq \mu(K_{\varepsilon} \cap C) > \mu(C) - \frac{\varepsilon}{2} > \mu(A) - \varepsilon. 
    \end{align*} 
    \qed
\end{proof*}

\begin{remark}
    Ein Wahrscheinlichkeitsmaß $\mu$ auf $\mathcal{B}(X)$ mit der Eigenschaft
    \begin{align*}
        \forall A \in \mathcal{B}(X): \quad \mu(A) = \sup\{\mu(K): K \subseteq A, \ K \text{ kompakt}\}. 
    \end{align*}
    wird auch als \textit{Radon-Wahrscheinlichkeitsmaß} oder \textit{Radon-Maß} bezeichnet.
\end{remark}

\begin{proposition}
    Sei $(X,d)$ ein vollständiger separabler metrischer Raum. Dann ist jedes Wahrscheinlichkeitsmaß $\mu$ auf $\mathcal{B}(X)$ straff.
\end{proposition}

Wir verwenden zum Beweis der Proposition die folgende Charakterisierung kompakter Teilmengen metrischer Räume. Ein Beweis findet sich etwa in \cite{amann}. 

\begin{lemma}
    Sei $(X,d)$ ein metrischer Raum. Eine Menge $K \subseteq X$ ist genau dann kompakt, wenn sie die folgenden beiden Eigenschaften erfüllt:
    \begin{enumerate}[(i)]
        \item $K$ ist vollständig,
        \item $K$ ist total-beschränkt, d.h.
        \begin{align*}
            \forall  \varepsilon > 0 \ \exists x_1,...,x_n \in K: \  K \subseteq \cup_{i=1}^n B(x_i, \varepsilon). 
        \end{align*} 
\end{enumerate}
\end{lemma}

\begin{proof*}
    \textbf{TODO}
    Sei $\varepsilon > 0$. Nach Voraussetzung existiert eine abzählbare dichte Teilmenge $D = \{x_1, x_2,...\}$ von $X$. Also gilt insbesondere für $q \in \N $
    \begin{align*}
        \bigcup_{i\in \N}\overline{B}(x_i, 2^{-q}) = X.
    \end{align*}
    Wegen der $\sigma$-Stetigkeit von $\mu$ existiert also ein $N_q \in \N$ mit 
    \begin{align*}
        \mu(\cup_{i=1}^{N_q}\overline{B}(x_i, 2^{-q}) \geq 1 - \varepsilon 2^{-q}. 
    \end{align*}
    Setze nun 
    \begin{align*}
        K := \bigcap_{q \in \N}\bigcup_{i=1}^{N_q}\overline{B}(x_i, 2^{-q}). 
    \end{align*}
    Dann ist $K$ als Schnitt abgeschlossener Teilmengen abgeschlossen, und da $X$ vollständig ist, folgt daraus bereits die Vollständigkeit von $K$. 
    Ferner ist $K$ total-beschränkt, denn zu $\varepsilon > 0$ existiert ein $q \in \N$ mit $2^{-q} < \varepsilon$ und $K \subseteq \cup_{i=1}^{N_q}B(x_i, 2^{-q}) \subseteq \cup_{i=1}^{N_q}B(x_i, \varepsilon)$. 
    Zudem gilt
    \begin{align*}
        \mu(K)  = 1 - \mu(\cup_{q \in \N}\cap_{i=1}^{N_q}\overline{B}(x_i, 2^{-q})^c) 
                &\geq 1 - \sum_{q=1}^{\infty} \mu(\cap_{i=1}^{N_q}\overline{B}(x_i, 2^{-q})^c) \\\
                &\geq 1 - \sum_{q=1}^{\infty} \varepsilon 2^{-q} = 1 - \varepsilon.
    \end{align*}
    Also ist $\mu$ straff. \qed
\end{proof*}

\begin{proposition}
    Sei $(X,d)$ ein vollständiger metrischer Raum und $\mu$ ein Wahrscheinlichkeitsmaß auf $\mathcal{B}(X)$. Dann sind äquivalent
    \begin{enumerate}[(i)]
        \item $\mu$ ist straff.
        \item Es gibt eine separable Teilmenge $E \subseteq X$ mit $\mu(E) = 1$. 
    \end{enumerate}
\end{proposition}
\begin{proof*}
    zu (i) $\Rightarrow$ (ii): Für alle $n \in \N$ existiert $K_n \subseteq X$ kompakt mit $\mu(K_n) \geq 1 - \frac{1}{n}$, o.E. gelte $K_n \subseteq K_{n+1}$. Es folgt 
    \begin{align*}
        \mu\big(\cup_{n=1}^{\infty}K_n\big) = \lim_{n \to \infty}\mu\big(K_{n+1}\big) = 1. 
    \end{align*}
    Da kompakte Teilmengen metrischer Räume insbesondere separabel sind, ist $E := \cup_{n=1}^{\infty}K_n$ als abzählbare Vereinigung separabler Mengen ebenso separabel. 
    \newline 
    zu (ii) $\Rightarrow$ (i): 
    Analog zum Beweis von Proposition 1.8. \qed
\end{proof*}