\section{Borelmengen in metrischen Räumen}
Für einen metrischen Raum $(X,d)$ bezeichne im Folgenden $\mathcal{B}(X)$ die Borel-$\sigma$-algebra in $X$. 
Zudem werden für $x \in X$ und $r>0$ mit $B(x, r)$ bzw. $\overline{B}(x,r)$ die offene bzw. abgeschlossene Kugel um $x$ mit Radius $r$ bezeichnet.
\begin{proposition}
    Sei $(X,d)$ ein separabler metrischer Raum. Dann gilt
    \begin{align*}
        \mathcal{B}(X) = \sigma(\{B(x,r): x \in X, r > 0 \}) = \sigma(\{\overline{B}(x,r): x \in X, r > 0 \}). 
    \end{align*}
\end{proposition}
\begin{proof*}
    Setze 
    \begin{align*}
        \mathcal{A}_1 &:= \sigma(\{B(x,r): x \in X, r > 0 \}), \\\ 
        \mathcal{A}_2 &:= \sigma(\{\overline{B}(x,r): x \in X, r > 0 \}). 
    \end{align*}
    Man sieht leicht ein, dass $\mathcal{A}_2 = \mathcal{A}_1 \subseteq \mathcal{B}(X)$. Zu zeigen bleibt also nur die Inklusion $\mathcal{B}(X) \subseteq \mathcal{A}_1$.
    Sei dazu $U \subseteq X$ offen und $x \in U$. Nach Voraussetzung existiert eine abzählbare dichte Teilmenge $D \subseteq X$. Definiere 
    \begin{align*}
        R := \{(y,r) : y \in U \cap D, r > 0, r \in \Q, B(y,r) \subseteq U \}.
    \end{align*}
    Dann ist $R$ abzählbar und da $D$ dicht in $X$ liegt gilt $U = \bigcup_{(y,r) \in R}B(y,r)$. 
    Also gilt $U \in \mathcal{A}_1$ und da $\mathcal{B}(X)$ von den offenen Teilmengen von $X$ erzeugt wird folgt die Behauptung. \qed
\end{proof*}

\begin{proposition}
    Für $i \in \N$ sei $(X_i, d_i)$ ein separabler metrischer Raum. Dann gilt
    \begin{align*}
        \mathcal{B}(X_1 \times X_2 \times ...) = \otimes_{i=1}^{\infty}\mathcal{B}(X_i)
    \end{align*}
\end{proposition}

\begin{proof*}
    Setze $X:= \times_{k \in \N}X_k$ und bezeichne $p_k: X \to X_k$ die Projektion auf die k-te Komponente. Betrachte das Mengensystem
    \begin{align*}
        \mathcal{E} :&= \{ \bigcap_{k \in K}p_k^{-1}(O) | \forall k \in K: O_k \subseteq X_k \text{ offen}, K \subseteq \N \text{ endlich}\}. 
    \end{align*}
    Offensichtlich gilt $\otimes_{k \in \N}\mathcal{B}(X_k) = \sigma(\mathcal{E})$. 
    Ferner ist $X$ ein separabler metrischer Raum und $\mathcal{E}$ eine Basis der Produkttopologie auf $X$, vgl. \cite{querenburg}[3.7]. 
    Also lässt sich jede offene Menge $O \subset X$ als abzählbare Vereinigung von Elementen aus $\mathcal{E}$ darstellen.  
    Dies impliziert 
    $$
    \mathcal{B}(X) = \sigma(\mathcal{E}) = \otimes_{k \in N}\mathcal{B}(X_k).
    $$
    \qed
\end{proof*}
