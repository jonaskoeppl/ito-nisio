\section{Flache Konzentrierung von Wahrscheinlichkeitsmaßen}
Für unsere spätere Betrachtung von Zufallsvariablen mit Werten in einem Banachraum lohnt es sich hier, noch eine weitere Charakterisierung der relativen Kompaktheit von Familien von Wahrscheinlichkeitsmaßen zu diskutieren. 
Der Begriff der \textit{flachen Konzentrierung} von Wahrscheinlichkeitsmaßen wurde zuerst von A.D. de Acosta betrachtet, vgl. \cite{acosta}. 
Die Darstellung hier orientiert sich an \cite{vakhania}. 
\newline \ \newline 
\textbf{Notation und Konventionen} \newline 
Im Folgenden sei $(E, \norm{\cdot})$ ein separabler reeller Banachraum und $(E', \norm{\cdot}_{op})$ der zugehörige Dualraum. Der metrische Raum der Wahrscheinlichkeitsmaße auf $\mathcal{B}(E)$ werde mit $(\mathcal{M}(E), \rho)$ bezeichnet, wobei 
$\rho$ die durch $(1.6)$ definierte Prokhorov-Metrik ist. Für ein Maß $\mu$ auf $\mathcal{B}(E)$, einen messbaren Raum $(X, \mathcal{X})$ und eine messbare Abbildung $f:E \to X$ bezeichne $\mu^f$ das \textit{Bildmaß} von $\mu$ unter $f$. Dieses ist definiert als 
$$
    \mu^f(A) := \mu(f^{-1}(A)), \quad A \in \mathcal{X}. 
$$
Es sei ferner daran erinnert, dass jeder endlichdimensionale Untervektorraum  $S \subseteq E$ abgeschlossen ist und eine Menge $A \subseteq S$ nach dem Satz von Heine-Borel genau dann kompakt ist, wenn sie abgeschlossen und beschränkt ist. 
Insbesondere besitzt jede beschränkte Folge in $S$ eine konvergente Teilfolge. 
\begin{mydef}
    Eine Menge $M \subseteq \mathcal{M}(E)$ von Wahrscheinlichkeitsmaßen auf $\mathcal{B}(E)$ heißt \textit{flach konzentriert}, falls es für alle $\varepsilon > 0$ einen endlichdimensionalen 
    Untervektorraum $S \subseteq E$ gibt mit 
    $$
        \forall \mu \in M: \quad \mu(S^{\varepsilon}) \geq 1 - \varepsilon.
    $$ 
\end{mydef}

\begin{lemma}
    Eine Teilmenge $A$ von $E$ ist genau dann relativ kompakt, wenn $A$ beschränkt ist und es für alle $\varepsilon > 0$ einen endlichdimensionalen Untervektorraum $S \subseteq E$ gibt mit 
    \begin{align*}
        A \subseteq S^{\varepsilon}. 
    \end{align*}
\end{lemma}

\begin{proof*}
    Zu $\Rightarrow$: Sei $A \subseteq E$ relativ kompakt. Dann ist $\overline{A}$ kompakt und folglich beschränkt, woraus wir direkt die Beschränktheit von $A$ erhalten. 
    Ferner ist $\overline{A}$ separabel, also existiert eine abzählbare dichte Teilmenge $\{x_1, x_2,...\} \subseteq \overline{A}$. 
    Für $\varepsilon > 0$ ist $(B(x_n, \varepsilon))_{n \in \N}$ somit eine offene Überdeckung von $\overline{A}$. Wegen der Kompaktheit von $\overline{A}$ existiert $I \subseteq \N$ endlich mit 
    $$
        \overline{A} \subseteq \bigcup_{i \in I}B(x_i, \varepsilon).
    $$
    Sei daher $S$ der von $\{x_i : i \in I\}$ erzeugte endlichdimensionale Untervektorraum von $E$. Dann gilt 
    $$
        A \subseteq \overline{A} \subseteq \bigcup_{i \in I}B(x_i, \varepsilon) \subseteq S^{\varepsilon}.
    $$
    Zu $\Leftarrow$: Wir zeigen, dass jede Folge in $A$ eine konvergente Teilfolge besitzt, der Grenzwert der Teilfolge muss hierbei nicht in $A$ liegen. 
    Sei dazu $(x^{(0)}_n)_{n \in \N}$ eine Folge in $A$, $\varepsilon > 0$ und $S \subseteq E$ ein endlichdimensionaler Untervektorraum mit $A \subseteq S^{\varepsilon}$. 
    Dann existiert insbesondere eine Folge $(y_n)_{n \in \N}$ in $S$ mit 
    $$
        \forall n \in \N: \quad d(x^{(0)}_n, y_n) \leq \varepsilon.
    $$
    Aus der Beschränktheit von $(x^{(0)}_n)_{n \in \N}$ erhalten wir direkt die Beschränktheit von $(y_n)_{n \in \N}$ und da $S$ endlichdimensional ist, existiert eine konvergente Teilfolge $(y_{n_k})_{k \in \N}$.
    Es existiert demnach $N(\varepsilon) \in \N$, sodass für $k,m \geq N(\varepsilon) \in \N$
    \begin{align*}
        \norm{x^{(0)}_{n_k} - x^{(0)}_{n_m}} \leq \norm{x^{(0)}_{n_k} - y_{n_k}} + \norm{y_{n_k} - y_{n_m}} + \norm{x^{(0)}_{n_m} - y_{n_m}} \leq 3\varepsilon
    \end{align*}
    gilt. 
    Durch Entfernen endlich vieler Folgenglieder können wir ohne Einschränkung annehmen, dass 
    $$
        \forall k,m \in \N: \quad \norm{x^{(0)}_{n_k} - x^{(0)}_{n_m}} \leq 3\varepsilon. 
    $$
    Durch obiges Verfahren können wir für $N \in \N$ und $\varepsilon_N = \frac{1}{N}$ induktiv eine Teilfolge $(x^{(N)}_n)_{n \in \N}$ von $(x^{(N-1)}_n)_{n \in \N}$ gewinnen mit
    $$
        \forall m,n \in \N: \quad \norm{x^{(N)}_n - x^{(N)}_m} \leq \frac{3}{N}.
    $$
    Durch Bilden der Diagonalfolge $(x^{(N)}_N)_{N \in \N}$ erhalten wir somit eine Teilfolge der Ausgangsfolge $(x^{(0)}_n)_{n \in \N}$, die eine Cauchy-Folge ist und daher in $E$ konvergiert. 
    \qed 
\end{proof*}

\begin{lemma}
    Sei $S \subseteq E$ ein endlichdimensionaler Untervektorraum und $f_1,...,f_n \in E'$ Funktionale mit 
    \begin{align}
        \forall x,y \in S: \ x \neq y \ \Rightarrow \ \exists k \in \{1,...,n\}: \quad f_k(x) \neq f_k(y).
    \end{align}
    Dann ist die Menge 
    $$
        B := \overline{S^{\varepsilon}} \cap \{x \in E: \abs{f_1(x)} \leq r_1,...,\abs{f_n(x)}\leq r_n\}
    $$
    für alle $\varepsilon, r_1,...,r_n \in (0, \infty)$ beschränkt. 
\end{lemma}

\begin{proof*}
    Wegen $(1.12)$ definiert 
    $$
        p(x) := \max_{1\leq k \leq n} \abs{f_k(x)}, \quad x \in S,
    $$
    eine Norm auf $S$. Da $S$ endlichdimensional ist, ist diese insbesondere äquivalent zur Einschränkung von $\norm{\cdot}$ auf $S$. 
    Angenommen die Menge $B$ ist nicht beschränkt. Dann existiert eine Folge $(x_m)_{m \in \N}$ in $B$ mit 
    $$
        \lim_{m \to \infty} \norm{x_m} = \infty. 
    $$
    Wegen $B \subseteq \overline{S^{\varepsilon}}$ gibt es somit eine Folge $(y_m)_{m \in \N}$ in $S$ mit 
    $$
        \forall m \in \N: \quad \norm{x_m - y_m} \leq \varepsilon. 
    $$
    Mittels der Dreiecksungleichung erhält man daraus
    $$
        \lim_{m \to \infty}\norm{y_m} = \infty. 
    $$
    Wegen der Normäquivalenz existiert folglich ein $k \in \{1,...,n\}$ mit 
    \begin{align}
        \lim_{m \to \infty}\abs{f_k(y_m)} = \infty. 
    \end{align}
    Da $f_1,...,f_n$ stetig und linear sind, existiert ferner ein $K > 0$ mit 
    $$
        \forall j \in \{1,...,n\} \ \forall x,y \in E: \quad \abs{f_j(x)-f_j(y)} \leq K \norm{x-y}. 
    $$
    Nach Definition von $B$ gilt 
    $$
        \forall m \in \N: \quad \abs{f_k(x_m)} \leq r_k.
    $$
    Also folgt für alle $m \in \N$ aus der Dreiecksungleichung
    $$
        \abs{f_k(y_m)} \leq \abs{f_k(y_m) - f_k(x_m)} + \abs{f_k(x_m)} \leq K \varepsilon + r_k. 
    $$
    Im Widerspruch zu $(1.13)$. \qed
\end{proof*}

\begin{remark}%TODO: evtl. weiter nach oben verschieben und noch erklären wieso es für S endlich viele derartige Funktionale gibt. 
    Sei $S \subseteq E$ ein endlichdimensionaler Untervektorraum von $E$ mit $dim(S) = r$. 
    Man sagt eine Familie von Abbildungen mit der Eigenschaft $(1.12)$ \textit{trenne die Punkte von S}. 
    Ist $\{x_1,...,x_r\}$ eine Basis von $S$, so lässt sich jedes Element $x \in S$ eindeutig darstellen als $x = \sum_{i=1}^r \lambda_i(x) x_i$, wobei $\lambda_1(x),...,\lambda_r(x) \in \R$.
    Definiere nun für $k \in \{1,...,r\}$
    $$
        \tilde{f}_k: S \to \R, \quad x \mapsto \lambda_k(x).
    $$
    Dann sind $\tilde{f}_1,...,\tilde{f}_r$ offensichtlich linear und da $S$ endlichdimensional ist, folgt daraus bereits die Stetigkeit. 
    Ferner trennen $\tilde{f}_1,...,\tilde{f}_r$ die Punkte von $S$, da $\{x_1,...,x_r\}$ eine Basis von $S$ ist. Mit Satz $A.9$ erhalten wir nun Funktionale $f_1,...,f_r \in E'$ mit 
    $f_j|S = \tilde{f}_j$ für alle $j =1,...,r$. Da $\tilde{f}_1,...,\tilde{f}_r$ die Punkte von S trennen, gilt dies insbesondere auch für $f_1,...,f_r \in E'$. 
    Wir haben also gezeigt, dass es zu jedem endlichdimensionalen Untervektorraum $S$ von $E$ tatsächlich endlich viele stetige lineare Funktionale gibt, die die Punkte von $S$ trennen. 
\end{remark}

\begin{theorem}[Satz von de Acosta]
    Eine Menge $M \subseteq \mathcal{M}(E)$ von Wahrscheinlichkeitsmaßen ist genau dann relativ kompakt in $(\mathcal{M}(E), \rho)$, wenn die folgenden beiden Bedingungen erfüllt sind:
    \begin{enumerate}[(a)]
        \item Für alle $f \in E'$ ist $\{\mu^f : \mu \in M\} \subseteq \mathcal{M}(\R)$ relativ kompakt und
        \item $M$ ist flach konzentriert. 
    \end{enumerate}
\end{theorem}

\begin{proof*}
    Zu $\Rightarrow$: 
    Sei $M \subseteq \mathcal{M}(E)$ relativ kompakt. Nach Satz $1.27$ ist $M$ dann insbesondere gleichmäßig straff. 
    Daher existiert zu $\varepsilon > 0$ eine kompakte Menge $K \subseteq E$ mit 
    $$
        \forall \mu \in M: \quad \mu(K) \geq 1 - \varepsilon.   
    $$ 
    Da alle $f \in E'$ stetig sind, ist jeweils auch $f(K)$ kompakt und es gilt
    $$
        \forall \mu \in M: \quad \mu^f(f(K)) = \mu(f^{-1}(f(K))) \geq \mu(K) \geq 1 - \varepsilon. 
    $$
    Also ist $\{\mu^f : \mu \in M\}$ für alle $f \in E'$ gleichmäßig straff. Erneutes Anwenden von Satz $1.27$  liefert $(a)$. 
    Da $K$ insbesondere relativ kompakt ist, liefert Lemma $1.30$ einen endlichdimensionalen Untervektorraum $S \subseteq E$ mit $K \subseteq S^{\varepsilon}$. Somit gilt
    $$
        \forall \mu \in M: \quad \mu(S^{\varepsilon}) \geq \mu(K) \geq 1 - \varepsilon.
    $$
    Folglich ist $M$ flach konzentriert. 
    \newline 
    Zu $\Leftarrow$: 
    Sei $\varepsilon > 0$. Dann existiert für alle $n \in \N$ ein endlichdimensionaler Untervektorraum $S_{n, \varepsilon} \subseteq E$ mit 
    \begin{align}
        \forall \mu \in M: \quad \mu(S_{n, \varepsilon}^{\frac{\varepsilon}{2^{n+1}}}) \geq 1 - \frac{\varepsilon}{2^{n+1}}.
    \end{align}
    Nach Bemerkung $1.32$ existieren $f_1^{(n)},...,f_{k_n}^{(n)} \in E'$ mit 
    $$
        \forall x,y \in S_{n, \varepsilon}: \quad x \neq y \Rightarrow \ \exists j \in \{1,...,k_n\}: \quad f_j(x) \neq f_j(y). 
    $$
    Nach Voraussetzung $(a)$ können wir zudem $r_1^{(n)},...,r_{k_n}^{(n)} \in (0, \infty)$ so wählen, dass
    \begin{align}
        \inf_{\mu \in M} \mu^{f_i}\big([-r_i^{(n)}, r_i^{(n)}]\big) \geq 1 - \frac{\varepsilon}{k_n 2^{n+1}}, \quad i=1,...,k_n.
    \end{align}
    Setze ferner
    $$
        F_{n,\varepsilon} := \overline{\big(S_{n,\varepsilon}^{\frac{\varepsilon}{2^{n+1}}}\big)}.
    $$
    Dann ist die Menge 
    $$
        K := \bigcap_{n =1}^{\infty}\big(F_{n, \varepsilon} \cap\{x \in E: \ \abs{f_1^{(n)}}\leq r_1^{(n)},...,\abs{f_{k_n}^{(n)}}\leq r_{k_n}^{(n)}\}\big)
    $$
    nach Lemma $1.31$ beschränkt und als Schnitt abgeschlossener Mengen abgeschlossen. Ferner gilt für alle $n \in \N$
    $$
        K \subseteq  F_{n, \varepsilon} \subseteq S_{n,\varepsilon}^{\frac{\varepsilon}{2^n}}.
    $$
    Also ist $K$ nach Lemma $1.30$ kompakt. Zudem gilt für $\mu \in M$ nach $(1.14)$ und $(1.15)$
    \begin{align*}
        \mu(K^c) &\leq \sum_{n=1}^{\infty}\mu(F_{n, \varepsilon}^c) + \sum_{i=1}^{k_n}\mu^{f_i^{(n)}}\big([-r_i^{(n)}, r_i^{(n)}]^c\big) \\\
                 &\leq \sum_{n=1}^{\infty}\mu(F_{n, \varepsilon}^c) + \frac{\varepsilon}{2^{n+1}} \\\
                 &\leq \sum_{n=1}^{\infty}(\frac{\varepsilon}{2^{n+1}} + \frac{\varepsilon}{2^{n+1}}) = \varepsilon. 
    \end{align*}
    Folglich ist $M$ gleichmäßig straff und somit nach Satz $1.27$ relativ kompakt. \qed
\end{proof*}

\begin{corollary}
    Eine Menge $M \subseteq \mathcal{M}(E)$ ist genau dann relativ kompakt, wenn $M$ flach konzentriert ist und für alle $\varepsilon > 0$ eine beschränkte Menge $L \subseteq E$ existiert mit
    $$
        \forall \mu \in M: \quad \mu(L) \geq 1 - \varepsilon. 
    $$
\end{corollary}

\begin{proof*}
    Zu $\Rightarrow$: Nach Satz $1.27$ und Satz $1.33$ ist $M$ flach konzentriert und gleichmäßig straff. Somit existiert für jedes $\varepsilon > 0$ eine kompakte Menge $K \subseteq E$ mit 
    $$
        \forall \mu \in M: \quad \mu(K) \geq 1 - \varepsilon. 
    $$
    Als kompakte Menge ist $K$ insbesondere beschränkt und die Behauptung ist gezeigt.
    \newline  
    Zu $\Leftarrow$: Nach Voraussetzung ist $M$ flach konzentriert, es genügt daher nach Satz $1.33$ zu zeigen, dass $\{\mu^f : \ \mu \in M\}$ für alle $f \in E'$ relativ kompakt ist. Sei dazu $\varepsilon > 0$ und $f \in E'$. 
    Dann existiert eine beschränkte Teilmenge $L \subseteq E$ mit
    $$
        \forall \mu \in M: \quad \mu(L) \geq 1 - \varepsilon.
    $$
    Da $f$ ein stetiger linearer Operator ist, existiert ein $R > 0$ mit 
    $$
        \forall x \in E: \quad \abs{f(x)} \leq R \norm{x}. 
    $$
    Somit folgt aus der Beschränktheit von $L$, dass $f(L)$ eine beschränkte Teilmenge von $\R$ ist und nach dem Satz von Heine-Borel ist $\overline{f(L)}$ kompakt. Es gilt ferner
    $$
        \forall \mu \in M: \quad \mu^f(\overline{f(L)}) = \mu(\{x \in E: \ f(x) \in \overline{f(L)}\}) \geq \mu(L) \geq 1 - \varepsilon. 
    $$
    Also ist $\{\mu^f : \ \mu \in M\}$ gleichmäßig straff und nach Satz $1.27$ relativ kompakt. \qed
\end{proof*}

