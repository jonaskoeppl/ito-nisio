\section{Flache Konzentrierung von Wahrscheinlichkeitsmaßen}
Da wir uns im Laufe der Arbeit hauptsächlich mit Zufallsvariablen auf Banachräumen beschäftigen werden, wollen wir noch eine weitere für diesen Kontext nützliche Charakterisierung der relativen Kompaktheit betrachten. 
Quelle: \cite{vakhania}, Originalpaper: de Acosta \textbf{(TODO: zitieren)}
Für eine nichtleere Teilmenge $A \subseteq E$ setze 
$$
    A^{\varepsilon} := \{x \in E: \inf_{y\in A}\norm{x-y} < \varepsilon \}.
$$
Ferner sei daran erinnert, dass jeder endlich dimensionale Untervektorraum  $S \subseteq E$ abgeschlossen ist und eine Menge $A \subseteq S$ genau dann kompakt ist, wenn sie abgeschlossen und beschränkt ist. 
Insbesondere besitzt jede beschränkte Folge in $S$ eine konvergente Teilfolge. 
\begin{mydef}
    Eine Familie $M \subseteq \mathcal{M}(E)$ von Wahrscheinlichkeitsmaßen auf $\mathcal{B}(E)$ heißt \textit{flach konzentriert}, falls es für alle $\varepsilon > 0$ einen endlichdimensionalen 
    Untervektorraum $S \subseteq E$ gibt mit 
    $$
        \forall \mu \in M: \quad \mu(S^{\varepsilon}) \geq 1 - \varepsilon.
    $$ 
\end{mydef}

\begin{lemma}
    Eine Teilmenge $A$ von $E$ ist genau dann relativ kompakt, wenn $A$ beschränkt ist und es für alle $\varepsilon > 0$ einen endlichdimensionalen Untervektorraum $S \subseteq E$ gibt mit 
    \begin{align*}
        A \subseteq S^{\varepsilon}
    \end{align*}
\end{lemma}

\begin{proof*}
    zu $\Rightarrow$: Sei $A \subseteq E$ relativ kompakt. Dann ist $\overline{A}$ kompakt und folglich beschränkt, woraus wir direkt die Beschränktheit von $A$ erhalten. 
    Ferner ist $\overline{A}$ separabel, also existiert eine abzählbare dichte Teilmenge $\{x_1, x_2,...\} \subseteq A$. 
    Für $\varepsilon > 0$ ist daher $(B(x_n, \varepsilon))_{n \in \N}$ eine offene Überdeckung von $\overline{A}$. Wegen der Kompaktheit existiert $I \subseteq \N$ endlich mit 
    $$
        \overline{A} \subseteq \bigcup_{i \in I}B(x_i, \varepsilon).
    $$
    Sei also $S$ der von $\{x_i : i \in I\}$ erzeugte endlich dimensionale Untervektorraum von $E$. Dann gilt 
    $$
        A \subseteq \overline{A} \subseteq \bigcup_{i \in I}B(x_i, \varepsilon) \subseteq S^{\varepsilon}.
    $$
    zu $\Leftarrow$: Wir zeigen, dass jede Folge in $A$ eine konvergente Teilfolge besitzt, der Grenzwert der Teilfolge muss hierbei nicht in $A$ liegen. 
    Sei dazu $(x^{(0)}_n)_{n \in \N}$ eine Folge in $A$, $\varepsilon > 0$ und $S \subseteq E$ ein endlichdimensionaler Untervektorraum mit $A \subseteq S^{\varepsilon}$. 
    Dann existieren insbesondere eine Folge $(y_n)_{n \in \N}$ in $S$ mit 
    $$
        \forall n \in \N: \quad d(x^{(0)}_n, y_n) \leq 2\varepsilon.
    $$
    Aus der Beschränktheit von $(x^{(0)}_n)_{n \in \N}$ erhalten wir direkt die Beschränktheit von $(y_n)_{n \in \N}$ und da $S$ endlichdimensional ist existiert eine konvergente Teilfolge $(y_{n_k})_{k \in \N}$.
    Es gilt also für $k,m \geq N(\varepsilon) \in \N$
    \begin{align*}
        \norm{x^{(0)}_{n_k} - x^{(0)}_{n_m}} \leq \norm{x^{(0)}_{n_k} - y_{n_k}} + \norm{y_{n_k} - y_{n_m}} + \norm{x^{(0)}_{n_m} - y_{n_m}} \leq 5\varepsilon. 
    \end{align*}
    Ohne Einschränkung können wir durch entfernen endlich vieler Folgenglieder annehmen, dass 
    $$
        \forall k,m \in \N: \quad \norm{x^{(0)}_{n_k} - x^{(0)}_{n_m}} \leq 5\varepsilon. 
    $$
    Durch obiges Verfahren können wir für $N \in \N$ und $\varepsilon_N = \frac{1}{N}$ induktiv eine Teilfolge $(x^{(N)}_n)_{n \in \N}$ von $(x^{(N-1)}_n)_{n \in \N}$ gewinnen mit
    $$
        \forall m,n \in \N: \quad \norm{x^{(N)}_n - x^{(N)}_m} \leq \frac{5}{N}.
    $$
    Durch bilden der Diagonalfolge $(x^{(N)}_N)_{N \in \N}$ erhalten wir somit eine Teilfolge der Ausgangsfolge $(x^{(0)}_n)_{n \in \N}$ die eine Cauchy-Folge ist und daher in $E$ konvergiert. 
    \qed 
\end{proof*}

\begin{lemma}
    Sei $S \subseteq E$ ein endlichdimensionaler Untervektorraum und $l_1,...,l_n \in E'$ Funktionale mit 
    \begin{align}
        \forall x,y \in S \ \exists k \in \{1,...,n\}: \quad l_k(x) \neq l_k(y).
    \end{align}
    Dann ist die Menge 
    $$
        B := S^{\varepsilon} \cap \{x \in E: \abs{l_1(x)} \leq r_1,...,\abs{l_n(x)}\leq r_n\}
    $$
    für alle $\varepsilon, r_1,...,r_n \in (0, \infty)$ beschränkt. 
\end{lemma}

\begin{proof*}
    Wegen $(1.4)$ definiert 
    $$
        p(x) := \max_{1\leq k \leq n} \abs{l_k(x)}, \quad x \in S,
    $$
    eine Norm auf $S$. Da $S$ endlichdimensional ist, ist diese insbesondere äquivalent zur Einschränkung von $\norm{\cdot}$ auf $S$. 
    Angenommen die Menge $B$ ist nicht beschränkt. Dann existiert eine Folge $(x_n)_{n \in \N}$ in $B$ mit 
    $$
        \lim_{n \to \infty} \norm{x_n} = \infty. 
    $$
    Wegen der Normäquivalenz existiert dann ein $k \in \{1,...,n\}$ mit 
    $$
        \lim_{n \to \infty}\abs{l_k(x_n)} = \infty. 
    $$
    Im Widerspruch zur Definition von $B$. \qed
\end{proof*}

\begin{theorem}
    Sei $\Gamma \subseteq E'$, sodass 
    \begin{align}
        \forall x,y \in E \ \exists l \in \Gamma: \quad l(x) \neq l(y).
    \end{align}
    Eine Menge $M \subseteq \mathcal{M}(E)$ von Wahrscheinlichkeitsmaßen ist genau dann relativ kompakt in $(\mathcal{M}, \rho)$ wenn die folgenden beiden Bedingungen erfüllt sind
    \begin{enumerate}[(a)]
        \item Für alle $l \in \Gamma$ ist $\{\mu^l : \mu \in M\} \subseteq \mathcal{M}(\R)$ relativ kompakt,
        \item $M$ ist flach konzentriert. 
    \end{enumerate}
\end{theorem}

\begin{proof*}
    zu $\Rightarrow$: 
    Sei $M \subseteq \mathcal{M}(E)$ relativ kompakt. Nach dem Satz von Prokhorov ist $M$ dann insbesondere gleichmäßig straff. 
    Also existiert zu $\varepsilon > 0$ eine kompakte Menge $K \subseteq E$ mit 
    $$
        \forall \mu \in M: \quad \mu(K) \geq 1 - \varepsilon.   
    $$ 
    Aus der Stetigkeit von $l \in \Gamma$ erhalten wir somit direkt die gleichmäßige Straffheit von $\{\mu^l : \mu \in \Gamma\}$. Erneutes anwenden des Satzes von Prokhorov liefert $(a)$. 
    Da $K$ insbesondere relativ kompakt ist liefert \textbf{Lemma zwei davor, TODO} einen endlichdimensionalen Untervektorraum $S \subseteq E$ mit $K \subseteq S^{\varepsilon}$. Somit gilt
    $$
        \forall \mu \in M: \quad \mu(S^{\varepsilon}) \geq \mu(K) \geq 1 - \varepsilon.
    $$
    Folglich ist $M$ flach konzentriert. 
    \newline 
    zu $\Leftarrow$: 
    \textbf{TODO}
\end{proof*}

\begin{corollary}
    Eine Menge $M \subseteq \mathcal{M}(E)$ ist genau dann relativ kompakt, wenn $M$ flach konzentriert ist und für alle $\varepsilon > 0$ eine beschränkte Menge $L \subseteq E$ existiert mit
    $$
        \forall \mu \in M: \quad \mu(L) \geq 1 - \varepsilon. 
    $$
\end{corollary}

\begin{proof*}
    Die Richtung $\Rightarrow$ ist klar und für $\Leftarrow$ reicht es anzumerken, dass beschränkte Mengen durch lineare Abbildungen auf beschränkte Mengen abgebildet werden. Nach dem Satz von Heine-Borel sind 
    beschränkte Teilmengen endlich dimensionaler normierter Vektorräume aber insbesondere relativ kompakt. \qed 
\end{proof*}

\begin{remark}%TODO: evtl. weiter nach oben verschieben und noch erklären wieso es für S endlich viele derartige Funktionale gibt. 
    Man sagt eine Familie von Abbildungen mit der Eigenschaft $(1.2)$ (bzw. $(1.1)$ ) \textit{trenne die Punkte von E} (bzw $S$). Die Existenz einer solchen Menge $\Gamma \subseteq E'$ ist durch den Satz von Hahn-Banach sichergestellt. 
    Insbesondere erfüllt $E'$ selbst$(1.1)$. 
\end{remark}