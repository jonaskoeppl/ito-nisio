\section{Die Prokhorov Metrik}
Nachdem wir im letzten Abschnitt damit begonnen haben uns mit der schwachen Konvergenz von Wahrscheinlichkeitsmaßen zu beschäftigen, 
wollen wir nun ein weiteres Hilfsmittel zur Untersuchung von schwacher Konvergenz einführen. 
\newline 
Für einen metrischen Raum $(X,d)$ bezeichne $\mathcal{M}(X)$ die Menge aller Wahrscheinlichkeitsmaße auf der Borelschen $\sigma$-Algebra $\mathcal{B}(X)$. 
\begin{mydef}
    Eine Familie $M \subseteq \mathcal{M}(X)$ von Wahrscheinlichkeitsmaßen heißt \textit{gleichmäßig straff}, 
    falls es für alle $\varepsilon > 0$ eine kompakte Menge $K \subseteq X$ gibt mit 
    $$
        \forall \mu \in M: \mu(K) \geq 1-\varepsilon. 
    $$
\end{mydef}
Ziel dieses Abschnitts ist der Satz von Prokhorov, der uns eine für spätere Beweise wichtige Charakterisierung der gleichmäßigen Straffheit liefert. 

Ein wichtiges Resultat ist die Folgende auf Prokhorov zurückgehende Charakterisierung. Ein Beweis findet sich etwa in \cite{parthasarathy}[Theorem 6.7]. 

\begin{theorem}[Satz von Prokhorov]
    Sei $(X,d)$ ein vollständiger separabler metrischer Raum und $M \subseteq \mathcal{M}(X)$. Dann sind äquivalent
    \begin{enumerate}[(i)]
        \item $M$ ist relativ kompakt,
        \item $M$ ist gleichmäßig straff. 
    \end{enumerate}
\end{theorem}