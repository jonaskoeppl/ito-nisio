Ein wichtiges Hilfsmittel zur Untersuchung von Verteilungskonvergenz sind die sogenannten \textit{charakteristischen Funktionale}. 
Sie sind eine Verallgemeinerung der aus dem skalaren Fall bekannten charakteristischen Funktionen.  

\begin{mydef}
    Sei $\mu$ ein Wahrscheinlichkeitsmaß auf $\mathcal{B}(E)$. 
    Das \textit{charakteristische Funktional} $\widehat{\mu}$ von $\mu$ ist definiert durch
    $$
        \widehat{\mu} : E' \to \C, \quad f \mapsto \int_{E}e^{if(x)}\mu(dx).
    $$
    Wegen der Stetigkeit von $e^{i\cdot}$ und $\abs{e^{ix}}\leq 1$ für alle $x \in \R$ ist $\widehat{\mu}$ wohldefiniert. 
    Für eine Zufallsvariable $X \in \mathcal{L}_0(E)$ bezeichne $\widehat{\mu_X}$ das charakteristische Funktional der Verteilung von $X$. In diesem Fall lässt sich die Abbildung auch schreiben als
    $$
        f \mapsto \mathbb{E}(e^{if(X)}), \quad f \in E'. 
    $$
\end{mydef}

\begin{remark}
    Im Fall $E = \R^d$ lässt sich $E'$ auf kanonische Weise mit $\R^d$ identifizieren, daher lassen sich die charakteristischen Funktionale hier in der Form 
    $$
        y \mapsto \int_{\R^d}e^{i\langle x, y\rangle}\mu(dx)
    $$
    schreiben. Nach dem Darstellungssatz von Riesz ist dies auch für allgemeine Hilberträume $E$ möglich. 
\end{remark}

\begin{proposition}
    Sei $\mu$ ein Wahrscheinlichkeitsmaß auf $\mathcal{B}(E)$. Dann ist $\widehat{\mu}:E' \to \C$ stetig bzgl. der Norm $\norm{\cdot}_{op}$. 
\end{proposition}
\begin{proof*}
    Sei $f \in E'$ und $(f_n)_{n \in \N}$ eine Folge in $E'$ mit $\lim_{n \to \infty}\norm{f-f_n}_{op} = 0$. 
    Dann konvergiert $(l_n)_{n \in \N}$ insbesondere punktweise gegen $l$ und mit dem Satz von der dominierten Konvergenz erhalten wir 
    $$
        \lim_{n \to \infty} \widehat{\mu}(f_n) = \int_E\lim_{n \to \infty}e^{if_n(x)}\mu(dx) = \int_E e^{if(x)}\mu(dx) = \widehat{\mu}(f). 
    $$
    \qed
\end{proof*}

Wie schon im skalaren Fall dienen die charakteristischen Funktionale als nützliches Hilfsmittel zur Untersuchung von Wahrscheinlichkeitsmaßen und schwacher Konvergenz.
Es sei daran erinnert, dass nach Proposition $1.34$ für einen separablen Banachraum $E$ die beiden $\sigma$-Algebren $\mathcal{B}(E)$ und $\sigma(E')$ übereinstimmen. Ferner ist 
$$
    \mathcal{C}(E) := \{ \bigcap_{i=1}^n f^{-1}(A): \ n \in \N, \ f_1,...,f_n \in E', \ A_1,...,A_n \in \mathcal{B}(\R) \}.
$$ 
ein schnitt-stabiler Erzeuger von $\sigma(E')$. 

\begin{theorem}[Eindeutigkeitssatz]
    Sei $E$ ein separabler Banachraum und seien $\mu$ und $\nu$ zwei Wahrscheinlichkeitsmaße auf $\mathcal{B}(E)$ mit $\widehat{\mu} = \widehat{\nu}$. Dann gilt $\mu = \nu$.
\end{theorem}

\begin{proof*}
    Bemerke zunächst, dass für ein Wahrscheinlichkeitsmaß $\mu$ auf $\mathcal{B}(X)$, $d \in  \N$, $t=(t_1,...,t_d) \in \R^d$ und $f_1,..., f_d \in E'$ gilt
    $$
        \widehat{\mu}(\sum_{j=1}^d t_j f_j) = \int_E e^{i\sum_{j=1}^dt_jf_j(x)}\mu(dx) = \int_{\R^d} e^{i\langle t, \xi \rangle}\mu^T(d\xi) = \widehat{\mu^T}(t),
    $$
    wobei 
    $$
        T: E \to \R^d, \quad x \mapsto (f_1(x),...,f_d(x)). 
    $$
    Nach Voraussetzung folgt also $\widehat{\mu^T} = \widehat{\nu^T}$ und aus \textbf{APPENDIX} folgt $\mu^T = \nu^T$. 
    Es gilt somit
    $$
        \forall d\in \N \ \forall f_1,...,f_d \in E' \ \forall A \in \mathcal{B}(\R^d): \ \mu((f_1,..f_d)^{-1}(A)) =  \nu((f_1,..f_d)^{-1}(A)). 
    $$
    Also stimmen $\mu$ und $\nu$ insbesondere auf dem schnittstabilen Erzeuger $\mathcal{C}(E)$ von $\mathcal{B}(E)$ und somit auf ganz $\mathcal{B}(E)$ überein. \qed
\end{proof*}

\begin{theorem}
    Sei $(X_n)_{n \in \N}$ eine Folge in $\mathcal{L}_0(E)$ und $X \in \mathcal{L}_0(E)$. 
    Dann konvergiert $(X_n)_{n \in \N}$ genau dann in Verteilung gegen $X$, wenn die folgenden beiden Bedingungen erfüllt sind
    \begin{enumerate}[(a)]
        \item $(\widehat{\mu_{X_n}})_{n \in \N}$ konvergiert punktweise gegen $\widehat{\mu_X}$ und
        \item $(X_n)_{n \in \N}$ ist gleichmäßig straff. 
    \end{enumerate}
\end{theorem}

\begin{proof*}
    Zu $\Rightarrow$: Für $n \in \N$ sei $\mu_n$ die Verteilung von $X_n$ und $\mu$ die Verteilung von $X$. Da die Folge $(\mu_n)_{n \in \N}$ nach Voraussetzung schwach gegen $\mu$ konvergiert, 
    ist $(\mu_n)_{n \in  \N}$ insbesondere relativ kompakt in $(\mathcal{M}(E), \rho)$, und nach dem Satz von Prokhorov gleichmäßig straff. 
    Ferner ist für $l \in E'$ die Abbildung 
    $$
        g_l: E \to \C,  \ x \mapsto e^{il(x)}
    $$ 
    beschränkt und stetig. Durch Zerlegung in Real und Imaginärteil liefern dadurch die Linearität des Integrals und die Definition der schwachen Konvergenz die puntkweise Konvergenz der charakteristischen Funktionale. 
    \newline 
    Zu $\Leftarrow$: Nach Voraussetzung ist  $\{\mu_n : n \in \N\}$ gleichmäßig straff, also nach dem Satz von Prokhorov relativ kompakt. Es genügt also zu zeigen, dass jede konvergente Teilfolge von $(\mu_n)_{n \in \N}$ gegen $\mu$ konvergiert. 
    Sei dazu $(\mu_{n_k})_{k \in \N}$ eine konvergente Teilfolge und $\nu \in \mathcal{M}(E)$ mit $\mu_{n_k} \rightharpoonup \nu$. Nach Voraussetzung und der Hinrichtung gilt also 
    $$
        \forall l \in E': \ \widehat{\mu}(l) = \lim_{k \to \infty} \widehat{\mu_{n_k}}(l) = \widehat{\nu}(l).
    $$
    Der Eindeutigkeitssatz liefert nun $\mu = \nu$ und somit die Behauptung. \qed
\end{proof*}

\begin{remark}
    Im endlichdimensionalen Fall muss die gleichmäßige Straffheit von $(\mu_n)_{n \in \N}$ nicht zusätzlich gefordert werden, vgl. \cite{gs}[Theorem 8.7.5, S.357]. 
\end{remark}

Bevor wir uns im Folgenden Kapitel mit der Konvergenz zufälliger Reihen in Banachräumen beschäftigen, halten wir noch zwei für den späteren Beweis des Satzes von Itô-Nisio nützliche Ergebnisse fest, die sich mit unserem bisher gesammelten Wissen über charakteristische Funktionale leicht beweisen lassen.  

\begin{proposition}
    Seien $(X_n)_{n \in \N}$ und $(Y_n)_{n \in \N}$ Folgen in $\mathcal{L}_0(E)$ und $X,Y \in \mathcal{L}_0(E)$ mit $X_n \stochastisch X$, sowie $Y_n \stochastisch Y$. 
    Falls für alle $n \in \N$ $X_n$ und $Y_n$ unabhängig sind, so sind auch $X,Y$ unabhängig. 
\end{proposition}

\begin{proof*}%TODO: Identifikation Dualraum und letzte Gleichheit in der Kette checken. Geht das evtl. auch elementarer?
    Wegen des Teilfolgenkriteriums für stochastische Konvergenz können wir ohne Einschränkung annehmen, dass $X_n \fastsicher X$ und $Y_n \fastsicher Y$. 
    Betrachte die $(E \times E)$-wertigen Zufallsvariablen $Z_n:=(X_n,Y_n)$, $n \in \N$, und $Z:=(X,Y)$. 
    Identifiziere $(E \times E)^* = E^* \times E^*$ und erhalte mittels dominierter Konvergenz und dem Satz von Fubini für $(f_1,f_2) \in E^* \times E^*$. 
    \begin{align*}
        \widehat{P^Z}(f_1,f_2) = \mathbb{E}(e^{i(f_1(X) + f_2(Y))}) &= \lim_{n \to \infty}\mathbb{E}(e^{i(f_1(X_n) + f_2(Y_n))})  \\\
                                                       &= \lim_{n \to \infty}\mathbb{E}(e^{if_1(X_n)})\mathbb{E}(e^{if_2(Y_n)}) \\\
                                                       &= \mathbb{E}(e^{if_1(X)})\mathbb{E}(e^{if_2(Y)})  \\\
                                                       &= \widehat{P^X}(f_1)\widehat{P^Y}(P)(f_2) = \widehat{P^X \times P^Y}(f_1, f_2).
    \end{align*}
    Der Eindeutigkeitssatz für charakteristische Funktionale liefert unmittelbar $P^X \times P^Y = P^{(X,Y)}$. Also sind $X,Y$ unabhängig. \qed
\end{proof*}

Sei $0_E$ das neutrale Element der Addition in $E$ und $\delta_0$ die Einpunktverteilung auf $0_E$, d.h. 
$$
    \delta_0(A) = 1_A(0), \quad A \in \mathcal{B}(E). 
$$

\begin{proposition}
    Sei $\mu$ ein Wahrscheinlichkeitsmaß auf $\mathcal{B}(E)$. Falls ein $r > 0$ existiert mit
    $$
       \forall f \in E': \  \norm{f}_{op} \leq r \ \Rightarrow \ \widehat{\mu}(f) = 1,
    $$
    so gilt $\mu = \delta_0$. 
\end{proposition}

\begin{proof*}
    Wähle $f \in E'$ beliebig aber fest und betrachte die Abbildung 
    $$
        \phi : \R \to \C, \ t \mapsto \widehat{\mu}(tf). 
    $$
    Dann ist $\phi$ die charakteristische Funktion von $\mu^{f}$ und nach Voraussetzung gilt $\phi(t) = 1$ für alle $t \in \R$ mit $\abs{t} \leq \frac{r}{\norm{f}_{op}}$. 
    Weiter gilt für $s,t \in \R$
    \begin{align*}
        \abs{\phi(t) - \phi(s)} = \bigg\lvert\int_E e^{isf(x)}(e^{i(t-s)f(x)} - 1)\mu(dx)\bigg\rvert \leq \int_E\big\lvert e^{isf(x)}(e^{i(t-s)f(x)} - 1)\big\rvert\mu(dx).
    \end{align*}
    Per Hölder-Ungleichung und der Definition des Absolutbetrags erhalten wir 
    \begin{align*}
        \int_E\big\lvert e^{isf(x)}(e^{i(t-s)f(x)} - 1)\big\rvert\mu(dx) &\leq \bigg(\int_E\big\lvert e^{isf(x)}(e^{i(t-s)f(x)} - 1)\big\rvert^2\mu(dx)\bigg)^{\frac{1}{2}} \\\
                                                          &= \sqrt{2 - \phi(t-s) - \overline{\phi(t-s)}} \\\
                                                          &\leq \sqrt{2\abs{1-\phi(t-s)}},
    \end{align*}
    und somit insgesamt 
    $$
        \abs{\phi(t) - \phi(s)} \leq \sqrt{2\abs{1-\phi(t-s)}}.
    $$
    Also muss $\phi$ konstant sein mit $\phi(t) = 1$ für alle $t \in \R$. Da $f$ beliebig gewählt war erhalten wir
    $$
        \forall f \in E': \quad \widehat{\mu}(f) = 1 = \int_E e^{if(x)}\delta_0(dx) = \widehat{\delta_0}(f). 
    $$
    Der Eindeutigkeitssatz für charakteristische Funktionale liefert nun $\mu = \delta_0$. \qed
\end{proof*}

