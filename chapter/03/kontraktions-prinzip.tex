\section{Das Kontraktions Prinzip}
Im Folgenden sei $(X_n)_{n \in \N}$ eine unabhängige Folge symmetrischer Zufallsvariablens aus $\mathcal{L}_0(E)$ und $(\lambda_n)_{n \in \N}$ eine beschränkte Folge in $\R$. 
Für $n \in \N$ setze
$$
    S_n := \sum_{i=1}^nX_i, \quad T_n = \sum_{i=1}^n\lambda_iX_i. 
$$
Seien ferner $S_0 = T_0 = 0$. 
\begin{lemma}
    Für alle $t > 0$ und $N \in \N$ gilt
    \begin{align}
        \prob{\norm{T_N} > t} \leq 2 \prob{\norm{S_N} > t}. 
    \end{align}
\end{lemma}

\begin{proof*}
    Wegen der Symmetrie von $(X_n)_{n \in \N}$ sind $(\lambda_n X_n)_{n \in \N}$ und $(\abs{\lambda_n}X_n)_{n \in \N}$ identisch verteilt. Ferner ist $(\lambda_n)_{n \in \N}$ nach Voraussetzung beschränkt. 
    Wir können also ohne Einschränkung annehmen, dass $0 \leq \lambda_n \leq 1$ für alle $n \in \N$. 
    \newline 
    \underline{Fall A}: $\lambda_1 \geq \lambda_2 \geq... \geq \lambda_N$. 
    \newline 
    Per Abelscher Summation erhalten wir 
    $$
        T_N = \sum_{n=1}^N \lambda_n(S_n - S_{n-1}) = \sum_{n=1}^{N-1}(\lambda_n - \lambda_{n+1})S_n + \lambda_N \S_N. 
    $$
    Mit der Dreiecksungleichung erhalten wir 
    \begin{align*}
        \norm{T_N} &\leq \sum_{n=1}^{N-1}(\lambda_n - \lambda_{n+1})\norm{S_n} + \lambda_N \norm{S_N} \\\
                   &\leq \max_{1 \leq n \leq N}\norm{S_n} \big(\sum_{n=1}^{N-1}(\lambda_n - \lambda_{n+1}) + \lambda_N\big) \\\
                   &= \max_{1\leq n \leq N}\norm{S_n} \lambda_1 \leq \max_{1\leq n \leq N}\norm{S_n}.
    \end{align*}
    Mit Lévys Maximalungleichung erhalten wir nun
    $$
        \prob{\norm{T_N} > t} \leq \prob{\max_{1 \leq n \leq N}\norm{S_n} > t} \leq 2 \prob{\norm{S_N} > t}.
    $$
    \underline{Fall B}: Allgemeiner Fall. 
    \newline 
    Mittels einer Permutation $\sigma$ erhält man $\lambda_{\sigma(1)} \geq ... \geq \lambda_{\sigma(N)}$. Man beachte schließlich, dass 
    $$
        \sum_{n=1}^N\lambda_{\sigma(n)}X_{\sigma(n)} = T_N \quad \text{ und } \quad \sum_{n=1}^N X_{\sigma(n)} = S_N.
    $$ 
    Aus Fall A folgt nun die Behauptung. \qed
\end{proof*}

\begin{theorem}[Kontraktions-Prinzip, Qualitative Version]
    Falls $(S_n)_{n \in \N}$ fast sicher konvergiert, dann konvergiert auch $(T_n)_{n \in \N}$ fast sicher. 
\end{theorem}
\begin{proof*}
    Sei $\varepsilon > 0$. Nach Lemma $3.11$ gilt für $m < n$
    $$
        \prob{\norm{T_n - T_m} > \varepsilon} \leq 2\prob{\norm{S_n - S_m} > \varepsilon}. 
    $$
    Nach dem Cauchy-Kriterium für stochastische Konvergenz konvergiert $(T_n)_{n \in \N}$ also stochastisch und nach dem Satz von Itô-Nisio insbesondere fast sicher. \qed
\end{proof*}

\begin{remark}
    Auf die Bedingung der Symmetrie kann nicht verzichtet werden, wie uns die deterministische Folge $X_n = (-1)^n \frac{1}{n}$ und $(\lambda_n)_{n \in \N} = ((-1)^{n})_{n \in \N}$ zeigen. 
    \newline 
    \textbf{TODO: Weitere Gegenbeispiele siehe Li,Queffelec}
\end{remark}