Wir wollen nun noch eine zweite Anwendung von Satz $3.9$ geben, welche in der Literatur teilweise als \textit{qualitative Version des Kontraktions-Prinzips} bezeichnet wird. Die Darstellung orientiert sich hierbei an \cite{li-queffelec}. 
Im Folgenden sei $(X_n)_{n \in \N}$ eine unabhängige Folge symmetrischer Radon-Zufallsvariablen aus $\mathcal{L}_0(E)$ und $(\lambda_n)_{n \in \N}$ eine beschränkte Folge in $\R$. 
Für $n \in \N$ setze
$$
    S_n := \sum_{i=1}^nX_i, \quad T_n = \sum_{i=1}^n\lambda_iX_i. 
$$
Aus technischen Gründen seien ferner $S_0 = T_0 = 0$. 
Unser Ziel ist es, zu zeigen, dass die fast sichere Konvergenz von $(S_n)_{n \in \N}$ auch die fast sichere Konvergenz von $(T_n)_{n \in \N}$ impliziert. 
Der Beweis beruht hauptsächlich auf der folgenden Abschätzung. 
\begin{lemma}
    Für alle $t > 0$ und $N \in \N$ gilt
    \begin{align}
        \prob{\norm{T_N} > t} \leq 2 \prob{\norm{S_N} > t}. 
    \end{align}
\end{lemma}

\begin{proof*}
    Wegen der Symmetrie von $X_n$ sind $\lambda_n X_n$ und $\abs{\lambda_n}X_n$ für $n \in \N$ identisch verteilt. Ferner ist $(\lambda_n)_{n \in \N}$ nach Voraussetzung beschränkt. 
    Wir können also ohne Einschränkung annehmen, dass $0 \leq \lambda_n \leq 1$ für alle $n \in \N$. 
    \newline 
    \underline{Fall A}: $\lambda_1 \geq \lambda_2 \geq... \geq \lambda_N$. 
    \newline 
    Wie man leicht nachrechnet, gilt  
    $$
        T_N = \sum_{n=1}^N \lambda_n(S_n - S_{n-1}) = \sum_{n=1}^{N-1}(\lambda_n - \lambda_{n+1})S_n + \lambda_N S_N. 
    $$
    Mit der Dreiecksungleichung erhalten wir somit
    \begin{align*}
        \norm{T_N} &\leq \sum_{n=1}^{N-1}(\lambda_n - \lambda_{n+1})\norm{S_n} + \lambda_N \norm{S_N} \\\
                   &\leq \max_{1 \leq n \leq N}\norm{S_n} \bigg(\sum_{n=1}^{N-1}(\lambda_n - \lambda_{n+1}) + \lambda_N\bigg) \\\
                   &= \max_{1\leq n \leq N}\norm{S_n} \lambda_1 \leq \max_{1\leq n \leq N}\norm{S_n}.
    \end{align*}
    Mit Lévys Maximalungleichung $(3.1)$ folgt daraus
    $$
        \prob{\norm{T_N} > t} \leq \prob{\max_{1 \leq n \leq N}\norm{S_n} > t} \leq 2 \prob{\norm{S_N} > t}.
    $$
    \underline{Fall B}: Allgemeiner Fall. 
    \newline 
    Mittels einer Permutation $\sigma$ erhält man $\lambda_{\sigma(1)} \geq ... \geq \lambda_{\sigma(N)}$. Man beachte schließlich, dass 
    $$
        \sum_{n=1}^N\lambda_{\sigma(n)}X_{\sigma(n)} = T_N \quad \text{ und } \quad \sum_{n=1}^N X_{\sigma(n)} = S_N.
    $$ 
    Aus Fall A folgt nun die Behauptung. \qed
\end{proof*}

\begin{theorem}[Kontraktions-Prinzip, Qualitative Version]
    Falls $(S_n)_{n \in \N}$ fast sicher konvergiert, dann konvergiert auch $(T_n)_{n \in \N}$ fast sicher. 
\end{theorem}
\begin{proof*}
    Sei $\varepsilon > 0$. Nach Lemma $3.13$ gilt für $m < n$
    \begin{align}
        \prob{\norm{T_n - T_m} > \varepsilon} \leq 2\prob{\norm{S_n - S_m} > \varepsilon}. 
    \end{align}
    Da $(S_n)_{n \in \N}$ fast sicher konvergiert, gilt nach Korollar $2.7$ und Satz $2.9$
    $$
        \lim_{N \to \infty} \sup_{n \geq N}\prob{\norm{S_n-S_N} > \varepsilon} = 0. 
    $$
    Wegen $(3.10)$ folgt also 
    $$
        \lim_{N \to \infty} \sup_{n \geq N}\prob{\norm{T_n-T_N} > \varepsilon} = 0.
    $$
    Da $\varepsilon$ beliebig gewählt war, konvergiert $(T_n)_{n \in \N}$ nach Satz $2.9$ stochastisch und nach Satz $3.9$ insbesondere fast sicher. \qed
\end{proof*}

\begin{remark}
    In Satz $3.15$ kann nicht auf die Symmetrie der Radon-Zufallsvariablen $X_1,X_2,...$ verzichtet werden. 
    Betrachte dazu etwa die Folge $(X_n)_{n \in \N}$, definiert durch
    $$
        X_n(\omega) := (-1)^n \frac{1}{n},  \quad \omega \in \Omega, \ n \in \N, 
    $$ 
    und die beschränkte Folge $((-1)^n)_{n \in \N}$. Dann ist die Folge $(\sum_{k=1}^n X_n)_{n \in \N}$ fast sicher konvergent, aber die Folge $(\sum_{k=1}^n\lambda_k X_k)_{n \in \N}$ divergiert fast sicher. 
    \qexampled
\end{remark}