\section{Maximalungleichungen}

Bezeichne $L_0(E)$ den Vektorraum der E-wertigen-Zufallsvariablen. 

\begin{mydef}
    Eine E-wertige Zufallsvariable $X$ heißt \textit{symmetrisch}, falls $-X$ die selbe Verteilung hat wie $X$, d.h.
    \begin{align*}
        \forall A \in \mathcal{B}(E): P(\{X \in A\}) = P(\{-X \in A\}). 
    \end{align*}
\end{mydef}

\begin{remark}
    Nach dem Eindeutigkeitssatz für charakteristische Funktionale ist eine Zufallsvariable $X \in L_0(E)$ genau dann symmetrisch, wenn 
    $$
        \forall z \in E': \quad \widehat{\mu_X}(z) = \widehat{\mu_{-X}}(z). 
    $$
\end{remark}

\begin{mydef}
    Eine Folge $(X_n)_{n \in \N}$ von E-wertigen Zufallsvariablen heißt \textit{symmetrisch}, 
    falls $(\varepsilon_1 X_1, \varepsilon_2 X_2,...)$ für jede Wahl von $\varepsilon_i = \pm 1$ 
    die gleiche Verteilung hat wie $(X_1,X_2,...)$. 
\end{mydef}

\begin{remark}
   Sind $X_1,X_2,...$ unabhängige E-wertige Zufallsvariablen, sodass $X_n$ für alle $n \in \N$ symmetrisch ist, dann ist $(X_1,X_2,...)$ symmetrisch. 
\end{remark}

\begin{theorem}[Lévys Maximal-Ungleichung]
    Seien $X_1,...,X_N \in L_0(E)$ unabhängige und symmetrische Zufallsvariablen und setze 
    \begin{align*}
        S_n := \sum_{i=1}^n X_i, \quad 1 \leq n \leq N. 
    \end{align*}
    Dann gilt für alle $t > 0$
    \begin{align}
        &P\big(\{ \max_{1 \leq n \leq N} \norm{S_n} > t \}\big) \leq 2 P\big(\{\norm{S_N} > t \}\big), \\\
        &P\big(\{ \max_{1 \leq n \leq N} \norm{X_n} > t \}\big) \leq 2 P\big(\{\norm{S_N} > t \}\big).
    \end{align}
\end{theorem}

\begin{proof*}
    \textbf{TODO}
\end{proof*}

Für nicht-symmetrische Zufallsvariablen erhalten wir mit einer ähnlichen Beweismethode die folgende auf Giuseppe Ottaviani und Anatoli Skorohod zurückgehende Maximal-Ungleichung, vgl. \cite{ledoux-talagrand}[Lemma 6.2]. 
\begin{theorem}[Maximal-Ungleichung von Ottaviani-Skorohod]
    Seien $X_1,...,X_N$ unabhängige $E$-wertige Zufallsvariablen, $N \in \N$. Setze 
    $$
        S_k := \sum_{i=1}^kX_i, \quad k = 1,...,N. 
    $$
    Dann gilt für alle $s,t > 0$
    \begin{align}
        P(\{ \max_{1 \leq k \leq N} \norm{S_k} > s + t \}) \leq \frac{P(\{\norm{S_N} > t \})}{1 - \max_{1 \leq k \leq N}P(\{ \norm{S_N - S_k} > s \})} \ . 
    \end{align}
\end{theorem}

\begin{proof*}
    \textbf{TODO}
\end{proof*}