\chapter{Maximal-Ungleichungen und Konvergenz zufälliger Reihen}
\textbf{Notation und Konventionen}\newline 
Im Folgenden sei $E$ ein separabler Banachraum und $(\Omega, \mathcal{A}, P)$ ein vollständiger Wahrscheinlichkeitsraum. Wegen der Separabilität von $E$ ist jede messbare Abbildung $X: (\Omega, \mathcal{A}) \to (E, \mathcal{B}(E))$ eine Radon-Zufallsvariable. 

\section{Maximalungleichungen}

Bezeichne $L_0(E)$ den Vektorraum der E-wertigen-Zufallsvariablen. 

\begin{mydef}
    Eine E-wertige Zufallsvariable $X$ heißt \textit{symmetrisch}, falls $-X$ die selbe Verteilung besitzt wie $X$, d.h.
    \begin{align*}
        \forall A \in \mathcal{B}(E): P(\{X \in A\}) = P(\{-X \in A\}). 
    \end{align*}
\end{mydef}

\begin{remark}
    Nach dem Eindeutigkeitssatz für charakteristische Funktionale ist eine Zufallsvariable $X \in L_0(E)$ genau dann symmetrisch, wenn 
    $$
        \forall z \in E': \quad \widehat{\mu_X}(z) = \widehat{\mu_{-X}}(z). 
    $$
\end{remark}

\begin{mydef}%TODO: Wird das überhaupt gebraucht? Ggf schöner formulieren. 
    Eine Folge $(X_n)_{n \in \N}$ von E-wertigen Zufallsvariablen heißt \textit{symmetrisch}, 
    falls $(\varepsilon_1 X_1, \varepsilon_2 X_2,...)$ für jede Wahl von $\varepsilon_i = \pm 1$ 
    die gleiche Verteilung hat wie $(X_1,X_2,...)$. 
\end{mydef}

\begin{remark}
   Sind $X_1,X_2,...$ unabhängige E-wertige Zufallsvariablen, sodass $X_n$ für alle $n \in \N$ symmetrisch ist, dann ist $(X_1,X_2,...)$ symmetrisch. 
\end{remark}

\begin{theorem}[Lévys Maximal-Ungleichung]
    Seien $X_1,...,X_N \in L_0(E)$ unabhängige und symmetrische Zufallsvariablen und setze 
    \begin{align*}
        S_n := \sum_{i=1}^n X_i, \quad 1 \leq n \leq N. 
    \end{align*}
    Dann gilt für alle $t > 0$
    \begin{align}
        &P\big(\{ \max_{1 \leq n \leq N} \norm{S_n} > t \}\big) \leq 2 P\big(\{\norm{S_N} > t \}\big), \\\
        &P\big(\{ \max_{1 \leq n \leq N} \norm{X_n} > t \}\big) \leq 2 P\big(\{\norm{S_N} > t \}\big).
    \end{align}
\end{theorem}

\begin{proof*}
    \textbf{TODO}
\end{proof*}

Für nicht-symmetrische Zufallsvariablen erhalten wir mit einer ähnlichen Beweismethode die folgende auf Giuseppe Ottaviani und Anatoli Skorohod zurückgehende Maximal-Ungleichung, vgl. \cite{ledoux-talagrand}[Lemma 6.2]. 
\begin{theorem}[Maximal-Ungleichung von Ottaviani-Skorohod]
    Seien $X_1,...,X_N$ unabhängige $E$-wertige Zufallsvariablen, $N \in \N$. Setze 
    $$
        S_k := \sum_{i=1}^kX_i, \quad k = 1,...,N. 
    $$
    Dann gilt für alle $s,t > 0$
    \begin{align}
        P(\{ \max_{1 \leq k \leq N} \norm{S_k} > s + t \}) \leq \frac{P(\{\norm{S_N} > t \})}{1 - \max_{1 \leq k \leq N}P(\{ \norm{S_N - S_k} > s \})} \ . 
    \end{align}
\end{theorem}

\begin{proof*}
    \textbf{TODO}
\end{proof*}
\newpage
\section{Der Satz von Itô-Nisio}
    Wir wollen nun damit beginnen die bisher erarbeitete Technik zur Untersuchung der Konvergenz zufälliger Reihen in Banachräumen anzuwenden.
    Die vorliegenden Beweise orientieren sich an \cite{ito-nisio}, \cite{ledoux-talagrand}, \cite{li-queffelec} und \cite{van-neerven}.  
    \newline \ \newline 
    Sei $(X_n)_{n \in \N}$ eine Folge unabhängiger Zufallsvariablen in $\mathcal{L}_0(E)$. Für $n \in \N$ setze
    \begin{align*}
    S_n := \sum_{i=1}^n X_i, 
    \quad 
    \mu_n := P^{S_n} .
    \end{align*}
\begin{theorem}[Itô-Nisio]
    Es sind äquivalent
    \begin{enumerate}[(i)]
        \item $(S_n)_{n \in \N}$ konvergiert fast sicher, 
        \item $(S_n)_{n \in \N}$ konvergiert stochastisch, 
        \item $(S_n)_{n \in \N}$ konvergiert in Verteilung. 
    \end{enumerate}
\end{theorem}

\begin{proof*}%TODO: Formatierung% 
    Die Implikationen $(i) \Rightarrow (ii) \Rightarrow (iii)$ wurden bereits in Kapitel 2 gezeigt, es genügt also $(ii) \Rightarrow (i)$ und $(iii) \Rightarrow (ii)$ zu zeigen. 
    \newline 
    Zu $(ii) \Rightarrow (i)$: \underline{Fall A}: Für alle $n \in \N$ ist $X_n$ symmetrisch verteilt. 
    \newline 
    Für $n, N \in \N$ mit $n < N$ setze 
    \begin{align*}
        Y_{n,N} &:= \max_{n < k \leq N}\norm{S_k - S_n}, \\\
        Y_n     &:= \lim_{N \to \infty} Y_{n,N} = \sup_{k > n} \norm{S_k - S_n}. 
    \end{align*}
    Seien $\varepsilon, t > 0$. Mit dem Cauchy-Kriterium für stochastische Konvergenz und Lévys Maximal-Ungleichung erhalten wir für $N > n \geq n_0 := n_0(\varepsilon,t) \in \N$
    $$
        P(\{ Y_{n, N} > t\}) \leq 2P(\{ \norm{S_N - S_n} > t\}) \leq \varepsilon . 
    $$
    Es folgt somit 
    $$
        P(\{Y_n > t \}) = \lim_{n \to \infty}P(\{Y_{n,N} > t \}) \leq \varepsilon
    $$
    für $n \geq n_0$. Also gilt $Y_n \stochastisch 0$, nach dem Cauchy-Kriterium für fast sichere Konvergenz folgt daraus die fast sichere Konvergenz von $(S_n)_{n \in \N}$. 
    \newline \underline{Fall B}: Allgemeiner Fall.
    \newline 
    Wir verwenden für diesen Fall die Beweistechnik der Symmetrisierung.  
    Betrachte hierzu den Produktraum $(\Omega \times \Omega, \mathcal{A}\otimes\mathcal{A}, P \times P)$. Für eine Zufallsvariable $X$ auf $(\Omega, \mathcal{A}, P)$ bezeichne
    $$
        \overline{X}(\omega_1, \omega_2) := X(\omega_1) - X(\omega_2), \quad (\omega_1, \omega_2) \in \Omega \times \Omega
    $$   
    die Symmetrisierung von $X$. Wie man mittels der charakteristischen Funktionale von $\overline{X}$ und $-\overline{X}$ leicht einsieht ist $\overline{X}$ tatsächlich symmetrisch. 
    Sei nun $S$ eine Zufallsvariable auf $(\Omega, \mathcal{A}, P)$ mit $S_n \stochastisch S$. 
    Dann folgt direkt $\overline{S_n} \stochastisch \overline{S}$, denn für $\varepsilon > 0$ gilt nach Konstruktion
    $$
        (P\times P)(\{ \norm{\overline{S_n} - \overline{S}} > \varepsilon \} ) \leq 2P(\{ \norm{S_n - S} > \frac{\varepsilon}{2} \}).
    $$
    Nach Fall A gilt also insbesondere $\overline{S_n} \fastsicher  \overline{S}$. Daher existiert eine Menge $\Omega^* \in \mathcal{A}\otimes\mathcal{A}$ mit \mbox{$(P\times P)(\Omega^*) = 1$} und
    $$  
        \forall (\omega_1, \omega_2) \in \Omega^*: \quad \overline{S_n}(\omega_1, \omega_2) \text{ konvergiert. }
    $$
    Mit dem Satz von Fubini erhalten wir
    $$
        0 = \int_{\Omega \times \Omega}1 - 1_{\Omega^*} d(P \times P) = \int_{\Omega}\int_{\Omega}1 - 1_{\Omega^*(\omega_2)}(\omega_1)dP(\omega_1)dP(\omega_2). 
    $$
    wobei $\Omega^*(\omega_2) =\{\omega_1\in\Omega: \ (\omega_1, \omega_2) \in \Omega^* \}$. Somit existiert ein $\omega_2 \in \Omega$ mit $P(\Omega^*(\omega_2)) = 1$ und es gilt
    $$
        \forall \omega_1 \in \Omega^*(\omega_2): \quad S_n(\omega_1) - S_n(\omega_2) \ \text{ konvergiert.}
    $$
    Setze nun $x_n := S_n(\omega_2)$, $n \in \N$. Dann existiert eine Zufallsvariable $L$ auf $(\Omega, \mathcal{A}, P)$ mit $S_n - x_n \fastsicher L$. Nach Voraussetzung erhalten wir also 
    $$
        x_n \stochastisch S - L,
    $$
    wobei wir $x_n$ für $n \in \N$ als konstante Zufallsvariable auf $(\Omega, \mathcal{A}, P)$ auffassen. Folglich existiert ein $x \in E$ mit 
    $$
        \lim_{n \to \infty}x_n = x
    $$
    und insgesamt erhalten wir $S_n \fastsicher L + x$. 
    \newline 
    Zu $(iii) \Rightarrow (ii)$: Für $1 \leq m < n$ setze
    \begin{align*}
        \mu_{m,n} :&= P^{S_n - S_m}
    \end{align*}
    Da $(\mu_n)_{n \in \N}$ nach Voraussetzung schwach gegen ein Wahrscheinlichkeitsmaß $\mu$ konvergiert
    ist die Menge $\{\mu_n: n \in \N \}$ relativ kompakt in $(\mathcal{M}(E), \rho)$ und somit nach dem Satz von Prokhorov gleichmäßig straff.
    Folglich existiert zu $\varepsilon > 0$ eine kompakte Teilmenge $K \subseteq E$ mit 
    $$
        \forall n \in \N: \quad \mu_n(K) \geq 1 - \varepsilon. 
    $$
    Wegen der Stetigkeit von 
    $$
        (x,y) \mapsto x - y, \quad (x,y) \in E \times E
    $$
    ist die Menge $\tilde{K} := \{x - y : x,y \in K \}$ wiederum kompakt, also insbesondere messbar, und es gilt
    \begin{align*}
        \mu_{m,n}(\tilde{K}) \geq P(\{S_n \in K, \ S_m \in K\}) \geq 1 - P(\{S_n \in K^c\}) - P(\{S_m \in K^c\}) \geq 1 - 2\varepsilon.
    \end{align*}
    Somit ist auch $\{\mu_{m,n}: m,n \in \N, m < n \}$ gleichmäßig straff und daher relativ kompakt in $(\mathcal{M}(E), \rho)$. 
    Wir zeigen nun
    \begin{align}
        \forall \varepsilon > 0 \ \exists N \in \N: \ \forall n > m \geq N: \quad \prob{\norm{S_n - S_m} < \varepsilon} = \mu_{m,n}(B(0,\varepsilon)) > 1 - \varepsilon.
    \end{align}
    Mit dem Satz $2.9$ folgt daraus die stochastische Konvergenz der Folge $(S_n)_{n \in \N}$. Angenommen $(3.6)$ ist nicht erfüllt, dann gilt
    \begin{align*}
        \exists \varepsilon > 0 \ \forall N \in \N: \ \exists n(N) > m(N) \geq N: \mu_{m(N), n(N)}(B(0, \varepsilon)) \leq 1 - \varepsilon.
    \end{align*}
    Da $\{\mu_{m,n}: m,n \in \N, m < n \}$ relativ kompakt ist existiert insbesondere eine Teilfolge von $(\mu_{m(N),n(N)})_{N \in \N}$ die schwach gegen ein Wahrscheinlichkeitsmaß $\nu$ auf $\mathcal{B}(E)$ konvergiert.
    Ohne Einschränkung können wir annehmen, dass bereits $\mu_{m(N),n(N)} \rightharpoonup \nu$ gilt. Da $B(0, \varepsilon)$ offen ist liefert das Portmanteau-Theorem
    \begin{align}
        \nu(B(0, \varepsilon)) \leq \liminf_{N \to \infty}\mu_{m(N),n(N)}(B(0,\varepsilon)) \leq 1 - \varepsilon. 
    \end{align}
    Andererseits gilt für $f \in E'$ wegen der Unabhängigkeit von $(X_n)_{n \in \N}$
    \begin{align*}
        \widehat{\mu_{n(N)}}(f) = \mathbb{E}(e^{if(S_{n(N)})})) &= \mathbb{E}(e^{if(S_{m(N)})}e^{if(S_{n(N)} - S_{m(N)})}) \\\
                                                   &= \mathbb{E}(e^{if(S_{m(N)})})\mathbb{E}(e^{if(S_{n(N)} - S_{m(N)})}) \\\
                                                   &= \widehat{\mu_{m(N}}(f)\widehat{\mu_{m(N),n(N)}}(f). 
    \end{align*}
    Mit Grenzübergang $N \to \infty$ folgt daraus wegen $\mu_N \rightharpoonup \mu$ und $\mu_{m(N),n(N)} \rightharpoonup \nu$
    \begin{align*}
        \forall z \in E': \quad \widehat{\mu}(z) = \widehat{\mu}(z) \widehat{\nu}(z).
    \end{align*}
    Wegen der Stetigkeit von $\widehat{\mu}$ und $\widehat{\mu}(0_{E'}) = 1$ existiert ein $r>0$ mit 
    \begin{align*}
        \widehat{\mu}(z) \neq 0, \quad \text{ für alle } z \in E' \text{ mit } \norm{z}_{op} \leq r. 
    \end{align*}
    Also gilt 
    \begin{align*}
        \widehat{\nu}(z) = 1, \quad \text{ für alle } z \in E' \text{ mit } \norm{z}_{op} \leq r. 
    \end{align*}
    Aus Proposition $2.19$ folgt nun $\nu = \delta_0$. Im Widerspruch zu $(3.7)$. Es gilt folglich $(3.6)$ und $(S_n)_{n \in \N}$ konvergiert demnach stochastisch. 
    \qed
\end{proof*}

\begin{remark}
    Mittels der Maximal-Ungleichung von Ottaviani-Skorohod ist auch ein direkter Beweis der Implikation $(ii) \Rightarrow (i)$ möglich. Betrachte hierzu etwa die Ereignisse 
    $$
        A_N := \bigcap_{m \in \N}\{\sup_{n > m} \norm{S_n - S_m} > \frac{1}{N}\}, \quad N \in \N.                                                                                                                        
    $$
    Dann ist $A := (\cup_{N=1}^{\infty} A_N)^c$ das Ereignis, dass $(S_n)_{n \in \N}$ eine Cauchy-Folge ist und man zeigt 
    $$
        \forall N \in \N: \quad P(A_N) = 0. 
    $$
    Ein Beweis mit dieser Vorgehensweise für den skalaren Fall findet sich etwa \cite{bauer}[Theorem 14.2, S.109]. Der allgemeine Fall funktioniert vollkommen analog. \qexampled
\end{remark}




\begin{theorem}
    Sei $(\mu_n)_{n \in \N}$ gleichmäßig straff. Dann existiert eine Folge $(c_n)_{n \in \N}$ in $E$ sodass $(S_n - c_n)_{n \in \N}$ fast sicher konvergiert.
\end{theorem}

\begin{proof*}
    Betrachte den Produktraum $(\Omega \times \Omega, \mathcal{A} \otimes \mathcal{A}, P \times P)$ und definiere 
    \begin{align*}
        \widetilde{X}_n&: \Omega \times \Omega \to E, \ (\omega_1, \omega_2) \mapsto X_n(\omega_1), \\\
        \widetilde{Y}_n&: \Omega \times \Omega \to E, \ (\omega_1, \omega_2) \mapsto X_n(\omega_2),
    \end{align*}
    sowie 
    \begin{align*}
        \widetilde{S}_n := \sum_{i = 1}^n \widetilde{X}_i, \quad \widetilde{T}_n := \sum_{i = 1}^n \widetilde{Y}_n, \quad U_n := \widetilde{S}_n - \widetilde{T}_n, \quad \mu_{U_n} := (P\times P)^{U_n}. 
    \end{align*}
    Nach Konstruktion sind $\widetilde{S}_n$,$\widetilde{T}_n$ und $S_n$ identisch verteilt. Wir zeigen zunächst, dass $(\mu_{U_n})_{n \in \N}$ gleichmäßig straff ist. 
    Sei dazu $\varepsilon > 0$ Dann existiert nach Voraussetzung eine kompakte Menge $K_{\varepsilon} \subseteq E$ mit 
    $$
        \forall n \in \N: \mu_n(K_{\varepsilon}) \geq 1 - \varepsilon. 
    $$
    Wegen der Stetigkeit der Abbildung 
    $$
        E \times E \to E, \ (x,y) \mapsto x - y
    $$
    ist auch die Menge $K := \{ x - y : x,y \in K_{\varepsilon} \}$ kompakt und somit insbesondere messbar. Ferner gilt
    \begin{align*}
        \mu_{U_n}(K) = \prob{\widetilde{S}_n - \widetilde{T}_n \in K} &\geq \prob{\widetilde{S}_n \in K_{\varepsilon}, \widetilde{T}_n \in \varepsilon} \\\
                                                              &\geq 1 - \prob{\widetilde{S}_n \notin K_{\varepsilon}} - \prob{\widetilde{T}_n \notin K_{\varepsilon}} \\\
                                                              &= 1 - 2\mu_n(K_{\varepsilon}^c) \\\
                                                              &\geq 1 - 2 \varepsilon.                                              
    \end{align*}
    Also ist $(\mu_{U_n})_{n \in \N}$ straff. 
    Als nächstes zeigen wir, dass $(\widehat{\mu_{U_n}}(f))_{n \in \N}$ für alle $f \in E'$ konvergiert. Sei dazu $f \in E'$ beliebig aber fest. 
    Die Unabhängigkeit von $(\widetilde{Y}_n)_{n \in \N}$ und $(\widetilde{X}_n)_{n \in \N}$ liefert direkt die Unabhängigkeit von $(\widetilde{X}_n - \widetilde{Y}_n)_{n \in \N}$
    und nach Konstruktion sind $Y_n$ und $X_n$ identisch verteilt. Also folgt 
    \begin{align*}
        \widehat{\mu_{U_n}}(f) = E(e^{if(U_n)}) = \mathbb{E}\big(\prod_{j=1}^n(e^{if(\widetilde{X}_j-\widetilde{Y}_j)})\big) &= \prod_{j=1}^n \mathbb{E}(e^{if(\widetilde{X}_j-\widetilde{Y}_j)}) \\\ 
                                                                                                            &= \prod_{j=1}^n \mathbb{E}(e^{if(\widetilde{X}_j)})\mathbb{E}(e^{-if(\widetilde{Y}_j)}) \\\
                                                                                                            &= \prod_{j=1}^n \abs{\mathbb{E}(e^{if(\widetilde{X}_j)})}^2
    \end{align*}
    Wegen $0 \leq \abs{E(e^{if(\widetilde{X}_j)})} \leq 1$ für alle $j \in \N$ folgt somit die Konvergenz von $(\widehat{\mu_{U_n}}(f))_{n \in \N}$. 
    Nach Kapitel 2 konvergiert $(U_n)_{n \in \N}$ also in Verteilung und somit nach dem Satz von Itô-Nisio insbesondere fast sicher. 
    Daher existiert eine Menge $\Omega^* \in \mathcal{A} \otimes \mathcal{A}$ mit $(P\times P)(\Omega^*) = 1$ und 
    $$
        \forall (\omega_1, \omega_2) \in \Omega^*: \quad U_n(\omega_1, \omega_2) = S_n(\omega_1) - S_n(\omega_2) \text{ konvergiert.}
    $$
    Mit dem Satz von Fubini erhalten wir wie im Beweis des Satzes von Itô-Nisio ein $\omega' \in \Omega$, sodass $S_n - S_n(\omega')$ fast sicher konvergiert. 
    Also erfüllt die Folge $(c_n)_{n \in \N}$ definiert durch $c_n := S_n(\omega')$, $n \in \N$, die gewünschte Eigenschaft. \qed

\end{proof*}

\begin{theorem}[Satz von Itô-Nisio für symmetrische Folgen]
    Sei $(X_n)_{n \in \N}$ eine Folge unabhängiger und symmetrischer Zufallsvariablen in $\mathcal{L}_0(E)$. Dann sind äquivalent
    \begin{enumerate}[(i)]
        \item $(S_n)_{n \in \N}$ konvergiert fast sicher, 
        \item $(S_n)_{n \in \N}$ konvergiert stochastisch, 
        \item $(S_n)_{n \in \N}$ konvergiert in Verteilung, 
        \item $(\mu_n)_{n \in \N}$ ist gleichmäßig straff, 
        \item Es gibt eine Zufallsvariable $S \in \mathcal{L}_0(E)$, sodass 
        $$
            \forall f \in E': \quad f(S_n) \stochastisch f(S),
        $$
        \item Es gibt ein Wahrscheinlichkeitsmaß $\mu$ auf $\mathcal{B}(E)$, sodass 
        $$
            \forall f \in E': \quad \lim_{n \to \infty}\widehat{\mu_n}(f) = \widehat{\mu}(f). 
        $$
    \end{enumerate}
\end{theorem}

\begin{proof*}
    Die Äquivalenz $(i) \iff (ii) \iff (iii)$ wurde bereits im allgemeinen Fall nicht-symmetrischer Zufallsvariablen gezeigt und die Implikationen $(iii) \Rightarrow (iv)$, $(i) \Rightarrow (v) \Rightarrow (vi)$ sind klar. 
    Wir zeigen noch $(vi) \Rightarrow (iv)$ und $(v) \Rightarrow (iv) \Rightarrow (i)$. 
    \newline
    zu $(iv) \Rightarrow (i)$:
    Nach Satz $3.8$ existiert eine Folge $(c_n)_{n \in \N}$ in $E$, sodass $(S_n - c_n)_{n \in \N}$ fast sicher konvergiert. Setze nun $P^X := P^{(X_1,X_2,...)}$ und $P^{-X} :=P^{(-X_1,-X_2,...)}$. 
    Wegen der Unabhängigkeit und Symmetrie von $(X_n)_{n \in \N}$ erhalten wir direkt $P^X = P^{-X}$. Weiter gilt für $N \in \N$ und $\varepsilon >0$
    \begin{align*}
        &\quad \ \prob{\sup_{n \geq N}\norm{(S_n - c_n) - (S_N- c_N)} > \varepsilon} \\\
                &= P^X(\{(y_n)_{n \in \N}\in E^{\N}: \  \sup_{n \geq N}\norm{y_n + y_{n-1} + ... + y_{N+1} + (c_n - c_N)} > \varepsilon \}) \\\
                &= P^{-X}(\{(y_n)_{n \in \N}\in E^{\N}: \ \sup_{n \geq N}\norm{y_n + y_{n-1} + ... + y_{N+1} + (c_n - c_N)} > \varepsilon \}) \\\
                &= \prob{\sup_{n \geq N}\norm{(-S_n - c_n) - (-S_N - c_N)} > \varepsilon}.
    \end{align*}
    Also konvergiert wegen dem Cauchy-Kriterium für fast sichere Konvergenz auch $(-S_n - c_n)_{n \in \N}$ fast sicher. Daraus folgt die fast sichere Konvergenz von $(S_n)_{n \in \N}$, denn für $n \in \N$ gilt
    $$
       S_n =  \frac{1}{2}((S_n - c_n) - (-S_n -c_n)). 
    $$
    zu $(v) \Rightarrow (iv)$: Wegen der Unabhängigkeit von $(X_n)_{n \in \N}$ sind für alle $f \in E'$ und $m \geq n$ die Zufallsvariablen $f(S_m - S_n)$ und $f(S_n)$ unabhängig. 
    Nach Proposition $2.18$ sind also $f(S-S_n)$ und $f(S_n)$ für alle $n \in \N$ unabhängig. Zusammen mit der Symmetrie von $S_n$ ergibt sich also für $f \in E'$
    \begin{align*}
        \widehat{P^S}(f) = \mathbb{E}(e^{if(S)}) &= \mathbb{E}(e^{if(S-S_n)})\mathbb{E}(e^{if(S_n)})  \\\
                                        &= \mathbb{E}(e^{if(S-S_n)})\mathbb{E}(e^{if(-S_n)}) = \mathbb{E}(e^{if(S-2S_n)}) = \widehat{P^{S-2S_n}}(f).
    \end{align*}
    Also sind $S$ und $S - 2S_n$ nach dem Eindeutigkeitssatz für alle $n \in \N$  identisch verteilt. Da $P^S$ straff ist existiert zu $\varepsilon >0$ eine kompakte Menge $K \subseteq E'$ mit $\prob{S \in K} \geq 1 - \varepsilon$. 
    Aus Stetigkeitsgründen ist auch die Menge $L := \{\frac{1}{2}(x-y): x,y \in K\}$ kompakt und es gilt für alle $n \in \N$
    $$
        \prob{S_n \in L} \geq \prob{S \in K, \ S-2S_n \in K} \geq 1 - \prob{S \notin K} - \prob{S-2S_n \notin K} \geq 1 - 2\varepsilon. 
    $$ 
    Also ist $(\mu_n)_{n \in \N}$ gleichmäßig straff. 
    \newline
    zu $(vi) \Rightarrow (iv)$: 
    Sei $f \in E'$ beliebig aber fest. Betrachte die Abbildungen 
    \begin{align*}
        \varphi_n &: \R \to \C, \quad t \mapsto \int_E e^{itf(x)}\mu_n(dx) = \widehat{\mu_n}(tf), \quad n \in \N, \\\
        \varphi   &: \R \to \C, \quad t \mapsto \int_E e^{itf(x)}\mu(dx) = \widehat{\mu}(tf). 
    \end{align*}
    Dann ist $\varphi_n$ für alle $n \in \N$ die charakteristische Funktion von $\mu_n^{f}$ und $\varphi$ die charakteristische Funktion von $\mu^f$. Nach Voraussetzung konvergiert $(\varphi_n)_{n \in \N}$ punktweise gegen $\varphi$ und nach 
    dem Stetigkeitssatz von Lévy, vgl. \cite{gs}[Satz 8.7.5, S.357], gilt also $\mu_n^f \rightharpoonup \mu^f$. Für festes $f \in E'$ ist $\{\mu_n^f : n \in \N\}$ also insbesondere relativ kompakt. Nach dem Satz von de Acosta genügt es folglich zu zeigen, dass $(\mu_n)_{n \in \N}$ flach konzentriert ist. 
    Da $\{\mu\}$ flach konzentriert ist genügt es dafür zu zeigen, dass für jeden endlich dimensionalen Untervektorraum $F \subseteq E$ und alle $\varepsilon > 0$ gilt 
    $$
        \mu_n((F^{\varepsilon})^c) \leq 2 \mu((F^{\varepsilon})^c).
    $$
    Sei also $F \subseteq E$ ein endlich dimensionaler Untervektorraum und $\varepsilon >0$. Nach dem Satz von Hahn-Banach existiert eine Folge $(f_n)_{n \in \N}$ in $E'$ mit 
    $$
        \forall x \in E: \quad d(x,F) := \inf_{y \in F}\norm{x-y} = \sup_{n \in \N}\abs{f_n(x)}. 
    $$
    Sei zunächst $m \in \N$ festgewählt. Mittels charakteristischer Funktionen prüft man leicht, dass $\mu_n^{(f_1,...,f_m)} \rightharpoonup \mu^{(f_1,...,f_m)}$. 
    Wegen der Linearität von $f_1,...,f_m$ können wir den Satz von Itô-Nisio auf die Folge $(T_n)_{n \in \N} := ((f_1,...,f_m)\circ S_n)_{n \in \N}$ anwenden und erhalten eine $\R^m$-wertige Zufallsvariable $T$ auf $(\Omega, \mathcal{A}, P)$ mit 
    $T_n \stochastisch T$. Insbesondere gilt also $T_n \schwach T$ und somit $P^T = \mu^{(f_1,...,f_,)}$. 
    Ferner erhält die Linearität von $f_1,...,f_m$ die Symmetrie und mit der Lévy Ungleichung für Grenzwerte in Verteilung angewendet auf $(T_n)_{n \in \N}$ erhalten wir im Banachraum $(\R^m, \norm{\cdot}_{\infty})$
    \begin{align*}
        \prob{\max_{1 \leq i \leq m}\abs{f_i(S_n)} > \varepsilon} &= \prob{\norm{T_n}_{\infty} > \varepsilon} \\\
                                                                  &\leq 2 \prob{\norm{T}_{\infty} > \varepsilon} = 2\mu\big(\{x \in E: \max_{1\leq i \leq m}\abs{f_i(x)} > \varepsilon\}\big)
    \end{align*}
    Es gilt folglich wegen der $\sigma$-Stetigkeit von Wahrscheinlichkeitsmaßen 
    \begin{align*}
        \mu_n((F^{\varepsilon})^c) = \prob{d(S_n, F) > \varepsilon} &= \lim_{m \to \infty}\prob{\max_{1 \leq i \leq m}\abs{f_i(S_n)} > \varepsilon} \\\
                                                                    &\leq \lim_{m \to \infty} 2 \mu(\{x \in E: \max_{1\leq i \leq m}\abs{f_i(x)} > \varepsilon\})
                                                                    = 2 \mu((F^{\varepsilon})^c).
    \end{align*}
    \qed
\end{proof*}

\begin{remark}
    Auf die Annahme der Symmetrie in Satz $3.10$ kann im Allgemeinen nicht verzichtet werden. Sei $E$ ein Hilbertraum mit Orthonormalbasis $(e_n)_{n \in \N}$.
    Nach dem Darstellungsatz von Riesz können wir den Dualraum $E'$ mit $E$ identifizieren. Setze nun 
    $$
        X_1(\omega) = e_1, \quad X_n(\omega) = e_n - e_{n-1}, \ n \in \N, \quad \omega \in \Omega. 
    $$
    Dann gilt offensichtlich $S_n = e_n$ und da $(e_n)_{n \in \N}$ eine Orthonormalbasis von $E$ ist gilt für alle $z \in E$
    $$
        \lim_{n \to \infty}\langle z,S_n \rangle = 0 = \langle z,S \rangle,
    $$
    wobei $S(\omega) = 0$ für alle $\omega \in \Omega$. Wegen $\norm{S_n} = \norm{e_n} =1$ für alle $n \in \N$ konvergiert $(S_n)_{n \in \N}$ aber weder fast-sicher noch stochastisch gegen $S$.
    \qexampled 
\end{remark}

Für Folgen $(a_n)_{n \in \N}$ in $[0, \infty)$ kennt man aus der reellen Analysis die Äquivalenz
$$
    \bigg(\sum_{i=1}^n a_i\bigg)_{n \in \N} \text{ konvergiert. } \iff \bigg(\sum_{i=1}^n a_i\bigg)_{n \in \N} \text{ ist beschränkt.}
$$
Mit Hilfe des Satzes von Itô-Nisio können wir nun ein ähnliches Resultat für symmetrische Zufallsvariablen formulieren. 

\begin{mydef}
    Eine Folge $(X_n)_{n \in \N}$ in $\mathcal{L}_0(E)$ heißt \textit{stochastisch beschränkt}, falls
    $$
        \forall \varepsilon >0 \ \exists R > 0: \quad \sup_{n \in \N}\prob{\norm{X_n} > R} < \varepsilon. 
    $$
\end{mydef}

\begin{corollary}
    Sei $d \in \N$ und $(X_n)_{n \in \N}$ eine unabhängige Folge $\R^d$-wertiger symmetrischer Zufallsvariablen. Dann sind äquivalent 
    \begin{enumerate}[(i)]
        \item $(S_n)_{n \in \N}$ konvergiert fast sicher.
        \item $(S_n)_{n \in \N}$ ist stochastisch beschränkt. 
    \end{enumerate} 
\end{corollary}
\begin{proof*}
    $(i) \Rightarrow (ii)$ ist klar. Aus $(ii)$ erhalten wir unmittelbar
    $$
        \forall \varepsilon > 0 \ \exists R > 0: \quad \inf_{n \in \N}\prob{S_n \in \overline{B}(0,R)} \geq 1 - \varepsilon.
    $$
    Da abgeschlossene und beschränkte Teilmengen des $\R^d$ nach dem Satz von Heine-Borel kompakt sind folgt daraus die gleichmäßige Straffheit von $(\mu_n)_{n \in \N}$. 
    Nach dem Satz von Itô-Nisio konvergiert $(S_n)_{n \in \N}$ also fast sicher. \qed 
\end{proof*}
Wir wollen nun noch eine zweite Anwendung von Satz $3.9$ geben, die Darstellung orientiert sich hierbei an \cite{li-queffelec}. 
Im Folgenden sei $(X_n)_{n \in \N}$ eine unabhängige Folge symmetrischer Zufallsvariablens aus $\mathcal{L}_0(E)$ und $(\lambda_n)_{n \in \N}$ eine beschränkte Folge in $\R$. 
Für $n \in \N$ setze
$$
    S_n := \sum_{i=1}^nX_i, \quad T_n = \sum_{i=1}^n\lambda_iX_i. 
$$
Aus technischen Gründen seien ferner $S_0 = T_0 = 0$. 
Unser Ziel ist es, zu zeigen, dass die fast sichere Konvergenz von $(S_n)_{n \in \N}$ auch die fast sichere Konvergenz von $(T_n)_{n \in \N}$ impliziert. 
Der Beweis beruht hauptsächlich auf der folgenden Abschätzung. 
\begin{lemma}
    Für alle $t > 0$ und $N \in \N$ gilt
    \begin{align}
        \prob{\norm{T_N} > t} \leq 2 \prob{\norm{S_N} > t}. 
    \end{align}
\end{lemma}

\begin{proof*}
    Wegen der Symmetrie von $X_n$ sind $\lambda_n X_n$ und $\abs{\lambda_n}X_n)$ für $n \in \N$ identisch verteilt. Ferner ist $(\lambda_n)_{n \in \N}$ nach Voraussetzung beschränkt. 
    Wir können also ohne Einschränkung annehmen, dass $0 \leq \lambda_n \leq 1$ für alle $n \in \N$. 
    \newline 
    \underline{Fall A}: $\lambda_1 \geq \lambda_2 \geq... \geq \lambda_N$. 
    \newline 
    Wie man leicht nachrechnet, gilt  
    $$
        T_N = \sum_{n=1}^N \lambda_n(S_n - S_{n-1}) = \sum_{n=1}^{N-1}(\lambda_n - \lambda_{n+1})S_n + \lambda_N S_N. 
    $$
    Mit der Dreiecksungleichung erhalten wir somit
    \begin{align*}
        \norm{T_N} &\leq \sum_{n=1}^{N-1}(\lambda_n - \lambda_{n+1})\norm{S_n} + \lambda_N \norm{S_N} \\\
                   &\leq \max_{1 \leq n \leq N}\norm{S_n} \big(\sum_{n=1}^{N-1}(\lambda_n - \lambda_{n+1}) + \lambda_N\big) \\\
                   &= \max_{1\leq n \leq N}\norm{S_n} \lambda_1 \leq \max_{1\leq n \leq N}\norm{S_n}.
    \end{align*}
    Mit Lévys Maximalungleichung $(3.1)$ folgt daraus
    $$
        \prob{\norm{T_N} > t} \leq \prob{\max_{1 \leq n \leq N}\norm{S_n} > t} \leq 2 \prob{\norm{S_N} > t}.
    $$
    \underline{Fall B}: Allgemeiner Fall. 
    \newline 
    Mittels einer Permutation $\sigma$ erhält man $\lambda_{\sigma(1)} \geq ... \geq \lambda_{\sigma(N)}$. Man beachte schließlich, dass 
    $$
        \sum_{n=1}^N\lambda_{\sigma(n)}X_{\sigma(n)} = T_N \quad \text{ und } \quad \sum_{n=1}^N X_{\sigma(n)} = S_N.
    $$ 
    Aus Fall A folgt nun die Behauptung. \qed
\end{proof*}

\begin{theorem}[Kontraktions-Prinzip, Qualitative Version]
    Falls $(S_n)_{n \in \N}$ fast sicher konvergiert, dann konvergiert auch $(T_n)_{n \in \N}$ fast sicher. 
\end{theorem}
\begin{proof*}
    Sei $\varepsilon > 0$. Nach Lemma $3.13$ gilt für $m < n$
    $$
        \prob{\norm{T_n - T_m} > \varepsilon} \leq 2\prob{\norm{S_n - S_m} > \varepsilon}. 
    $$
    Nach dem Cauchy-Kriterium für stochastische Konvergenz konvergiert $(T_n)_{n \in \N}$ somit stochastisch und nach Satz $3.8$ insbesondere fast sicher. \qed
\end{proof*}

\begin{remark}
    In Satz $3.15$ kann nicht auf die Symmetrie der Zufallsvariablen $X_1,X_2,...$ verzichtet werden. Betrachte dazu etwa die Folge $(X_n)_{n \in \N}$ definiert durch
    $$
        X_n(\omega) := (-1)^n \frac{1}{n},  \quad \omega \in \Omega, n \in \N, 
    $$ 
    und die beschränkte Folge $((-1)^n)_{n \in \N}$. Dann ist die Folge $(\sum_{k=1}^n X_n)_{n \in \N}$ fast sicher konvergent, aber die Folge $(\sum_{k=1}^n\lambda_k X_k)_{n \in \N}$ divergiert fast sicher. 
    \qexampled
\end{remark}
