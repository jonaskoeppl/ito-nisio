\section{Separabilit"at}
In diesem Abschnitt wollen wir einige nützliche Fakten über separable Vektorräume sammeln, die vor allem in Abschnitt 1.6 Anwendung finden. 
Wir beginnen zunächst mit ein paar grundlegenden Definitionen. Sei dazu $(X, \norm{\cdot})$ ein normierter Vektorraum. 

\begin{mydef}
    Eine Teilmenge $U \subseteq X$ heißt \textit{separabel}, falls eine abzählbare Menge $A \subseteq U$ mit $\overline{A} = U$ existiert.
\end{mydef}

\begin{mydef}
    Sei $A \subseteq X$. Dann heißt 
    $$
        lin(A) := \left\{\sum_{i=1}^n \alpha_i x_i \  | \ n \in \N, \alpha_1,...,\alpha_n \in \R, x_1,...,x_n \in A \right\}
    $$
    die \textit{lineare Hülle von $A$}. 
\end{mydef}

Eines der für diese Arbeit wichtigsten Resultate ist die folgende Charakterisierung separabler normierter Räume. Den Beweis findet man etwa in \cite[Lemma I.2.9]{werner}.

\begin{theorem}
    Es sind äquivalent:
    \begin{enumerate}[(i)]
        \item $X$ ist separabel.
        \item Es gibt eine abzählbare Menge $A \subseteq X$ mit $\overline{lin(A)} = X$. 
    \end{enumerate}
\end{theorem}

\begin{corollary}
    Sei $(E_n)_{n \in \N}$ eine Folge separabler Untervektorräume von $X$. Dann ist auch 
    $$
        \overline{lin(\bigcup_{n \in \N}E_n)}
    $$
    ein separabler Untervektorraum von $X$. 
\end{corollary}

\begin{proof*}
    Nach Definition existiert für alle $n \in \N$ eine abzählbare Teilmenge $A_n \subset E_n$ mit $\overline{A_n} = E_n$. Wir zeigen: 
    $$
    \overline{lin(\bigcup_{n \in \N}E_n)} = \overline{lin(\bigcup_{n \in \N}A_n)}. 
    $$
    Zu $\supseteq$: Klar, weil $\bigcup_{n \in \N}A_n \subseteq \bigcup_{n \in \N}E_n$. \newline 
    Zu $\subseteq$: Es reicht zu zeigen, dass
    $$
        lin(\bigcup_{n \in \N}E_n) \subseteq \overline{lin(\bigcup_{n \in \N}A_n)}.
    $$
    Sei also $y \in lin(\bigcup_{n \in \N}E_n)$. Dann existiert ein $N \in \N$, $\lambda_1,...,\lambda_N \in \R$ und $y_1,...,y_N \in \bigcup_{n \in \N}E_n$ mit 
    $$
        y = \sum_{k=1}^N\lambda_ky_k. 
    $$
    Durch eventuelles Umnummerieren können wir ohne Einschränkung annehmen, dass $y_k \in E_k$ für $k=1,...,N$. 
    Da $A_n$ für $n \in \N$ dicht in $E_n$ liegt, existiert für $k \in \{1,...,N\}$ eine Folge $(x_m^{(k)})_{m \in \N}$ in $A_k$ mit $\lim_{m \to \infty}x_m^{(k)} = y_k$. 
    Also ist die Folge $(\sum_{k=1}^N\lambda_k x_m^{(k)})_{m \in \N}$ in $lin(\bigcup_{n \in \N}A_n)$ und nach den Rechenregeln für Grenzwerte gilt 
    $$
        \lim_{m \to \infty}(\sum_{k=1}^N\lambda_k x_m^{(k)}) = \sum_{k=1}^N\lambda_ky_k = y. 
    $$
    Also gilt $y \in \overline{lin(\bigcup_{n \in \N}A_n)}$. 
    Insgesamt folgt also $\overline{lin(\bigcup_{n \in \N}E_n)} = \overline{lin(\bigcup_{n \in \N}A_n)}$ und da $\bigcup_{n \in \N}A_n$ abzählbar ist, liefert Satz $A.3$ die Behauptung. \qed
\end{proof*}

\begin{corollary}
    Sei $E \subseteq X$ eine separable Teilmenge von $X$. Dann ist auch $\overline{lin(E)}$ separabel. 
\end{corollary}
\begin{proof*}
    Nach Voraussetzung existiert eine dichte abzählbare Menge $A \subseteq E$. Wir zeigen 
    $$
        lin(E) \subseteq \overline{lin(A)}. 
    $$
    Zusammen mit $A \subseteq E$ und der Definition des topologischen Abschlusses einer Menge folgt daraus
    $$
        \overline{lin(A)} \subseteq \overline{lin(E)} \subseteq \overline{lin(A)}
    $$
    und somit die Behauptung. Sei also $x \in lin(E)$. Dann existieren $n \in \N, x_1,...,x_n \in E$ und $\lambda_1,...,\lambda_n \in \R$ mit
    $$
        x = \sum_{i=1}^n \lambda_i x_i. 
    $$ 
    Da $A$ dicht in $E$ liegt, existiert für jedes $i \in \{1,...,n\}$ eine Folge $(x_m^{(i)})_{m \in \N}$ in $A$ mit $\lim_{m \to \infty}x_m^{(i)} = x_i$. 
    Für alle $m \in \N$ gilt zudem $\sum_{i=1}^n \lambda_i x_m^{(i)} \in lin(A)$ und wegen 
    $$
        \lim_{m \to \infty} \sum_{i=1}^n \lambda_i x_m^{(i)} = \sum_{i=1}^n \lambda_i \lim_{m \to \infty}x_m^{(i)} = \sum_{i=1}^n \lambda_i x_i
    $$
    folgt schließlich $x \in \overline{lin(A)}$. \qed 
\end{proof*}

