\section{Der Satz von Hahn-Banach}
\textbf{Notation und Konventionen}\newline
Für einen normierten Vektorraum $(X, \norm{\cdot})$ sei $(X', \norm{\cdot}_op)$ der zugehörige Dualraum. Ferner bezeichne
$$
    B_{X'} := \{f \in X' \ : \ \norm{f}_op \leq 1 \}
$$
die abgeschlossene Einheitskugel in $X'$. 

\begin{mydef}
    Sei $X$ ein Vektorraum. Eine Abbildung $p:X \to \R$ heißt \textit{sublinear}, falls
    \begin{enumerate}[(a)]
        \item $p(\lambda x) = \lambda p(x)$ für alle $\lambda \geq 0, x \in X$,
        \item $p(x+y) \leq p(x) + p(y)$ für alle $x,y \in X$. 
    \end{enumerate}
\end{mydef}

\begin{proposition}
    Sei $X$ ein normierter Vektorraum und $F \subseteq X$ ein abgeschlossener Untervektorraum. Dann ist die Abbildung
    $$
        d(\cdot, F):X \to \R, y \mapsto d(y,F) := \inf_{x \in F}\norm{x-y}
    $$
    sublinear. 
\end{proposition}

\begin{proof*}
    Da $F$ ein Untervektorraum von $X$ ist, gilt für $\lambda \geq 0$ und $y \in X$ 
    $$
        d(\lambda y, F) = \inf_{x \in F}\norm{\lambda y - x} = \inf_{x \in F}\norm{\lambda y - \lambda x} = \lambda \inf_{x \in F}\norm{y-x} = \lambda d(y,F). 
    $$
    Ferner liefert die Dreiecksungleichung für $y,z \in X$
    $$
        d(y+z,F) = \inf_{x \in F}\norm{y+z - x} \leq \inf_{x \in F}(\norm{y - \frac{x}{2}} + \norm{z - \frac{x}{2}}) \leq \inf_{x \in F}\norm{y-x} + \inf_{x \in F}\norm{z-x} = d(y,F) + d(z,F). 
    $$
    Also ist $d(\cdot, F)$ sublinear. \qed 
\end{proof*}

\begin{theorem}[Hahn-Banach, Version der Linearen Algebra]
    Sei $X$ ein Vektorraum und $U$ ein Untervektorraum von $X$. Ferner seien $p: X \to \R$ sublinear und $f: U \to \R$ linear mit 
    $$
       \forall x \in U: \quad f(x) \leq p(x).
    $$
    Dann existiert eine lineare Fortsetzung $F:X \to \R$, mit $F|_U = f$ und
    $$
    \forall x \in X: \quad F(x) \leq p(x). 
    $$ 
\end{theorem}

\begin{theorem}[Hahn-Banach, Fortsetzungsversion]
    Sei $X$ ein normierter Raum und $U$ ein Untervektorraum. 
    Zu jedem stetigen linearen Funktional $f:U \to \R$ existiert ein stetiges lineares Funktional $F:X \to \R$ mit 
    $$
        F|_U = f \text{ und } \norm{F}_{op} = \norm{f}_{op}.
    $$
    Jedes stetige Funktional kann also normgleich fortgesetzt werden.
\end{theorem}

\begin{corollary}
    In jedem normierten Raum $X$ existiert zu jedem $x \in X$, $x \neq 0$, ein Funktional $f_x \in X'$ mit $\norm{f_x}_{op} = 1$ und $f_x(x) = \norm{x}$. 
    Speziell trennt $X'$  die Punkte von $X$, d.h.
    $$
        \forall x_1, x_2 \in X \text{ mit } x_1 \neq x_2 \ \exists f \in X': \quad f(x_1) \neq f(x_2). 
    $$
\end{corollary}

\begin{corollary}
    In jedem normierten Raum $X$ gilt 
    $$
        \forall x \in X : \quad \norm{x} = \sup_{f \in B_{X'}}\abs{f(x)}.
    $$
\end{corollary}

\begin{corollary}
    Falls $X$ separabel ist, so existiert eine abzählbare Menge $D \subseteq B_{X'}$ mit 
    $$
        \forall x \in X : \quad \norm{x} = \sup_{f \in D}\abs{f(x)}. 
    $$
\end{corollary}

\begin{proof*}
    Sei $E \subseteq X$ eine abzählbare dichte Teilmenge von $X$. Nach Korollar $A.10$ existiert für jedes $x \in E$ ein $f_x \in X'$ mit 
    $\norm{f_x}_{op} = 1$ und $f_x(x) = \norm{x}$. Setze also $D:=\{f_x : x \in E\}$. Sei nun $x \in X$ beliebig und $\varepsilon > 0$. Dann gilt zunächst 
    $$
        \forall n \in \N: \quad \abs{f_n(x)} \leq \norm{f_n}_{op} \norm{x} = \norm{x}. 
    $$
    Also
    $$
        \sup_{f \in D}\abs{f(x)} \leq \norm{x}. 
    $$
    Ferner existiert eine Folge $(y_n)_{n \in \N}$ in $E$ mit $\lim_{n \to \infty}\norm{x-y_n} = 0$. Insbesondere existiert $N \in \N$, sodass für alle $n \geq N$ 
    \begin{align}
        \norm{y_n - x} < \frac{\varepsilon}{2}.
    \end{align}
    Zudem gilt für alle $n \geq N$
    \begin{align}
        \abs{f_{y_n}(y_n)} \leq \abs{f_{y_n}(y_n)-f_{y_n}(x)} + \abs{f_{y_n}(x)} \leq \frac{\varepsilon}{2} + \abs{f_{y_n}(x)}.
    \end{align}
    Insgesamt ergibt sich also wegen $(A.1)$ und $(A.2)$ für $n \geq N$
    $$
        \norm{x} \leq \norm{x - y_n} + \norm{y_n} \leq \frac{\varepsilon}{2} + \abs{f_{y_n}(y_n)} \leq \abs{f_{y_n}(x)} + \varepsilon. 
    $$
    Da $\varepsilon > 0$ und $x \in X$ beliebig gewählt waren, gilt also 
    $$
        \forall x \in X: \ \norm{x} = \sup_{f \in D}\abs{f(x)}.  
    $$
\end{proof*}

\begin{corollary}
    Sei $X$ separabel und $F \subseteq X$ ein abgeschlossener Untervektorraum.
    Dann existiert eine Folge $(f_n)_{n \in \N}$ in $X'$ mit $\norm{f_n}_{op} \leq 1$ für alle $n \in \N$ und 
    $$
        \forall x \in X: \quad d(x,F) = \inf_{y \in F}\norm{x-y} = \sup_{n \in \N}\abs{f_n(x)}. 
    $$
\end{corollary}
\begin{proof*}
    Da $X$ separabel ist, existiert eine abzählbare dichte Teilmenge $D = \{x_1,x_2,...\} \subseteq X$. Sei nun $n \in \N$. Dann gilt $lin(\{x_n\}) = \{\lambda x_n: \lambda \in \R\}$ und die Abbildung
    $$
        g_n: lin(\{x_n\}) \to \R, \quad \lambda x_n \mapsto \lambda d(x_n,F)
    $$
    ist, wie man leicht nachrechnet, linear. Nach Proposition $A.7$ ist ferner die Abbildung 
    $$
        d(\cdot, F):X \to \R, y \mapsto d(y,F) := \inf_{x \in F}\norm{x-y}
    $$
    sublinear. Für $z = \lambda x_n \in lin(\{x_n\})$ gilt zudem 
    $$
        g_n(z) = \lambda d(x_n,F) \leq \abs{\lambda} \inf_{y \in F}\norm{x-y} = \inf_{y \in F}\norm{\lambda x - \lambda y} = d(z,F). 
    $$
    Also existiert nach Satz $A.8$ für alle $n \in \N$ ein Funktional $f_n \in X'$ mit $f_n|_{lin(\{x_n\})} = g_n$ und 
    $$
        \forall y \in X: \quad f_n(y) \leq d(y,F) = \inf_{x \in F}\norm{x-y} \leq \norm{y}. 
    $$
    Somit gilt insbesondere $\norm{f_n}_{op} \leq 1$. Betrachte nun die so gewonnene Folge $(f_n)_{n \in \N}$ in $X'$. Für $x \in X$ gilt dann nach Konstruktion
    $$
        \forall n \in \N: \quad f_n(x) \leq d(x,F). 
    $$
    Daraus folgt direkt
    $$
        \sup_{n \in \N}\abs{f_n(x)} \leq d(x,F).
    $$
    Sei nun $\varepsilon > 0$. Da $D$ dicht in $X$ liegt, gibt es ein $n \in \N$, sodass $\norm{x_n - x} < \frac{\varepsilon}{2}$. Daher gilt wegen $\norm{f_n}_{op} \leq 1$
    $$
        \abs{f_n(x_n)} \leq \abs{f_n(x_n) - f_n(y)} + \abs{f_n(y)} \leq \abs{f_n(y)} + \frac{\varepsilon}{2}. 
    $$
    Daraus folgt schließlich wegen der sublinearität von $d(\cdot,F)$
    $$
        d(x,F) \leq d(x-x_n,F) + d(x_n,F) \leq \norm{x-x_n} + \abs{f_n(x_n)} \leq \abs{f_n(y)} + \varepsilon. 
    $$
    Da $\varepsilon > 0$ beliebig gewählt war, folgt insgesamt
    $$
        \sup_{n \in \N}\abs{f_n(x)} = d(x,F).
    $$
    \qed
\end{proof*}
