\section{Der Satz von Hahn-Banach}
Die Beweise der folgenden Sätze findet man etwa in \cite{werner}[Theorem III.1.5]
\begin{theorem}[Hahn-Banach, Fortsetzungsversion]
    Sei $X$ ein normierter Raum und $U$ ein Untervektorraum. 
    Zu jedem stetigen linearen Funktional $u':U \to \R$ existiert ein stetiges lineares Funktional $x':X \to \R$ mit 
    $$
        x'|_U = u' \text{ und } \norm{x'}_{op} = \norm{u'}_{op}.
    $$
    Jedes stetige Funktional kann also normgleich fortgesetzt werden
\end{theorem}

\begin{corollary}
    In jedem normierten Raum $X$ existiert zu jedem $x \in X$, $x \neq 0$, ein Funktional $x' \in X'$ mit $\norm{x'}_{op} = 1$ und $x'(x) = \norm{x}$. 
    Speziell trennt $X'$  die Punkte von $X$, d.h.
    $$
        \forall x_1, x_2 \in X \text{ mit } x_1 \neq x_2 \ \exists x' \in X': \quad x'(x_1) \neq x'(x_2). 
    $$
\end{corollary}

\begin{corollary}
    In jedem normierten Raum $X$ gilt 
    $$
        \norm{x} = \sup_{x' \in B_{X'}}\abs{x'(x)}, \quad x \in X. 
    $$
\end{corollary}

\begin{corollary}
    Falls $X$ separabel ist so existiert eine abzählbare Menge $D \subseteq B_{X'}$ mit 
    $$
        \norm{x} = \sup_{x' \in D}\abs{x'(x)}, \quad x \in X. 
    $$
\end{corollary}

\begin{proof*}
    \textbf{TODO}
\end{proof*}

\begin{corollary}
    Sei $X$ separabel und $F \subseteq X$ ein endlich-dimensionaler Untervektorraum.
    Dann existiert eine Folge $(f_n)_{n \in \N}$ in $X'$ mit $\norm{f_n}_{op} \leq 1$ für alle $n \in \N$ und 
    $$
        \forall x \in X: \quad \inf_{y \in F}\norm{x-y} = \sup_{n \in \N}\abs{f_n(x)}. 
    $$
\end{corollary}
\begin{proof*}
    \textbf{TODO}
\end{proof*}
